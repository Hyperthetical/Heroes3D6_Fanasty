%Copyright (C) 2021-2023  Geoff Beck
%
%This document is open-source: you can redistribute it and/or modify
%it under the terms of the GNU General Public License as published by
%the Free Software Foundation, either version 3 of the License, or
%(at your option) any later version.
%
%This program is distributed in the hope that it will be useful,
%but WITHOUT ANY WARRANTY; without even the implied warranty of
%MERCHANTABILITY or FITNESS FOR A PARTICULAR PURPOSE.  See the
%GNU General Public License for more details.
%
%See https://www.gnu.org/licenses/ for more details

\documentclass[a4paper,11pt,oneside]{book}

\usepackage{fullpage}
\usepackage{amsmath}
\usepackage{import}
\usepackage{tocbibind}
\usepackage[bookmarks=true,plainpages=false]{hyperref}
\usepackage{titletoc,titlecaps}
\usepackage{caption}
\usepackage[a4paper,margin=2cm]{geometry}

\usepackage[T1]{fontenc}
\usepackage{lmodern}


\newcommand{\dicediffbase}{9}
\newcommand{\dicenoprof}{-1}
\newcommand{\dicecritlvl}{4}

\newcommand{\textlf}[1]{\textbf{\titlecap{#1}}}
\newcommand{\textlfirst}[1]{\textbf{\textit{\titlecap{#1}}}}

\title{\textbf{\huge Heroes3D6\\Fantasy role-playing rules add-on}}
\author{Geoff Beck}
\date{}

\begin{document}
\maketitle
\frontmatter
\tableofcontents
\mainmatter

%\chapter{The world and the `Last City'}
%The world of Krell is a devastated wasteland. An event in the distant past commonly referred to as the `Catastrophe' unleashed vast quantities of warping magical energies that laid waste to cities and ecosystems across the planet. The cause of this disaster is not clearly remembered, however, it has left a long lasting suspicion of magic in the few survivors that cling to this wrecked world. Over time the lingering magic, called `the residual', has produced new mutant life-forms by warping the small surviving populations. The `Last City' is the only major settlement in existence, built after the Catastrophe and housing peoples of all species, it teeters constantly the brink of collapse as a result of supply shortages as well as internal tensions. What holds the city together is the shelter it provides from the residual, this being a great wall built of a patchwork of `untainted' iron. This being a metal of great value, often used as currency outside the city, as it survived the Catastrophe without being infused with dangerous residual energies and now repels these same influences.    
%
%Small settlements dot the wastelands outside the Last City, these struggle constantly against the dangers of residual, mutant wild-life, bandits, and the terrifying changekin. These last are feral people twisted by the residual energies into nightmarish monsters. Life outside the city is a constant struggle to find food and water that are untainted by the residual, as too much exposure can eventually twist one into a changekin.
%
%\section{The last city}
%The only bastion of civilisation in a blighted world, the Last City stands as monument to the endurance of the people of Krell. Surrounded by a patch-work wall of untainted iron the city is largely protected from the warping power of the residual. Entrances to the city are guarded and those too contaminated by the residual are denied entrance, especially as a common punishment for crimes is being exposed to contamination and cast out of the city. Such is the suspicion of magic and its users that all mages must be registered and are branded with forehead marking, those who refuse are cast out as criminals. The mage brand brings much suspicion upon its bearers with common folk often spitting in the street at their passing, and some being unwilling to even do business with them. Often children with magical talent are disowned by their families and turned out onto the streets. 
%
%The city is ruled over by council of exclusively human `founding families', constituting a ruling aristocracy whose power is enforced by a small soldier corps. Low-level crime is endemic in the city, as it contains many desperate people of all species, and criminal gangs act as the defacto rulers of some districts and trades. The existence of the city is precarious, as it depends strongly on food grown in surrounding settlements, which are constantly under pressure from raiding bandits, changekin, and the residual itself. Despite this, it is the most sure shelter against the terrors of the wastelands.
%
%Within the city coins are used as currency but the alternative of untainted iron is also widely accepted (it is often the only currency used in the wastelands). A coin-sized disk of this metal is exchanged for 3 gold coins, such is its value in warding off the effects of residual when travelling outside the city.   
%
%\section{The changed and the changekin}  
%The changed are a phenomenon that began after the Catastrophe. They are beings that have been warped by the residual, and exhibit anything from minor changes like scales or feathers on their skin, up to limbs or heads being a mixture of animal and human (bird claws, goats hooves, beaks, animal jaws). The changed are, however, still people and although often viewed with suspicion and prejudice they are not attacked on sight. 
%
%Changekin are somewhat like the changed but have become feral, their minds rotted away by the residual. Their mutations tend to be even more extreme, often multiple additional limbs, or large bloated bodies. Changekin are savage and attack anything on sight, except for other changekin (most of the time). Such creatures form into roaming hunting bands that seek out anything weak enough to be preyed upon, attacking travellers or small settlements if their band grows large enough. Changekin seldom wield weapons or wear armour, their minds are too damaged for such rationality.  
%
%
%\section{The residual}
%The residual is remnant warping energy left over from the Catastrophe. Exposure to it is dangerous as it causes damage to exposed tissues and can, with sufficiently high doses, twist the bodies of the exposed in monstrous forms. The residual is undetectable to normal senses except in very high concentrations, where it feels like a heated crackle in the air (these concentration levels are immediately dangerous as well). Untainted iron is commonly used to ward off the effects of the residual and skilled alchemists can produce potions brewed with it that can protect from, or cleanse a body of, the residuals effects. 
%
%Whenever exposed to a residual infused environment (or consuming tainted food/drink) a creature must make a \textlf{resolve} check against a \textlf{difficulty} set by how saturated the area is. For example low contamination is \textlf{difficulty} 8, dangerous is 11, and deadly is 14. If the creature fails they gain a Residual point (\textlf{critical failure} adds extra points). The number of points is added to the \textlf{difficulty} of subsequent checks against residual contamination. At 5 points the creature begins to feel light-headed, weak, and sweaty (has \textlf{edge-} on all actions). At 10 points a creature dies. Residual points last until removed. Armour made of untainted iron grants an \textlf{edge} on all such \textlf{resolve} checks and weapons of this metal have \textlf{edge+} to \textlf{damage checks} against changekin. 
\chapter{Introduction}
This set of rules supplements the core rules for \textit{Heroes 3D6} by providing the rules for playing in a fantasy setting of my own invention. This rule book is open-source~\footnote{Copyright (C) 2021-2023  Geoff Beck}: you can redistribute it and/or modify
it under the terms of the GNU General Public License as published by
the Free Software Foundation, either version 3 of the License, or any later version (see \url{https://www.gnu.org/licenses/}).


\chapter{The highlands of Atla}
This area of land (within the world of Krell) is entirely mountainous, with the major city of Stormheight occupying the tallest peak in the region. Smaller towns and villages also dot surrounding valleys. This terrain is rich in aether, with great deposits of crystals being mined in many of the taller peaks. This richness of aether has lead to Stormheight being a seat of technological advancement. In turn, the study of aether has become an integral part of its powerful religious structures.  

\section{The structure of a year}
A year on the world of Krell is referred to as a `prime cycle' due to the fact that Krell orbits a binary star system. This means their seasons have two inter-linked cycles. The prime cycle tracks the world's motion around the twin suns (taking roughly 320 days). The minor cycle is a result of the alignment of the suns with respect to Krell (this has a period of around 20 days). Thus, prime seasons contain a 20 day cycle of minor seasons. For example: prime summer will contain periods of both prime and minor summer (peak summer) as well as periods where the minor cycle is in winter (low summer). Each prime season on Krell contains a full cycle of minor seasons. This means the weather can vary quite drastically on the time scale of 20 days. Note that Krell has no moons, so the minor cycle is their analogue of a month. A date is conventionally structured by the number of the prime cycle, the prime season, and the day number within the season (1-80). For example: the 42nd day of prime winter in cycle 457.

\section{Climate and ecosystem}
The highlands are generally cold and stormy. Winds lash the peaks all year round and powerful storms occur regularly in any season. During peak winter there are howling gales and snow storms that often isolate mountain valleys. The rest of prime winter tends towards chill winds with these often developing into powerful thunderstorms that can scour the peaks. Prime summer has slightly milder temperatures and a steady drizzle of rain. Rain is far less common, and the weather warmer, during peak summer. 

This is a difficult climate, as such the majority of native plant life is lichen or scrub. The fauna of the highlands are well adapted to their harsh and variable conditions. Krell's low gravity (60\% of Earth's) means that many inhospitable places, like the highlands, are dominated by warm-blooded creatures that would be arthropods (invertebrates with exoskeletons) except that many have both exo and endoskeletons. For larger creatures, these skeletons tend to be composed of a hybrid or bone and a chitin-like polymer with far greater strength. A common highland sight are small herds of `skippers', metre-long grass-hopper-like grazers. These are preyed upon by species of large, pack-hunting, spiders (up to 30 cm across) and `skulkers', long, many-legged, low-slung, and covered in jointed armour. These have a body-size between 60 cm and 4 m, based on age, and are fierce predators with powerful, and venomous, biting mandibles up to 60 cm long. The highlands host many varieties of crab, mainly herbivorous or scavengers. Giant stone crabs, for example, measure around 3 m across and are clad in thick armour plates that are nearly as tough as steel. These are omnivores, being mainly herbivorous but content to scavenge carcasses or eat particularly stupid/slow smaller creatures. A vast array of small arthropod-like creatures fill almost all the niches of the ecosystem. For instance there are no rats in the highlands, but there are rat-sized pests called `skrik', which look like lobsters and are prolific scavengers, even being willing to eat crab chitin in lean times. There is also an array of migratory bird-life that visit the highlands during prime summer. 

\section{Religion and society}
Stormheight itself is home to the Synod of Inquiry, an organisation dedicated to the reverence of the natural world as a matchless system of harmony (the religion itself is known as The Harmony). The synod is extremely influential, being the ruling body of the city in all but name. The heart of this religious body is a grand campus, where all study of the natural world and its aether is conducted in a suitably reverent manner. Needless to say, scholarship and aethermancy outside of this body is illegal. The rationale for this being that unrestricted inquiry might lead to disastrous abuses of nature. Under the Harmony, the duty of thinking beings is to match themselves to the harmony of nature, part of this is by achieving ``inner harmony'', the other being in harmony with society and the world. To better achieve this, everyone in Stormheight is allocated a role and occupation according to the assessment of the synod. Those who cannot find harmony with their allocated position are confined to asylums or exiled. The influence of the synod is significantly lesser in the countryside, where folk festivals tend to match the religious calender of the Harmony, but the people aren't really invested in the wider system. Their only encounter with the synod being a once per prime cycle arrival of tax men, along with assessors who decide on the future occupations of the village youth. The Harmony's symbol is a field of countless, mutually-interlocking rings.

\subsection{Class and wealth}
Stormheight has no explicit aristocracy, as the structure of the synod is meant to be meritocratic. However, this tends not to be true in practice, with many influential families rather over-represented in the upper echelons of the synod. Any such familial influence cannot be wielded openly, due to the avowed commitment to merit before birth. This means that politics between families plays out in the decisions and policies of the synod itself. 

Stormheight is spatially segregated, with those of higher prestige within the Harmony living closer to the peak of the mountain (and the grand campus of the synod). Since the Harmony is so inter-woven into every aspect of society, those who rise high within its ranks are also those who accrue the most wealth. The upper ranks of city are populated by those who are given roles of `leadership' or `organisation' by the synod assessors. These are coveted roles, as they are viewed as being entrusted with coordinating the maintenance of harmony between elements of society. As one rises in the Harmony by displaying prowess in ones' given duties, there are a large number of well-off skilled workers and professionals who occupy the middle slopes of the city. The lower slopes are densely populated slums and rigorously policed by the Keepers of Order (town guards). Here dwell those who are given manual labour as a life-path by the synod assessors. This offers little scope for advancement. Thus, the slums are saturated with the dirt and crime that follow in the wake of such dead-end living.   

\subsubsection{Outlying villages}
The towns and villages that dot the highlands are agriculturally focussed. They have grown vastly over the last century due to the development of aethertech, as this has made agriculture far more viable in the damp, cold, and stormy highlands. The smaller villages survive by trading agricultural products with merchants from stormheight. The larger towns have more diverse economies, a result of their increasingly large populations. This also means that such settlements have attracted more interest from the synod, having appointed governors and bureaucrats. 

\subsubsection{Highlanders}
The term `highlander' is reserved for those who still follow the semi-nomadic lifestyle of the pre-aethertech highlands. They survive by foraging and hunting in the harsh wilderness of the peaks. The highlanders do not follow the Harmony, instead having a culture based around reverence for their ancestors. This means they are persona non grata as far as the synod are concerned. Despite this, highlanders are often closely linked with small villages, being welcome there when the synod isn't watching and often inter-marrying with the village folk.   

Highlanders are organised into clans, each having a territorial area for foraging and hunting. These areas are fluid and change as clans become more or less powerful as well as during inter-clan conflicts. Clans are lead by two people, one man and one woman (they need not be either related or married). These leaders are the symbolic parents of the entire clan (which is regarded as a kind of extremely extended family). Both leaders have an equal role in the governance of the clan and in dispute resolution. Clan leaders are elected, every few prime-cycles, by popular vote.  

The highlanders refer to themselves as the ``atladarra'', or ``people of the spear''. To them the spear represents the chief values of their culture: defending and providing for oneself and the clan. Every highlander owns at least a ceremonial spear that is marked with various designs and ornaments to reflect their personal history and achievements. Highlanders look upon the use of weapons that are designed purely for war as cowardly, the dual use of provision and battle being equally essential to them. Their preferred mode of combat is spear and shield, with either a bow, javelins, or a sling for use at range. The highlands mineral resources are either buried deep, or found on the highest and most inhospitable peaks. Thus, highlanders tend to use the more available resource: crab chitin. Giant stone crabs in particular provide a material more flexible and nearly as strong as steel. Spear heads, however, are always made of metal. Older highlanders shun aethertech and its by-products, they believe aether is the material that composes the spirits of their ancestors (aethermancy is therefore blasphemous). This opposition has been slowly eroding among younger clan members as the utility becomes increasingly apparent.  



\section{Living in the highlands}
This section will provide some information about how people in the highlands get by day to day.

\subsection{What people grow and eat}
The highlands grow barley and grapes as staple food crops, with lichen and moss being cultivated as animal feed. Dietary protein tends to be sourced from wild bird-life in summer, as well as both hunted and raised crabs. Meat from cattle or pigs is expensive and imported. Wine is the beverage of choice, it is mostly drunk watered as it is produced with a high alcohol content to aid in preservation. 

\subsubsection{Currency}
The currency in the highlands is called the `dinar'. These are bronze coins, a lower denomination called `cestis' (made of copper) is also widely used. There are 100 cestis (c) to a dinar (d). To give an idea of how much the coins are worth you can note that manual labourers make 15 - 25 c per day. A solider receives a salary of 45 c per day, this is similar to a more skilled labourer. 

The currency is not backed by the value of the coins themselves, as in many other lands. Rather, the value of the dinar is controlled and guaranteed by the synod itself. As such, all coins are stamped with a symbol of the Harmony. 

\subsubsection{Cost of living}
A loaf of bread, that would constitute a staple food, costs around 12 copper cestis (12 c). A cheaper alternative being edible varieties of lichen (nutritious, but barely flavourful), costing about 4 c for a similar amount to a loaf of bread. A cup of wine ranges in price from 3 c at the low end to around 30 c at the high end. Finally, a hot meal goes for 10 to 40 c, depending on the tavern. Renting an inn room for a night costs between 30 and 60 c. Inns and taverns tend slightly towards the more expensive end of the price ranges in Stormheight itself.  

\begin{table}[ht!]
	\centering
	\begin{tabular}{|l|l|}
		\hline
		Edible lichen loaf & 4 c\\ 
		Loaf of bread & 12 c \\
		Tavern meal & 10 - 40 c \\
		Inn room (per night) & 30 - 60 c \\
		Brandy (per cup) & 7 - 12 c \\
		Wine (per cup) & 3 - 30 c \\
		\hline		
	\end{tabular}
\caption{Prices of common goods}
\end{table}

\subsubsection{Trade with the outside world}
The highlands trade extensively with the outside world. Trade routes are hard going, as one must negotiate many winding mountain passes. This makes imported goods expensive for common folk. Imported food, for instance, is usually four times as expensive as local fare, rising even further when winter makes trade routes impassable. 

  



\chapter{Aether}
Aether is the term used to describe the natural energy that flows throughout the world. All creatures are attuned to aether to some extent, but study and practice can allow one to directly manipulate their own aether, as well as draw it in from their environment. Aether manipulation can create a variety of effects, either aligned to the natural forces or altering the aetheric fields of living creatures.

Aether is thought to be composed of fields, with the physical world being the result of many thousands of such fields interacting. Aetheric fields pulse and vibrate with energy, in turn they are characterised by the frequency of their vibrations. This frequency seems directly linked to the function of the field. For instance, some naturally occurring iron has a particular vibrational frequency that pulls other metal towards it, adding aether to this frequency increases the strength of this effect. A skilled aethermancer could even `flip' the effect so that the metal is pushed away instead. The ability to reverse field functions was conjectured by philosophers to result from a `symmetry of fundamental opposites' that remains a central pillar of aethermantic studies. New vibrations can even be added to existing fields, thus one could even cause a piece of wood to attract metal. An aethermancer performs these feats using their own aetheric field as a tuning fork for other fields, first initiating a vibration in their own field and then imposing it on another field of interest.

Aethermancy is difficult, as projecting and controlling one's own aetheric field is non-trivial and requires extensive practice. Note, however, that anyone can become an aethermancer, just as they could become a swordsman with enough training. An aethermancer is just someone who has learned the skill to actually control the interactions of their body's own fields. 

The constant interaction of aetheric fields means that an aethermancer's control is always tenuous. They can easily lose control of processes they attempt to change or initiate, sometimes with surprising results. Rules for aethermancy will be discussed in Section~\ref{sec:aethermancy}.

\section{Aether points}
Every character's body has a maximum of 1 aether point at any one time. Any character can spend this on the \textlf{aether surge} ability. Holding additional points requires a device called an aether capacitor (see Section~\ref{sec:capacitor}).

\subsection{Aether surge}
The character concentrates their aether to grant themselves a burst of power. At a cost of 1 aether point they may gain \textlf{edge+} on a roll of their choice.

\subsection{The energy of life}
A character's body uses aetheric energies to function, if they run out of aether points they can spend \textlf{endurance} in place of it.

\subsection{Recharging aether}
A one hour rest restores 1 aether point to any character that had none remaining. A hearty meal also restores up to 1 aether point. A full night's sleep restores up to 3 aether points. A character cannot gain aether if they would exceed their allowed maximum.

\subsection{Aether crystals}
Sometimes aether crystallises of its own accord. These crystals can be consumed by anyone to gain aether points (which can be in excess of their normal limit). The number gained depends on the strength of the crystal. In addition to this, a character consuming a crystal experiences a rush of life and enhanced sensations for 10 minutes.
\subsubsection{Crystal jitters}
The consumption of aether crystals is addictive, withdrawal manifesting in the form of uncontrolled shaking motions throughout the victim's body. Each time a character uses a crystal they gain crystal points equal to the gained aether. Crystal points decay at a rate of 1 per week. A character with any crystal points must make a \textlf{resolve} check each day vs 9 + crystal points. If they fail, they are overcome with a need to acquire and use more crystals. This means they have \textlf{edge-} on all checks until they consume a crystal.   




\chapter{Character creation}
These are additional rules providing the options for character species and additional backgrounds in this fantasy setting.

\section{Character species}

\subsection{Humans}
Humans are a dynamic species: adaptive, resilient, and highly cooperative.

Humans are the most prevalent species in the highlands of Atla, making up the majority of the rural population. Stormheight, however, is considerably more diverse, with lots of peoples from other species both resident and passing through.

The average human lives for up to 80 prime cycles, 30 being considered a mature adult, though humans are considered adults at the age of 16 in most societies. Despite lacking the patience or long practice of the Elves, they make up for it with a natural ability to quickly adapt to new situations or tasks. 

The elves' popular view of humans is that they are flighty and lacking in focus. Mus-folk stereotype humans as grim and unscrupulous. Trogs stereotype humankind as deceitful and selfish.  

\subsubsection*{Flexibility}
Humans are extremely adaptable and may spend an extra 2 points when choosing \textlf{perks} and \textlf{proficiencies} at character creation.


\subsection{Elves}
Though elves somewhat resemble humans at first glance, there is a world of difference on closer examination. Firstly, elves are incapable of digesting meat. They are vegetarian with a preferred diet of sweet, sugary things. Second, their manner of reproduction is completely different, elven eggs gestate externally in a similar manner to those of fish. Additionally elven blood is clear, as it contains no iron, and their hearts are far smaller than that of a man, as capillary action plays a major role in their circulatory systems.

Elves have high cheek bones and very narrow, angular features. Their skin varies in shade between green and light brown with a similar texture to smooth tree bark. Their eyes range in colour from yellow, through green and blue to white, the pupils of which are narrow and cat-like. Elves are short by human standards, and are also of a slighter build. All of them posses the curious ability to alter their hair colour at will. However, if they are not careful it will change on its own to reflect their mood. 

Elves' lives and metabolisms move more slowly than those of humans, and they are `saplings' till the age of about 40. They grow far more slowly than other species and tend only to reach their full growth by 60 prime cycles. This slowness of biology heavily influences the perspective of elves. They are more relaxed and far slower to decision and judgement than the frenetic, hasty pace they observe in other species. Additionally, they only need to eat or sleep once every two or three days. Elves are common in Stormheight and are often enthusiastic followers of the Harmony. 

Humans tend to stereotype elves as slow, pedantic, and more interested in pleasurable living than work. Mus-folk and Trogs popularly assert elves to be crazy and lacking in any kind of sense. 

\subsubsection*{Perfectionists}
Elves live long lives and thus have time to pursue their skills to a point few others can reach. As such, elves may choose one additional skill \textlf{proficiency}.


\subsection{Mus-folk}
The mus-folk (or `Musmus' in their own language) resemble large rats. They reach a maximum height/length of 3 - 4 ft and tend to inhabit the margins of human settlements where they are regarded as a mixture of pest and second-class citizen (rude terms for them include `squeakers' and `ratties'). Their own language consists of chirps and squeaks rather than words. However, they are fully capable of producing human and elvish speech. When they do, it comes out in a rapid torrent of words infused with poor grammar and often features repetition when they are excited. Mus-folk prefer underground dwellings and make burrows throughout the highland peaks, with a significant population in Stormheight itself. Their prolific digging means they gravitate towards mining for minerals and aether. They are often stereotyped as light-fingered, insular, and mischievous by other species.  

\subsubsection*{Sneak \& squeak}
Mus-folk are naturally stealthy. They can always choose proficiency in \textlf{stealth}, even if it does not fit their background otherwise. In addition, they have the \textlf{cunning plan perk} (add \textlf{cunning} to \textlf{power} for \textlf{sneak attacks}.)

\subsubsection*{Dark-dwellers}
Mus-folk ignore the penalties associated with \textlf{low light} conditions.

\subsubsection*{Second-class citizens}
Mus-folk will often experience prejudicial treatment from humans and elves. 


\subsection{Trogs}
Trogs are large (usually 7 ft tall) and bulky, but their most striking feature is their large eyes. Trog facial features tend to be heavy, with wide square jaws and large beak-like noses. Trogs have tough greyish, slightly scaly skin and are in general quite difficult to kill or injure. Traditionally, trog groups are familial and organised around a mother trog, a very large and formidable female (female trogs tend to be larger and stronger than males), with a harem of males who care for the young trogs. Plenty of young trogs find this authoritarian system stifling, so they form their own more egalitarian, but heavily community oriented, bands which tend to migrate towards settlements like Stormheight. This form of living is scorned and belittled by their elders. Trogs typically prefer dark places, due to their sensitive eyes. This means they tend to live in close proximity to the Mus-folk, often forming a symbiotic system where Trogs provide the muscle and Mus-folk the cunning.

Other species have a stereotyped view of Trogs as stupid and violent (Mus-folk view them as endearingly simple).  

\subsubsection*{Tough as nails}
Trogs are extremely durable and thus have + 1 \textlf{endurance}. 

\subsubsection*{Dark-dwellers}
Trogs ignore the penalties associated with \textlf{low light} conditions.

\subsubsection*{Second-class citizens}
Trogs will often experience prejudicial treatment from humans and elves.



\section{Starting equipment}
A new character may choose one piece of armour and up to three weapons from Table~\ref{tab:start-gear}. All characters get a set of clothes. Default starting money is 2 d, this can be increased for some backgrounds (suggested maximum 6 d).
\begin{table}[ht!]
	\centering
	\caption{New adventurers starting gear.}
	\label{tab:start-gear}
	\begin{tabular}{|l|l|l|l|l|}
		\hline
		Weapon & Power & Hands &  Lethality & Notes\\
		\hline
		Rusty sword & +1/- & 2/1 & N & -\\
		Corroded knife & - & 1 & N & Small\\
		Notched axe & +1/- & 2/1 & N & -\\
		Worn crossbow & +1 & 2 & N & Reload 2, Range 3\\
		Pitted hammer & - & 2/1 & N & Penetration 2/1 \\
		Aged spear & - & 2/1 & N & Rending 1\\
		Creaky short bow & - & 2 & N & Range 2\\
		Shabby pistol & - & 1 & N & Penetration 1,Reload 1, Range 2 \\
		Ramshackle musket & +1 & 2 & N & Penetration 1, Range 4, Reload 2 \\
		Dented shield & - & 1 & - & Deflect +1 \\
		\hline
	\end{tabular}
	\begin{tabular}{|l|l|l|l|}	
		\hline
		Armour & Toughness & Type & Notes\\
		\hline
		Battered breastplate and mail & +4 & Heavy & - \\
		Rusted mail hauberk & +2 & Medium & - \\
		Tattered gambeson & +1 & Light & - \\
		Travelling clothes & - & - & - \\
		\hline
	\end{tabular}
\end{table}


\section{Additional backgrounds}

\subsection{Student of the synod}
The character has spent time studying aether and its powers under the auspices of synod in Stormheight. An important aspect of this background is the degree to which the character believes in the Harmony and its social order. As individuals who stand high up in the heirachy, they may be tempted to justify the status quo based on their own success. Skill suggestions: Aethermancy, Mechanical, Aethertech, Infusary, History, Religion, Plants, Animals.

\subsection{Savant}
The character has learned some of the mysteries of aether on their own. Such individuals often hide their talents from the synod for fear of being forced into conformity or having to leave their home village. You should consider why the character hides their abilities, as well as their attitude to the synod and Harmony. Skill suggestions: Aethermancy, Infusary, Aethertech, Survival, Persuade, Decieve. 

\subsection{Exile}
The character has been exiled by the synod for repeated defiance of the Harmony's dictates. They now skulk on the margins of society, keeping away from the eyes of law enforcement in the far-flung rural districts of the highlands. Some such individuals find sanctuary among the Mus-folk, who have little regard for the Harmony or synod, others turn to less than legal ways of making a living in the aethertech black-market. You should decide how your character has got by and why they were exiled. Skill suggestions: Stealth, Disguise, Deceive, Survival, Slight of hand, Aethertech, Awareness.  

\subsection{Highlander}
The character is belongs to the partially nomadic people who form the original inhabitants of the highlands. They have lived a difficult life in the harsh environments of the highlands, constantly needing to avoid the synod's agents (due to religious differences). A highlander will belong to a given tribe whose hunting grounds are staked out over certain region (the exact boundaries are fluid and often fought over by the tribes). Highlander society values strength in the form of the ability to provide and defend. In the past the tribes have shunned aethertech but some younger highlanders have begun to adopt this new technology. An important aspect of this background is the character's attitude to highland villagers who live under the synod's rule. Are they pathetic weaklings or just highlanders who have lost their way? Additionally, do they follow the old ways or are they interested in aethertech? Skill suggestions: Survival, Athletics, Awareness, Ride, Animal, Plant.  


\section{Additional convictions}

\subsection{The Harmony of all}
The character is firmly convinced of the teachings of the Harmony. They live their life in pursuit of balance within themselves and with society as a whole. They tend to regard the synod and its laws as an essential ingredient in achieving harmony with the natural world.

\subsection{Free spirit}
The character detests the enforced conformity that the Harmony brings with it. Having these views publicly known can be dangerous. 

\subsection{Aethermancy unlimited}
Aethermancy and aethertech are forces for good, their exploration brings nothing but enlightenment. Some radical individuals might also hold that it should be free from the synod's interference. 

\subsection{Aether exploited}
Aethermancy and aethertech go too far in their rampant consumption of the world's aether. One cannot achieve the claimed Harmony while mercilessly extracting resources.



\section{Character templates}
Here we provide a few templates for character building. This does not involve any elements of personality or characterisation, just suggestions that provide characters that can build towards certain combat roles. 

\subsection{The commander}
The commander is a front-line fighter who uses their leadership skill to empower allies. Their choice of armour makes them hard to shift and a shield negates the \textlf{deflect} penalty on \textlf{heavy armour}. The attribute choice reflects a forceful personality who is quick of thought and attentive to detail. This  enhances their leadership and endurance, as well as making them harder to trick and better at landing hits in combat. Their \textlf{perk} choice lets them use their \textlf{leadership} for free whenever they defeat a foe in combat. They would aim to acquire the \textlf{perk} \textlf{Inspiring oratory} to make them ultra-effective as a leader. They could also build out \textlf{perks} to increase their combat prowess like \textlf{shield bash} and its upgrades.

\begin{tabular}{|l|l|}
	\hline
	Name & Commander\\
	Weapons & Rusty sword and shield\\
	Armour & Battered breastplate and mail\\
	Skills & Leadership\\
	Proficiencies & Swords\\
	Perks & Deeds not words\\
	Attributes & M 0, W 1, C 0, R 2\\  
	\hline
\end{tabular}

\subsection{The gunslinger}
The gunslinger aims to rapidly eliminate their foes in a hail of gun shots. They are lightly armoured as their skills lend themselves towards ending fights quickly and being free to get the jump on slower targets. The attributes are chosen to reflect a sly and quick-witted personality. This also maximises accuracy, skill synergy, and avoiding being hit (\textlf{cunning} can also enhance \textlf{sneak attacks}). Their strategy would be to use \textlf{quick-draw} to unload all three pistols, one at a time, before having to do any reloading. They could advance the build in several ways: lean into \textlf{sneak attacks} with the \textlf{shoot first perk} and \textlf{cunning plan}, go for two pistols at a time with \textlf{Two-weapon fighting proficiency}, or choose utility like \textlf{too hot to handle}, \textlf{arrow to the knee}, or \textlf{rifle drill}. The \textlf{evasive perk} is also a great choice for increased defences.
 
\begin{tabular}{|l|l|}
	\hline
	Name & Gunslinger\\
	Weapons & 3 Shabby pistols\\
	Armour & Tattered gambeson\\
	Skills & Awareness, Stealth\\
	Proficiencies & Firearms\\
	Perks & Quick-draw\\
	Attributes & M 0, W 1, C 2, R 0\\  
	\hline
\end{tabular}

\subsection{The aethermancer}
The aethermancer focuses on aetheric manifestations to control the battlefield. They carry a musket for when they run out of aether points, or do not wish to spend them. Their choice of \textlf{medium armour} means some protection without the downsides of \textlf{heavy armour}. \textlf{discharge} was chosen as their manifestation but there are a wide variety of options here. Their attributes are chosen to maximise the effectiveness of their manifestations (but will also benefit their weapon attacks). This build could be made more defensive by moving a point to \textlf{cunning} or \textlf{resolve}. Further experience points would be invested into new manifestations (see Section~\ref{sec:manifestations}).

\begin{tabular}{|l|l|}
	\hline
	Name & Aethermancer\\
	Weapons & Ramshackle musket\\
	Armour & Rusted mail hauberk\\
	Skills & Aethermancy\\
	Proficiencies & Firearms, Aetheric learning\\
	Perks & Discharge\\
	Attributes & M 1, W 2, C 0, R 0\\
	\hline  
\end{tabular}

\subsection{The highlander}
The highlander comes from a nomadic people that travel the peaks with their herded animals. They have chosen a pole weapon for versatility (and simplicity to fit the background). They come from a warrior culture, so their armour is strong without being encumbering. Their attributes reflect warrior aggression and the tenacity of a nomadic people. Finally, \textlf{berserker} is chosen to represent the fury with which they fight to defend their people. The build could be continued by purchasing upgrades to \textlf{berserker} or with things like \textlf{reckless attack} or \textlf{power attack} to get the most out of each swing.

\begin{tabular}{|l|l|}
	\hline
	Name & Highlander\\
	Weapons & Aged spear\\
	Armour & Rusted mail hauberk\\
	Skills & Survival, Animal\\
	Proficiencies & Pole weapons\\
	Perks & Berserker\\
	Attributes & M 2, W 0, C 0, R 1\\  
	\hline
\end{tabular}




\chapter{Perks and proficiencies}

\section{General proficiencies}
These perks do \textbf{not} occupy equipment slots for passive or active \textlf{perk}, their effect is always active.


\subsection{Aetheric learning [3]}
(Requires \textlf{skill proficiency: aethermancy}). This may only be chosen at character creation. This grants the character an aether capacitor that can hold 1 aether point (see Section~\ref{sec:capacitor}). Additionally, they may choose one aether manifestation with a cost of 3 experience or less.


\subsection{Weapon proficiency: X [2]}
In this setting, X can be drawn from: unarmed, swords, daggers, axes, blunt weapons, pole weapons, extended weapons, bows, crossbows, slings, and firearms. A character who is not \textlf{proficient} with his equipped weapon has \textlf{edge-} on \textlf{aim} and \textlf{damage checks}.


\section{Combat styles}
These open up new bonuses to different modes of combat specific to this setting, they must occupy an equipment slot for active/passive \textlf{perks} to be usable.

\subsection{Aetheric savant [3]}
\textlf{Passive perk}. The character can spend an aether point to gain the benefits of any \textlf{perk} for 1 turn.


\subsection{Archer's rhythm [4]}
\textlf{Passive perk} (Requires \textlf{Weapon proficiency: Bows}). If the character hits an enemy with a ranged attack from a bow, their next such action has $+1$ \textlf{aim}.


\subsection{Expansive knowledge [6]}
\textlf{Passive perk} (Requires \textlf{Skill proficiency: aethermancy}). The character can equip 2 aether manifestations per \textlf{active perk} slot.


\subsection{Hammer time [4]}
\textlf{Passive perk} (Requires \textlf{Weapon proficiency: Blunt}). \textlf{Critical hits} from blunt weapons inflict \textlf{knock down} on the victim.


\subsection{Lethal flourish [3]}
\textlf{Passive perk} (Requires \textlf{Weapon proficiency: Swords}). The character delivers their finishing blows with the point of a blade. While wielding a sword, the character's attacks gain \textlf{Rending} 1 against victims with missing \textlf{endurance}.


\subsection{Loose aether [2]}
\textlf{Passive perk}. The character has learned to regather dissipating aether when a manifestation unravels on them. If the character fails to produce an aether manifestation, they do not expend the aether point. This does not include \textlf{critical failures}.

\subsubsection{Upgrade: Aetheric frugality [2]}
(Requires \textlf{loose aether}). \textlf{loose aether} now applies to \textlf{critical failures} as well.


\subsection{Main gauche [3]}
\textlf{Active perk} (Requires one of \textlf{Two-weapon fighting}, \textlf{pole position}, or \textlf{unarmed proficiency}). There is an art to pairing weapons for fighting with both hands, this character has mastered it. When dual-wielding, the character may elect to use a close-combat weapon defensively. The chosen weapon may not make \textlf{damage checks} this round, but grants \textlf{edge+} on their next \textlf{deflect}. This also applies when fighting unarmed or using the \textlf{pole position perk}.


\subsection{Momentum [4]}
\textlf{Passive perk} (Requires \textlf{Weapon proficiency: Axes}). When wielding an axe, the character gets \textlf{cleave} + 1 on their next attack after scoring a \textlf{critical hit}.


\subsection{Multi-tasker [5]}
\textlf{Passive perk} (Requires \textlf{Skill proficiency: aethermancy}). The character has learned to concentrate on many things at once. This allows them to maintain two \textlf{persistent} effects at the same time.


\subsection{One in the sleeve [4]}
\textlf{Active perk} (Requires \textlf{Weapon proficiency: daggers}). If the character causes damage with a dagger, they can draw and throw an additional one at the cost of one reaction point (this effect is not recursive).

\subsubsection{Upgrade: Two in the face [2]}
(Requires \textlf{One in the sleeve}). If the character spends X reaction points on \textlf{one in the sleeve}, their dagger throw makes X \textlf{damage checks}. 


\subsection{Pole position [3]}
\textlf{Passive perk} (Requires \textlf{Weapon proficiency: Pole weapons}). The character may use a quarter staff (or other pole weapon) to make an extra attack with the haft. Thus, gaining an extra \textlf{damage check} (with no weapon bonuses apart from Trip and Disarm). Note that \textlf{dual-wielding} penalties apply when using this \textlf{perk}.


\subsection{Reach of the aether [3]}
\textlf{Passive perk} (Requires \textlf{Skill proficiency: aethermancy}). Aetheric manifestations have + 1 range.


\subsection{Reloading drill [4]}
\textlf{Passive perk} (Requires \textlf{Weapon proficiency: crossbow}). The character is well practised at rapidly preparing crossbows to fire. This reduces crossbow \textlf{reload} action costs by 1, to a minimum of 1.


\subsection{Rifle drill [4]}
\textlf{Passive perk} (Requires \textlf{Weapon proficiency: Firearms}). Firearms \textlf{reload} action costs are reduced by 1, to a minimum of 1.


\subsection{Shield bash [2]}
\textlf{Passive perk}. The character may use a shield to \textlf{shove} (add the \textlf{deflect} bonus to such checks).

\subsubsection{Upgrade: Reactive bash [2]}
(Requires: \textlf{Shield bash}) When the character \textlf{deflects} an attack, while using a shield, they can use \textlf{shield bash} at the cost of 1 reaction point.

\subsubsection{Upgrade: Shield combo [4]}
(Requires: \textlf{Shield bash}) The character may use \textlf{shield bash} for free as part of an \textlf{all-out attack}.


\subsection{Signature manifestation [4]}
\textlf{Passive perk} (Requires \textlf{Skill proficiency: aethermancy}). The character may choose one aetheric manifestation that they know. This has \textlf{edge+} on \textlf{aethermancy} checks when manifesting it. This can only be chosen once by a character.


\subsection{Yeoman bowman [6]}
\textlf{Passive perk} (Requires \textlf{Weapon proficiency: bows}). Removes the \textlf{reload} time from longbows. 




\section{Aether manifestations}
\label{sec:manifestations}

All manifestations are \textlf{active perks}. See section~\ref{sec:aethermancy} for rules on how to produce manifestations. Experience costs are listed in square brackets [], whereas \textlf{aethermancy} difficulty is given in round brackets (). A P indicates a \textlf{persistent} manifestation.


\subsection{Aether siphon (10) [2]}
An aethermancer can establish a resonance with another creature such that the aether is stripped from its body. For 2 action points (but no aether points) a chosen target within range 0 (5 m) must \textlf{resist(C)} or lose an aether point. If the target has no aether points, it loses an \textlf{endurance} instead (this cannot be increased by \textlf{critical success}). If the aethermancer wins the \textlf{resist} check, they gain 1 aether point, failure incurs a \textlf{moment of weakness}.
\subsubsection{Upgrade: Voracious siphon [4]}
The action point cost of \textlf{aether siphon} is reduced to 1.  


\subsection{Aetheric wall (10,P) [3]}
For 1 action point you manifest a wall of shimmering aether around a radius of 5 m (1 combat area) within a range of 2 (25 m). Passing through the wall costs any creature 1 action point and requires a \textlf{resist(M)} check, failure means they cannot traverse the wall and suffer damage with \textlf{normal lethality} (\textlf{critical success} for the aethermancer on the \textlf{resist} increments the damage severity). This lasts for 10 minutes and is a \textlf{persistent} manifestation.


\subsection{Animal aether (7) [2]}
Every living creature hums with an aetheric field. By carefully tuning your own aetheric field into resonance you can hear the mood and emotions of animals. This lasts for up to 1 hour.
\subsubsection{Upgrade: Animal resonance [2]}
Your mastery of the aetheric fields of animals allows you to communicate emotions to them as well as attempt to \textlf{persuade} or \textlf{deceive} them while \textlf{animal aether} is active.


\subsection{Animate homunculus (9) [2]}
You can shape a small creature/object out of any given material (30 cm is the maximum size), this is then imbued with aether (you need only make the aethermancy check when creating a homunculus). This grants it a kind of limited life, it can be controlled directly by you (this requires your full concentration). Otherwise it remains inert. A homunculus has no skill \textlf{proficiencies} and cannot inflict damage in combat.
\subsubsection{Upgrade: Automaton [3]}
Your homunculi are always animated and will obey simple verbal commands. You may endow a homunculus with a single \textlf{skill proficiency} on creation.
\subsubsection{Upgrade: Golem (+3) [4]}
The maximum size, for a single homunculus, is increased to medium (man-sized) creatures/objects, these can make unarmed attacks in combat (with \textlf{weapon proficiency}). 


\subsection{Arcing aether (10) [3]}
You mould your aether into a resonance with elemental lightning. For 2 action points, aetheric lighting leaps from your body and strikes a target within range 2 (25 m). The target must \textlf{resist(R)} or suffer damage with \textlf{normal lethality}. The lightning then leaps randomly to a second target within range 1 of the first (preferring those nearer by), and a third within range 1 of the second. Any individual target can only be struck once by this manifestation.
\subsubsection{Upgrade: Resonant arcing [4]}
The lightning gains an extra leap every time it causes damage and targets can now be struck more than once (but cannot be struck twice in a row).
\subsubsection{Upgrade: Paralytic arcs [2]}
Victims damaged by this manifestation are \textlf{immobilised} in their next turn.


\subsection{Befuddle (10) [2]}
A discordant aetheric projection removes your target's ability to distinguish friends from foes. If the target fails a \textlf{resist(W)} check, they regard all creatures as hostile and dangerous. This means they can panic, run away, or lash-out at anything nearby. The GM should choose combat targets and behaviour for the victim randomly. This lasts up to 1 minute. Each round, after the first, the victim may re-attempt to \textlf{resist}.
\subsubsection{Alternate: Bewildering barrage (12,P) [3]}
\textlf{Befuddle} becomes a \textlf{persistent} effect. Thus, if the victim \textlf{resists} they do not experience the effects for that round. On subsequent rounds they must continue to \textlf{resist} until the effect is ended.


\subsection{Bend light (10,P) [3] }
\label{spell:bend-light}
Aether flows from your fingers creating an aetheric lattice that bends light. At a distance up to range 2 (25 m), you can create a stationary illusion up to medium size. Anyone looking at this effect must \textlf{resist(W)} to decide if they are fooled. This is a \textlf{persistent} effect. The illusion is only visual, it makes no sounds, smells, and cannot be touched. 
\subsubsection{Upgrade: Major illusion [3]}
You can bend light with such dexterity that you can create illusions up to 5 m in size. 


\subsection{Blinding ray (8) [2]}
\label{spell:blind-ray}
You pour aether into a light source you touch, focusing its illumination into a searingly bright beam. The beam strikes a chosen 5 m radius (1 combat area). If any victim fails a \textlf{resist(R)} check, it is \textlf{blind} until the end of its next turn. Additionally, the normal lighting from the light source is removed for the next round, but the target area is fully illuminated for this time. The blindness lasts 1 additional round per level of \textlf{critical success} for the aethermancer on \textlf{resist}.


\subsection{Clinging aether (11) [3]}
For 1 action point you blast a 5 m radius (1 combat area) within a range of 1 (15 m) with crackling aether that anchors everything to the earth. All creatures in the area must \textlf{resist(C)} or be \textlf{immobilised} for their next two turns (1 additional turn per level of aethermancer \textlf{critical success} on the \textlf{resist} check).


\subsection{Compel (10) [3]}
You project aether into a target by touch, forcing them to make an action chosen by you if they fail a \textlf{Resist(W)} check. This fails if the action would harm the target. Specify one extra action per level of \textlf{critical success} for the aethermancer on \textlf{resist}. The victim is aware that they are being forced to act.
\subsubsection{Upgrade: Dominate [3]}
Your control is so strong that Compel will not fail if the action would be harmful to the victim itself.
\subsubsection{Upgrade: Project aether [2]}
You can now manifest this power at range 1.


\subsection{Concussive wave (9) [2]}
Your aether blasts out in a wave of concussive force. This \textlf{knocks back} all adjacent creatures and objects, size large or smaller, if they fail \textlf{resist (M)}. 
\subsubsection{Upgrade: Violent concussion [3]}
Victims \textlf{knocked back} are also \textlf{knocked down}. If they enter \textlf{rough terrain} or \textlf{cover} as a result of the \textlf{knock back}, they suffer damage with \textlf{normal lethality}.


\subsection{Consuming darkness (10,P) [3]}
Light has its own aetheric field and your aether can be used to create a counter vibrating field that extinguishes lights. All light disappears within a chosen radius of 1 (around 15 m). Creatures within the region fight using the \textlf{dark} lighting rules. This lasts until you cancel it.


\subsection{Damping field (7,P) [2]}
\label{spell:arcane-armour}
For 1 action point you envelope yourself in a high-pressure field of aether that depletes the force of incoming attacks. This manifests as a shimmering in the air around you and increases your \textlf{toughness} by 1. This manifestation lasts for 10 minutes and is a \textlf{persistent} effect.
\subsubsection{Upgrade: Ablation [2]}
When \textlf{damping field} is active you can spend a reaction point, when hit by an attack, to add an additional 1+\textlf{Wit} to your \textlf{Toughness} until the attack is resolved.


\subsection{Discharge (8) [2]}
\label{spell:sorc-blast}
This costs 1 action point and projects your aether out in a searing jet. A chosen target within range 2 (25 m) must make two \textlf{resist(C)} checks or suffer damage with \textlf{normal lethality} for each failure (\textlf{critical success} for the aethermancer on \textlf{resist} increments damage severity). 
\subsubsection{Upgrade: Resounding blast [2]}
This manifestation now inflicts \textlf{knocked down} if it causes damage. This only applies to \textlf{large creatures} or smaller.


\subsection{Earthquake (11) [4]}
You flood the ground with aether, making it roll and buck as though it was alive. For 2 action points you make the ground violently shake. All creatures adjacent to you must \textlf{resist(M)} or be knocked down (you are unaffected). This can be maintained as a \textlf{persistent} effect for up to 1 minute. If the aethermancer moves this manifestation moves with them.
\subsubsection{Upgrade: Project aether [2]}
You can now manifest this power at range 2 (25 m), instead of shaking the ground adjacent to you. You can change the target area on subsequent turns (once per turn only) for 1 action point.


\subsection{Embrace of iron (9) [3]}
The aethermancer tunes the aetheric field of a chosen object within range 2 (25 m) to repel/attract metal. Any creature adjacent to the target must \textlf{resist(M)} or be \textlf{disarmed} of any held metal objects each turn. Dropped metal objects are pushed/pulled a distance 1 away/towards the object. If attract was chosen, then there is \textlf{edge+} on \textlf{aim} to hit it with metal implements. The reverse applies for repel. Creatures in \textlf{medium} or \textlf{heavy armour} count the area adjacent to the target as \textlf{rough terrain}.


\subsection{Eyes of aether (7) [2]}
You can alter your senses to perceive flows of the aether. This renders you blind, instead you view the world as the flow of the aether. This costs no aether points to use and may be cancelled at any time.


\subsection{Fearful frequencies (8) [2]}
You bombard a target, within range 2 (25 m), with aetheric vibrations, causing a cloud of fear to pass across their mind. If they fail a \textlf{resist(R)} check, they cannot approach the aethermancer, or remain adjacent to them, for 1 minute. Each round, after the first, the victim may re-attempt the \textlf{resist(R)}. 
\subsubsection{Upgrade: Terrible tremors [3]}
You have learned to afflict the mind with absolute terror. The target of this manifestation suffers \textlf{edge-} to all rolls while this effect persists. 
\subsubsection{Upgrade: Frightful field [2]}
\textlf{Fearful frequencies} now affects all selected targets within a chosen combat area (5 m radius). 



\subsection{Fiery surge (10) [3]}
You can fuel a flame via aether infusion, greatly increase the intensity of an existing fire within range 2 (25 m). Any creature that makes contact with such a flame (i.e. enters or is in the same combat area) must \textlf{resist(C)} or suffer damage with \textlf{normal lethality}. This fire burns for 2 rounds before going out. This manifestation sets all flammable material within radius 0 (5 m) on fire.
\subsubsection{Upgrade: Ravenous flames [3]}
The fires burn with greater hunger, gaining \textlf{burst} + 1.
\subsubsection{Alternate: Explosive emanation [4]}
This increases the action point cost of the manifestation by 1. Instead of intensifying a fire you can cause it to expend all its aether at once in an explosive blast of radius 0 (5 m). Creatures in the blast must make a \textlf{resist(C)} check. Failure inflicts damage with \textlf{crushing lethality} (\textlf{lethality} scales with levels of \textlf{critical success} for the aethermancer).  


\subsection{Flare (10) [2]}
\label{spell:flare}
Projecting aether suddenly into a flame within range 2 (25 m), you cause it to burst into in blinding white flash. Anyone who can see the flare is \textlf{blind} until the end of the next round, and 1 additional round per level of \textlf{critical failure} on \textlf{resist(M)}. The explosion of the flare itself is harmless.
\subsubsection{Upgrade: Burning flare [3]}
Flare ignites all adjacent creatures, who suffer damage with \textlf{normal lethality} if they failed \textlf{resist(M)} (\textlf{critical success} for the aethermancer on \textlf{resist} increments damage severity).


\subsection{Freezing eruption (11,P) [4]}
An icey aetheric field creates a vortex of intense cold. For 2 action points you choose a region of radius 5 m radius (1 combat area) within a range of 2 (25 m) and freezing shards of ice erupt from the ground. Creatures within the area must \textlf{resist(R)} or suffer damage with \textlf{normal lethality}. For 10 minutes the area is \textlf{rough terrain}, anyone entering or leaving the area must \textlf{resist(R)} or suffer damage \textlf{normal lethality}. This is a \textlf{persistent} effect.
\subsubsection{Upgrade: Sub-zero [2]}
The cost of this manifestation is reduced by 1 action point.
\subsubsection{Upgrade: Glacial creep [3]}
In each subsequent round that the manifestation persists, you may choose an additional 5 m radius within range 1 of a frozen area to freeze over. All of these expire after 10 minutes.


\subsection{Glacial ray (10) [3]}
This costs 2 action points. A ray of freezing aether strikes a target within range 2 (25 m). They must \textlf{resist(C)}, failure means they are \textlf{immobilised} and \textlf{vulnerable} until damaged. 
\subsubsection{Upgrade: Creeping frost [3]}
\textlf{Glacial ray} applies its effects to a different target after it ends on the initial target.
\subsubsection{Upgrade: Deep-freeze [2]}
After the effect of \textlf{Glacial ray} ends, the ice shatters causing all creatures adjacent to the target take a hit with \textlf{power} +0 and \textlf{normal lethality}.


\subsection{Grip of the ground (9,P) [3]}
The aetheric field of objects can be adjusted so they respond more or less strongly to the pull of ground beneath their feet. If used to reduce a target's weight then their movement speed increases by 1 and they gain \textlf{edge+} on \textlf{deflect}. Such targets also have \textlf{damage checks} from falling made with \textlf{edge-}. 

If used to increase an objects weight, they must succeed on a \textlf{resist(M)} check each turn or be \textlf{immobilised}. If the aethermancer achieves \textlf{critical success} then the target suffers damage with \textlf{normal lethality}. Such targets also suffer \textlf{edge+} on \textlf{damage checks} from falling.

This lasts a maximum of 10 minutes.


\subsection{Guided motion (9) [3]}
Aether streams out from you to enhance the movements of another body within range 2 (25 m). This confers \textlf{edge+} to an allied target's \textlf{Athletics} or \textlf{deflect} checks until they fail such a check (maximum 1 hour).
\subsubsection{Alternate: Inhibit motion [2]}
Guided motion can target an enemy and confer \textlf{edge-} instead, which ends if they pass a penalised check.


\subsection{Ignite (8) [2]}
\label{spell:ignite}
Your aether can generate a resonance between your targets and the element of fire. This manifestation can be used on all chosen objects/creatures in a radius 0 region within range 2. The fire created by \textlf{Ignite} also suffers from normal physical restrictions, i.e. you may not set fire to a creature unless it is naturally flammable, covered in oil, or circumstantially vulnerable. Materials like metals can be heated by this effect to burn their bearer (this disarms any damaged victims). Setting fire to the clothes of a foe does no great harm to them, and they can extinguish the fire for 1 action point on their own turn.


\subsection{Illuminate (7) [2]}
\label{spell:illuminate}
You imbue an object with aether, causing it to resonate with the aetheric field of light. A single small object you touch begins to emit a soft aetheric glow. This provides low-light illumination over a single combat area (around a 5 m radius). This lasts for up to 6 hours.


\subsection{Iron arm (9,P) [4]}
You touch a creature with an aetheric field that resonates with the musculature of their body. All chosen creatures, within range 0 of you, gain +1 \textlf{power}. This is a \textlf{persistent} effect that lasts up to 1 minute.


\subsection{Ley of the land (9) [2]}
You can attune your own aether to the fields within the terrain around you, allowing effortless avoidance of obstructions. This manifestation allows the aethermancer to ignore natural forms of \textlf{rough} or \textlf{dangerous terrain} effects as well as \textlf{edge+} on \textlf{stealth} and \textlf{awareness} checks while in wilderness. This lasts for 1 hour.


\subsection{Mimic sound (10) [2]}
You subtly leak aether into the air, causing it to vibrate in a chosen pattern. This allows you to produce any sound you can imagine, at volumes between a whisper and a shout. The convincingness of this is decided by results of listeners' \textlf{resist(W)} checks.


\subsection{Orb of aether (8,P) [3]}
\label{spell:orb-light}
You can imbue your aether into the aetheric field of light itself, creating a condensed orb of luminosity. This manifests as floating orb of aether that can move a distance of 15 m (range 1) each turn. The orb provides full illumination within 5 m (1 combat area) of itself and \textlf{low} light within 15 m (radius 1). This lasts up to 1 hour and is a \textlf{persistent} effect. 


\subsection{Overgrowth (11) [3]}
\label{spell:overgrowth}
Plants have their own variety of aetheric field, you can project your aether into this field to create a resonance. You may choose a region of radius 5 m radius (1 combat area) within a range of 2 (25 m) and make local plant growth explode from the ground. This region is now \textlf{rough terrain}.


\subsection{Prediction (13) [4]}
Time itself resonates with aetheric frequencies. You can use this to attempt to peer into possible futures. This manifestation takes 10 minutes to complete. Afterwards, roll 3d6 and put them to one side. At any point within the next 24 hours you may replace one 3d6 roll (made by any creature or character) with the 3d6 you set aside. On a \textlf{critical failure} the GM may instead choose when to make the roll substitution. You may only have one such set of predicted dice available at once.
\subsubsection{Upgrade: Forecasting [3]}
You can store 2 predictions at once.


\subsection{Resonant motion (7) [2]}
\label{spell:min-tele}
You imbue aether into objects so that their aetheric fields resonate with motion. This allows you to exert the force of a single hand to perform simple actions on an object visible to you within range 2 (25 m).
\subsubsection{Upgrade: Major resonance (+2) [5]}
The aethermancer is far more attuned to the resonance of motion, allowing for the manipulation of an object with weight up to 10 kg plus 20 kg per point of \textlf{might}. If the object is used as a weapon, it uses the aethermancer's \textlf{aethermancy} for \textlf{aim}. The bonus \textlf{Power} of such a weapon is the aethermancers \textlf{Might}. The \textlf{power} for grabbing objects is calculated in the same manner. If multiple objects are controlled, the \textlf{Power} bonus from excess lifting force may be divided between them at the aethermancer's discretion. This manifestation requires that the aethermancer to maintain full concentration, in combat they must spend an action point to manipulate objects with this power.


\subsection{Root weaving (11,P) [3]}
By creating a complex aetheric resonance between yourself and nearby plants, you can call on their assistance to entrap foes. Provided there are plants or other natural growths nearby, this can be used to make roots \textlf{grapple} or \textlf{shove} a chosen creature or object within the foliage (this can be used once per turn but costs no action points). The roots use your \textlf{aethermancy} power score when making opposed checks. This is a \textlf{persistent} effect that can be maintained for up to 10 minutes.
\subsubsection{Upgrade: Bushwhack [3]}
Plants and roots can be made to strike at your foes, using your \textlf{aethermancy} for \textlf{aim}. For an action point you can activate a tree to strike at a nearby foe, this attack has crushing \textlf{lethality}.


\subsection{Sapping aether (10,P) [4]}
You project a discordant aetheric field into a target, weakening their body. If a target within range 2 (25 m) fails a \textlf{resist(R)} check then whenever they lose \textlf{endurance}, they lose 1 additional point. This lasts for up to 1 minute and  is a \textlf{persistent} effect.
\subsubsection{Upgrade: Field [2]}
If used for 2 action points this affects all creatures within a radius of 5 m (1 combat area).


\subsection{Shape earth (7) [3]}
The ground around you has an aetheric field that can be manipulated. A cunning flow of aether allows one to twist the very earth into any desired shape. The aethermancer may touch earth or stone and then manipulate up to 30 kg of earth, or 15 kg of stone, plus 10 kg earth or stone per point of \textlf{wit}. 


\subsection{Sonic boom (8) [3]}
By tuning your aether to resonate with the field of the air around you, a localised over-pressured pocket of air escapes outwards with a booming crash. This \textlf{Staggers} all creatures within a chosen area if they fail a \textlf{resist(R)} check. The chosen area must be within range 2 (25 m).
\subsubsection{Upgrade: Thunderous blast [2]}
This power now additionally inflicts damage with \textlf{normal lethality} on victims if they failed \textlf{resist} (\textlf{critical success} for the aethermancer on \textlf{resist} increments damage severity).


\subsection{Stone sense (7) [2]}
\label{spell:stone-sense}
Your expert knowledge of aetheric vibrations allows you to extend your hearing through a continuous stone (or earthen) surface or structure. For this purpose, a wall of stone bricks is continuous but soil and dirt do not count (they do not conduct vibrations coherently enough). This power lasts for up to 1 hour.


\subsection{Stone skin (8) [3]}
Manipulating the aetheric field of nearby rocks, you draw stone from the ground to clad you in armour. Until the end of the next round, \textlf{damage checks} against you suffer a \textlf{lethality} downgrade (to a minimum of normal). This can be cast with a either an action or reaction point.
\subsubsection{Upgrade: Earth-clad [2] (P)}
This makes \textlf{Stone skin} a \textlf{persistent} effect that lasts up to 10 minutes.


\subsection{Strange attractor (11,P) [4]}
An intricately twisted pattern of aetheric fields creates a point that draws in all nearby objects. Choose a point within range 2, this point becomes an aetheric singularity. Any creature (\textlf{large creatures} or smaller) within range 1 of the singularity must succeed on a \textlf{resist(M)} check at the start of their turn or be pulled a distance of 1 towards it and be \textlf{immobilised} on their next turn. On a \textlf{critical failure} any held weapons or objects are pulled from their grip. Any creature that spends a whole round within the singularity suffers damage with \textlf{normal lethality}. All loose objects of size large or smaller are also pulled in.


\subsection{Suggestion (10) [3]}
A creature's own aetheric field can be altered with aether to make its mind more open and malleable. Make a single-sentence suggestion of an action to the target, on a failed \textlf{Resist(W)} check they follow the suggestion willingly and rationalise it to themselves. This will automatically fail if it would
be harmful to the victim or their friends/allies. Specify 1 additional suggestion per level of \textlf{critical success} for the aethermancer on \textlf{resist}. 		
\subsubsection{Upgrade: Manipulation [3]}
Your mastery of the aetheric fields in brains allows you to implant beliefs or feeling into a target via suggestion. In addition suggestion no longer fails if it would be harmful to the victims friends/allies.


\subsection{Time warp (12) [5]}
Projecting your aether towards another creature, you make its own aetheric field discordant with the flow of time. Choose a creature within range 2 (25 m) to either gain or lose an action point (to a minimum of 1) each turn until the effect ends (maximum duration 1 minute). Targets may \textlf{resist(R)} each turn to ignore the effect (it does not end however). Only one action point may be gained or lost each turn in this way. \textlf{critical success} for the aethermancer on \textlf{resist} means the target cannot attempt \textlf{resist} next turn.
\subsubsection{Upgrade: Field [2]}
If used for 2 action points this affects all creatures within a targeted radius of 5 m (1 combat area).


\subsection{Transfix (10) [3]}
You project your aether into a creature, disrupting its aetheric field to fill its mind with a deluge of sensory information. A single target within range 2 (25 m) must \textlf{resist(R)} or stand transfixed (no actions allowed) for 10 minutes. A \textlf{resist} attempt may be made at the end of each turn to end the effect. This is \textlf{persistent} effect and any damage ends the effect immediately. 
\subsubsection{Upgrade: Field [2]}
If used for 2 action points this manifestation affects all creatures within a radius of 5 m (1 combat area).
\subsubsection{Upgrade: Amplification [3]}
If the victim of this effect is damaged or successfully \textlf{resist}, the manifestation lingers for 1 round before ending.


\subsection{Vortex (8) [3]}
Infusing a sudden burst of aether into the air creates a resonance that roars out as a powerful vortex of wind. This stops the flight of projectiles entering a target 5 m (1 combat area) region within range 2 (25 m) until the end of the next round. The stopped projectiles hit any creature inside the chosen region (allocate hits randomly). 
\subsubsection{Upgrade: Howling winds [2]}
\textlf{Vortex} can be  manifested with ferocious winds that knock down all creatures in the target area if they fail a \textlf{resist(M)} check.
\subsubsection{Upgrade: Swift-wind [2]}
\textlf{Vortex} can be cast for 1 reaction point during enemy action resolution.


\subsection{Warp space (12) [5]}
You create an aetheric resonance around a creature, warping the aetheric field of space itself. For 1 action point you can move a target creature, which can attempt to \textlf{resist(C)} if it wants to. This moves the target a distance up to range 2 (25 m). This cannot affect targets more than 1 size category larger than the aethermancer without 1 level of \textlf{critical success} for the aethermancer on \textlf{resist} per size additional category difference.


\subsection{Weather sense (7) [2]}
You attune to the aetheric fields of air and water, thus you can predict the weather up to 1 day + \textlf{wit} in advance.
\subsubsection{Upgrade: Weather touch [4]}
Your mastery of air and water allows you to nudge the weather. This lets you alter the weather, with the effect occurring after 5 - X days where X is the number of aether points spent on this manifestation. 



\chapter{Character skills}

\section{Aethermancy (Wit)}
\label{sec:aethermancy}
This is the skill invoked to produce aether manifestations. 

Any character with sufficient knowledge of the flow and hum of the aether can alter the very fabric of the world.  The energy for this is drawn from the inner strength of the character themselves. Thus, one can only exercise this power sparingly, lest they exhaust the aether powering their body. The use of aether to create external effects is known as ``manifestation''. Characters learn new manifestations either through being taught them or by spending experience points to unlock them.

\subsection{Aether capacitor}
\label{sec:capacitor}
A character's body can only hold so much aether. To transcend this limitation, aethermancers have developed a device called an `aether capacitor' which stores additional aether for them. This device is bulky and has wires running from it various points on the aethermancer's body. A standard aether capacitor costs 20 d and can hold 1 aether point. For each additional capacity point the cost increases by 60 d. The maximum possible capacity is 4 points.


\subsection{Producing a manifestation}
Manifestations do not require the use of gestures or words, they are produced by the character's fine control of their own aetheric field. By attuning the vibrations of their bodies own aetheric field, the aethermancer can cascade changes through other nearby fields to produce physical effects. Self-taught aethermancers often use gestures, as the spotty nature of their learning can lead to a personal association between gestures and manifestations. The synod requires its trainees do not use them at all. A list of manifestations can be found in Section~\ref{sec:manifestations}.

Manifestations are actions like any other, costing 1 action point unless otherwise specified. However, they also cost 1 point of aether, unless otherwise specified. Manifestations require the user to make an \textlf{aethermancy} check against a \textlf{difficulty} given in brackets on the manifestation description. If the aethermancer succeeds, they and the victim make a \textlf{resist} check to decide the effects. A \textlf{critical failure} during on the \textlf{aethermancy} check releases aether in an uncontrolled reaction (some suggestions are listed below).
\begin{itemize}
	\item Manifestation occurs adjacent to the aethermancer instead of targeted area
	\item Manifestation occurs but changes from targeting enemies to allies or vice-a-versa
	\item Manifestation occurs but picks a random target
	\item Manifestation fails producing an aether explosion (radius 0, \textlf{Power}+2)
\end{itemize} 

\subsubsection{Manifestation time}
An aethermancer can elect to reduce the action point cost of a manifestation by 1 (to a minimum of 1), at the cost of \textlf{edge-} on the aethermancy check. Conversely they can increase the cost by 1 to gain \textlf{edge+}. Note that 1 action point equates to roughly 3 seconds of real time.

\subsection{Resist(X)}
A Resist(X) check involves the victim making an opposed roll with their natural attribute X against the aethermancer, who adds their \textlf{Might} to the roll. If the aethermancer is not \textlf{proficient} in \textlf{aethermancy}, they experience \textlf{edge-} on this check.

\subsection{Disrupt}
An aethermancer can use their own aether to nullify the manifestations of another. This can be used in place of \textlf{Resist} or at a cost of 1 reaction point. \textlf{disrupt} involves an \textlf{aethermancy} check opposed with the other aethermancer. If the check succeeds, the manifestation is nullified.

\subsection{Persistent manifestations}
These have a long duration but require that concentration be maintained to keep them functioning. Thus, only one can be employed at a time. If a character is disabled or killed then their \textlf{persistent} manifestation ends. Otherwise, if they suffer a successful \textlf{damage check}, they must make a \textlf{resolve} check against $\dicediffbase$ plus the \textlf{power} of the attack. If they fail, the \textlf{persistent} manifestation ends. \textlf{Persistent} manifestations are marked with a P after their difficulty in Section~\ref{sec:manifestations}.



\section{Aethertech (Cunning)}
This represents a character's knowledge of aether-based technology and other sophisticated devices. A character does \textbf{not} need to be an \textlf{aethermancer} to use this skill. Using this skill a character can
\begin{itemize}
	\item Operate aethertech devices
	\begin{itemize}
		\item \textlf{critical failure} means the device fails catastrophically
	\end{itemize}
	\item Identify functions of aethertech devices
	\begin{itemize}
		\item Success means the functions are correctly identified
		\item \textlf{critical success} gives the character \textlf{edge+} on subsequent checks with the device
		\item Failure means the character is unsure about the device
		\item \textlf{critical failure} means the character is confident but wrong about the device
	\end{itemize}
\end{itemize}

\subsection{Weapontech}
This is an additional function of the aethertech skill, it allows characters to modify equipment with aethertech enhancements. The required materials are aether crystals and precious metals (to conduct aether), the cost of which are listed with each enhancement (the cost is split 50/50 between crystals and metals). A single piece of equipment can only ever have one enhancement. These enhancements cost double the material cost from professional aethertechs.

The enhancement process involves including additional metal structures or pathways to direct or harness the wielder's aether. This can be anything from strange wires and antennae to narrow, inlaid gold and silver veins, or even implanted aether crystals to supply the needed energy.  

\subsubsection{Arcing (13)}
The weapon carries complicated metal architecture that enhances aether emitted by the bearer. Aether manifestations of the wielder may add one extra target. Material cost: 150 d.

\subsubsection{Aggression (8)}
The weapon feels alive in your hand as aether thrums through inlaid metal veins. The weapon grants + 1 \textlf{Aim} on its first attack against a given target. Material cost: 50 d.

\subsubsection{Culling (11)}
The weapon unleashes a burst of aether when it strikes weakened targets. Attacks made against targets which are \textlf{stunned}, \textlf{Knocked Down}, or \textlf{Bleeding} benefit from a \textlf{lethality} upgrade. Material cost: 110 d.

\subsubsection{Deflection (11)}
The aether circuits in this weapon create a shimmering field around the weapon that absorbs energy from deflected attacks. This allows the weapon's wielder to gain a reaction point if they score a \textlf{Critical Deflect}. Material cost 110 d.

\subsubsection{Devastation (13)}
The weapon emits blasts of energy whenever it strikes. The weapon has \textlf{Burst} +1. Material cost: 150 d.

\subsubsection{Executioner (11)}
The weapon has auxiliary systems that kick-in when it inflicts a telling strike to increase its power even further. When the weapon inflicts a \textlf{Critical Hit} the wielder is granted an extra \textlf{damage check}. Material cost: 110 d.

\subsubsection{Finesse (9)}
The weapon is enhanced with guidance mechanisms to keep the wielder's aim true. \textlf{Critical Failures} on \textlf{aim} rolls with the weapon may be re-rolled. Material cost: 70 d.

\subsubsection{Impaling (11)}
The weapons stabbing or cutting edges are augmented with active aether blades that bore into the target with a well-placed strike. The weapon gains \textlf{Rending} + 1. Material cost: 110 d.

\subsubsection{Malevolent (12)}
Aether circuits zap the nervous system of its victims. When it causes damage, the weapon inflicts \textlf{edge-} to the next \textlf{Deflect} check made by its victim. Material cost: 130 d.

\subsubsection{Masterful (14)}
The weapon generates a resonance with aether manifestations produced by the wearer. The weapon applies \textlf{edge-} to \textlf{Resist} attempts made against the wielder's aetheric manifestations. Material cost: 150 d.

\subsubsection{Penetrating (9)}
The weapon's edge uses discordant aetheric fields to slice through even the thickest armour. The weapon gains \textlf{Penetration} + 1. Material cost: 70 d.

\subsubsection{Searing (9)}
The weapon unleashes searing aether into existing wounds on the target. \textlf{Bleeding} targets are \textlf{Vulnerable} to the next attack after being damaged by the weapon. Material cost: 70 d.

\subsubsection{Thirsting (13)}
Aetheric circuits draw energy out of targets wounded by this weapon. When the weapon causes at least 1 \textlf{endurance} damage to a victim, the wielder regains 1 missing \textlf{endurance} or aether point. Material cost: 150 d. 

\subsubsection{Thunderous (9)}
Powerful blows from this weapon unleash a booming aetheric disturbance. \textlf{Critical Hits} from this weapon knock the victim down. Material cost: 70 d.

\subsubsection{Vengeful (12)}
The aetheric circuits in the weapon register the signature of targets who have damaged the wielder. The weapons gains \textlf{edge+} to \textlf{aim} and \textlf{damage checks} against a foe that has damaged you within the last round of combat. Material cost: 130 d. 

\subsubsection{Vorpal (15)}
The weapon is imbued with aetheric fields that disrupt the target, making it far more lethal. Upgrades \textlf{Lethality} of the weapon on \textlf{critical hits}. Material cost: 190 d.


\subsection{Armourtech}
This is an additional function of the aethertech skill, it allows characters to modify equipment with aethertech enhancements. The required materials are aether crystals and precious metals (to conduct aether), the cost of which are listed with each enhancement. A single piece of equipment can only ever have one enhancement. The \textlf{difficulty} of the enhancement is given in brackets after the name. When acquired from professional aethertechs, these enhancements cost double the listed material cost.

The enhancement process involves including additional metal structures or pathways to direct or harness the wielder's aether. This can be anything from strange wires and antennae to narrow, inlaid gold and silver veins, or even implanted aether crystals to supply the needed energy.    

\subsubsection{Resilient (9)}
Aetheric circuitry runs through the armour, disrupting aetheric manifestations. Grants the wearer + 1 to \textlf{Resist} checks made against \textlf{aether manifestations}. Material cost: 70 d.

\subsubsection{Feather-light (11)}
The aetheric field of the armour makes its wearer lighter. This grants its wearer +1 range to their normal movement. Material cost: 110 d.

\subsubsection{Unflinching (13)}
An aetheric field generator softens blows dealt to the armour. This armour now negates the first \textlf{endurance} point lost each turn. Material cost: 150 d.

\subsubsection{Evasion (10)}
A flickering aetheric field surrounds the wearer, making it harder to determine their movements. The armour grants + 1 \textlf{Deflect}. Material cost: 90 d.

\subsubsection{Mirrored (10)}
This can only be used upon a shield, it uses aether ciruits to make the shield reflect aetheric manifestations. While using the shield, a wielder may add their \textlf{Deflect} bonus when making \textlf{Resist} attempts against aetheric manifestations. Material cost: 90 d.



\section{Infusary (Cunning)}
\label{sec:infuse}
\textlf{Infusary} is the craft of creating liquids that resonate with certain aetheric frequencies, achieving dramatic effects when drunk/applied. A character does \textbf{not} need to be an \textlf{aethermancer} to use this skill.

An infusion needs a container to hold it as well as a aether crystals and a signature ingredient to shape the purpose of the aether. To create infusions a character needs set of glass vessels and equipment for measuring, grinding and heating ingredients (infuser's tools) that can be purchased from an aethertech for 30 silver.
Table~\ref{tab:alch} displays some suggested infusions.

\begin{table}[ht!]
\caption{Aether infusions. D is the \textlf{difficulty}, time is how long the effects last and is given in minutes. The cost reflects that of the ingredients for a single draught of potion and the bottle to hold it, double this for the price charged by most infusers.}
\begin{tabular}{|l|l|l|l|l|l|}
\hline
Potion & D & Effects & Time & Ingredients & Cost \\
\hline
Might &  11 & +1 Might & 10 & Skulker mandibles & 5 d \\
Cunning & 11 & +1 Cunning & 10 & Fresh mus-folk whiskers & 5 d \\
Iron-flesh & 8 & +1 Toughness & 10 & Fine granite power & 2 d \\
Fire-blood & 9 & Enrage & - & Trog blood & 4 d \\
Hawk-eye & 10 & \textlf{edge+} on Aim & 20 & Eagle feathers & 6 d \\
Warding & 10 & \textlf{edge+} on \textlf{resist} checks & 10 & Powdered silver & 6 d \\ 
Invigoration & 11 & Restore 1 \textlf{endurance} & - & Mend-well root & 4 d \\
Restoration & 9 & Cure 1 Condition & - & Common herbs & 4 d \\
Peace & 11 & Cure all Conditions & - & Nightshade,  & 12 d \\
 & & & & Mend-well leaves & \\
Competence & 12 & \textlf{edge+} on a chosen skill & 60 & Gold, Silver & 40 d \\
Camouflage & 10 & \textlf{Stealth edge+} & 20 & Nightshade, Ivy root & 6 d \\
Invisibility & 13 & Invisibility & 5 & Distilled aether & 50 d \\
Haste & 13 & +1 action point per round & 5 & Adrenaline & 50 d \\ 
Explosive & 11 & radius 0, Burst 1, Power +1 & - & Acid, coal, salt-peter & 6 d \\
Aethershot & 12 & 1 Aethershot ammunition & - & Black powder & 4 d \\ 
\hline
\end{tabular}
\label{tab:alch}
\end{table}




\chapter{Arms and armour}
\label{chap:arms}

\begin{table}[ht!]
	\centering
	\caption{Close-combat weapons. Note that L signifies \textlf{lethality}, A is \textlf{aim}, D is \textlf{deflect}, and P is \textlf{power}. Pen indicates the \textlf{penetration} effect. ``Hands'' signifies whether the weapon requires one or two hands to operate.}
	\label{tab:weps}
	\begin{tabular}{|l|l|l|l|l|l|l|l|}
		\hline
		Name & Cost & Hands & L & A & D & P & Special\\ [0.5ex]
		\hline 
		\textbf{Daggers} & & & & & & & \\
		\hline
		Dagger & 3 d & 1 & N & - & - & - & Small, Throw 2\\
		Stiletto & 5 d & 1 & N & - & - & - & Small, Rending 1 \\
		Sword breaker & 5 d & 1 & N & - & - & - & Small, Disarm\\
		\hline 
		\textbf{Swords} & & & & & & & \\
		\hline
		Side-sword & 10 d & 1 & N & +1 & - & - & \\
		Longsword & 15 d & 2/1 & N & +1 & - & +1/- & \\
		Saber & 25 d & 1 & N & +1 & - & +1 & \\ 
		Rapier & 25 d & 1 & N & +1 & - & - & Rending 2\\ 
		Greatsword & 25 d & 2 & N & +1 & - & - & \textlf{damage edge+}\\
		Zweihander & 40 d & 2 & C & +1 & - & - & Cumbersome\\ 
		\hline 
		\textbf{Axes} & & & & & & & \\
		\hline
		Battle axe & 6 d & 1 & N & - & - & +1 & - \\
		Bearded axe & 7 d & 1 & N & - & - & +1 & Disarm \\
		Tabar & 10 d & 1 & N & - & - & +1 & Pen 1 \\
		Throwing axe & 8 d & 1 & N & - & - & +1 & Throw 3 \\
		Long axe & 25 d & 2 & N & - & - & +1 & \textlf{damage edge+} \\
		\hline
		\textbf{Pole weapons} & & & & & & & Trip \\
		\hline
		Javelin & 30 c & 1 & N & - & - & - & Rending 2, Throw 3\\
		Quarter Staff & 20 c & 2 &  N & - & - & - & Disarm\\
		Spear & 1 d & 2/1 & N & - & - & +1/- & Rending 2 \\
		Pike & 2 d & 2 & N & - & - & - & Rending 2, Reach\\
		Glaive & 10 d & 2 & N & - & - & - & \textlf{damage edge+} \\
		Halberd & 25 d & 2 & N & - & - & +1 & Rending 1, Disarm\\
		Pole-hammer & 20 d & 2 & N & - & - & +1 & Pen 1, Disarm\\ 
		Pole-axe & 40 d & 2 & N & - & - & +1 & Rending 1, Pen 1 \\
		Lance & 6 d & 2 & N & - & - & - & Rending 2\\
		\hline
		\textbf{Blunt weapons} & & & & & & & \\
		\hline
		Cudgel & 5 c & 1  & N & - & - & - &  \\
		Club & 30 c & 2 & N & - & - & +1 & Pen 1 \\
		War hammer & 15 d & 1/2 & N & - & - & -/+1 & Pen 2 \\ 
		Maul & 35 d & 2 & C & - & - & - & Cumbersome, Pen 2 \\
		Mace & 4 d & 1 & N & - & - & - & Pen 1 \\
		\hline
		\textbf{Extended weapons} & & & & & & & \\
		\hline
		Flail & 12 d & 1 & N & +1 & - & - & Trip, Disarm \\ 
		Grand flail & 17 d & 2 & N & +1 & - & +1 & Trip, Disarm\\
		Whip & 1 d & 1 & N & - & - & -1 & Trip, Disarm, Reach\\
		\hline
	\end{tabular}
\end{table}
\begin{table}[ht!]
	\centering
	\caption{Ranged weapons. Note that L signifies \textlf{lethality} and P is \textlf{power}. Pen indicates the \textlf{penetration} effect. ``Hands'' signifies whether the weapon requires one or two hands to operate.}
	\begin{tabular}{|l|l|l|l|l|l|l|l|}
		\hline
		Name & Cost & Hands & Range & Reload & L & P & Special\\ [0.5ex]
		\hline
		\textbf{Slings} & & & & & & & \\
		\hline
		Short sling & 10 c & 1 & 3 & 1 & N & +1 & \\ 
		Long sling & 10 c & 1 & 5 & 1 & N & - & \\
		\hline
		\textbf{Bows} & & & & & & & \\
		\hline
		Short bow & 10 d & 2 & 4 & - & N & - & \\
		Long bow & 15 d & 2 & 6 & 1 & N & +1 & \\
		\hline
		\textbf{Cross bows} & & & &  & &  & \\
		\hline
		Hand crossbow & 15 d & 1 & 1 & - & N & -1 & Small \\ 
		War crossbow &  25 d & 2 & 5 & 2 & N & - & \textlf{damage edge+}\\
		Repeater crossbow & 30 d & 2 & 4 & 2 & N & - & Burst 1 \\  
		Heavy crossbow & 50 d & 2 & 5 & 3 & C & - & \\
		\hline
		\textbf{Firearms} & & & & & & & \\
		\hline
		Pistol & 25 d & 1 & 3 & 1 & N & - & Pen 2, Small \\
		Musket & 50 d & 2 & 6 & 2 & N & +1 & Pen 2 \\
		Blunderbuss & 50 d & 2 & 4 & 3 & N & - & Cone, Pen 2\\
		Bomb & 10 d & 1 & 3 & - & N & - & Burst 1, Thrown, Radius 0, \\
		& & & & & & & \textlf{damage edge+} \\
		\hline 
	\end{tabular}
	\label{tab:range-weps}    
\end{table}

\begin{table}[ht]
	\centering
	\caption{Basic armour.}
	\begin{tabular}{|l|l|l|}
		\hline
		Name  & Cost & Toughness\\
		\hline
		\textbf{Light armour} & &  \\
		\hline
		Gambeson & 5 d & + 1 \\
		Mail shirt & 25 d & + 2 \\  
		\hline
		\textbf{Medium armour} & &  \\
		\hline
		Mail hauberk & 40 d & +3  \\ 
		Brigandine and mail  & 60 d & +4  \\
		\hline
		\textbf{Heavy armour} & &  \\
		\hline
		Plate cuirass and mail & 90 d & +5 \\
		Full plate  & 130 d & +6  \\
		\hline
		Barding (Horse)& 150 d & +4 \\
		\hline
	\end{tabular}
	\caption{Shields}
	\begin{tabular}{|l|l|l|l|}
		\hline
		Name & Cost & Deflect & Special\\   
		\hline
		Shield & 10 d & +1 &  \\
		\hline
	\end{tabular}
\end{table}





\section{Weapon special rules}

\subsection{Pike}
Pikes are so long that they receive \textlf{edge-} on \textlf{aim} against targets within range 0.

\subsection{Lance}
A lance gains bonus \textlf{lethality} if the user is mounted and moved a distance of at least 1 in the same turn.

\subsection{Slings}
These are rotational sling weapons consisting of a long loop of cord and a pouch that holds and launches a stone or lead projectile. They count as throwing-type ranged weapons.

\subsection{Bows}
Bows are shooting-type weapons. %Firing two arrows together incurs \textlf{edge-} to \textlf{Aim} but adds an extra projectile to the shot.

\subsection{Crossbows}
All crossbows are shooting weapons.

\subsubsection{Hand crossbow}
These diminutive weapons are designed to be easily concealed and to unloaded at close range into an unsuspecting victim. As such, they are \textlf{small} and fast firing but are still tricky to reload when used in pairs (requiring a 1 action \textlf{Reload}).


\subsection{Musket}
A musket is a heavy object and can be used as a close combat weapon in its own right. It must be used in two hands. If it has a bayonet it gains + 1 \textlf{rending} in close combat.

\subsection{Blunderbuss}
This weapon fires a scatter of shot in a cone, hitting up to 2 targets in a single combat area. The user can also declare using it point blank within range 0, then it hits one target only but has \textlf{Burst} +1.

\subsection{Aethershot ammunition}
This is special ammunition for firearms, infused with aetheric energies to increase its destructive potential. A shot made with such ammunition gains \textlf{edge+} on \textlf{damage checks}.





\section{Armour special rules}
\label{sec:armspec}


\subsection{Putting on and removing armour}
Even heroes seldom sleep in full-plate armour, so there are then times when the speed at which a character puts on or removes armour might matter. It requires four action points to put on \textlf{light armour} or one to strap on a shield. However, it takes 1 minute to put on \textlf{medium armour} but it takes 5 minutes to put on \textlf{heavy armour}, which also requires that the wearer has assistance in putting it on. \textlf{Medium} or \textlf{light armour} can be put on in a rush, doing so means that armour might not be precisely adjusted in order function to it's full effect, this takes only four action points for medium armour or two action point for \textlf{light armour}, but reduces the \textlf{Toughness} of the armour by 1 while it is worn in this sloppy fashion.

Removing \textlf{light armour} takes 1 minute, \textlf{medium armour} takes 2 minutes, and \textlf{heavy armour} takes 5 minutes to remove.


\subsection{Shields}
\label{sec:shields}
Shields add a + 1 bonus to the bearer's \textlf{Deflect} score and negate the \textlf{edge-} on \textlf{deflect} when wearing \textlf{heavy armour}. 


\section{Weapon enhancements}
A weapon may have one of the enhancements listed below. 

\subsection{Masterful quality}
This grants the item + 1 \textlf{Power} or \textlf{aim}. This costs 15 d + thrice the base cost of the weapon.

\subsection{Expanded bolt rack}
(Crossbow only) The weapon is fitted with expanded armatures and space for two bolts. It can thus fire two bolts with a single trigger release. This grants \textlf{Burst} + 1 on attacks made with the crossbow. Cost: Base cost x 3 + 10 d. 

\subsection{Special payload}
(Crossbow only) The armatures and bolt track of the bow are modified to carry larger and heavier bolts. This configures the crossbow to fire specialised ammunition that can carry explosives (bomb from infusary - Section~\ref{sec:infuse}), chains/ropes/nets, or poisons and other chemicals that spray in a radius of 0 around the impact point. Cost: 70 d.

\subsection{Experimental chamber}
(Firearms only) The weapon can hold reserve ammunition that it is not currently firing. This reduces the \textlf{Reload} time of the weapon by one rank (reduce paired pistols to \textlf{Reload} 2) to a minimum of 1. However, if the user scores a raw 5- with the weapon, roll 3d6. On a score of $8-$ the weapon needs 50 d worth of repairs before it works again, on $9+$ it jams requiring 1 action point be spent to clear the jam. Cost: 100 d.

\subsection{Extra barrel}
(Firearms only) The weapon is fitted with an extra barrel. The gun may be fired twice before re-loading but reloading time increases by 1 action point. Cost: 50 d.




\section{Armour enhancements}
Armour may have one of the enhancements listed below.

\subsection{Masterful quality}
This grants the item + 1 Toughness. Cost: double the base cost plus 10 d.

\subsection{Heavily reinforced}
The armour is carefully enhanced to maximally redistribute impact. This incurs a -1 \textlf{Burst} penalty (minimum 1 \textlf{damage check}) to any attacks made against the wearer. Cost: 50 d. \textlf{Medium} or \textlf{heavy armour} only.

\subsection{Cunning strappage}
Careful use of straps and weight distribution makes the armour feel so light you could dance in it (this is still not recommended). This negates a chosen \textlf{edge-} penalty incurred by the armour. Cost: 30 d. \textlf{Medium} or \textlf{heavy armour} only.

\subsection{Knuckle-blades}
Punching people is easier with steel blades on the knuckles of your gloves. This grants + 1 \textlf{power} to unarmed \textlf{damage checks}. Cost: 10 d.

\subsection{Armour spikes}
Sharpened protrusions or heavy studs set in the surface of the armour can make attacking the wearer a real hazard. This can be applied to any \textlf{Medium} or \textlf{Heavy armour} piece, enemies attempting to grapple with the wearer have \textlf{edge-} to \textlf{Grapple} checks. Note this does not apply if the wearer is the one attempting the grapple. Cost 12 d.



\chapter{Other equipment}


\section{Cost}
This cost reflects an average price that the item would be purchasable for, from a merchant. This cost can go up and down dependent on how well a player haggles and on the circumstances of the city and/or merchant; that is, if something is in shortage it costs a lot more, or if the merchant is desperate to sell, the price goes down. A player can sell an item for a price dependent on the condition of the item, a merchant will offer him half the normal cost for a used item in good condition.

\section{Equipment}
\subsection{Tools}

\subsubsection*{Bandages}
Required by a healer to use his healing skill, each use of the skill consumes a bandage.

\subsubsection*{Infuser's tools}
A complex piece of equipment that allows an infuser to purify and distil various liquids, this is required for the creation of infusions (see Section~\ref{sec:infuse}).

\subsubsection*{Glass vial}
A simple glass vial with a stopper, used to hold liquids, this is needed to hold any brew created by an infuser (see Section~\ref{sec:infuse}).

\subsubsection*{Lock-pick}
A small piece of cunningly bent metal that can be used to open a lock with a Mechanical skill check opposed to the \textlf{difficulty} of the lock, on a failure the lock-pick breaks.

\subsubsection*{Musical instrument}
A simple instrument for making music.

\subsubsection*{Wood-axe}
An axe for chopping up wood.

\subsubsection*{Master-crafted tools}
These are finely crafted versions of any type of tool, this adds a + 1 bonus to any rolls made using the tools.

\subsection{Adventuring gear}

\subsubsection*{Backpack}
A large but otherwise conventional satchel.

\subsubsection*{Barrel}
A large wooden barrel, bound with iron rings. The barrel can hold about 50 litres.

\subsubsection*{Basket}
A simple wicker basket for carrying small loads.

\subsubsection*{Candle}
A small candle that that produces \textlf{low} light in a 5 m radius, this candle can burn for up to 3 hours.

\subsubsection*{Canvas}
A sheet of water-proof canvas. The price depends on the size of the sheet.

\subsubsection*{Chain}
A metal chain made up of heavy steel links, the chain is strong enough to support very large weights, up to 500 kg.

\subsubsection*{Crowbar}
A simple steel crowbar used for levering open doors or hinges. Using a crowbar adds \textlf{edge+} to rolls made using your Might to force open any hinged container or door.

\subsubsection*{Firewood}
Wood for keeping a fire burning for one day.

\subsubsection*{Fishing tackle}
A fishing line, a hook, sinkers and lures. This can be used to attempt to catch fish with the \textlf{Survival} skill, the \textlf{difficulty} of this is dependent on the speed of the water and type of the fish.

\subsubsection*{Grappling hook}
A heavy iron hook used for scaling near vertical surfaces. This hook will hold the attached rope in place after being successfully thrown to the point you wish to climb to. The hook also makes an effective weapon.

\subsubsection*{Hammer}
A simple hammer for hitting stuff and pitching tents.

\subsubsection*{Ladder}
A length of ladder which can be climbed without needing an Athletics check.

\subsubsection*{Lantern}
A hooded lantern used for projecting \textlf{full} light directionally, up to 13 m from the bearer. This uses 0.5 kg of oil to burn for 24 hours.

\subsubsection*{Lantern oil}
Enough oil to power a standard lantern for 24 hours. Highly flammable.

\subsubsection*{Mirror}
A simple shiny surface (glass or metal) that reflects light. Useful for looking around corners.

\subsubsection*{Needle}
A sharp needle for use in sewing or stitching wounds.

\subsubsection*{Paper}
A single sheet of paper for writing on.

\subsubsection*{Pick}
A simple pick used for breaking earth or skulls.

\subsubsection*{Pole}
A length of wooden or metal pole.

\subsubsection*{Pot}
A cooking pot, about 30 cm in diameter.

\subsubsection*{Quill}
A feather quill used for writing.

\subsubsection*{Rope}
A length of rope capable of supporting the weight of three people ($\sim$250 kg).

\subsubsection*{Sack}
A large rough cloth bag.

\subsubsection*{Spade}
A digging implement (can be used to hit people as well).

\subsubsection*{Spyglass}
A telescope capable of magnifying things from up to a mile away.

\subsubsection*{Tent}
A canvas tent to keep you dry at night, this can house up to two man-sized creatures.

\subsubsection*{Tinder box}
A small box containing fast lighting twigs, used to rapidly start a fire.

\subsubsection*{Torch}
Provides \textlf{full} light up to a 5 m radius and \textlf{low} light within 12 m. The torch can burn for 4 hours.

\subsubsection*{Trail rations}
Generally dried meat, bread and cheese. Long lasting food-stuffs to nourish a traveller.

\subsubsection*{Water-skin}
A hide bag used for carrying water, this carries water for one person for 5 days.

\subsection{Mounts}
\subsubsection*{Pony}
A small hard working pony, capable of carrying an full grown small creature or the child of a man sized creature. A pony has a combat movement distance of 2. Such a creature can carry weights of up to 100 kg while still being to walk or trot all day with no appreciable strain. Carrying weights of 150 kg or more risks injuring the horse if done for prolonged periods. This horse is not trained for war and may panic, it has Deflect 0 and Toughness 8.

\subsubsection*{Light horse}
A swift if none too tough horse, a Light Horse has a combat movement distance of 4. A light horse can gallop up to 2 hours a day at a speed of five times faster than a man, any longer risks injuring the horse. This horse can only move a distance of 2 per movement action while carrying a heavily-armoured rider. A light horse can carry weights of up to 80 kg while still being to walk or trot all day with no appreciable strain. Carrying weights of 130 kg or more risks injuring the horse if done for prolonged periods. This horse is not trained for war and may panic, it has Deflect 0 and Toughness 9.

\subsubsection*{Cart horse}
A heavy farm horse capable of carrying heavy loads and working all day, however, it is not capable of maintaining very rapid speeds. A Cart Horse has a combat movement distance of 2. A cart horse can carry weights of up to 200 kg while still being to walk or trot all day with no appreciable strain. Carrying weights of 300 kg or more risks injuring the horse if done for prolonged periods. This horse is not trained for war and may panic, it has Deflect -1 and Toughness 10.

\subsubsection*{Heavy horse}
A strong and reasonably quick horse. This horse can carry a man in heavy armour while moving at full speed. This horse has a combat movement distance of 3. Such a creature can carry weights of up to 120 kg while still being to walk or trot all day with no appreciable strain. Carrying weights of 200 kg or more risks injuring the horse if done for prolonged periods. This horse is not trained for war and may panic, it has Deflect -1 and Toughness 10.

\subsubsection*{Destrier (war horse)}
A heavy horse trained for battle, the destrier is by far the largest breed of horse, standing over 2 metres at the shoulder. Its war training makes it a savage weapon in its own right and it has close-combat Aim +1, Deflect 0, Toughness 10, Power +1, and combat movement range 3. A destrier can carry weights of up to 200 kg while still being to walk or trot all day with no appreciable strain. Carrying weights of 300 kg or more risks injuring the horse if done for prolonged periods.

\subsubsection*{Simple saddle}
A rough worked leather saddle, needed for sitting astride your mount. A saddle is fastened with girth buckles and has attached stirrups.

\subsubsection*{War saddle}
A heavy saddle needed to support a fully-armoured knight, this saddle is designed also to allow the knight to wield a lance one handed.

\subsubsection*{Racing saddle}
A saddle built for race horses, it is light weight and fine craftsmanship add 1 to the horses' combat movement distance.

\subsubsection*{Tack}
Reins, bit, bridle and other equipment necessary for riding a mount. Riding bareback increases the \textlf{difficulty} of all ride checks by 2.

\subsubsection*{Feed}
Food for a mount for one day.

\chapter{Equipment tables}
\begin{table}[ht]
	\centering
	\caption{Tools}
	\begin{tabular}{|l|l|}
		\hline
		Item & Cost\\ [0.5ex]
		\hline
		Bandages (5)& 50 c\\
		Infuser's tools & 30 d\\
		Glass Vials (5)& 5 d\\
		Lock-picks (5) & 2 d\\
		Musical instrument & 20 d\\
		Wood-axe & 3 d\\		
		\hline
		Master-crafted tools & + 50 d\\
		\hline
	\end{tabular}
\end{table}

\begin{table}[ht]
	\centering
	\caption{Adventuring gear}
	\begin{tabular}{|l|l|l|}
		\hline
		Item & Cost & Weight\\ [0.5ex]
		\hline
		Backpack & 50 c & 1 kg\\
		Barrel & 30 c & \\
		Basket & 20 c & 0.2 kg\\
		Candle & 10 c & -\\
		Canvas (per sq. m) & 1 d & 4 kg\\
		Chain, 1 m & 2 d & 3 kg\\
		Crowbar & 5 d & 5 kg\\
		Firewood, 1 day & 20 c & 5 kg\\
		Fishing tackle & 20 c & -\\
		Grappling hook & 10 d & 5 kg\\
		Hammer & 1 d & 4 kg\\
		Ladder, 1 m & 1 d & 3 kg\\
		Lantern & 5 d & 1 kg\\
		Lantern oil & 1 d & 0.5 kg \\
		Mirror & 1 d & -\\
		Needle & 20 c & -\\
		Paper 1 sheet & 2 c & -\\
		Pick & 2 d & 8 kg\\
		Pole, 1 m & 30 c & 2 kg\\
		Pot & 3 d & 0.5 kg\\
		Quill pen & 10 c & -\\
		Rope, 1 m & 20 c & 0.1 kg\\
		Sack & 10 c & 0.2 kg\\
		Spade & 5 d & 4 kg\\
		Spyglass & 50 d & 1 kg\\
		Tent & 5 d & 3 kg\\
		Tinder box & 2 d & -\\
		Torch & 2 c & 0.1 kg\\
		Trail rations 1 day & 10 c & 0.1 kg\\
		Water-skin (full) & 50 c & 3 kg\\		
		\hline
	\end{tabular}
\end{table}

\begin{table}[ht]
	\centering
	\caption{Mounts}
	\begin{tabular}{|l|l|}
		\hline
		Item & Cost\\ [0.5ex]
		\hline
		Pony & 60 d\\
		Light horse & 60 d\\
		Cart horse & 80 d\\
		Heavy horse & 80 d\\
		War horse & 160 d\\
		\hline
		Simple saddle & 5 d\\
		War saddle & 15 d\\
		Racing saddle & 10 d\\
		Tack & 2 d\\
		Feed, 1 day & 5 c\\
		\hline
	\end{tabular}
\end{table}

\begin{table}[ht]
	\centering
	\caption{Clothes}
	\begin{tabular}{|l|l|}
		\hline
		Item & Cost\\ [0.5ex]
		\hline
		Peasant & 10 c\\
		Scholar & 1 d\\
		Cold weather & 1 d\\
		Noble & 30 d\\
		Merchant & 5 d\\
		\hline
	\end{tabular}
\end{table}





\listoftables


\end{document}
