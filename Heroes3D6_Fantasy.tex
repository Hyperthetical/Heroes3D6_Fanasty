%Copyright (C) 2021-2025 Geoff Beck. This work is licensed under a Creative Commons Attribution-ShareAlike 4.0 License. To view a copy of this license visit: \url{https://creativecommons.org/licenses/by-sa/4.0/legalcode

\documentclass[a4paper,11pt,oneside]{book}

\usepackage{fullpage}
\usepackage{amsmath}
\usepackage{import}
\usepackage{tocbibind}
\usepackage[bookmarks=true,plainpages=false]{hyperref}
\usepackage{titletoc,titlecaps}
\usepackage{caption}
\usepackage[a4paper,margin=2cm]{geometry}
\usepackage[T1]{fontenc}
\usepackage{lmodern}
\usepackage{fancyhdr}
\usepackage{titling}
\title{\textbf{\huge Heroes3D6\\Fantasy role-playing rules add-on}}
\author{Geoff Beck}
\predate{\centering}
\date{}
\postdate{\vfill{\copyright 2021-2025 Geoff Beck. This work is licensed under a Creative Commons Attribution-ShareAlike 4.0 License. To view a copy of this license visit: \url{https://creativecommons.org/licenses/by-sa/4.0/legalcode}}\hfill}
\usepackage{lipsum}
\pagestyle{plain}


\newcommand{\dicediffbase}{9}
\newcommand{\dicecritlvl}{3}

\newcommand{\textlf}[1]{\textbf{\titlecap{#1}}}
\newcommand{\textlfirst}[1]{\textbf{\textit{\titlecap{#1}}}}
\newcommand{\dist}{2 m}


\begin{document}
\frontmatter
\maketitle
\tableofcontents
\mainmatter

%\chapter{The world and the `Last City'}
%The world of Krell is a devastated wasteland. An event in the distant past commonly referred to as the `Catastrophe' unleashed vast quantities of warping magical energies that laid waste to cities and ecosystems across the planet. The cause of this disaster is not clearly remembered, however, it has left a long lasting suspicion of magic in the few survivors that cling to this wrecked world. Over time the lingering magic, called `the residual', has produced new mutant life-forms by warping the small surviving populations. The `Last City' is the only major settlement in existence, built after the Catastrophe and housing peoples of all species, it teeters constantly the brink of collapse as a result of supply shortages as well as internal tensions. What holds the city together is the shelter it provides from the residual, this being a great wall built of a patchwork of `untainted' iron. This being a metal of great value, often used as currency outside the city, as it survived the Catastrophe without being infused with dangerous residual energies and now repels these same influences.    
%
%Small settlements dot the wastelands outside the Last City, these struggle constantly against the dangers of residual, mutant wild-life, bandits, and the terrifying changekin. These last are feral people twisted by the residual energies into nightmarish monsters. Life outside the city is a constant struggle to find food and water that are untainted by the residual, as too much exposure can eventually twist one into a changekin.
%
%\section{The last city}
%The only bastion of civilisation in a blighted world, the Last City stands as monument to the endurance of the people of Krell. Surrounded by a patch-work wall of untainted iron the city is largely protected from the warping power of the residual. Entrances to the city are guarded and those too contaminated by the residual are denied entrance, especially as a common punishment for crimes is being exposed to contamination and cast out of the city. Such is the suspicion of magic and its users that all mages must be registered and are branded with forehead marking, those who refuse are cast out as criminals. The mage brand brings much suspicion upon its bearers with common folk often spitting in the street at their passing, and some being unwilling to even do business with them. Often children with magical talent are disowned by their families and turned out onto the streets. 
%
%The city is ruled over by council of exclusively human `founding families', constituting a ruling aristocracy whose power is enforced by a small soldier corps. Low-level crime is endemic in the city, as it contains many desperate people of all species, and criminal gangs act as the defacto rulers of some districts and trades. The existence of the city is precarious, as it depends strongly on food grown in surrounding settlements, which are constantly under pressure from raiding bandits, changekin, and the residual itself. Despite this, it is the most sure shelter against the terrors of the wastelands.
%
%Within the city coins are used as currency but the alternative of untainted iron is also widely accepted (it is often the only currency used in the wastelands). A coin-sized disk of this metal is exchanged for 3 gold coins, such is its value in warding off the effects of residual when travelling outside the city.   
%
%\section{The changed and the changekin}  
%The changed are a phenomenon that began after the Catastrophe. They are beings that have been warped by the residual, and exhibit anything from minor changes like scales or feathers on their skin, up to limbs or heads being a mixture of animal and human (bird claws, goats hooves, beaks, animal jaws). The changed are, however, still people and although often viewed with suspicion and prejudice they are not attacked on sight. 
%
%Changekin are somewhat like the changed but have become feral, their minds rotted away by the residual. Their mutations tend to be even more extreme, often multiple additional limbs, or large bloated bodies. Changekin are savage and attack anything on sight, except for other changekin (most of the time). Such creatures form into roaming hunting bands that seek out anything weak enough to be preyed upon, attacking travellers or small settlements if their band grows large enough. Changekin seldom wield weapons or wear armour, their minds are too damaged for such rationality.  
%
%
%\section{The residual}
%The residual is remnant warping energy left over from the Catastrophe. Exposure to it is dangerous as it causes damage to exposed tissues and can, with sufficiently high doses, twist the bodies of the exposed in monstrous forms. The residual is undetectable to normal senses except in very high concentrations, where it feels like a heated crackle in the air (these concentration levels are immediately dangerous as well). Untainted iron is commonly used to ward off the effects of the residual and skilled alchemists can produce potions brewed with it that can protect from, or cleanse a body of, the residuals effects. 
%
%Whenever exposed to a residual infused environment (or consuming tainted food/drink) a creature must make a \textlf{resolve} check against a \textlf{difficulty} set by how saturated the area is. For example low contamination is \textlf{difficulty} 8, dangerous is 11, and deadly is 14. If the creature fails they gain a Residual point (\textlf{critical failure} adds extra points). The number of points is added to the \textlf{difficulty} of subsequent checks against residual contamination. At 5 points the creature begins to feel light-headed, weak, and sweaty (has \textlf{edge-} on all actions). At 10 points a creature dies. Residual points last until removed. Armour made of untainted iron grants an \textlf{edge} on all such \textlf{resolve} checks and weapons of this metal have \textlf{edge+} to \textlf{damage checks} against changekin. 

\chapter{The highlands of Atla}
This area of land (within the world of Krell) is entirely mountainous, with the major city of Stormheight occupying the tallest peak in the region. Smaller towns and villages also dot surrounding valleys. This terrain is rich in aether, with great deposits of crystals being mined in many of the taller peaks. This richness of aether has lead to Stormheight being a seat of technological advancement. In turn, the study of aether has become an integral part of its powerful religious structures.  

\section{The structure of a year}
A year on the world of Krell is referred to as a `prime cycle' due to the fact that Krell orbits a binary star system. This means their seasons have two inter-linked cycles. The prime cycle tracks the world's motion around the twin suns (taking roughly 320 days). The minor cycle is a result of the alignment of the suns with respect to Krell (this has a period of around 20 days). Thus, prime seasons contain a 20 day cycle of minor seasons. For example: prime summer will contain periods of both prime and minor summer (peak summer) as well as periods where the minor cycle is in winter (low summer). Each prime season on Krell contains a full cycle of minor seasons. This means the weather can vary quite drastically on the time scale of 20 days. Note that Krell has no moons, so the minor cycle is their analogue of a month. A date is conventionally structured by the number of the prime cycle, the prime season, and the day number within the season (1-80). For example: the 42nd day of prime winter in cycle 457.

\section{Climate and ecosystem}
The highlands are generally cold and stormy. Winds lash the peaks all year round and powerful storms occur regularly in any season. During peak winter there are howling gales and snow storms that often isolate mountain valleys. The rest of prime winter tends towards chill winds with these often developing into powerful thunderstorms that can scour the peaks. Prime summer has slightly milder temperatures and a steady drizzle of rain. Rain is far less common, and the weather warmer, during peak summer. 

This is a difficult climate, as such the majority of native plant life is lichen or scrub, with trees being rare and wood an expensive luxury (peat is the preferred fuel for fires). The fauna of the highlands are well adapted to their harsh and variable conditions. Krell's low gravity (60\% of Earth's) means that many inhospitable places, like the highlands, are dominated by warm-blooded creatures that would be arthropods (invertebrates with exoskeletons) except that many have both exo and endoskeletons. For larger creatures, these skeletons tend to be composed of a hybrid of bone and a chitin-like polymer with far greater strength. A common highland sight are small herds of `skippers', metre-long grass-hopper-like grazers. These are preyed upon by species of large, pack-hunting, spiders (up to 1 m across) and `skulkers', long, many-legged, low-slung, and covered in jointed armour. These have a body-size between 60 cm and 4 m, based on age, and are fierce predators with powerful, and venomous, biting mandibles up to 60 cm long. The highlands host many varieties of crab, mainly herbivorous or scavengers. Giant stone crabs, for example, measure around 3 m across and are clad in thick armour plates that are nearly as tough as steel. These are omnivores, being mainly herbivorous but content to scavenge carcasses or eat particularly stupid/slow smaller creatures. A vast array of small arthropod-like creatures fill almost all the niches of the ecosystem. For instance there are no rats in the highlands, but there are rat-sized pests called `skrik', which look like lobsters and are prolific scavengers, even being willing to eat crab chitin in lean times. Higher on the food-chain are `karks', dog-sized creatures with a lobster or crayfish-like anatomy that are often used as watch-dogs. `Grey backs', reptilian creatures with six powerful hoofed legs, grey scaly skin, and strong jaws fill a wolf-like niche in the ecosystem. Despite being ill-tempered, grey backs are often domesticated by Highlanders as hunting animals. This requires raising the creature from an egg. There is also an array of migratory bird-life that visit the highlands during prime summer. 

\section{Religion and society}
Stormheight itself is home to the Synod of Inquiry, an organisation dedicated to the reverence of the natural world as a matchless system of harmony (the religion itself is known as The Harmony). The synod is extremely influential, being the ruling body of the city in all but name. The heart of this religious body is a grand campus, where all study of the natural world and its aether is conducted in a suitably reverent manner. Needless to say, scholarship and aethermancy outside of this body is illegal. The rationale for this being that unrestricted inquiry might lead to disastrous abuses of nature. Under the Harmony, the duty of thinking beings is to match themselves to the harmony of nature, part of this is by achieving ``inner harmony'', the other being in harmony with society and the world. To better achieve this, everyone in Stormheight is allocated a role and occupation according to the assessment of the synod. Those who cannot find harmony with their allocated position are confined to asylums or exiled. The influence of the synod is significantly lesser in the countryside, where folk festivals tend to match the religious calender of the Harmony, but the people aren't really invested in the wider system. Their only encounter with the synod being a once per prime cycle arrival of tax men, along with assessors who decide on the future occupations of the village youth. The Harmony's symbol is a field of countless, mutually-interlocking rings.

\subsection{Class and wealth}
Stormheight has no explicit aristocracy, as the structure of the synod is meant to be meritocratic. However, this tends not to be true in practice, with many influential families rather over-represented in the upper echelons of the synod. Any such, familial influence cannot be wielded openly, due to the avowed commitment to merit before birth. This means that politics between families plays out in the decisions and policies of the synod itself. 

Stormheight is spatially segregated, with those of higher prestige within the Harmony living closer to the peak of the mountain (and the grand campus of the synod). Since the Harmony is so inter-woven into every aspect of society, those who rise high within its ranks are also those who accrue the most wealth. The cities upper levels are populated by those who are given roles of `leadership' or `organisation' by the synod assessors. These are coveted positions, as they are viewed as being entrusted with coordinating and maintaining harmony between elements of society. As one rises in the Harmony by displaying prowess in ones' given duties, there are a large number of well-off skilled workers and professionals who occupy the middle slopes of the city. The lower slopes are densely populated slums and rigorously policed by the Keepers of Order (town guards). Here dwell those who are given manual labour as a life-path by the synod assessors. This offers little scope for advancement. Thus, the slums are saturated with the dirt and crime that follow in the wake of such dead-end living.   

\subsubsection{Outlying villages}
The towns and villages that dot the highlands are agriculturally focussed. They have grown vastly over the last century due to the development of aethertech, as this has made agriculture far more viable in the damp, cold, and stormy highlands. The smaller villages survive by trading agricultural products with merchants from stormheight. The larger towns have more diverse economies, a result of their increasingly large populations. This also means that such settlements have attracted more interest from the synod, having appointed governors and bureaucrats. 

\subsubsection{Highlanders}
The term `highlander' is reserved for those who still follow the semi-nomadic lifestyle of the pre-aethertech highlands. They survive by foraging and hunting in the harsh wilderness of the peaks. The highlanders do not follow the Harmony, instead having a culture based around reverence for their ancestors. This means they are persona non grata as far as the synod are concerned. Despite this, highlanders tend to be closely linked with small villages, being welcome there when the synod isn't watching and often inter-marrying with the village folk.   

Highlanders are organised into clans, each having a territorial area for foraging and hunting. These areas are fluid and change as clans become more or less powerful as well as during inter-clan conflicts. Clans are lead by two people, one man and one woman (they need not be either related or married). These leaders are the symbolic parents of the entire clan (which is regarded as a kind of extremely extended family). Both leaders have an equal role in the governance of the clan and in dispute resolution. Clan leaders are elected, every few prime-cycles, by popular vote.  

The highlanders refer to themselves as the ``atladarra'', or ``people of the spear''. To them the spear represents the chief values of their culture: defending and providing for oneself and the clan. Every highlander owns at least a ceremonial spear that is marked with various designs and ornaments to reflect their personal history and achievements. Wood is hard to come by in the highlands, so is reserved only for spears with cultural significance. More utilitarian weapons have their haft composed of crab sinew and chitin, stretched and wound into the form of a pole. The highlanders' preferred mode of combat is spear and shield, with either a bow, javelins, or a sling for use at range. The highlands mineral resources are either buried deep, or found on the highest and most inhospitable peaks. Thus, highlanders tend to use the more available resource: crab chitin. Giant stone crabs in particular provide a material more flexible and nearly as strong as steel. Spear heads, however, are always made of metal. Older highlanders shun aethertech and its by-products like firearms, they believe aether is the material that composes the spirits of their ancestors (aethermancy is therefore blasphemous). This opposition has been slowly eroding among younger clan members as the utility becomes increasingly apparent.  



\section{Living in the highlands}
This section will provide some information about how people in the highlands get by day to day.

\subsection{What people grow and eat}
The highlands climate supports abundant lichen and moss that provide grazing for its crab-like fauna. Cultivation of other crops in the highlands has historically been difficult, due to the altitude and inclement weather. Notable native plants include `tzak', a root vegetable with a spicy flavour, and a variety of coral-like plants (`hard plants') whose hard shells conceal firm fibrous material that is tasty when cooked. With the rise of technology based on aethermancy, the number of viable crops has greatly increased and sedentary farming has become more viable. Barley is the most widely farmed grain with beans, turnips, onions, carrots, and even strawberries, in prime summer, making up the bulk of agriculture. Grapes have recently become a viable crop due to aetheric devices that regulate air temperature around them. Dietary protein tends to be sourced from wild bird-life in summer, as well as both hunted and raised crabs. Meat from cattle or pigs is expensive and imported. 

Those lower in income tend to live largely off barley porridge, onions, and beans. These folk would tend to drink water and grain alcohols. Bread is more expensive and makes up a vital component of a better-off citizen's diet. In addition, those with more means would eat the meat of various farmed or hunted crustacean-like creatures. Wine is the beverage of choice for those with the money, it is mostly drunk watered as it is produced with a high alcohol content to aid in preservation.   

\subsubsection{Currency}
The currency in the highlands is called the `dinar' (plural `dinari'). These are bronze coins, a lower denomination called `cestis' (made of copper) is also widely used. There are 100 cestis (c) to a dinar (d). To give an idea of how much the coins are worth you can note that manual labourers make 36 - 48 c per day. A common soldier receives a salary of 60 c per day, and a skilled labourer gets 80 - 120 c per day. 

The currency is not backed by the value of the coins themselves, as in many other lands. Rather, a dinar's value is controlled and guaranteed by the synod itself. As such, all coins are stamped with a symbol of the Harmony. 

\subsubsection{Cost of living}
A loaf of bread, that would constitute a staple food, costs around 4 copper cestis (4 c). A cheaper alternative being edible varieties of lichen (nutritious, but barely flavourful), costing about 2 c for a similar amount to a loaf of bread. A cup of wine ranges in price from 5 c at the low end to around 30 c at the high end. Finally, a bowl of stew goes for 6 to 40 c, depending on the tavern. Renting an inn room for a night costs between 30 and 60 c (a bunk-house or hostel costs 10 to 15 c a night). Inns and taverns tend slightly towards the more expensive end of the price ranges in Stormheight itself.  

\begin{table}[ht!]
	\centering
	\begin{tabular}{|l|l|}
		\hline
		Edible lichen (500 g)  & 2 c\\
		Grain porridge (bowl) & 2 c \\ 
		Bread (500 g) & 4 c \\
		Tavern meal & 6 - 40 c \\
		Inn room (per night) & 30 - 60 c \\
		Brandy (cup) & 9 - 60 c \\
		Wine (cup) & 5 - 30 c \\
		Whisky (cup) & 1 - 6 c \\
		\hline		
	\end{tabular}
\caption{Prices of common goods}
\end{table}

\subsubsection{Trade with the outside world}
The highlands trade extensively with the outside world. Trade routes are hard going, as one must negotiate many winding mountain passes. This makes imported goods expensive for common folk. Imported food, for instance, is usually four times as expensive as local fare, rising even further when winter makes trade routes impassable. 

  



\chapter{Aether}
Aether is the term used to describe the natural energy that flows throughout the world. All creatures are attuned to aether to some extent, but study and practice can allow one to directly manipulate their own aether, as well as draw it in from their environment. Aether manipulation can create a variety of effects, either aligned to the natural forces or altering the aetheric fields of living creatures.

Aether is thought to be composed of fields, with the physical world being the result of many thousands of such fields interacting. Aetheric fields pulse and vibrate with energy, in turn they are characterised by the frequency of their vibrations. This frequency seems directly linked to the function of the field. For instance, some naturally occurring iron has a particular vibrational frequency that pulls other metal towards it, adding aether to this frequency increases the strength of this effect. A skilled aethermancer could even `flip' the effect so that the metal is pushed away instead. The ability to reverse field functions was conjectured by philosophers to result from a `symmetry of fundamental opposites' that remains a central pillar of aethermantic studies. New vibrations can even be added to existing fields, thus one could even cause a piece of wood to attract metal. An aethermancer performs these feats using their own aetheric field as a tuning fork for other fields, first initiating a vibration in their own field and then imposing it on another field of interest.

Aethermancy is difficult, as projecting and controlling one's own aetheric field is non-trivial and requires extensive practice. Note, however, that anyone can become an aethermancer, just as they could become a swordsman with enough training. An aethermancer is just someone who has learned the skill to actually control the interactions of their body's own fields. 

The constant interaction of aetheric fields means that an aethermancer's control is always tenuous. They can easily lose control of processes they attempt to change or initiate, sometimes with surprising results. Rules for aethermancy will be discussed in Section~\ref{sec:aethermancy}.

\section{Aether points}
Every character's body has a maximum of 1 aether point at any one time. Holding additional points requires a device called an aether capacitor (see Section~\ref{sec:capacitor}). %Any character can spend this on the \textlf{aether surge} ability.

\subsection{Aether surge}
The character concentrates their aether to grant themselves a burst of power. The character can spend an aether point to gain the benefits of any \textlf{perk} for 1 round.
%At a cost of 1 aether point they may gain \textlf{edge+} on a roll of their choice. If the character is proficient in \textlf{aethermancy}, they may choose two rolls instead of one.

\subsection{The energy of life}
A character's body uses aetheric energies to function, if they run out of aether points they can spend \textlf{endurance} in place of it.

\subsection{Recharging aether}
A one hour rest restores 1 aether point to any character that had none remaining. A hearty meal also restores up to 1 aether point. A full night's sleep restores up to 3 aether points. A character cannot gain aether if they would exceed their allowed maximum.

\subsection{Aether crystals}
Sometimes aether crystallises of its own accord. These crystals can be consumed by anyone to gain aether points (which can be in excess of their normal limit). The number gained depends on the strength of the crystal. In addition to this, a character consuming a crystal experiences a rush of life and enhanced sensations for 10 minutes.
\subsubsection{Crystal jitters}
The consumption of aether crystals is addictive, withdrawal manifesting in the form of uncontrolled shaking motions throughout the victim's body. Each time a character uses a crystal they gain crystal points equal to the gained aether. Crystal points decay at a rate of 1 per week. A character with any crystal points must make a \textlf{resolve} check each day vs 9 + crystal points. If they fail, they are overcome with a need to acquire and use more crystals. This means they have \textlf{edge-} on all checks until they consume a crystal.   




\chapter{Character creation}
These are additional rules providing the options for character species and additional backgrounds in this fantasy setting.

\section{Character species}

\subsection{Humans}
Humans are a dynamic species: adaptive, resilient, and highly cooperative.

Humans are the most prevalent species in the highlands of Atla, making up the majority of the rural population. Stormheight, however, is considerably more diverse, with lots of peoples from other species both resident and passing through.

The average human lives for up to 80 prime cycles (if they survive infancy), 30 being considered a mature adult, though humans are legally adults at the age of 16 in most societies. Despite lacking the patience or long practice of the Elves, they make up for it with a natural ability to quickly adapt to new situations or tasks. 

The elves' popular view of humans is that they are flighty and lacking in focus. Mus-folk stereotype humans as grim and unscrupulous. Trogs stereotype humankind as deceitful and selfish.  

\subsubsection*{Flexibility}
Humans are extremely adaptable and may spend an extra 2 points when choosing \textlf{perks} and \textlf{proficiencies} at character creation.


\subsection{Elves}
Though elves somewhat resemble humans from a distance, there is a world of difference on closer examination. Their skin varies in shade between green and light brown with a similar texture to tree bark. Their hair is like a cascade of leaves and their eyes range in colour from yellow, through green and blue, to white, the pupils of which are narrow and cat-like. Their fingers are like long flexible twigs with knots for joints and their feet resemble spreading tree roots. Elves start out short by human standards, and are also of a slighter build. However, they never stop growing. All of them posses the curious ability to alter their hair colour at will. However, if they are not careful it will change on its own to reflect their mood. 

In terms of more exotic differences with humans: elves are incapable of digesting meat. They are vegetarian with a preferred diet of sweet, sugary things. Elven eggs gestate externally in a similar manner to those of fish. Male-female differences in elves are entirely internal, there is no way to tell by sight. Additionally, elven blood is clear, as it contains no iron, and their hearts are far smaller than that of a man, as capillary action plays a major role in their circulatory systems.

Elves' lives and metabolisms move more slowly than those of humans, and they are `saplings' till the age of about 40 where the reach an average of 5 ft tall. This slowness of biology heavily influences the perspective of elves. They are more relaxed and far slower to decision and judgement than the frenetic, hasty pace they observe in other species. Additionally, they only need to eat or sleep once every two or three days (they absorb a small amount of passive nutrition from the soil through their feet). In fact, elves metabolism slows down and they become more tree-like as they age. If an elf reaches several centuries of age they become almost indistinguishable from a tree, as they move so slowly and have grown so large (these ancestor trees are greatly revered in elven culture). Elves are common in Stormheight and are often enthusiastic followers of the Harmony. 

Humans tend to stereotype elves as slow, pedantic, and more interested in pleasurable living than work. Mus-folk and Trogs popularly assert elves to be crazy and lacking in any kind of sense. 

\subsubsection*{Perfectionists}
Elves live long lives and thus have time to pursue their skills to a point few others can reach. As such, elves may choose one additional skill \textlf{proficiency}.

\subsubsection*{Elven aging}
Younger elves (less than 100 years) are under 7 ft tall and have no special rules. Older elves (under 150 years) gain \textlf{large creature} rules but have 1 fewer \textbf{AP} in combat. Elves any older are too slow to make good adventurers.


\subsection{Mus-folk}
The mus-folk (or `Musmus' in their own language) resemble large rats (note that they are not humanoid rats). They reach a maximum height/length of 3 - 4 ft and tend to inhabit the margins of human settlements where they are regarded as a mixture of pest and second-class citizen (rude terms for them include `squeakers' and `ratties'). Their own language consists of chirps, squeaks, and pheromone signals; rather than words. In particular, a mus can smell the emotional state of other mus. However, they are fully capable of producing human and elvish speech. When they do, it comes out in a rapid torrent of words infused with poor grammar and word repetition. 

Musmus are highly community oriented creatures and live in large, dense groups. They live relatively short lives (around 40 years) but breed rapidly (a mus pregnancy lasts about 3 months and produces a litter of 2 to 4 pups). Mus pups mature quickly, being able to fend for themselves by 3 years old. Their families are tight-knit with settlements being composed of many sprawling dwellings that contain vast extended families. Mus-folk are famed for their willingness to eat nearly anything, coveting cheese most of all (which they also ferment to produce beer). In fact, when mus-folk die it is customary for their family to consume the corpse. However, eating another mus who is not related to you is deeply taboo.

Mus-folk prefer underground living and make burrows throughout the highland peaks, with a significant population in Stormheight itself. Their nest-houses are built of any material that is to hand, mostly dried mud. Mus-folk do not tend to use hard furnishings, like chairs, instead their dwellings are filled with piles of cushions, blankets, or soft grass (into which they can burrow). Their prolific digging means they gravitate towards mining for minerals and aether. They are often stereotyped as light-fingered, insular, and mischievous by other species. These suspicions are largely well-founded.

\subsubsection*{Sneak \& squeak}
Mus-folk are naturally stealthy. They have \textlf{stealth proficiency}, even if it does not fit their background otherwise. 

\subsubsection*{Sniff-sniff smell-smell}
Mus-folk have very sensitive noses, granting \textlf{edge+} on \textlf{awareness} checks that can benefit from their sense of smell.

\subsubsection*{Dark-dwellers}
Mus-folk treat \textlf{low} light conditions as \textlf{full} illumination and \textlf{dark} situations as \textlf{low} lighting.

\subsubsection*{Second-class citizens}
Mus-folk will often experience prejudicial treatment from humans and elves. NPCs from these species might have negative \textlf{attitude} scores towards mus-folk characters  and those in their company.


\subsection{Trogs}
Trogs are large (usually 7 ft tall) and bulky, but their most striking feature is their large eyes. Trog facial features tend to be heavy, with wide square jaws and large beak-like noses. Trogs have tough, hairless skin that is grey and slightly scaled. In general they are powerfully built and quite difficult to kill or injure.  Traditionally, trog groups are familial and organised around a mother trog, a very large and formidable female (female trogs tend to be larger and stronger than males), with a harem of males who care for the young trogs. Plenty of young trogs find this authoritarian system stifling, so they form their own more egalitarian, but heavily community oriented, bands which tend to migrate towards settlements like Stormheight. This form of living is scorned and belittled by their elders. 

Trogs typically prefer dark places, due to their sensitive eyes. This means they tend to live in close proximity to the Mus-folk, often forming a symbiotic system where Trogs provide the muscle and Mus-folk the cunning. They live for slightly longer than humans and tend to grow larger and stronger, but slower, as they age. The largest (and oldest) trogs reaching around 8 ft tall. Their natural strength, and prejudice of other species, means they are often found working as guards, labourers, or mercenaries. Trogs are omnivores, but their diets have a heavy slant towards meat supplemented by the mushrooms they farm underground. Their biology makes them largely inured to alcohol, with recreational drugs instead taking the form of hallucinogenic mushrooms. 

Other species have a stereotyped view of Trogs as stupid and violent. While Trogs do think and speak slowly, they are not actually stupid. Mus-folk, in particular, appreciate their slow, considered honesty as well as their community orientation. In terms of violence, trogs are slow to anger but experience this emotion as a powerful and passionate outburst when pushed too far (hence popular beliefs of other species). 

\subsubsection*{Size matters}
Trogs are \textlf{large creatures}, having \textlf{cleave} and \textlf{power} $+ 1$ on close-combat attacks as well as $+1$ \textlf{endurance}. Their great physical strength allows them to lift and carry twice as much as \textlf{medium} sized creatures. Their slow thought process means they cannot start the game with a \textlf{Wit} score greater than $0$. At character creation, a player may choose to be a runty trog instead, where they lose \textlf{cleave} and \textlf{power} bonuses, but the starting \textlf{wit} limit changes to $+1$. 

\subsubsection*{Dark-dwellers}
Trogs treat \textlf{low} light conditions as \textlf{full} illumination and \textlf{dark} situations as \textlf{low} lighting.

\subsubsection*{Second-class citizens}
Trogs will often experience prejudicial treatment from humans and elves. NPCs from these species might have negative \textlf{attitude} scores towards trog characters and those in their company.



\section{Starting equipment}
A new character may choose one piece of armour and up to three weapons from Table~\ref{tab:start-gear}. All characters get a set of clothes. Default starting money is 2 d, this can be increased for some backgrounds (suggested maximum 6 d).
\begin{table}[ht!]
	\centering
	\caption{New adventurers starting gear.}
	\label{tab:start-gear}
	\begin{tabular}{|l|l|l|l|l|}
		\hline
		Weapon & \textlf{Power} & Hands & Lethality & Notes\\
		\hline
		Rusty sword & $+1$/- & 2/1 & 1 & \textlf{Aim} $+1$\\
		Corroded dagger & - & 1 & 1 & \textlf{Small}\\
		Notched axe & $+2$/$+1$ & 2/1 & 1 & -\\
		Worn crossbow & - & 2 & 1 & \textlf{Reload} 1, Range 6, \textlf{Aim} $+1$\\
		Pitted hammer & $+1$/- & 2/1 & 1 & \textlf{Penetration} 1 \\
		Aged spear & - & 2/1 & 1 & \textlf{Rending} 2/1\\
		Creaky short bow & - & 2 & 1 & Range 4\\
		Shabby pistol & $+1$ & 1 & 1 & \textlf{Reload} 2, Range 4, \textlf{Aim} $-1$ \\
		Ramshackle musket & $+1$ & 2 & 2 & Range 6, \textlf{Reload} 3, \textlf{Aim} $-1$ \\
		Dented shield & - & 1 & - & Ignore \textlf{heavy armour defence} penalty \\
		\hline
	\end{tabular}
	\begin{tabular}{|l|l|l|l|}	
		\hline
		Armour & \textlf{Toughness} & Type & Notes\\
		\hline
		Chainmail hauberk & $12$ & Heavy & \textlf{edge-} on \textlf{stealth}, jumping, swimming, and climbing. \\
		Leather lamellar & $10$ & Medium & \textlf{edge-} on \textlf{stealth} \\
		Gambeson & $9$ & Light & - \\
		Travelling clothes & - & - & - \\
		\hline
	\end{tabular}
\end{table}


\section{Additional backgrounds}

\subsection{Student of the synod}
The character has spent time studying aether and its powers under the auspices of synod in Stormheight. An important aspect of this background is the degree to which the character believes in the Harmony and its social order. As individuals who stand high up in the heirachy, they may be tempted to justify the status quo based on their own success. Skill suggestions: Aethermancy, Mechanical, Aethertech, Infusary, History, Religion, Plants, Animals.

\subsection{Savant}
The character has learned some of the mysteries of aether on their own. Such individuals often hide their talents from the synod for fear of being forced into conformity or having to leave their home village. You should consider why the character hides their abilities, as well as their attitude to the synod and Harmony. Skill suggestions: Aethermancy, Infusary, Aethertech, Survival, Persuade, Decieve. 

\subsection{Exile}
The character has been exiled by the synod for repeated defiance of the Harmony's dictates. They now skulk on the margins of society, keeping away from the eyes of law enforcement in the far-flung rural districts of the highlands. Some such individuals find sanctuary among the Mus-folk, who have little regard for the Harmony or synod, others turn to less than legal ways of making a living in the aethertech black-market. You should decide how your character has got by and why they were exiled. Skill suggestions: Stealth, Disguise, Deceive, Survival, Slight of hand, Aethertech, Awareness.  

\subsection{Highlander}
The character is belongs to the partially nomadic people who form the original inhabitants of the highlands. They have lived a difficult life in the harsh environments of the highlands, constantly needing to avoid the synod's agents (due to religious differences). A highlander will belong to a given tribe whose hunting grounds are staked out over certain region (the exact boundaries are fluid and often fought over by the tribes). Highlander society values strength in the form of the ability to provide and defend. In the past the tribes have shunned aethertech but some younger highlanders have begun to adopt this new technology. An important aspect of this background is the character's attitude to highland villagers, who live under the synod's rule. Are they pathetic weaklings or just highlanders who have lost their way? Additionally, does the character follow the old ways or are they interested in aethertech? Skill suggestions: Survival, Athletics, Awareness, Ride, Animal, Plant.  


\section{Additional convictions}

\subsection{The Harmony of all}
The character is firmly convinced of the teachings of the Harmony. They live their life in pursuit of balance within themselves and with society as a whole. They tend to regard the synod and its laws as an essential ingredient in achieving harmony with the natural world.

\subsection{Free spirit}
The character detests the enforced conformity that the Harmony brings with it. Having these views publicly known can be dangerous. 

\subsection{Aethermancy unlimited}
Aethermancy and aethertech are forces for good, their exploration brings nothing but enlightenment. Some radical individuals might also hold that it should be free from the synod's interference. 

\subsection{Aether exploited}
Aethermancy and aethertech go too far in their rampant consumption of the world's aether. One cannot achieve the claimed Harmony while mercilessly extracting resources.



\section{Character templates}
Here we provide a few templates for character building. This does not involve any elements of personality or characterisation, just suggestions that provide characters that can build towards certain combat roles. 

\subsection{The commander}
The commander is a front-line fighter who uses their leadership skill to empower allies. Their choice of armour makes them hard to shift and a shield enhances their \textlf{defence}. The attribute choice reflects a forceful personality who is quick of thought and attentive to detail. This  enhances their leadership and endurance, as well as making them harder to trick and better at landing hits in combat. Their \textlf{perk} choice grants them their \textlf{resolve} as a bonus to \textlf{power} after making \textlf{Leadership} actions. They could either aim to buy more \textlf{leadership} perks, those to increase their combat prowess like \textlf{shield bash}, or some \textlf{defensive perks} like \textlf{indomitable}.

\begin{tabular}{|l|l|}
	\hline
	Name & Commander\\
	Weapons & Rusty sword and shield\\
	Armour & Chainmail hauberk\\
	Skills & Leadership\\
	Proficiencies & Swords\\
	Perks & Lead the charge\\
	Attributes & M 0, W 1, C 0, R 2\\  
	\hline
\end{tabular}

\subsection{The gunslinger}
The gunslinger aims to rapidly eliminate their foes in a hail of gun shots. They are lightly armoured as their skills lend themselves towards ending fights quickly and being free to get the jump on slower targets. The attributes are chosen to reflect a sly and quick-witted personality. This also maximises accuracy, skill synergy, and avoiding being hit (\textlf{cunning} can also enhance \textlf{sneak attacks}). Their strategy would be to use \textlf{quick-draw} to unload all three pistols, one at a time, before having to do any reloading. They could advance the build in several ways: lean into \textlf{sneak attacks} with the \textlf{shoot first perk} and \textlf{cunning plan}, go for two pistols at a time with \textlf{Two-weapon fighting proficiency}, or choose utility like \textlf{too hot to handle}, \textlf{arrow to the knee}, or \textlf{rifle drill}. The \textlf{evasive perk} is also a great choice for increased defences.
 
\begin{tabular}{|l|l|}
	\hline
	Name & Gunslinger\\
	Weapons & 3 Shabby pistols\\
	Armour & Gambeson\\
	Skills & Awareness, Stealth\\
	Proficiencies & Firearms\\
	Perks & Quick-draw\\
	Attributes & M 0, W 1, C 2, R 0\\  
	\hline
\end{tabular}

\subsection{The aethermancer}
The aethermancer focuses on aetheric manifestations to control the battlefield. They carry a musket for when they run out of aether points, or do not wish to spend them. Their choice of \textlf{medium armour} means some protection without the full downsides of \textlf{heavy armour}. \textlf{discharge} was chosen as their manifestation but there are a wide variety of options here. Their attributes are chosen to maximise the effectiveness of their manifestations (but will also benefit their weapon attacks). This build could be made more defensive by moving a point to \textlf{cunning} or \textlf{resolve}. Further experience points would be invested into new manifestations (see Section~\ref{sec:manifestations}).

\begin{tabular}{|l|l|}
	\hline
	Name & Aethermancer\\
	Weapons & Ramshackle musket\\
	Armour & Leather lamellar\\
	Skills & Aethermancy\\
	Proficiencies & Firearms, Aetheric learning\\
	Perks & Discordant blast\\
	Attributes & M 1, W 2, C 0, R 0\\
	\hline  
\end{tabular}

\subsection{The highlander}
The highlander comes from a nomadic people that travel the peaks with their herded animals. They have chosen a pole weapon for versatility (and simplicity to fit the background). They come from a warrior culture, so their armour is strong without being encumbering. Their attributes reflect warrior aggression and the tenacity of a nomadic people. Finally, \textlf{berserker} is chosen to represent the fury with which they fight to defend their people. The build could be continued by purchasing upgrades to \textlf{berserker} or with things like \textlf{reckless attack} or \textlf{power attack} to get the most out of each swing.

\begin{tabular}{|l|l|}
	\hline
	Name & Highlander\\
	Weapons & Aged spear\\
	Armour & Leather lamellar\\
	Skills & Survival, Animal\\
	Proficiencies & Pole weapons\\
	Perks & Berserker\\
	Attributes & M 2, W 0, C 0, R 1\\  
	\hline
\end{tabular}




\chapter{Perks and proficiencies}

\section{General proficiencies}
These perks do \textbf{not} occupy equipment slots for passive or active \textlf{perk}, their effect is always active.


\subsection{Origin: aetheric learning [3]}
(Requires \textlf{skill proficiency: aethermancy}). This may only be chosen at character creation. This grants the character an aether capacitor that can hold 1 aether point (see Section~\ref{sec:capacitor}). Additionally, they may choose one aether manifestation with a cost of 3 experience or less.

\subsection{Origin: veteran [3]}
(Requires at least 2 \textlf{weapon proficiencies}). This may only be chosen at character creation. The character has seen some combat in their time. As such, they gain one \textlf{weapon expertise} and may choose weapons from Tables~\ref{tab:weps} or \ref{tab:range-weps} up to a cost of 5 d. 

\subsection{Origin: professional [3]}
(Requires at least 2 \textlf{skill proficiencies}). This may only be chosen at character creation. The character ranks a chosen skill up to \textlf{skill adept} and may select tools from Table~\ref{tab:tools} up to the value of 5 d.


\subsection{Weapon proficiency: X [2]}
In this setting, X can be drawn from: unarmed, swords, daggers, axes, blunt weapons, pole weapons, extended weapons, bows, crossbows, slings, and firearms. A character who is not \textlf{proficient} with his equipped weapon has \textlf{edge-} on \textlf{aim} and \textlf{damage checks}.

\subsection{Yeoman bowman [3]}
(Requires \textlf{Weapon adept: bows}). Removes the \textlf{cumbersome} penalty from longbows. 


\section{Combat styles}
These open up new bonuses to different modes of combat specific to this setting, they must occupy an equipment slot for active/passive \textlf{perks} to be usable.

%\subsection{Aetheric savant [3]}
%\textlf{Active perk}. The character can spend an aether point to gain the benefits of any \textlf{perk} for 1 round.


\subsection{Archer's rhythm [2]}
\textlf{Passive perk} (Requires \textlf{Weapon adept: Bows}). If the character hits an enemy with a ranged attack from a bow, their next such action has $+1$ \textlf{aim}.


\subsection{Expansive knowledge [6]}
\textlf{Passive perk} (Requires \textlf{Skill expert: aethermancy}). The character can equip 2 aether manifestations per \textlf{active perk} slot.


\subsection{Hammer time [4]}
\textlf{Passive perk} (Requires \textlf{Weapon adept: Blunt}). \textlf{Critical hits} from blunt weapons inflict \textlf{knock down} on the victim. This only applies to creatures up to one size category larger than attacker.


\subsection{Lethal flourish [3]}
\textlf{Passive perk} (Requires \textlf{Weapon adept: Swords}). The character delivers their finishing blows with the point of a blade. While wielding a sword, the character's attacks gain \textlf{Rending} 1 against victims with missing (or zero) \textlf{endurance}.


\subsection{Loose aether [3]}
\textlf{Passive perk}. The character has learned to regather dissipating aether when a manifestation fails. If the character aether manifestations are successfully produced but fail to cause any effect, 1 spent aether point is regained.


\subsection{Main gauche [3]}
\textlf{Active perk} (Requires one of \textlf{Two-weapon fighting}, \textlf{pole position}, or \textlf{snake fist}). There is an art to pairing weapons for fighting with both hands, this character has mastered it. When dual-wielding, the character may elect to use a close-combat weapon defensively. The chosen weapon may not be used attack to this round, but grants \textlf{edge+} on the wielder's next \textlf{Defence} roll. This also applies when fighting unarmed or using the \textlf{pole position perk}.


\subsection{Momentum [4]}
\textlf{Passive perk} (Requires \textlf{Weapon adept: Axes}). When wielding an axe, the character gets \textlf{cleave} + 1 on their next attack after scoring a \textlf{critical hit}.


\subsection{Multi-tasker [5]}
\textlf{Passive perk} (Requires \textlf{Skill adept: aethermancy}). The character has learned to concentrate on many things at once. This allows them to maintain two \textlf{channelled} effects at the same time for the cost of just 1 \textbf{AP}. The benefits of this perk can be used only once per turn.


\subsection{One in the sleeve [4]}
\textlf{Active perk} (Requires \textlf{Weapon adept: daggers}). If the character is targeted by an attack, they can spend 1 \textbf{RP} to draw and throw a dagger at the attacker.

\subsubsection{Upgrade: Two in the face [2]}
(Requires \textlf{One in the sleeve}). If the character hits with a \textlf{one in the sleeve} attack, they may spend X additional \textbf{RP} to grant their dagger throw X extra \textlf{damage checks}. 


\subsection{Overcharged manifestation [3]}
\textlf{Passive perk} (Requires \textlf{Skill adept: Aethermancy}). The character can spend an extra Aether point on a manifestation to gain \textlf{edge} on \textlf{resist} checks involved in the power.


\subsection{Pike wall [3]}
\textlf{passive perk} (Requires \textlf{Weapon adept: Pole weapons}). Pole weapon attacks made with \textlf{reach} also inflict a \textlf{shove} attempt on the target if they cause damage . 


\subsection{Pole position [3]}
\textlf{Passive perk} (Requires \textlf{Weapon adept: Pole weapons}). The character may use a quarter staff (or other pole weapon) to make an extra attack with the haft (with no weapon bonuses). Note that the rules for attacking with multiple weapons still apply when using this \textlf{perk}.


\subsection{Reach of the aether [3]}
\textlf{Passive perk} (Requires \textlf{Skill proficiency: aethermancy}). Aetheric manifestations have + 2 range when selecting targets.


\subsection{Reloading drill [4]}
\textlf{Passive perk} (Requires \textlf{Weapon adept: crossbow}). The character is well practised at rapidly preparing crossbows to fire. This reduces crossbow \textlf{reload} action costs by 1.


\subsection{Rifle drill [4]}
\textlf{Passive perk} (Requires \textlf{Weapon adept: Firearms}). Firearms \textlf{reload} action costs are reduced by 1.


\subsection{Shield bash [2]}
\textlf{Passive perk}. The character may use a shield to \textlf{shove} (add the \textlf{Defence} bonus to such checks).

\subsection{Upgrade: Shield slam [3]}
(Requires: \textlf{Shield bash}) The character's \textlf{shield bash shove} actions inflict damage with \textlf{lethality} 1 on \textlf{critical success}.

\subsubsection{Upgrade: Reactive bash [3]}
(Requires: \textlf{Shield bash}) When the character succeeds on \textlf{all-out defence} against an attack, while using a shield, they can immediately use \textlf{shield bash} for free.

%\subsubsection{Upgrade: Shield combo [4]}
%(Requires: \textlf{Shield bash}) The character may use \textlf{shield bash} for free as part of an \textlf{all-out attack}.


\subsection{Signature manifestation [4]}
\textlf{Passive perk} (Requires \textlf{Skill adept: aethermancy}). The character may choose one aetheric manifestation that they know. This has \textlf{edge+} on \textlf{aethermancy} checks when manifesting it. This can only be chosen once by a character.







\section{Aether manifestations}
\label{sec:manifestations}

All manifestations are \textlf{perks} and can be purchased even if the character does not have \textlf{aethermancy proficiency}. See section~\ref{sec:aethermancy} for rules on how to produce manifestations. Experience costs are listed in square brackets, e.g. [3], whereas \textlf{aethermancy} difficulty is given in round brackets, e.g. (8). A `C' indicates a \textlf{channelled} manifestation, e.g. (8,C). 


\subsection{Aether siphon (11) [2]}
\textlf{Active perk}. An aethermancer can establish a resonance with another creature such that the aether is stripped from its body. For 2 \textbf{AP} (but no aether points) a chosen target within range 1 (4 m) must \textlf{resist(C)} or lose an aether point. If the target has no aether points, it loses an \textlf{endurance} instead (this cannot be increased by \textlf{critical success}). If the aethermancer wins the \textlf{resist} check, they gain 1 aether point, \textlf{critical failure} means the intended victim can use the \textlf{exploit weakness} reaction on the aethermancer.
\subsubsection{Upgrade: Voracious siphon [4]}
The \textbf{AP} cost of \textlf{aether siphon} is reduced to 1. 
\subsubsection{Alternate: Transference [2]}
You may manifest \textlf{aether siphon} to transfer 1 point of your aether to the target. The target may \textlf{resist(C)} if they wish to.   


\subsection{Aetheric wall (10,C) [3]}
\textlf{Active perk}. For 2 \textbf{AP} you manifest a wall of shimmering aether up to range 5 (12 m) long. The wall must start within range 5 of the aethermancer. Passing through the wall costs any creature 1 \textbf{AP} and requires a \textlf{resist(M)} check, failure means they cannot traverse the wall and suffer damage with \textlf{lethality} 1 (X \textlf{critical success} thresholds for the aethermancer adds X \textlf{lethality}). This lasts for 10 minutes and is a \textlf{channelled} manifestation.


\subsection{Animal aether (-) [3]}
\textlf{Passive perk}. Every living creature hums with an aetheric field. By carefully tuning your own aetheric field into resonance you can hear the mood and emotions of animals. 
\subsubsection{Upgrade: Animal resonance [3]}
Your mastery of the aetheric fields of animals allows you to communicate emotions to them as well as attempt to \textlf{persuade} or \textlf{deceive} them.


\subsection{Animate homunculus (9) [2]}
\textlf{Active perk}. You can shape a small creature/object out of any given material (30 cm is the maximum size), this is then imbued with aether over 10 minutes. This grants it a kind of limited life, any/all of your homunculi can be controlled directly by you while they are within range 25 (52 m) (this requires your full concentration). Otherwise homunculi remain inert. A homunculus cannot speak, has no skill \textlf{proficiencies}, and cannot inflict damage in combat (they have \textlf{toughness} 7, \textlf{Defence} +1, and are destroyed by any damage). You can create 1 extra homunculus per level of \textlf{critical success} on the \textlf{aethermancy} check.
\subsubsection{Upgrade: Automaton [3]}
Your homunculi are always animated and will obey simple verbal commands until the task is complete or they are told to stop. You may endow a homunculus with a single \textlf{skill proficiency} (that you have) on creation.
\subsubsection{Alternate: Golem (12,C) [4]}
You can create a medium (man-sized) homunculus with a manifestation that requires increased concentration. These creatures can make unarmed attacks in combat (with \textlf{weapon proficiency}). Use your own \textlf{wit} and \textlf{might} to determine \textlf{aim} and \textlf{power}. The golem is a \textlf{mindless} creature, has \textlf{toughness} 10, 2 \textlf{endurance}, and is destroyed if the latter runs out. 


\subsection{Arcing aether (12) [3]}
\textlf{Active perk}. You mould your aether into a resonance with elemental lightning. For 2 \textbf{AP}, aetheric lighting leaps from your body and strikes a target within range 5 (12 m). The target must \textlf{resist(R)} or suffer damage with \textlf{lethality} 1 (X \textlf{critical success} thresholds for the aethermancer adds X \textlf{lethality}). The lightning leaps randomly to a second target within range 1 of the first (preferring those nearer by), the same leap can happen once more. Any individual target can only be struck once by this manifestation.
\subsubsection{Upgrade: Resonant arcing [4]}
The lightning can leap one extra time and targets can now be struck more than once (but cannot be struck twice in a row).
\subsubsection{Upgrade: Paralytic arcs [2]}
Victims damaged by this manifestation are \textlf{immobilised} in their next turn.


\subsection{Befuddle (11) [2]}
\textlf{Active perk}. A discordant aetheric projection removes your target's ability to distinguish friends from foes. Spend 2 \textbf{AP} and choose a target within range 5 (12 m), if they fail a \textlf{resist(W)} check, they regard all creatures as hostile and dangerous. This means they can panic, run away, or lash-out at anything nearby. The Game Master should choose combat targets and behaviour for the victim randomly. This lasts up to 1 minute. Each round, after the first, the victim may re-attempt to \textlf{resist} (\textlf{critical failure} by X thresholds means they cannot re-attempt for X rounds).
\subsubsection{Alternate: Bewildering barrage (12,C) [3]}
\textlf{Befuddle} becomes a \textlf{channelled} effect. Thus, if the victim \textlf{resists} they do not experience the effects for that round. On subsequent rounds they must continue to \textlf{resist} until the effect is ended. This lasts while the aethermancer maintains it and the target is within range 10 (22 m).


\subsection{Bend light (10,C) [3] }
\textlf{Active perk}. Aether flows from your fingers creating an aetheric lattice that bends light. At a distance up to range 10 (22 m), you can create a stationary illusion up to medium size. Anyone looking at this effect must \textlf{resist(W)} to decide if they are fooled. This is a \textlf{channelled} effect. The illusion is only visual, it makes no sounds, smells, and cannot be touched. 
\subsubsection{Upgrade: Major illusion [3]}
You can bend light with such dexterity that you can create illusions up to 4 m in size. 


\subsection{Blinding ray (9) [3]}
\label{spell:blind-ray}
\textlf{Active perk}. You pour aether into a light-source you touch, focusing its illumination into a searingly bright beam. The beam strikes a chosen \dist{} radius (1 combat area) within range 10 (22 m). If any victim fails a \textlf{resist(R)} check, it is \textlf{blind} until the end of its next turn. Additionally, the normal lighting from the light source is removed for the next round, but the target area is fully illuminated for this time. The blindness lasts 1 additional round per level of \textlf{critical success} for the aethermancer on \textlf{resist}.


\subsection{Clinging aether (11) [3]}
\textlf{Active perk}. For 2 \textbf{AP} you blast a \dist{} radius (1 combat area) within a range of 5 (12 m) with crackling aether that anchors everything to the earth. All creatures in the area must \textlf{resist(C)} or be \textlf{immobilised} for their next two turns (1 additional turn per level of aethermancer \textlf{critical success} on the \textlf{resist} check).


\subsection{Compel (12) [3]}
\textlf{Active perk}. For 2 \textbf{AP} you project aether into a target by touch, forcing them to make an action chosen by you if they fail a \textlf{Resist(W)} check. This fails if the action would harm the target. Specify one extra action per level of \textlf{critical success} for the aethermancer on \textlf{resist}. The victim is aware that they are being forced to act. The command is obeyed until the action is complete or 10 minutes have expired. The victim receives \textlf{edge+} to \textlf{resist} subsequent uses of this manifestation during the same day.
\subsubsection{Upgrade: Dominate [3]}
Your control is so strong that Compel will not fail if the action would be harmful to the victim itself.
\subsubsection{Upgrade: Project aether [2]}
You can now manifest this power at range 2.


\subsection{Concussive wave (10) [2]}
\textlf{Active perk}. Your aether blasts out in a wave of concussive force. This costs 2 \textbf{AP} and \textlf{knocks back} all creatures and objects within range 1, size large or smaller, if they fail \textlf{resist (M)}. 
\subsubsection{Upgrade: Violent concussion [3]}
Victims \textlf{knocked back} are also \textlf{knocked down}. If they enter \textlf{rough terrain} or \textlf{cover} as a result of the \textlf{knock back}, they suffer damage with \textlf{lethality} 1.


\subsection{Consuming darkness (11,C) [4]}
\textlf{Active perk}. Light has its own aetheric field and your aether can be used to create a counter vibrating field that extinguishes lights. All light disappears within a chosen radius of 2 (around 4 m). Creatures within the region fight using the \textlf{dark} lighting rules. This lasts until you cancel it.


\subsection{Damping field (10,C) [3]}
\textlf{Active perk}. For 1 \textbf{AP} you envelope yourself in a high-pressure field of aether that depletes the force of incoming attacks. This manifests as a shimmering in the air around you and means you have $+1$ \textlf{toughness}. This manifestation lasts for 10 minutes and is a \textlf{channelled} effect.
\subsubsection{Upgrade: Ablation [2]}
When \textlf{damping field} is active you can spend 1 \textbf{RP}, when hit by an attack, to add an additional 1+\textlf{Wit} to your \textlf{Toughness} until the attack is resolved.


\subsection{Discordant blast (10) [3]}
\textlf{Active perk}. This costs 2 \textbf{AP} and projects a focussed aetheric vibration that disrupts solid matter. A chosen target within range 5 (12 m) must make a \textlf{resist(C)} check or suffer damage with \textlf{lethality} 2 on failure (X \textlf{critical success} thresholds for the aethermancer adds X \textlf{lethality}). 
\subsubsection{Upgrade: Resounding blast [2]}
This manifestation now inflicts \textlf{knocked down} if it causes damage. This only applies to \textlf{large creatures} or smaller.
\subsubsection{Upgrade: Discordant coherence (+X) [2]}
The aethermancer learns to collimate their aether into a narrow beam to increase its range of effect. This adds $X$ to the range of manifestation.


\subsection{Earthquake (13,C) [4]}
\textlf{Active perk}. You flood the ground with aether, making it roll and buck as though it was alive. For 2 \textbf{AP} you make the ground violently shake. All creatures within range 1 of you must \textlf{resist(M)} each round or be knocked down (you are unaffected). All affected creatures are \textlf{stunned}, even if they pass the \textlf{resist(M)}. This can be maintained as a \textlf{channelled} effect for up to 1 minute. If the aethermancer moves this manifestation moves with them.
\subsubsection{Upgrade: Project aether [2]}
You can now manifest this power at range 5 (12 m), instead of shaking the ground around you. You can change the target area on subsequent turns (once per turn only) for 1 \textbf{AP}.


\subsection{Embrace of iron (10) [3]}
\textlf{Active perk}. For 2 \textbf{AP} the aethermancer tunes the aetheric field of a chosen object within range 5 (12 m) to repel/attract metal (chosen on manifestation). Any creature within range 2 (6 m) to the target must \textlf{resist(M)} or be \textlf{disarmed} of any held metal objects each turn. Dropped metal objects are pushed/pulled a distance 1 away/towards the object. If attract was chosen, then there is \textlf{edge+} on \textlf{aim} to hit the target object with metal implements. The reverse applies for repel. Creatures in \textlf{medium} or \textlf{heavy armour} (made of metal) count the area within range 2 of the target as \textlf{rough terrain}.

\subsubsection{Upgrade: Lingering embrace (11,C) [2]}
\textlf{embrace of iron} is now a \textlf{channelled manifestation}. Each turn the polarity and target can be changed for 1 \textbf{AP}. This lasts up to 1 minute.

\subsubsection{Upgrade: Fleeting embrace [2]}
\textlf{embrace of iron} can have its target and polarity changed for 1 \textbf{RP} at any time.


\subsection{Eyes of aether (-) [2]}
\textlf{Passive perk}. You can alter your senses to perceive flows of the aether. This renders you blind, instead you view the world as the flow of the aether. This costs no aether points to use and may be cancelled at any time.


\subsection{Fearful frequencies (9) [3]}
\textlf{Active perk}. You bombard a target, within range 5 (12 m), with aetheric vibrations, causing a cloud of fear to pass across their mind. If they fail a \textlf{resist(R)} check, they cannot approach the aethermancer, or remain within range 1 of them, for 1 minute. In each subsequent round the victim may re-attempt the \textlf{resist(R)}. \textlf{critical failure} by X thresholds on \textlf{resist} mean the victim cannot re-attempt \textlf{resist} for X rounds. 
\subsubsection{Upgrade: Terrible tremors [3]}
You have learned to afflict the mind with absolute terror. The target of this manifestation suffers \textlf{edge-} to all rolls while this effect persists. 
\subsubsection{Upgrade: Frightful field [2]}
\textlf{Fearful frequencies} now costs 2 \textbf{AP} but affects all selected targets within a chosen combat area (\dist{} radius). 



\subsection{Fiery surge (11) [3]}
\textlf{Active perk}. You can fuel a flame via aether infusion, greatly increase the intensity of an existing fire within range 5 (12 m) at the cost of 2 \textbf{AP}. Any creature that makes contact with such a flame (i.e. enters or is in the same combat area) must \textlf{resist(C)} or suffer damage with \textlf{lethality} 1 (X \textlf{critical failure} thresholds for the target adds X \textlf{lethality}). This fire burns for 2 rounds before going out. This manifestation sets all flammable material within radius 0 (\dist{}) on fire.
\subsubsection{Upgrade: Ravenous flames [3]}
The fires burn with greater hunger, gaining \textlf{burst} + 1.
\subsubsection{Alternate: Explosive emanation [4]}
This increases the \textbf{AP} cost of the manifestation by 1. Instead of intensifying a fire you can cause it to expend all its aether at once in an explosive blast of radius 0 (\dist{}). Creatures in the blast must make a \textlf{resist(C)} check. Failure inflicts damage with \textlf{lethality} 2 (X \textlf{critical failure} thresholds for the target adds X \textlf{lethality}).  


\subsection{Flare (11) [3]}
\textlf{Active perk}. Projecting aether suddenly into a flame within range 5 (12 m), you cause it to burst into in blinding white flash. Anyone who can see the flare must roll \textlf{resist(M)} or be \textlf{staggered} and \textlf{stunned} until the end of the next round, and 1 additional round per level of \textlf{critical failure} on \textlf{resist(M)}. The explosion of the flare itself is harmless.
\subsubsection{Upgrade: Burning flare [3]}
Flare ignites all creatures that failed the \textlf{resist(M)}, inflicting the \textlf{burning} condition.


\subsection{Freezing eruption (12,C) [4]}
\textlf{Active perk}. An icey aetheric field creates a vortex of intense cold. For 2 \textbf{AP} you choose a region of radius \dist{} radius (1 combat area) within a range of 5 (10 m) and freezing shards of ice erupt from the ground. Creatures within the area must \textlf{resist(R)} or suffer damage with \textlf{lethality} 1 (X \textlf{critical failure} thresholds for the target adds X \textlf{lethality}). For 10 minutes the area is \textlf{rough terrain}, anyone entering or leaving the area must \textlf{resist(R)} or suffer damage \textlf{lethality} 1 (X \textlf{critical failure} thresholds for the target adds X \textlf{lethality}). This is a \textlf{channelled} effect.
\subsubsection{Upgrade: Sub-zero [2]}
The cost of this manifestation is reduced by 1 \textbf{AP}.
\subsubsection{Upgrade: Glacial creep [3]}
In each subsequent round that the manifestation persists, you may choose an additional \dist{} radius within range 1 of a frozen area to freeze over. All of these expire after 10 minutes.


\subsection{Glacial chill (11) [3]}
\textlf{Active perk}. This costs 2 \textbf{AP}. You strike a target within range 5 (12 m) with an aetheric vibration that causes them to rapidly emit heat and freeze over. They must \textlf{resist(C)}, failure means they are \textlf{immobilised} and \textlf{vulnerable} until damaged. 
\subsubsection{Upgrade: Creeping frost [3]}
\textlf{Glacial chill} applies its effects to a different target after it ends on the initial target.
\subsubsection{Upgrade: Deep-freeze [2]}
After the effect of \textlf{Glacial chill} ends, the ice shatters causing all creatures adjacent to the target suffer a \textlf{damage check} with \textlf{power} +0 and \textlf{lethality} 1.


\subsection{Grip of the ground (10,C) [3]}
\textlf{Active perk}. The aetheric field of objects can be adjusted so they respond more or less strongly to the pull of ground beneath their feet. If used to reduce a target's weight then their movement speed increases by 1 and they gain \textlf{edge+} on \textlf{Defence}. Such targets also have \textlf{damage checks} from falling made with \textlf{edge-}. 

If used to increase an objects weight, they must succeed on a \textlf{resist(M)} check each turn or be \textlf{immobilised} and \textlf{stunned}. If the aethermancer achieves X \textlf{critical success} thresholds then the target suffers damage with \textlf{lethality} X. Such targets also suffer \textlf{edge+} on \textlf{damage checks} from falling.

This lasts a maximum of 10 minutes.


\subsection{Guided motion (10) [3]}
\textlf{Active perk}. Aether streams out from you to enhance the movements of another body within range 5 (12 m). This confers \textlf{edge+} to an allied target's \textlf{Athletics} and \textlf{Defence} checks until they fail such a check (maximum 1 hour). You may select one extra target per level of \textlf{critical success} on the \textlf{aethermancy} check.
\subsubsection{Alternate: Inhibit motion (+2) [4]}
Guided motion can target an enemy and confer \textlf{edge-} instead, which ends if they pass a penalised check (no \textlf{resist} attempts can be made against this manifestation).


\subsection{Ignite (9) [2]}
\textlf{Active perk}. Your aether can generate a resonance between your target and the element of fire, triggering a sudden burst of flame. This manifestation can be used on a chosen object/creature within range 5 (12 m). Add 1 extra target per level of \textlf{critical success} on the \textlf{aethermancy} check. Fire created by \textlf{Ignite} suffers from normal physical restrictions, i.e. you may not set fire to a creature unless it is naturally flammable, covered in oil, or circumstantially vulnerable (such targets are afflicted by \textlf{burning}). Setting fire to the clothes of a foe means they must pass a \textlf{Resist(W)} check each turn or spend the turn panicking, if they succeed then the fire can be extinguished for 2 \textbf{AP} on their own turn.
\subsubsection{Alternate: Quench [2]}
You can consume fire to restore aether. This costs no aether and extinguishes all fires in a chosen \dist{} radius (within range 5). If at least enough fire to constitute a larger camp fire (1 square metre) is extinguished, you regain 1 point of missing aether. \textlf{critical success} on the \textlf{aethermancy} check allows you to select an additional \dist{} radius (you may regain 1 missing aether per extinguished area).


\subsection{Illuminate (8) [2]}
\textlf{Active perk}. You imbue an object with aether, causing it to resonate with the aetheric field of light. A single small object you touch begins to emit a soft aetheric glow. This provides low-light illumination over a single combat area (around a \dist{} radius). \textlf{critical success} on the \textlf{aethermancy} check allows you to pick extra targets or increase the brightness to full lighting in 2 m and low in 4 m. This lasts for up to 6 hours.
\subsubsection{Alternate: Extinguish (7+X) [1]}
You can extinguish a chosen light source within range X. Select an extra target per level of \textlf{critical success} on the \textlf{aethermancy} check.



\subsection{Ley of the land (10) [2]}
\textlf{Active perk}. You can attune your own aether to the fields within the terrain around you, allowing effortless avoidance of obstructions. This manifestation allows the aethermancer to ignore natural forms of \textlf{rough} or \textlf{dangerous terrain} effects as well as gaining \textlf{edge+} on \textlf{stealth} and \textlf{awareness} checks while in wilderness. This lasts for 1 hour.


\subsection{Mend flesh (11+X) [4]}
\textlf{Active perk}. An aethermancer can adjust the vibrations of a body's aetheric fields to repair damage. For 2 \textbf{AP} you can reduce the severity of a \textlf{wound} effect on target within range 0 by one level, i.e. from \textlf{badly wounded} to \textlf{wounded}. The difficulty is determined by the wound severity, X = 0 for \textlf{wounded}, X = 2 for \textlf{badly wounded}, etc. You cannot restore a target that has died.
\subsubsection{Alternate: Rend flesh (12) [3]}
You can inflict a wound effect on a victim within range 0. The target must \textlf{resist(R)} or increase their \textlf{wound level} by 1 (scaling with \textlf{critical failure}). This does no \textlf{endurance} damage.  


\subsection{Mimic sound (11) [2]}
\textlf{Active perk}. You subtly leak aether into the air, causing it to vibrate in a chosen pattern. This allows you to produce any sound you can imagine, at volumes between a whisper and a shout. The convincingness of this is decided by results of listeners' \textlf{resist(W)} checks. The sound can last a maximum of 10 minutes.


\subsection{Orb of aether (9,C) [3]}
\textlf{Active perk}. You can imbue your aether into the aetheric field of light itself, creating a condensed orb of luminosity. This manifests as floating orb of aether that can move a distance of 2 m (range 1) each turn. The orb provides full illumination within 4 m (range 1) of itself and \textlf{low} light within 8 m (radius 3). This lasts up to 1 hour and is a \textlf{channelled} effect. 


\subsection{Overgrowth (12) [3]}
\textlf{Active perk}. Plants have their own variety of aetheric field, you can project your aether into this field to create a resonance. You may choose a region of radius \dist{} radius (1 combat area) within a range of 5 (12 m) and make local plant growth explode from the ground. This region is now \textlf{rough terrain}.
\subsubsection{Alternate: Wither [1]}
You may choose a \dist{} radius within range 5 to wither the plant life in.


\subsection{Pulsing fury (11,C) [4]}
\textlf{Active perk}. You radiate an aetheric field that resonates with aggression, enhancing the potency of your allies. While this is in effect, all allied creatures within range 1 of you gain $+1$ \textlf{power}. This is a \textlf{channelled} effect that lasts up to 1 minute.
\subsubsection{Alternate: Pacifying pulses [1]}
The effect of \textlf{Pulsing fury} is -1 \textlf{power} to enemies instead of its usual effects.


\subsection{Prediction (14) [4]}
\textlf{Active perk}. Time itself resonates with aetheric frequencies. You can use this to attempt to peer into possible futures. This manifestation takes 10 minutes to complete. Afterwards, roll 3d6 and put them to one side. At any point within the next 24 hours you may replace one 3d6 roll (made by any creature or character) with the 3d6 you set aside. On a \textlf{critical failure} the GM may instead choose when to make the roll substitution. You may only have one such set of predicted dice available at once.
\subsubsection{Upgrade: Forecasting [3]}
You can store 2 predictions at once.


\subsection{Pyromancy (10) [3]}
\textlf{Active perk}. A careful net of aetheric fields can nudge fire to move, as if of its own accord. The aethermancer can transport the energies of a fire within range 5 (12 m). The fire can move up to range 3 from its origin. If it is aimed as a projectile then the target must \textlf{resist(C)} or suffer damage with \textlf{lethality} 1 (X \textlf{critical failure} thresholds for the target adds X \textlf{lethality}). Materials like metals can be heated by this effect to burn their bearer if they fail \textlf{resist(R)} (this \textlf{disarms} any damaged victims).


\subsection{Resonant motion (8) [2]}
\textlf{Active perk}. You imbue aether into objects so that their aetheric fields resonate with motion. This allows you to exert the force of a single hand to perform simple actions on an object visible to you within range 5 (12 m). You may manipulate one extra object per level of \textlf{critical success} on the \textlf{aethermancy} check.
\subsubsection{Upgrade: Major resonance (+3) [4]}
The aethermancer is far more attuned to the resonance of motion, allowing for the manipulation of an object with weight up to 10 kg plus 20 kg per level of \textlf{critical success} on the \textlf{aethermancy} check. If the object is used as a weapon, it uses the aethermancer's \textlf{aethermancy} for \textlf{aim}. The bonus \textlf{Power} of such a weapon is the aethermancer's \textlf{Might}. The \textlf{power} for grabbing objects is calculated in the same manner. In combat the aethermancer must spend 1 \textbf{AP} to manipulate objects with this power.


\subsection{Root weaving (12,C) [3]}
\textlf{Active perk}. By creating a complex aetheric resonance between yourself and nearby plants, you can call on their assistance to entrap foes. Provided there are plants or other natural growths nearby, this can be used to make roots \textlf{grapple} or \textlf{shove} a chosen creature or object within the foliage (this costs 1 \textbf{AP}). The roots use your \textlf{aethermancy} in place of any other skills. This is a \textlf{channelled} effect that can be maintained for up to 10 minutes.
\subsubsection{Upgrade: Bushwhack [3]}
Plants and roots can be made to strike at your foes, using your \textlf{aethermancy} for \textlf{aim}. For 2 \textbf{AP} you can activate a tree to strike at a nearby foe, this attack has \textlf{lethality} 2.


\subsection{Sapping aether (11,C) [3]}
\textlf{Active perk}. You project a discordant aetheric field into a target, weakening their body. If a chosen target within range 5 (12 m) fails a \textlf{resist(R)} check, then, whenever they lose \textlf{endurance} they lose 1 additional point. This lasts for up to 1 minute and  is a \textlf{channelled} effect.
\subsubsection{Upgrade: Field [4]}
If used for 2 \textbf{AP} this affects all creatures within a radius of 2 m (1 combat area).


\subsection{Shape earth (8) [3]}
\textlf{Active perk}. The ground around you has an aetheric field that can be manipulated. A cunning flow of aether allows one to twist the very earth into any desired shape. The aethermancer may touch earth or stone and then manipulate up to 30 kg of earth, or 15 kg of stone, plus 10 kg earth or stone per point of \textlf{wit}. 
\subsubsection{{Moulding (9,C) [2]}}
\textlf{Shape earth} is now a \textlf{channelled} manifestation and can manipulate the given mass of earth or stone each turn (6 seconds). This lasts up to 1 hour. 


\subsection{Stone sense (8) [2]}
\textlf{Active perk}. Your expert knowledge of aetheric vibrations allows you to extend your hearing through a continuous stone (or earthen) surface or structure. For this purpose, a wall of stone bricks is continuous but soil and dirt do not count (they do not conduct vibrations coherently enough). The distance you can move your hearing is 10 m + 5 m per level of \textlf{critical success} on the \textlf{aethermancy} check. This power lasts for up to 1 hour.


\subsection{Stone skin (12) [4]}
\textlf{Active perk}. Manipulating the aetheric field of nearby rocks, you draw stone from the ground to clad you in armour. Until the end of the next round you have \textlf{hardened} 1 + 1 per level of \textlf{critical success}. This can be manifested with a either an \textbf{AP} or \textbf{RP}.
\subsubsection{Upgrade: Earth-clad [2] (C)}
This makes \textlf{Stone skin} a \textlf{channelled} effect that lasts up to 10 minutes and grants \textlf{Hardened} 1 in each subsequent round.


\subsection{Strange attractor (12,C) [4]}
\textlf{Active perk}. An intricately twisted pattern of aetheric fields creates a point that draws in all nearby objects. Choose a point within range 5, this point becomes an aetheric singularity. Any creature (\textlf{large creatures} or smaller) within range 2 of the singularity must succeed on a \textlf{resist(M)} check at the start of their turn or be pulled a distance of 1 towards it and be \textlf{immobilised} on their next turn. On a \textlf{critical failure} any held weapons or objects are pulled from their grip. Any creature that spends a whole round within the singularity suffers damage with \textlf{lethality} 1. All loose objects of size \textlf{large} or smaller are also pulled in.


\subsection{Suggestion (11) [4]}
\textlf{Active perk}. A creature's own aetheric field can be altered to make its mind more open and malleable. Spend 2 \textbf{AP} to make a single-sentence suggestion of an action to the target, on a failed \textlf{Resist(W)} check they follow the suggestion willingly and rationalise it to themselves. This will automatically fail if it would
be harmful to the victim or their friends/allies. Specify 1 additional suggestion per level of \textlf{critical success} for the aethermancer on \textlf{resist}. 		
\subsubsection{Upgrade: Manipulation [3]}
Your mastery of the aetheric fields in brains allows you to implant beliefs or feeling into a target via suggestion. In addition, \textlf{suggestion} no longer fails if it would be harmful to the victims friends/allies.


\subsection{Time warp (13,C) [5]}
\textlf{Active perk}. Projecting your aether towards another creature, you make its own aetheric field discordant with the flow of time. Choose a creature within range 5 (12 m) to either gain or lose an \textbf{AP} (to a minimum of 1) each turn until the effect ends (maximum duration 1 minute). Targets may \textlf{resist(R)} each turn to ignore the effect (it does not end however). Only one \textbf{AP} may be gained or lost each turn in this way. \textlf{critical success} for the aethermancer on \textlf{resist} means the target cannot attempt \textlf{resist} next turn.
\subsubsection{Upgrade: Field [3]}
If used for 2 \textbf{AP} this affects all creatures within a targeted radius of \dist{} (1 combat area).


\subsection{Transfix (11,C) [3]}
\textlf{Active perk}. You project your aether into a creature, disrupting its aetheric field to fill its mind with a deluge of sensory information. A single target within range 5 (12 m) must \textlf{resist(R)} or stand transfixed (no actions allowed) for 10 minutes. A \textlf{resist} attempt may be made at the end of each turn to end the effect. This is \textlf{channelled} manifestation and any damage to the victim ends the effect immediately. 
\subsubsection{Upgrade: Field [4]}
If used for 2 \textbf{AP} this manifestation affects all creatures within a radius of \dist{} (1 combat area).
\subsubsection{Upgrade: Amplification [3]}
If the victim of this effect is damaged or successfully \textlf{resists}, the manifestation lingers for 1 round before ending.


\subsection{Vortex (9) [3]}
\textlf{Active perk}. Infusing a sudden burst of aether into the air creates a resonance that roars out as a powerful vortex of wind. This stops the flight of projectiles entering a target 4 m radius within range 5 (12 m) until the end of the next round. The stopped projectiles can hit creatures inside the vortex region (allocate hits randomly). 
\subsubsection{Upgrade: Howling winds [2]}
\textlf{Vortex} can be manifested with ferocious winds that knock down all creatures in the target area if they fail a \textlf{resist(M)} check.
\subsubsection{Upgrade: Swift-wind [2]}
\textlf{Vortex} can be manifested for 1 \textbf{RP} during enemy action resolution.


\subsection{Warp space (13) [5]}
\textlf{Active perk}. You create an aetheric resonance around a creature, warping the aetheric field of space itself. For 1 \textbf{AP} you can move a target creature within range 5, which can attempt to \textlf{resist(C)} if it wants to. This moves the target a distance up to range 5 (12 m). This cannot affect unwilling targets more than 1 size category larger than the aethermancer without 1 level of \textlf{critical success} for the aethermancer on \textlf{resist} per size additional category difference.


\subsection{Weather sense (-) [2]}
\textlf{Passive perk}. You attune to the aetheric fields of air and water, thus you can predict the weather up to 1 day + \textlf{wit} in advance.
\subsubsection{Upgrade: Weather touch (9+X) [4]}
Your mastery of air and water allows you to nudge the weather. This lets you alter the weather, with the effect occurring after 5 - X days where X is the number of aether points spent on this manifestation. 



\chapter{Character skills}

\section{Aethermancy (Wit)}
\label{sec:aethermancy}
This is the skill invoked to produce aether manifestations. 

Any character with sufficient knowledge of the flow and hum of the aether can alter the very fabric of the world.  The energy for this is drawn from the inner strength of the character themselves. Thus, one can only exercise this power sparingly, lest they exhaust the aether powering their body. The use of aether to create external effects is known as ``manifestation''. Characters learn new manifestations either through being taught them or by spending experience points to unlock them.

\subsection{Aether capacitor}
\label{sec:capacitor}
A creature's body can only hold so much aether. To transcend this limitation, aethermancers have developed a device called an `aether capacitor' which stores additional aether for them. This device is bulky and has wires running from it various points on the aethermancer's body. A standard aether capacitor costs 5 d and can hold 1 aether. The maximum possible capacity is 4 points and cost progression is 5 d, 15 d, 45 d, and 135 d.


\subsection{Producing a manifestation}
Manifestations do not require the use of gestures or words, they are produced by the character's fine control of their own aetheric field. By attuning the vibrations of their bodies own aetheric field, the aethermancer can cascade changes through other nearby fields to produce physical effects. Self-taught aethermancers often use gestures, as the spotty nature of their learning can lead to a personal association between gestures and manifestations. The synod requires its trainees do not use them at all. A list of manifestations can be found in Section~\ref{sec:manifestations}.

Manifestations are actions like any other, costing 1 \textbf{AP} unless otherwise specified. Manifestations require the user to make an \textlf{aethermancy} check against a \textlf{difficulty} given in brackets on the manifestation description. If the aethermancer succeeds, they must spend the required number of aether points (1 by default) and then apply the described effects. A \textlf{critical failure} during on the \textlf{aethermancy} check releases aether in an uncontrolled reaction (some suggestions are listed below).
\begin{itemize}
	\item Manifestation occurs adjacent to the aethermancer instead of targeted area
	\item Manifestation occurs but changes from targeting enemies to allies or vice-a-versa
	\item Manifestation occurs but picks a random target
	\item Manifestation fails producing an aether explosion (radius 0, \textlf{Power} +2)
\end{itemize} 
On the other hand, \textlf{critical success} on the \textlf{aethermancy} check grants \textlf{edge+} (scaling with the number of critical thresholds) on any further rolls linked to the manifestation.

\subsubsection{Manifestation time}
An aethermancer can elect to reduce the \textbf{AP} cost of a manifestation by 1 (to a minimum of 1), at the cost of \textlf{edge-} on the aethermancy check. Conversely they can increase the cost by 1 to gain \textlf{edge+}. Note that 1 \textbf{AP} equates to roughly 2 seconds of real time.

\subsection{Resist(X)}
A Resist(X) check involves the victim making an opposed roll with their natural attribute X against the aethermancer, who adds their \textlf{Might} to the roll. If the aethermancer is not \textlf{proficient} in \textlf{aethermancy}, they experience \textlf{edge-} on this check. 

\subsubsection{Notes}
An non-sentient object, or huge creature, follows the core rules for attacking such targets when it comes to making \textlf{resist(C)}.

\subsection{Disrupt}
An aethermancer can use their own aether to nullify the manifestations of another. This can be used at a cost of 1 \textbf{RP} or \textbf{AP} (depending whose turn it is). \textlf{disrupt} involves an \textlf{aethermancy} check opposed with the other aethermancer. If the check succeeds, the disrupter expends 1 aether point and the manifestation is nullified. \textlf{critical success} for the disrupter removes the aether cost.

\subsection{Channelled manifestations}
These have a long duration but require that concentration be maintained to keep them functioning. If a character is \textlf{mortally wounded} or killed, then their \textlf{channelled} manifestation ends. Otherwise, they must spend 1 \textbf{AP} in each subsequent round to maintain the effect, it ends if they cannot do so. \textlf{channelled} manifestations are marked with a C after their difficulty in Section~\ref{sec:manifestations}.



\section{Aethertech (Cunning)}
This represents a character's knowledge of aether-based technology and other sophisticated devices. A character does \textbf{not} need to be an \textlf{aethermancer} to use this skill. Using this skill a character can
\begin{itemize}
	\item Operate aethertech devices
	\begin{itemize}
		\item \textlf{critical failure} means the device fails catastrophically
	\end{itemize}
	\item Identify functions of aethertech devices
	\begin{itemize}
		\item Success means the functions are correctly identified
		\item \textlf{critical success} gives the character \textlf{edge+} on subsequent checks with the device
		\item Failure means the character is unsure about the device
		\item \textlf{critical failure} means the character is confident but wrong about the device
	\end{itemize}
\end{itemize}

\subsection{Weapontech}
This is an additional function of the aethertech skill, it allows characters to modify equipment with aethertech enhancements. These can be applied to weapons via a \textlf{skill check}, the difficulty of each one is listed is brackets after the name, e.g. (11). A single piece of equipment can only ever have one enhancement. The required materials are aether crystals and precious metals (to conduct aether). The costs are listed in units of \textlf{aethertech materials}, each of which costs 5 d. These enhancements cost double the material cost from professional aethertechs. 

The enhancement process involves including additional metal structures or pathways to direct or harness the wielder's aether. This can be anything from strange wires and antennae to narrow, inlaid gold and silver veins, or even implanted aether crystals to supply the needed energy. 

\subsubsection{Aggression (10)}
(Consumes 2 \textlf{aethertech materials}). The weapon feels alive in your hand as aether thrums through inlaid metal veins. The wielder may spend 1 aether point during action declaration, the weapon gains \textlf{edge+} on \textlf{aim} this round.  

\subsubsection{Arcing (14)}
(Consumes 7 \textlf{aethertech materials}). The weapon carries complicated metal architecture that enhances aether emitted by the bearer. The wielder may spend 1 aether point after successfully producing an aether manifestation, they may select twice as many targets for this manifestation. 

\subsubsection{Culling (12)}
(Consumes 4 \textlf{aethertech materials}). The weapon unleashes a burst of aether when it strikes weakened targets. The wielder may spend 1 aether point when they successfully damage a target which is \textlf{stunned}, \textlf{Knocked Down}, \textlf{burning}, or \textlf{Bleeding}. The victim suffers damage with 1 \textlf{lethality}. 

\subsubsection{Devastation (15)}
(Consumes 7 \textlf{aethertech materials}). The weapon can convert the bearer's aether into a powerful blast on contact. The wielder may spend 1 aether point after successfully hitting a target. The weapon has \textlf{Burst} + 1 this round. 

\subsubsection{Executioner (12)}
(Consumes 4 \textlf{aethertech materials}). The weapon has auxiliary systems that kick-in when it inflicts a telling strike to increase its power even further. The wielder may spend 1 aether when inflicting a \textlf{Critical Hit} with the weapon and then make an extra \textlf{damage check} against the same target. 

\subsubsection{Lancing (12)}
(Consumes 4 \textlf{aethertech materials}). An aetheric spear lances forth from the weapon, extending its range. The user may spend 1 aether during action declaration to grant the weapon \textlf{reach} $+1$ for this round.

\subsubsection{Impaling (12)}
(Consumes 4 \textlf{aethertech materials}). The weapons stabbing or cutting edges are augmented with active aether blades that bore into the target with a well-placed strike. The wielder may spend 1 aether after scoring a \textlf{Penetrating hit} with the weapon. The weapon gains $+2$ \textlf{Power} this round. 

\subsubsection{Malevolent (13)}
(Consumes 5 \textlf{aethertech materials}). Aether circuits zap the nervous system of its victims. The wielder may spend 1 aether point after causing damage with this weapon. The victim then suffers \textlf{edge-} on \textlf{Defence} for 1 round.

\subsubsection{Masterful (16)}
(Consumes 8 \textlf{aethertech materials}). The weapon generates a resonance with the wielder's own aetheric field. The wielder may spend 1 aether point when targeting an attack or aether manifestation. The victim has \textlf{edge-} to \textlf{Resist} and \textlf{Defence} against you this round. 

\subsubsection{Penetrating (10)}
(Consumes 2 \textlf{aethertech materials}). The weapon's edge can generate discordant aetheric fields to slice through even the thickest armour. The wielder may spend 1 aether after successfully hitting a target with the weapon. The weapon gains \textlf{Penetration} + 2 this round. 

\subsubsection{Searing (10)}
(Consumes 2 \textlf{aethertech materials}). The weapon unleashes searing aether into existing wounds on the target. The wielder may spend 1 aether point after damaging a target. The victim is \textlf{Vulnerable} for 1 round. 

\subsubsection{Thirsting (14)}
(Consumes 6 \textlf{aethertech materials}). Aetheric circuits draw energy out of targets wounded by this weapon. The wielder may spend 1 aether point, if the weapon causes damage, to regain 1 missing \textlf{endurance}. 

\subsubsection{Thunderous (10)}
(Consumes 2 \textlf{aethertech materials}). Powerful blows from this weapon unleash a booming aetheric disturbance. The wielder may spend 1 aether point after causing damage with this weapon to inflict \textlf{knocked down} on the victim if they fail \textlf{resist(R)}. This can only affect creatures up to one size category larger than the wielder.

\subsubsection{Vengeful (13)}
(Consumes 5 \textlf{aethertech materials}). The aetheric circuits in the weapon produce a powerful resonance against targets who have damaged the wielder. The wielder may spend 1 aether point after suffering damage. For 1 round the weapon gains \textlf{edge+} to \textlf{aim} and \textlf{damage checks} against any foe that has damaged you within the last round of combat. 

\subsubsection{Vorpal (17)}
(Consumes 9 \textlf{aethertech materials}). The bearer's aether fuels fields that disrupt the target, making the weapon far more lethal. The wielder may spend 1 aether point after succeeding on a \textlf{damage check} with this weapon. The weapon's \textlf{Lethality} is increased by 1 on this check. 


\subsection{Armourtech}
This is an additional function of the aethertech skill, it allows characters to modify equipment with aethertech enhancements. These can be applied to armour via a \textlf{skill check}, the difficulty of each one is listed is brackets after the name, e.g. (11). A single piece of equipment can only ever have one enhancement. The required materials are aether crystals and precious metals (to conduct aether). The costs are listed in units of \textlf{aethertech materials}, each of which costs 5 d. These enhancements cost double the material cost from professional aethertechs.

The enhancement process involves including additional metal structures or pathways to direct or harness the wielder's aether. This can be anything from strange wires and antennae to narrow, inlaid gold and silver veins, or even implanted aether crystals to supply the needed energy. 

\subsubsection{Absorption (12)}
(Consumes 4 \textlf{aethertech materials}). The armour contains circuits that, when stimulated with aether, can absorb hostile manifestations. If the wearer scores a \textlf{critical success} to \textlf{resist}/\textlf{disrupt} an aether manifestation, they regain one missing aether point.

\subsubsection{Adamant (13)}
(Consumes 5 \textlf{aethertech materials}). The armour uses the bearer's aether to project a field that is discordant with piercing and crushing. Spend 1 aether point after being hit by an attack. The armour ignores \textlf{penetration} for 1 round. 

\subsubsection{Ascension (12)}
(Consumes 4 \textlf{aethertech materials}). The armour can briefly generate a powerful aetheric field allowing it to soar above the ground. The wearer can spend an aether point to make their next move-type action while hovering up to 4 m in the air. 

\subsubsection{Redirection (13)}
(Consumes 5 \textlf{aethertech materials}). Activating the aetheric circuits of the armour can rebound projectiles upon the firer. Spend 1 aether point when you \textlf{Defend} against a ranged attack. Make the ranged weapon's \textlf{damage check} against the attacker.    

\subsubsection{Reflex (11)}
(Consumes 3 \textlf{aethertech materials}). The aetheric field of this armour can be triggered to grant a burst of inhuman reflex. The wearer may spend 1 aether point to gain an extra \textbf{RP}.

\subsubsection{Refraction (10)}
(Consumes 2 \textlf{aethertech materials}). The aetheric circuitry of the armour can briefly blur the wearer's motion, making them hard to get to grips with. Spend 1 aether point when targeted by an aether manifestation or attack. You gain \textlf{edge+} on \textlf{Resist} and \textlf{Defence} checks linked to the attack/manifestation.

\subsubsection{Steadfast (13)}
(Consumes 5 \textlf{aethertech materials}). The armour's aetheric circuits can project a field that is discordant with attacks that would debilitate the wearer. The wearer may spend 1 aether point when afflicted by a \textlf{condition} to negate this effect.

\subsubsection{Swift stride (11)}
(Consumes 3 \textlf{aethertech materials}) The armour can briefly generate an aetheric field that makes its wearer move swiftly. The wearer may spend 1 aether when making a \textlf{run} action. This grants $+1$ range to their movement.

\subsubsection{Unflinching (16)}
(Consumes 7 \textlf{aethertech materials}). An aetheric field generator can be briefly activated to soften blows dealt to the armour. The wearer may spend 1 aether point when damaged by an attack. The attack's \textlf{lethality} is reduced to 1.



\section{Infusary (Cunning)}
\label{sec:infuse}
\textlf{Infusary} is the craft of creating liquids that resonate with certain aetheric frequencies, achieving dramatic effects when drunk/applied. A character does \textbf{not} need to be an \textlf{aethermancer} to use this skill.

An infusion needs a container to hold it as well as a aether crystals and a signature ingredient to shape the purpose of the aether. To create infusions a character needs set of glass vessels and equipment for measuring, grinding and heating ingredients (infuser's tools) that can be purchased from an aethertech for 10 dinari. 
Table~\ref{tab:alch} displays some suggested infusions. Poisons can also be created but do not require aether crystals, they are listed in Table~\ref{tab:poison}.

\textlf{Failure} to create a given infusion allows an immediate re-attempt, however if this also fails the ingredients are spoiled. \textlf{critical failure} always results in the loss of ingredients and high levels of this might also produce a volatile explosion. When creating many drafts of the same infusion, follow the rules for a \textlf{multi-stage check} (a guideline is to require half as many successes as infusions being made, with a maximum requirement of 5 to avoid tedium). \textlf{critical success} means the infuser saves half of the required aether crystals.  

\begin{table}[ht!]
\caption{Aether infusions. D is the \textlf{difficulty}, time is how long the effects last and is given in minutes. The total cost is the sum of the special ingredients, crystal cost, and 20 c for a glass vial, double this total for the price charged by most infusers.}
\begin{tabular}{|l|l|l|l|l|l|}
\hline
Potion & D & Effects & Time & Special ingredients & Crystals \\
\hline
Might &  12 & $+1$ Might & 10 & Skulker mandibles (1 d) & 70 c \\
Cunning & 12 & $+1$ Cunning & 10 & Fresh mus-folk whiskers (20 c) & 70 c \\
Iron-flesh & 9 & $+1$ \textlf{Toughness} & 10 & Fine granite power (30 c) & 40 c \\
Fire-blood & 10 & Enrage & - & Trog blood (1 d) & 50 c \\
Hawk-eye & 11 & \textlf{edge+} on \textlf{Aim} & 20 & Eagle feathers (1 d) & 60 c \\
Warding & 11 & \textlf{edge+} on \textlf{resist} checks & 10 & Powdered silver (2 d) & 60 c \\ 
Invigoration & 12 & Restore 2 \textlf{endurance} & - & Mend-well root (30 c) & 70 c \\
Restoration & 10 & Cure 1 Condition\footnotemark[1] & - & Common herbs (10 c) & 50 c \\
Peace & 13 & Cure all Conditions\footnotemark[1] & - & Nightshade (50 c) & 70 c \\
 & & & & Mend-well leaves (20 c) & \\
Competence & 14 & \textlf{edge+} on a chosen skill & 60 & Gold (3 d), Silver (1 d 50 c) & 80 c \\
Camouflage & 11 & \textlf{Stealth edge+} & 20 & Nightshade (50 c), Ivy root (30 c) & 60 c \\
Invisibility & 15 & Invisibility (while stationary) & 5 & Distilled aether (16 d) & - \\
Haste & 15 & $+1$ \textbf{AP} per round & 5 & Quicksilver (9 d) & 90 c \\ 
Explosive & 12 & Blast 0, Burst 1 & - & Black powder (1 d) & 70 c \\
Aethershot & 13 & 1 Aethershot ammunition & - & Black powder (5 c) & 80 c \\
Aetherdraft & 13 & Gain 1 aether point & - &  Pure alcohol (50 c) & 80 c \\ 
Daze water & 10 & \textlf{resist(R)} or \textlf{dazed} (radius 0) & - & Daze cap mushrooms (10 c) & 30 c \\
\hline
\end{tabular}
\label{tab:alch}
\end{table}
\footnotetext[1]{\textlf{Poisoned, Bleeding, Blind, Immobilised, Slowed, Stunned, Staggered, Dazed}}

\begin{table}
	\centering
	\caption{Poisons, D is the difficulty to create the poison, and the cost is that of materials per concentration level.}
	\label{tab:poison}
	\begin{tabular}{|l|l|l|l|l|}
		\hline
		Name & D & Effect & Critical effect & Ingredients \\
		\hline
		Lead limb & 11 & No run, attacks cost 2 \textbf{AP} & \textlf{staggered} & Green alkaloid (10 c) \\
		Jitter juice & 12 &  \textlf{edge-} on all rolls & Increment \textlf{edge-} & Nightshade (20 c)  \\
		Mangle mind & 13 & \textlf{stunned} for duration &  \textlf{staggered} & Daze cap mushrooms (30 c) \\
		Death dealer & 14 & Damage with \textlf{lethality} 1 & Increment \textlf{lethality} & Black shriek mushrooms (40 c)\\
		\hline
	\end{tabular}
\end{table}



\chapter{Arms and armour}
\label{chap:arms}

Every adventurer ends up in situations which can only be resolved with violence now and then. Here you will find the rules for pokey bits and hopefully poke-proof clothing that aid you in such situations. 

\section{Weapons}

Tables~\ref{tab:weps} and \ref{tab:range-weps} provide a summary of the available weapon types. More details about each weapon are provided in the following sections.

\begin{table}[ht!]
	\centering
	\caption{Close-combat weapons. Note that L signifies \textlf{lethality}, A is \textlf{aim}, D is \textlf{Defence}, and P is \textlf{power}. ``Hands'' signifies whether the weapon requires one or two hands to operate.}
	\label{tab:weps}
	\begin{tabular}{|l|l|l|l|l|l|l|l|}
		\hline
		Name & Cost & Hands & L & A & D & P & Special\\ [0.5ex]
		\hline
		Shield & 2 d & 1 & 1 & - & - & - & $+2$ \textlf{all-out defence} \\
		\hline 
		\textbf{Daggers} & & & & & & & \\
		\hline
		Throwing knife & 50 c & 1 & 1 & - & - & - & \textlf{Small}, \textlf{Throw} 3\\
		Rondel dagger & 1 d & 1 & 1 & - & - & - & \textlf{Small}, \textlf{Rending} 1 \\
		Stiletto & 70 c & 1 & 1 & - & - & - & \textlf{Small}, \textlf{Penetration} 1 \\
		Sword breaker & 70 c & 1 & 1 & - & - & - & \textlf{Small}, $+1$ \textlf{Disarm}\\
		\hline 
		\textbf{Swords} & & & & & & & \\
		\hline
		Shortsword & 2 d & 1 & 1 & $+1$ & - & - & \\
		Longsword & 11 d & 2/1 & 1 & $+1$ & - & $+1$/- & \textlf{Rending} 1 \\
		Sabre & 8 d & 1 & 1 & $+1$ & - & $+1$ & \\ 
		Rapier & 7 d & 1 & 1 & $+1$ & - & - & \textlf{Rending} 1\\ 
		Greatsword & 11 d & 2 & 1 & $+1$ & - & - & \textlf{damage edge+}\\
		Zweihander & 15 d & 2 & 2 & $+1$ & - & - & \textlf{Cumbersome}\\ 
		\hline 
		\textbf{Axes} & & & & & & & \\
		\hline
		Battle axe & 2 d & 1 & 1 & - & - & $+1$ & - \\
		Bearded axe & 3 d & 1 & 1 & - & - & $+1$ & $+1$ \textlf{Disarm} \\
		Throwing axe & 3 d & 1 & 1 & - & - & $+1$ & \textlf{Throw} 3 \\
		Long axe & 9 d & 2 & 1 & - & - & $+1$ & \textlf{damage edge+} \\
		\hline
		\textbf{Pole weapons} & & & & & & & $+1$ \textlf{Trip} \\
		\hline
		Javelin & 30 c & 1 & 1 & - & - & -  & \textlf{Rending} 1, \textlf{Throw} 4\\
		Quarter Staff & 20 c & 2 & 1 & - & $+1$ & - & $+1$ \textlf{Disarm}\\
		Spear & 1 d & 2/1 & 1 & - & - & - & \textlf{Rending} 2, \textlf{reach} 1/0 \\
		Pike & 2 d & 2 & 1 & - & $-1$ & - & \textlf{Rending} 2, \textlf{Reach} 2\\
		Glaive & 9 d & 2 & 1 & - & - & - & \textlf{damage edge+}, \textlf{Reach} 1 \\
		Halberd & 10 d & 2 & 1 & - & - & - & \textlf{Rending} 1, $+1$ \textlf{Disarm}, \textlf{Reach} 1\\
		Pole-hammer & 7 d & 2 & 1 & - & - & - & \textlf{Penetration} 1, $+1$ \textlf{Disarm}, \textlf{Reach} 1\\ 
		Pole-axe & 13 d & 2 & 1 & - & - & $+1$ & \textlf{Rending} 1, \textlf{Penetration} 1 \\
		Lance & 3 d & 2 & 1 & - & - & - & \textlf{Rending} 2, \textlf{cumbersome}, \textlf{Reach} 1\\
		\hline
		\textbf{Blunt weapons} & & & & & & & \\
		\hline
		Cudgel & 10 c & 1  & 1 & - & - & - &  \\
		Club & 30 c & 2 & 1 & - & - & $+1$ & \textlf{Penetration} 1 \\
		War hammer & 11 d & 2/1 & 1 & - & - & $+1$ & \textlf{Penetration} 2/1 \\ 
		Maul & 15 d & 2 & 2 & - & - & - & \textlf{Cumbersome}, \textlf{Penetration} 2 \\
		Mace & 1 d & 1 & 1 & - & - & - & \textlf{Penetration} 1 \\
		Great mace & 7 d & 2 & 1 & - & - & - & \textlf{Penetration} 1, \textlf{damage edge+}\\
		Flail & 4 d & 1 & 1 & $+1$ & - & - & $+1$ \textlf{Trip}, $+1$ \textlf{Disarm}, \textlf{Penetration} 1 \\ 
		Grand flail & 13 d & 2 & 1 & $+1$ & - & $+1$ & $+1$ \textlf{Trip}, $+1$ \textlf{Disarm}, \textlf{Penetration} 1\\
		\hline
	\end{tabular}
\end{table}

\begin{table}[ht!]
	\centering
	\caption{Ranged weapons. Note that L signifies \textlf{lethality}, P is \textlf{power}, and A is \textlf{Aim}. ``Hands'' signifies whether the weapon requires one or two hands to operate.}
	\begin{tabular}{|l|l|l|l|l|l|l|l|l|}
		\hline
		Name & Cost & Hands & Range & Reload & L & P & A & Special\\ [0.5ex]
		\hline
		\textbf{Simple weapons} & & & & & & & & \\
		\hline
		Sling & 5 c & 1 & 6 & - & 1 & - & $+1$ & \\ 
		Whip & 50 c d & 1 & 2 & - & 1 & $-1$ & - & \textlf{Tripping}, \textlf{Disarming}\\
		\hline
		\textbf{Bows} & & & & & & & & \\
		\hline
		Short bow & 2 d & 2 & 6 & - & 1 & - & - & \textlf{Rending} 1 \\
		Long bow & 6 d & 2 & 9 & - & 1 & $+1$ & - & \textlf{cumbersome}, \textlf{Rending} 1\\
		\hline
		\textbf{Cross bows} & & & &  & &  & & \\
		\hline
		Hand crossbow & 7 d & 1 & 4 & - & 1 & $-1$ & $+1$ & \textlf{Small} \\ 
		War crossbow &  10 d & 2 & 8 & 1 & 1 & - & $+1$ & \\
		Repeater crossbow & 16 d & 2 & 6 & 2 & 1 & - & - & \textlf{Burst} 1 \\  
		Heavy crossbow & 13 d & 2 & 9 & 2 & 1 & $+1$ & $+1$ & \textlf{Penetration} 1 \\
		\hline
		\textbf{Firearms} & & & & & & & & \\
		\hline
		Pistol & 20 d & 1 & 6 & 2 & 1 & $+1$ & $-1$ & \textlf{Penetration} 2, \textlf{Small} \\
		Musket & 25 d & 2 & 9 & 3 & 2 & $+1$ & $-1$ & \textlf{Penetration} 2 \\
		Jezail & 30 d & 2 & 12 & 3 & 2 & $+1$ & - & \textlf{Penetration} 2\\
		Blunderbuss & 30 d & 2 & 6 & 3 & 1 & - & - & Cone, \textlf{Penetration} 2\\
		Bomb & 4 d & 1 & 4 & - & 1 & $+1$ & - & \textlf{Burst} 1, \textlf{Blast} 1\\
		Trog cannon & 35 d & 2 & 6 & 4 & 1 & $+1$ & $-1$ & \textlf{Blast} 0, \textlf{cumbersome}\\
		\hline 
	\end{tabular}
	\label{tab:range-weps}    
\end{table}

\subsection{Daggers}
All daggers are \textlf{small} weapons, with blades typically shorter than 30 cm.

\subsubsection{Throwing knife}
Carefully weighted knives that make efficient projectiles as well as sharp close-combat weapons. These can be \textlf{Thrown} at range 3.

\subsubsection{Rondel dagger}
A common sidearm for an armoured warrior, this is a broad but viciously sharp dagger. This has \textlf{rending} 1.

\subsubsection{Stiletto}
A dagger with a long slim blade, perfect for sliding into gaps in armour. This has \textlf{penetration} 1. 

\subsubsection{Sword-breaker}
A long dagger with notched blade designed to catch and lock an opponent's weapon. This has $+1$ on \textlf{disarm} checks.



\subsection{Swords}
All swords heave great versatility in how they strike, making it harder for a foe to parry or avoid you. Thus, they have $+1$ \textlf{aim}.

\subsubsection{Shortsword}
A sword designed for use in one hand. These can vary in length from 45 to 80 cm (this causes no difference for game mechanics).   

\subsubsection{Longsword}
A longer blade (around 1.2 m) designed to be swapped between one and two-handed use. This has \textlf{rending} 1 as well as $+1$ \textlf{power} when used in two hands.

\subsubsection{Sabre}
A heavy, single-edged blade for cut-and-thrust combat. This has a blade length of around 80 cm and is used in one hand. The heft of this weapon gives it $+1$ \textlf{power}.

\subsubsection{Rapier}
A long (1 m), thin stabbing sword. The tapered blade grants this weapon excellent penetration when carefully aimed, giving it \textlf{rending} 1.

\subsubsection{Greatsword}
A long, heavy blade only usable with two hands. This weapon's powerful strokes mean it has \textlf{edge+} on \textlf{damage checks}. 

\subsubsection{Zweihander}
A monstrously long and heavy blade that can inflict devastating wounds but is barely usable by most humans. This weapon is \textlf{cumbersome} but has \textlf{lethality} 2.



\subsection{Axes}
Axes effectively concentrate the force of the blow on their cutting edge, giving them $+1$ \textlf{power}.

\subsubsection{Battle axe}
An axe designed for use in one hand, its wedge-shaped head cleaves meat and metal with equal ease.

\subsubsection{Bearded axe}
A battle axe with a hooked blade, allowing it to catch an opponent's weapon or shield and granting $+1$ to \textlf{disarm}.

%\subsubsection{Tabar}
%A weapon with an axe blade balanced by a hammer head. This versatile weapon gains \textlf{penetration} 1 as the hammer makes it effective versus heavier armour.

\subsubsection{Throwing axe}
A weighted axe designed to strike the target blade first. This can be \textlf{thrown} at range 3.

\subsubsection{Long axe}
A heavy axe blade mounted on a long haft. The length grants great power to the weapon's swings in the form of \textlf{edge+} on \textlf{damage checks}.



\subsection{Pole weapons}
These weapons are characterised by a long wooden or chitin haft, which is well suited to tripping up targets with $+1$ to \textlf{trip}. 

\subsubsection{Javelin}
A short, throwing spear. The aerodynamics allow it to be \textlf{thrown} at range 4 while the sharp spear point grants \textlf{rending} 1.
 
\subsubsection{Quarter staff}
A heavy wooden/chitin staff which doubles as a walking stick. This versatile weapon is used in two hands and is good at striking weapons from foes' hands, giving it $+1$ to \textlf{disarm} and \textlf{defence}.

\subsubsection{Spear}
Humanities' oldest friend in hunting and war. The sharp point grants \textlf{Rending} 2, and the user gets $+1$ \textlf{reach} when using this in two hands.

\subsubsection{Pike}
When a spear is too short for you, choose a pike. This weapon has \textlf{rending} 2 as well as \textlf{reach} 2. The massive length makes it awkward, inflicting $-1$ \textlf{defence}.

\subsubsection{Glaive}
A pole weapon with a broad cleaving blade which grants \textlf{edge+} on \textlf{damage checks}.

\subsubsection{Halberd}
A flexible weapon with spiked point and an axe blade balanced by a sharpened hook. This combination grants the halberd $+1$ \textlf{power}, \textlf{rending} 1, and $+1$ to \textlf{disarm}. The long haft gives it \textlf{reach} 1.

\subsubsection{Pole-hammer}
Like a halberd but with a hammer in place of the axe blade. This grants \textlf{penetration} 1, \textlf{rending} 1, and $+1$ to \textlf{disarm}. The long haft gives it \textlf{reach} 1.

\subsubsection{Pole-axe}
A mighty weapon with a heavy axe blade balanced by a hammer and topped by a sharp spear point. The pole-axe gains $+1$ \textlf{power} \& \textlf{defence}, \textlf{rending} 1, and \textlf{Penetration} 1.

\subsubsection{Lance}
A long pole topped with a sharp point. This is designed to be used in a mounted charge. A lance has \textlf{rending} 2, \textlf{reach} 1, and gains $+1$ \textlf{lethality} when the user makes a \textlf{charge} action. This weapon can be used one-handed when mounted but is otherwise \textlf{cumbersome} as it is very long and heavy. 



\subsection{Blunt weapons}
Blunt weapons are usually effective versus heavier armour, delivering crushing impact through padding and plates alike. 

\subsubsection{Cudgel}
A short wooden/chitin club designed for use in one hand. \textlf{Sneak attacks} from a cudgel render their victim unconscious if they fail a \textlf{resist(M)} check vs 10 +  the wielder's \textlf{Power}.

\subsubsection{Club}
A heavy implement of wood or chitin for use in two hands. Cheap and good at bashing, this weapon has $+1$ \textlf{power} and \textlf{penetration} 1. 

\subsubsection{War hammer}
A flexible anti-armour weapon that can be used in one or two hands. This grants $+1$ \textlf{power} and \textlf{penetration} always, but an extra \textlf{Penetration} when used two-handed.

\subsubsection{Maul}
A massive sledgehammer that delivers terrific blows but is heavy and slow to swing. This weapon is \textlf{cumbersome} but has \textlf{lethality} 2 and \textlf{penetration} 2.

\subsubsection{Mace}
A heavy metal head mounted on a wooden/chitin handle. This is used in one hand and has \textlf{penetration} 1.

\subsubsection{Great mace}
A bigger mace for use in two hands. This grants \textlf{penetration} 1 and \textlf{edge+} on \textlf{damage checks}.

\subsubsection{Flail}
This weapon is composed of a heavy mace head on short chain linked to a handle. Such a weapon delivers blows at strange angles and even around a parrying weapon or shield. This grants $+1$ \textlf{aim}, \textlf{trip}, and \textlf{disarm}.

\subsubsection{Grand flail}
A heavy two-handed flail that grants $+1$ on \textlf{aim}, \textlf{power}, \textlf{trip}, and \textlf{disarm}.





\subsection{Simple weapons}
These do not require proficiency to use effectively

\subsubsection{Sling}
These are rotational sling weapons consisting of a long loop of cord and a pouch that holds and launches a stone or lead projectile. They count as throwing-type ranged weapons.
This has range 6, is used in one hand and grants $+1$ \textlf{power}.

\subsubsection{Whip}
A long leather whip that can land blows at range 2 but lacks in killing-power, having $-1$ \textlf{power} as a result but $+1$ to \textlf{trip} and \textlf{disarm}.

\subsubsection{Shield}
A solid wooden or chitin shield to deflect incoming blows. This is a weapon with no bonuses. However, it adds $+2$ to \textlf{all-out defence} reactions. 



\subsection{Bows and slings}
A length of wood, chitin, or horn that is bent to store elastic energy and launch arrows from a string that spans the bow. 

\subsubsection{Short bow}
A versatile, fast firing weapon with range 6 and \textlf{rending} 1.

\subsubsection{Longbow}
Longbows are as tall as their user and launch arrows up to range 9 with $+1$ \textlf{power} and \textlf{Rending} 1. However, they require arduous training in order to use with any lethality. Thus, they are \textlf{cumbersome} without the \textlf{yeoman bowman proficiency}.



\subsection{Crossbows}
A strong bow of wood, horn, chitin, or even steel is mounted on a length of wood. A latch mechanism allows the bow to be held in tension without exertion, this makes for accurate shooting with $+1$ \textlf{aim} but often comes at the cost of a time-consuming reload. 

\subsubsection{Hand crossbow}
These diminutive weapons are designed to be easily concealed and unloaded at close range into an unsuspecting victim. As such, they are \textlf{small} and make no noise, firing a tiny bolt at range 4 with $-1$ \textlf{power}. If used in a pair they require a 1 \textbf{AP} reload between shots.

\subsubsection{War crossbow}
This fires a short bolt up to range 8 and requires a 1 \textbf{AP} reload between shots using a goatsfoot lever. 

\subsubsection{Repeater crossbow}
A crossbow with lower tension but a mechanical magazine that allows it to fire rapidly. This fires at range 6, loses the $+1$ \textlf{aim} and gains \textlf{burst} 1. The magazine is slightly tricky to load, requiring a 2 \textbf{AP} \textlf{reload} between shots.

\subsubsection{Heavy crossbow}  
A powerful bow that is tensioned with a complicated windlass mechanism. This fires at range 9 with $+1$ \textlf{power} and \textlf{penetration} 1 but requires a 2 \textbf{AP} reload between shots.



\subsection{Firearms}
These weapons use the detonation of compressed black powder to propel a lead ball to terrifying speeds. They are lethal but inaccurate and their muzzle loading takes time. Several firearms incur a $-1$ on \textlf{aim}.

\subsubsection{Aethershot ammunition}
This is special ammunition for firearms, infused with aetheric energies to increase its destructive potential. A shot made with such ammunition gains \textlf{edge+} on \textlf{damage checks}.

\subsubsection{Pistol}
This is a one-handed firearm that lacks the power of its larger cousins. It has range 6, benefits from $+1$ \textlf{power} and \textlf{penetration} 2 but has $-1$ \textlf{aim} and requires a 2 \textbf{AP} reload between shots. Additionally, this weapon is a robust object and can be used as a close combat weapon without incurring \textlf{edge-} on aim.

\subsubsection{Musket}
A long barrelled firearm that delivers a large lead ball at lethal velocities. This fires up to range 9, has \textlf{lethality} 2, $+1$ \textlf{power}, \textlf{penetration} 2, but suffers $-1$ \textlf{aim} and requires a 3 \textbf{AP} reload between shots. Additionally, a musket is a robust and heavy object. Thus, it can be used as a close combat weapon without incurring \textlf{edge-} on aim. It must be used in two hands for this purpose. If it has a bayonet it gains $+1$ \textlf{rending} in close combat.

\subsubsection{Jezail}
This is a type of hand-crafted, long-barrelled musket. Its length and weight allow it to fire higher calibre ammunition without severe recoil. As such, it is highly suited to long range engagements, where its long barrel also affords great accuracy. The long, heavy barrel means this weapon requires a bi-pod to stabilise it, one \textbf{AP} must be spent to set up the bi-pod (it must be set-up again if the wielder moves). Without this, the weapon incurs \textlf{edge-} on \textlf{aim}. This is otherwise a musket with range 12 and no $-1$ \textlf{aim} penalty. 

\subsubsection{Blunderbuss}
This weapon fires a scatter of shot in a cone giving it \textlf{cleave} 1 with a range of 6. If fired at a target within range 1 it loses \textlf{cleave} but has \textlf{Burst} 1 instead. This requires a 3 \textbf{AP} reload between shots. Additionally, it is a robust object and can be used as a close combat weapon without incurring \textlf{edge-} on aim. It must be used in two hands for this purpose.

\subsubsection{Bomb}
The bomb is a thrown weapon with a range of 4 that hits all targets within range 1 of the target combat area. The attacker rolls \textlf{aim} once and each target rolls \textlf{Defence}. This represents the effectiveness of targets taking cover against the thrower's placement of the bomb.

\subsubsection{Trog cannon}
This weapon is a large pipe made of heavily reinforced iron. It is muzzle loaded and fires a spray of lead shot across a wide area. Anything smaller than a \textlf{large creature} is \textlf{knocked down} when firing this. 










\section{Armour}
\label{sec:armspec}

Table~\ref{tab:armour} provides a summary of available armour types. The following sections provide more detail and additional special rules.
	
	
\begin{table}[ht]
	\centering
	\caption{Basic armour.}
	\label{tab:armour}
	\begin{tabular}{|l|l|l|}
		\hline
		Name  & Cost & \textlf{Toughness}\\
		\hline
		\textbf{Light armour} & &  \\
		\hline
		Gambeson & 1 d & $9$ \\
		Light carapace & 3 d & $10$ \\
		Padded jack & 4 d & $10$ \\
		Jack of plates & 15 d & $11$ \\  
		\hline
		\textbf{Medium armour} & &  \\
		\hline
		Greyback lamellar & 2 d & $10$\\
		Carapace & 6 d & $11$ \\
		Mail and leather & 7 d & $11$ \\
		Brigandine & 20 d & $12$  \\
		\hline
		\textbf{Heavy armour} & &  \\
		\hline
		Mail hauberk & 4 d & $12$  \\
		Heavy carapace & 12 d & $13$ \\
		Hauberk and plate & 13 d & $13$\\
		Plate cuirass and mail & 30 d & $14$ \\
		Full plate & 90 d & $15$  \\
		\hline
		Barding (Horse) & 60 d & $13$ \\
		\hline
	\end{tabular}
%	\caption{Shields}
%	\begin{tabular}{|l|l|l|l|}
%		\hline
%		Name & Cost & Defence & Special\\   
%		\hline
%		Shield & 2 d & $+1$ & - \\
%		\hline
%	\end{tabular}
\end{table}


\subsection{Putting on and removing armour}
Even heroes seldom sleep in full-plate armour, so there are then times when the speed at which a character puts on or removes armour might matter. It requires 4 \textbf{AP} to put on \textlf{light armour} or one to strap on a shield. However, it takes 1 minute to put on \textlf{medium armour} but it takes 5 minutes to put on \textlf{heavy armour}, which also requires that the wearer has assistance in putting it on. \textlf{Medium} or \textlf{light armour} can be put on in a rush, doing so means that armour might not be precisely adjusted in order function to it's full effect, this takes only four \textbf{AP} for medium armour or two \textbf{AP} for \textlf{light armour}, but reduces the \textlf{Toughness} of the armour by 1 while it is worn in this sloppy fashion.

Removing \textlf{light armour} takes 1 minute, \textlf{medium armour} takes 2 minutes, and \textlf{heavy armour} takes 5 minutes to remove.








\subsection{Light armour}
Light armour provides less protection but does not noticeably encumber the wearer.

\subsubsection{Gambeson}
A thick padded cloth garment that provides some minor protection with $9$ \textlf{toughness}.

\subsubsection{Padded jack}
A heavy gambeson combining many layers of cloth and leather. \textlf{toughness} 10.

\subsubsection{Light carapace}
A selection of plates made from giant crab carapace to cover vital spots. This has 10 \textlf{toughness}.

\subsubsection{Jack of plates}
A padded jack with thin metal plates sewn between the layers of cloth. This has 11 \textlf{toughness}.



\subsection{Medium armour}
\textlf{medium armour} provides good protection but incurs \textlf{edge-} penalties on \textlf{stealth} and \textlf{athletics}.

\subsubsection{Greyback lamellar}
Layered plates of leather made from greyback hide are sewn together to form a protective coat. This has 10 \textlf{toughness}.

\subsubsection{Carapace}
A heavier covering of thick crab carapace plates with 11 \textlf{toughness}. 

\subsubsection{Mail and leather}
A mix of chainmail and flexible leather, this provides \textlf{toughness} 11.

\subsubsection{Brigandine}
A heavy gambeson with thicker overlapping metal plates riveted onto it to grant 12 \textlf{toughness}.



\subsection{Heavy armour}
\textlf{heavy armour} provides the best protection but incurs \textlf{edge-} penalties on \textlf{stealth}, \textlf{athletics}, and \textlf{Defence}.

\subsubsection{Mail hauberk}
A heavy coat of tiny interlocking metal rings. This is cheap and provides reasonable defence, with 12 \textlf{toughness}, but it is very heavy as it hangs from the shoulders. 

\subsubsection{Heavy carapace}
A comprehensive covering of crab-carapace plates supplemented with greyback leather in the joints and gaps. This has \textlf{toughness} 13.

\subsubsection{Hauberk and plate}
A chainmail hauberk with plate armour on arms and legs. Additionally, a grid of small lamellar plates reinforce the chest area. \textlf{toughness} 13. 

\subsubsection{Plate cuirass and mail}
A steel plate breastplate worn over a full coat of mail with some strategic plate on arms and legs. This has 14 \textlf{toughness}. 

\subsubsection{Full plate}
The peak of the armourer's art. A fully enclosed suit of steel plates, supplemented by mail and leather in vulnerable joints. Its great weight is more easily carried due to a clever leather harness distributing the load across a wearer's body. This has 15 \textlf{toughness}. 



\section{Weapon enhancements}
A weapon may have one quality enhancement and one other. 

\subsection{High quality}
A fine weapon, expertly crafted. This grants the weapon $+1$ \textlf{Power}. It costs 5 d + the base cost of the weapon to add this enhancement.

\subsection{Aether-forged quality}
Made with the aid of an aethermancer, this weapon is finely tuned to deal death. This grants the weapon $+2$ \textlf{Power}. It costs 10 d + twice the base cost of the weapon to add this enhancement.

\subsection{Expanded bolt rack}
(Crossbow only) The weapon is fitted with expanded armatures and space for two bolts. It can thus fire two bolts with a single trigger release. This grants \textlf{Burst} + 1 on attacks made with the crossbow. Cost: Base cost x 3 + 5 d. 

\subsection{Special payload}
(Crossbow only) The armatures and bolt track of the bow are modified to carry larger and heavier bolts. This configures the crossbow to fire specialised ammunition that can carry explosives (bomb from infusary - Section~\ref{sec:infuse}), chains/ropes/nets, or poisons and other chemicals that spray in a radius of 0 around the impact point. Cost: 15 d.

\subsection{Experimental chamber}
(Firearms only) The weapon can hold reserve ammunition that it is not currently firing. This reduces the \textlf{Reload} time of the weapon by one rank (reduce paired pistols to \textlf{Reload} 2) to a minimum of 1. However, if the user scores a raw 5- with the weapon, roll 3d6. On a score of $8-$ the weapon needs 10 d worth of repairs before it works again, on $9+$ it jams requiring 1 \textbf{AP} be spent to clear the jam. Cost: 40 d.

\subsection{Extra barrel}
(Firearms only) The weapon is fitted with an extra barrel. The gun may be fired twice before reloading, but the reload time increases by 1 \textbf{AP}. Cost: 20 d.



\section{Armour enhancements}
Armour may have one quality enhancement and one other.

\subsection{Fine quality}
A well-made piece of protective gear. This grants the item + 1 \textlf{Toughness}. It costs 5 d + the base cost of the armour to add this enhancement

\subsection{Aether-forged quality}
Armour made with the help of an aethermancer tuning its vibrational state to maximise its defences. This grants the item + 2 \textlf{Toughness} . It costs 10 d + twice the base cost of the armour to add this enhancement.

\subsection{Heavily reinforced}
The armour is carefully enhanced to maximally redistribute impact. This incurs a -1 \textlf{Burst} penalty (minimum 1 \textlf{damage check}) to any attacks made against the wearer. Cost: 10 d. \textlf{Medium} or \textlf{heavy armour} only.

\subsection{Impact padding}
Heavy layers of linen cloth beneath the armour make piercing and concussive attacks less effective. This reduces the \textlf{penetration} of incoming attacks by 1. Cost: 10 d. \textlf{Medium} or \textlf{heavy armour} only.

\subsection{Cunning strappage}
Careful use of straps and weight distribution makes the armour feel so light you could dance in it (this is still not recommended). This negates the \textlf{edge-} penalty incurred on \textlf{athletics} by the armour. Cost: 10 d. \textlf{heavy armour} only.

\subsection{Knuckle-blades}
Punching people is easier with steel blades on the knuckles of your gloves. This grants + 1 \textlf{power} to unarmed \textlf{damage checks}. Cost: 5 d.

\subsection{Armour spikes}
Sharpened protrusions or heavy studs set in the surface of the armour can make the wearer look really edgy. This can be applied to any \textlf{Medium} or \textlf{Heavy armour} piece, enemies attempting to grapple with the wearer have \textlf{edge-} to \textlf{Grapple} checks. Note this does not apply if the wearer is the one attempting the grapple. Cost 7 d.



\chapter{Other equipment}


\section{Cost}
This cost reflects an average price that the item would be purchasable for, from a merchant. This cost can go up and down dependent on how well a player haggles and on the circumstances of the city and/or merchant; that is, if something is in shortage it costs a lot more, or if the merchant is desperate to sell, the price goes down. A player can sell an item for a price dependent on the condition of the item, a merchant will offer him half the normal cost for a used item in good condition.

\section{Equipment}
\subsection{Tools}

\subsubsection{Aethertech's tools}
A set of instruments needed to measure the flow of aether as well as create aethertech circuits.

\subsubsection{Bandages}
Required by a healer to use his healing skill, each use of the skill consumes a bandage.

\subsubsection{Burglar's tools}
A crowbar, 10 lockpicks, 4 m of rope, a hammer, and chisel.

\subsubsection{Glass vial}
A simple glass vial with a stopper, used to hold liquids, this is needed to hold any brew created by an infuser (see Section~\ref{sec:infuse}).

\subsubsection{Infuser's tools}
A complex piece of equipment that allows an infuser to purify and distil various liquids, this is required for the creation of infusions (see Section~\ref{sec:infuse}).

\subsubsection{Lock-pick}
A small piece of cunningly bent metal that can be used to open a lock with a Mechanical skill check opposed to the \textlf{difficulty} of the lock, on a failure the lock-pick breaks.

\subsubsection{Master-crafted tools}
These are finely crafted versions of any type of tool, this adds a + 1 bonus to any rolls made using the tools.

\subsubsection{Musical instrument}
A simple instrument for making music. Wooden instruments are more expensive than those made of chitin.

\subsubsection{Survivalist's tools}
A pot, 2 full water-skins, a tinder box, 5 days of trail rations, and a tent.

\subsection{Adventuring gear}

\subsubsection{Backpack}
A large but otherwise conventional satchel.

\subsubsection{Barrel}
A large barrel made of wood (cost doubled) or crab chitin, bound with iron rings. The barrel can hold about 50 litres.

\subsubsection{Basket}
A simple wicker basket for carrying small loads.

\subsubsection{Candle}
A small candle that that produces \textlf{low} light in a 2 m radius, this candle can burn for up to 3 hours.

\subsubsection{Canvas}
A sheet of water-proof canvas. The price depends on the size of the sheet.

\subsubsection{Chain}
A metal chain made up of heavy steel links, the chain is strong enough to support very large weights, up to 500 kg.

\subsubsection{Crowbar}
A simple steel crowbar used for levering open doors or hinges. Using a crowbar adds \textlf{edge+} to rolls made using your Might to force open any hinged container or door.

\subsubsection{Fuel}
Enough peat to keeping a fire burning for one day.

\subsubsection{Fishing tackle}
A fishing line, a hook, sinkers and lures. This can be used to attempt to catch fish with the \textlf{Survival} skill, the \textlf{difficulty} of this is dependent on the speed of the water and type of the fish.

\subsubsection{Grappling hook}
A heavy iron hook used for scaling near vertical surfaces. This hook will hold the attached rope in place after being successfully thrown to the point you wish to climb to. This reduces the difficulty of lasso attempts by 2. The hook also makes an effective weapon.

\subsubsection{Hammer}
A simple hammer for hitting stuff and pitching tents.

\subsubsection{Ladder}
A length of ladder which can be climbed without needing an \textlf{Athletics} check.

\subsubsection{Lantern}
A hooded lantern used for projecting \textlf{full} light directionally, up to 13 m from the bearer. This uses 0.5 kg of oil to burn for 24 hours.

\subsubsection{Lantern oil}
Enough oil to power a standard lantern for 24 hours (0.5 kg). Highly flammable.

\subsubsection{Mirror}
A simple shiny surface (glass or metal) that reflects light. Useful for looking around corners.

\subsubsection{Needle}
A sharp needle for use in sewing or stitching wounds.

\subsubsection{Paper}
A single sheet of paper for writing on.

\subsubsection{Pick}
A simple pick used for breaking earth or skulls.

\subsubsection{Pole}
A length of wooden or metal pole.

\subsubsection{Pot}
A cooking pot, about 30 cm in diameter.

\subsubsection{Quill}
A feather quill used for writing.

\subsubsection{Rope}
A length of rope capable of supporting the weight of three people ($\sim$250 kg).

\subsubsection{Sack}
A large rough cloth bag.

\subsubsection{Spade}
A digging implement (can be used to hit people as well).

\subsubsection{Spyglass}
A telescope capable of magnifying things from up to a mile away.

\subsubsection{Tent}
A canvas tent to keep you dry at night, this can house up to two man-sized creatures.

\subsubsection{Tinder box}
A small box containing fast lighting twigs, used to rapidly start a fire.

\subsubsection{Torch}
Provides \textlf{full} light up to a 4 m radius and \textlf{low} light within 8 m. The torch can burn for 4 hours.

\subsubsection{Trail rations}
Generally dried meat, bread and cheese. Long lasting food-stuffs to nourish a traveller.

\subsubsection{Water-skin}
A hide bag used for carrying water, this carries water for one person for 5 days.

\subsubsection{Wood-axe}
An axe for chopping up wood.

\subsection{Mounts}

\subsubsection{Cart crab}
A crab the size of a small horse with a thick shell and powerful legs. These creatures can pull heavy loads but are slow moving. A cart crab has a combat movement distance of 1, it can carry weights of up to 400 kg while still being to walk all day with no appreciable strain. Carrying weights of 400 kg or more risks injuring the crab if done for prolonged periods. This creature is not trained for war and will likely remain stationary and lash out at approaching foes, it has \textlf{Defence} -2 and \textlf{Toughness} 12.

\subsubsection{Pony}
A small hard working pony, capable of carrying an full grown small creature or the child of a man sized creature. A pony has a combat movement distance of 2. Such a creature can carry weights of up to 100 kg while still being to walk or trot all day with no appreciable strain. Carrying weights of 150 kg or more risks injuring the horse if done for prolonged periods. This horse is not trained for war and may panic, it has \textlf{Defence} 0 and \textlf{Toughness} 8.

\subsubsection{Light horse}
A swift if none too tough horse, a Light Horse has a combat movement distance of 4. A light horse can gallop up to 2 hours a day at a speed of five times faster than a man, any longer risks injuring the horse. This horse can only move a distance of 2 per movement action while carrying a heavily-armoured rider. A light horse can carry weights of up to 80 kg while still being to walk or trot all day with no appreciable strain. Carrying weights of 130 kg or more risks injuring the horse if done for prolonged periods. This horse is not trained for war and may panic, it has \textlf{Defence} 0 and \textlf{Toughness} 9.

\subsubsection{Cart horse}
A heavy farm horse capable of carrying heavy loads and working all day, however, it is not capable of maintaining very rapid speeds. A cart horse has a combat movement distance of 2. A cart horse can carry weights of up to 200 kg while still being to walk or trot all day with no appreciable strain. Carrying weights of 300 kg or more risks injuring the horse if done for prolonged periods. This horse is not trained for war and may panic, it has \textlf{Defence} 0 and \textlf{Toughness} 10.

\subsubsection{Heavy horse}
A strong and reasonably quick horse. This horse can carry a man in heavy armour while moving at full speed. This horse has a combat movement distance of 3. Such a creature can carry weights of up to 120 kg while still being to walk or trot all day with no appreciable strain. Carrying weights of 200 kg or more risks injuring the horse if done for prolonged periods. This horse is not trained for war and may panic, it has \textlf{Defence} 0 and \textlf{Toughness} 10.

\subsubsection{Destrier (war horse)}
A heavy horse trained for battle, the destrier is by far the largest breed of horse, standing over 2 metres at the shoulder. Its war training makes it a savage weapon in its own right and it has close-combat \textlf{Aim} 1, \textlf{Defence} 0, \textlf{Toughness} 10, \textlf{Power} +1, and combat movement range 3. A destrier can carry weights of up to 200 kg while still being to walk or trot all day with no appreciable strain. Carrying weights of 300 kg or more risks injuring the horse if done for prolonged periods.

\subsubsection{Simple saddle}
A rough worked leather saddle, needed for sitting astride your mount. A saddle is fastened with girth buckles and has attached stirrups.

\subsubsection{War saddle}
A heavy saddle needed to support a fully-armoured knight, this saddle is designed also to allow the knight to wield a lance one handed.

\subsubsection{Racing saddle}
A saddle built for race horses, it is light weight and fine craftsmanship add 1 to the horses' combat movement distance.

\subsubsection{Tack}
Reins, bit, bridle and other equipment necessary for riding a mount. Riding bareback increases the \textlf{difficulty} of all ride checks by 2.

\subsubsection{Feed}
Food for a mount for one day.

\chapter{Equipment tables}
\begin{table}[ht]
	\parbox{0.45\hsize}{
	\centering
	\caption{Tools}
	\label{tab:tools}
	\begin{tabular}{|l|l|}
		\hline
		Item & Cost\\ [0.5ex]
		\hline
		Bandage & 5 c\\
		Infuser's tools & 5 d\\
		Aethertech tools & 5 d \\
		Burglar's tools & 3 d \\
		Survivalist tools & 5 d \\
		Glass Vials (5)& 1 d\\
		Lock-picks (5) & 50 c\\
		Musical instrument & 4 d\\		
		\hline
		Master-crafted tools & + 20 d\\
		\hline
	\end{tabular}}
	\parbox{0.45\hsize}{
	\centering
	\caption{Clothes}
	\begin{tabular}{|l|l|}
		\hline
		Item & Cost\\ [0.5ex]
		\hline
		Peasant & 50 c\\
		Scholar & 2 d\\
		Cold weather & 3 d\\
		Noble & 30 d\\
		Merchant & 5 d\\
		\hline
	\end{tabular}}

	\centering
	\caption{Mounts}
	\begin{tabular}{|l|l|}
		\hline
		Item & Cost\\ [0.5ex]
		\hline
		Cart crab & 50 d \\ 
		Pony & 180 d\\
		Light horse & 180 d\\
		Cart horse & 240 d\\
		Heavy horse & 240 d\\
		War horse & 480 d\\
		\hline
		Simple saddle & 2 d\\
		War saddle & 15 d\\
		Racing saddle & 10 d\\
		Tack & 1 d\\
		Feed, 1 day & 5 c\\
		\hline
	\end{tabular}
\end{table}

\begin{table}[ht]
	\centering
	\caption{Adventuring gear}
	\begin{tabular}{|l|l|l|}
		\hline
		Item & Cost & Weight\\ [0.5ex]
		\hline
		Backpack & 50 c & 1 kg\\
		Barrel & 30 c & \\
		Basket & 20 c & 0.2 kg\\
		Candle & 10 c & -\\
		Canvas (per sq. m) & 30 c & 4 kg\\
		Chain, 1 m & 50 c & 3 kg\\
		Crowbar & 1 d & 5 kg\\
		Fishing tackle & 20 c & -\\
		Grappling hook & 3 d & 5 kg\\
		Hammer & 50 c & 4 kg\\
		Ladder, 1 m & 1 d & 3 kg\\
		Lantern & 5 d & 1 kg\\
		Lantern oil & 1 d & 0.5 kg \\
		Mirror & 1 d & -\\
		Needle & 5 c & -\\
		Paper 1 sheet & 1 c & -\\
		Peat, 1 day & 6 c & 5 kg\\
		Pick & 2 d & 8 kg\\
		Pole, 1 m & 30 c & 2 kg\\
		Pot & 1 d & 0.5 kg\\
		Quill pen & 10 c & -\\
		Rope, 1 m & 20 c & 0.1 kg\\
		Sack & 10 c & 0.2 kg\\
		Spade & 1 d & 4 kg\\
		Spyglass & 17 d & 1 kg\\
		Tent & 2 d & 3 kg\\
		Tinder box & 70 c & -\\
		Torch & 2 c & 0.1 kg\\
		Trail rations 1 day & 30 c & 0.1 kg\\
		Water-skin (full) & 20 c & 3 kg\\	
		Wood-axe & 1 d & 2 kg\\	
		\hline
	\end{tabular}
\end{table}


\chapter{Fearsome foes}

\section{Humanoids}

\subsection{Outlaws}

\subsubsection{Thug}
A thug is street-tough, poorly equipped but keen for the fight. 

\begin{tabular}{|l|l|}
	\hline
	Name & Thug\\
	\hline
	Weapons & Club or cudgel and dagger\\
	Armour & Gambeson T 9\\
	Skills & Intimidate\\
	Attributes & M 1, W 0, C 0, R 0, End 0\\ 
	\hline
	Attacks & \\
	\hline
	Club & A 0, P 2, \textlf{penetration} 1\\
	Dagger & A 0, P 1, \textlf{Rending} 1 \\
	Cudgel & A 0, P 1\\ 
	\hline
\end{tabular}

\begin{tabular}{|l|l|}
	\hline
	Name & Cut-purse\\
	\hline
	Weapons & Two daggers\\
	Armour & Gambeson T 9\\
	Skills & Stealth, Slight of hand\\
	Attributes & M 0, W 0, C 1, R 0, End 0\\ 
	\hline
	Attacks & \\
	\hline
	Dagger & A 0, P 1, \textlf{Rending} 1 \\
	\hline
\end{tabular}

\begin{tabular}{|l|l|}
	\hline
	Name & Highwayman\\
	\hline
	Weapons & Side-sword and 2 pistols\\
	Armour & Mail and Leather(M) T 11\\
	Skills & Intimidate, Persuade\\
	Perks & Quick-draw, Deceptive draw \\
	Attributes & M 0, W 1, C 1, R 0, End 1\\ 
	\hline
	Attacks & \\
	\hline
	Pistol & A 0, P 1, \textlf{penetration} 2, range 6\\
	Side-sword & A 2, P 0 \\
	\hline
\end{tabular}

\begin{tabular}{|l|l|}
	\hline
	Name & Bandit\\
	\hline
	Weapons & Spear\\
	Armour & Lamellar (M) T 10\\
	Skills & Intimidate\\
	Attributes & M 1, W 0, C 0, R 0, End 1\\ 
	\hline
	Attacks & \\
	\hline
	Spear & A 0, P 2, \textlf{Rending} 2, \textlf{Defence} 1, \textlf{Reach} 1 \\
	\hline
\end{tabular}

\begin{tabular}{|l|l|}
	\hline
	Name & Bandit archer\\
	\hline
	Weapons & Short bow, dagger\\
	Armour & Lamellar (M) T 10\\
	Skills & Intimidate\\
	Attributes & M 0, W 1, C 0, R 0, End 1\\ 
	\hline
	Attacks & \\
	\hline
	Short bow & A 1, P 0, range 6 \\
	Dagger & A 1, P 0, \textlf{Rending} 1 \\
	\hline
\end{tabular}

\begin{tabular}{|l|l|}
	\hline
	Name & Bandit elite\\
	\hline
	Weapons & Longsword, war crossbow\\
	Armour & Chainmail (H) T 11\\
	Skills & Intimidate\\
	Attributes & M 1, W 1, C 0, R 0, End 1\\ 
	\hline
	Attacks & \\
	\hline
	War crossbow & A 2, P 0, \textlf{Reload} 1, range 8 \\
	Longsword & A 2, P 2, \textlf{Rending} 1 \\
	\hline
\end{tabular}

\subsection{Highlanders}

\begin{tabular}{|l|l|}
	\hline
	Name & Highland warrior\\
	\hline
	Weapons & Spear, 2 javelins and shield\\
	Armour & Carapace (M) T 11\\
	Skills & Survival\\
	Perks & Spear specialist, shield bash \\
	Attributes & M 1, W 1, C 0, R 1, End 2\\ 
	\hline
	Attacks & \\
	\hline
	Spear and shield & A 2, P 1, \textlf{Rending} 2, \textlf{Defence} 1 \\
	Javelin & A 1, P 1, \textlf{Rending} 1, range 4 \\
	\hline
\end{tabular}

\begin{tabular}{|l|l|}
	\hline
	Name & Highland hunter\\
	\hline
	Weapons & Spear, longbow\\
	Armour & Carapace (M) T 11\\
	Skills & Survival\\
	Perks & Spear specialist, Eagle eye \\
	Attributes & M 1, W 1, C 0, R 1, End 2\\ 
	\hline
	Attacks & \\
	\hline
	Spear & A 2, P 2, \textlf{Rending} 2, \textlf{Defence} 1 \\
	Long-bow & A 1, P 2, range 9\\
	\hline
\end{tabular}

\subsection{Mercenaries}

\begin{tabular}{|l|l|}
	\hline
	Name & Mercenary trooper\\
	\hline
	Weapons & Spear and shield\\
	Armour & Mail and leather (M) T 11\\
	Skills & Intimidate\\
	Attributes & M 1, W 0, C 0, R 0, End 1\\
	\hline
	Attacks & \\
	\hline
 	Spear & A 1, P 1, D 1, \textlf{Rending} 2\\
	Spear and Shield & A 1, P 0, D 2, \textlf{Rending} 2\\
	\hline
\end{tabular}

\begin{tabular}{|l|l|}
	\hline
	Name & Mercenary archer\\
	\hline
	Weapons & Longbow and dagger\\
	Armour & Mail and leather (M) T 11\\
	Skills & Intimidate\\
	Attributes & M 0, W 1, C 0, R 0, End 1\\
	\hline
	Attacks & \\
	\hline
	Dagger & A 1, P 0, \textlf{Rending} 1\\
	Longbow & A 1, P 1, range 9\\
	\hline
\end{tabular}


\begin{tabular}{|l|l|}
	\hline
	Name & Mercenary cleaver\\
	\hline
	Weapons & Greatsword, long axe\\
	Armour & Medium\\
	Skills & Intimidate\\
	Perks & Reckless attack \\
	Attributes & M 1, W 0, C 0, R 0, End 1\\
	\hline
	Attacks & \\
	\hline
	Greatsword & A 1, P 1, \textlf{dmg edge+}\\
	Long axe & A 0, P 2, \textlf{dmg edge+}\\
	\hline
\end{tabular}

\begin{tabular}{|l|l|}
	\hline
	Name & Mercenary elite\\
	\hline
	Weapons & Rapier, sabre, longsword, zweihander, pole-axe\\
	Armour & Heavy\\
	Skills & Intimidate\\
	Perks & Aimed shot/Power attack, Reactive defence \\
	Attributes & M 1, W 1, C 1, R 0, End 2\\
	\hline
	Attacks & \\
	\hline
	Rapier & A 2, P 1, \textlf{Rending} 1 \\
	Sabre & A 2, P 2 \\
	Longsword & A 2, P 2/1, \textlf{Rending} 1\\
	Zweihander & A 2, P 1, \textlf{Lethality} 2\\
	Pole-axe & A 1, P 2, \textlf{Rending} 1, \textlf{penetration} 1 \\
	Musket & A 0, P 2, \textlf{penetration} 2, \textlf{lethality} 2, range 9, reload 3 \\
	\hline
\end{tabular}

\subsection{Trogs}

\begin{tabular}{|l|l|}
	\hline
	Name & Trog stomper\\
	\hline
	Weapons & Maul\\
	Armour & Trog plate (H) T 12 \\
	Skills & Intimidate\\
	Perks & Hardy, True grit \\
	Attributes & M 1, W 0, C 0, R 1, End 4\\
	\hline
	Attacks & \\
	\hline
	Maul & A 0, P 2, \textlf{Penetration} 2, \textlf{lethality} 2, \textlf{cleave} 1, \textlf{cumbersome}\\
	\hline
\end{tabular}

\begin{tabular}{|l|l|}
	\hline
	Name & Trog thunderer\\
	\hline
	Weapons & Long axe and trog cannon\\
	Armour & Trog plate (H) T 12 \\
	Skills & Intimidate\\
	Perks & Rifle drill, desperate mechanisms \\
	Attributes & M 1, W 0, C 0, R 1, End 3\\
	\hline
	Attacks & \\
	\hline
	Trog cannon & A $-1$, P 3, \textlf{blast} 0, range 6, reload 4, \textlf{cumbersome}\\
	Long axe & A 0, P 3, \textlf{cleave} 1, \textlf{dmg edge+}\\
	\hline
\end{tabular}

\begin{tabular}{|l|l|}
	\hline
	Name & Trog line-breaker\\
	\hline
	Weapons & Heavy shield and battle axe\\
	Armour & Trog plate (H) T 12 \\
	Skills & Intimidate\\
	Perks & Bull headed, Stampede, Pain train \\
	Attributes & M 1, W 0, C 0, R 1, End 3\\
	\hline
	Attacks & \\
	\hline
	Battle axe and heavy shield & A $-1$, P 3, \textlf{cleave} 1, \textlf{Defence} 2\\
	\hline
\end{tabular}

\subsection{Synod}

\begin{tabular}{|l|l|}
	\hline
	Name & Keeper of order\\
	\hline
	Weapons & Halberd and war crossbow\\
	Armour & Hauberk and plate (H) T 12\\
	Skills & Intimidate\\
	Perks & Pole position, reloading drill \\
	Attributes & M 1, W 0, C 0, R 1, End 2\\
	\hline
	Attacks & \\
	\hline
	Halberd & A 0, P 1, \textlf{rending} 1, $+1$ \textlf{disarm} \\
	War crossbow & A 1, P 1, range 8 \\
	\hline
\end{tabular}

\begin{tabular}{|l|l|}
	\hline
	Name & Hunter tracker\\
	\hline
	Weapons & Musket with bayonet\\
	Armour & Mail and leather (M) T 11\\
	Skills & Intimidate, Survival\\
	Perks & Rifle drill \\
	Attributes & M 0, W 1, C 1, R 0, End 1\\
	\hline
	Attacks & \\
	\hline
	Musket & A 0, P 1, \textlf{penetration} 2, \textlf{lethality} 2, range 9, reload 2 \\
	Bayonet & A 1, P 1, \textlf{rending} 1 \\
	\hline
\end{tabular}

\begin{tabular}{|l|l|}
	\hline
	Name & Hunter guardian\\
	\hline
	Weapons & Pike and mace\\
	Armour & Hauberk and plate (H) T 12\\
	Skills & Intimidate\\
	Perks & Pike wall \\
	Attributes & M 0, W 1, C 0, R 1, End 2\\
	\hline
	Attacks & \\
	\hline
	Pike & A 1, P 0, \textlf{rending} 2, \textlf{reach} 2 \\
	Mace & A 1, P 0, \textlf{penetration} 1 \\
	\hline
\end{tabular}

\begin{tabular}{|l|l|}
	\hline
	Name & Hunter stalker\\
	\hline
	Weapons & Dagger and 2 pistols\\
	Armour & Leather (L) T 10\\
	Skills & Stealth, Slight of hand \\
	Perks & Choose your targets, Made you look, Arrow to the knee \\
	Attributes & M 0, W 1, C 1, R 0, End 1\\
	\hline
	Attacks & \\
	\hline
	Dagger & A 1, P 0, \textlf{rending} 1 \\
	Pistol & A 0, P 1, \textlf{penetration} 2, range 6, reload 2 \\
	\hline
\end{tabular}

\begin{tabular}{|l|l|}
	\hline
	Name & Hunter officer\\
	\hline
	Weapons & Rapier and 2 pistols\\
	Armour & Hauberk and plate (H) T 12\\
	Skills & Intimidate, Leadership\\
	Perks & Decisive leadership, Duelist, Plan b \\
	Attributes & M 1, W 1, C 0, R 1, End 2\\
	\hline
	Attacks & \\
	\hline
	Rapier & A 2, P 1, \textlf{rending} 1, \\
	Pistol & A 0, P 2, \textlf{penetration} 2, range 6, reload 2 \\
	\hline
\end{tabular}



\listoftables


\end{document}
