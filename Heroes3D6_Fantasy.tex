%Copyright (C) 2021  Geoff Beck
%
%This document is falls under the category of free software: you can redistribute it and/or modify
%it under the terms of the GNU General Public License as published by
%the Free Software Foundation, either version 3 of the License, or
%(at your option) any later version.
%
%This program is distributed in the hope that it will be useful,
%but WITHOUT ANY WARRANTY; without even the implied warranty of
%MERCHANTABILITY or FITNESS FOR A PARTICULAR PURPOSE.  See the
%GNU General Public License for more details.
%
%See https://www.gnu.org/licenses/ for more details

\documentclass[a4paper,11pt,oneside]{book}

\usepackage{fullpage}
\usepackage{amsmath}
\usepackage{import}
\usepackage{tocbibind}
\usepackage[bookmarks=true,plainpages=false]{hyperref}
\usepackage{titletoc,titlecaps}
\usepackage{caption}

%\usepackage{helvet}
%\usepackage{sansmath}
%\usepackage{sfmath}
%\renewcommand{\familydefault}{\sfdefault}

\usepackage[T1]{fontenc}
%\usepackage{palatino,mathpazo}
\usepackage{lmodern}


\newcommand{\dicediffbase}{11}
\newcommand{\dicenoprof}{-1}
\newcommand{\dicecritlvl}{4}

\newcommand{\textlf}[1]{\textbf{\titlecap{#1}}}
\newcommand{\textlfirst}[1]{\textbf{\textit{\titlecap{#1}}}}

\title{\textbf{\huge Heroes3D6\\Fantasy Role-playing Rules Add-on}}
\author{Geoff Beck}
\date{}

\begin{document}
\maketitle
\frontmatter
\tableofcontents
\mainmatter

%\chapter{The world and the `Last City'}
%The world of Krell is a devastated wasteland. An event in the distant past commonly referred to as the `Catastrophe' unleashed vast quantities of warping magical energies that laid waste to cities and ecosystems across the planet. The cause of this disaster is not clearly remembered, however, it has left a long lasting suspicion of magic in the few survivors that cling to this wrecked world. Over time the lingering magic, called `the residual', has produced new mutant life-forms by warping the small surviving populations. The `Last City' is the only major settlement in existence, built after the Catastrophe and housing peoples of all species, it teeters constantly the brink of collapse as a result of supply shortages as well as internal tensions. What holds the city together is the shelter it provides from the residual, this being a great wall built of a patchwork of `untainted' iron. This being a metal of great value, often used as currency outside the city, as it survived the Catastrophe without being infused with dangerous residual energies and now repels these same influences.    
%
%Small settlements dot the wastelands outside the Last City, these struggle constantly against the dangers of residual, mutant wild-life, bandits, and the terrifying changekin. These last are feral people twisted by the residual energies into nightmarish monsters. Life outside the city is a constant struggle to find food and water that are untainted by the residual, as too much exposure can eventually twist one into a changekin.
%
%\section{The last city}
%The only bastion of civilisation in a blighted world, the Last City stands as monument to the endurance of the people of Krell. Surrounded by a patch-work wall of untainted iron the city is largely protected from the warping power of the residual. Entrances to the city are guarded and those too contaminated by the residual are denied entrance, especially as a common punishment for crimes is being exposed to contamination and cast out of the city. Such is the suspicion of magic and its users that all mages must be registered and are branded with forehead marking, those who refuse are cast out as criminals. The mage brand brings much suspicion upon its bearers with common folk often spitting in the street at their passing, and some being unwilling to even do business with them. Often children with magical talent are disowned by their families and turned out onto the streets. 
%
%The city is ruled over by council of exclusively human `founding families', constituting a ruling aristocracy whose power is enforced by a small soldier corps. Low-level crime is endemic in the city, as it contains many desperate people of all species, and criminal gangs act as the defacto rulers of some districts and trades. The existence of the city is precarious, as it depends strongly on food grown in surrounding settlements, which are constantly under pressure from raiding bandits, changekin, and the residual itself. Despite this, it is the most sure shelter against the terrors of the wastelands.
%
%Within the city coins are used as currency but the alternative of untainted iron is also widely accepted (it is often the only currency used in the wastelands). A coin-sized disk of this metal is exchanged for 3 gold coins, such is its value in warding off the effects of residual when travelling outside the city.   
%
%\section{The changed and the changekin}  
%The changed are a phenomenon that began after the Catastrophe. They are beings that have been warped by the residual, and exhibit anything from minor changes like scales or feathers on their skin, up to limbs or heads being a mixture of animal and human (bird claws, goats hooves, beaks, animal jaws). The changed are, however, still people and although often viewed with suspicion and prejudice they are not attacked on sight. 
%
%Changekin are somewhat like the changed but have become feral, their minds rotted away by the residual. Their mutations tend to be even more extreme, often multiple additional limbs, or large bloated bodies. Changekin are savage and attack anything on sight, except for other changekin (most of the time). Such creatures form into roaming hunting bands that seek out anything weak enough to be preyed upon, attacking travellers or small settlements if their band grows large enough. Changekin seldom wield weapons or wear armour, their minds are too damaged for such rationality.  
%
%
%\section{The residual}
%The residual is remnant warping energy left over from the Catastrophe. Exposure to it is dangerous as it causes damage to exposed tissues and can, with sufficiently high doses, twist the bodies of the exposed in monstrous forms. The residual is undetectable to normal senses except in very high concentrations, where it feels like a heated crackle in the air (these concentration levels are immediately dangerous as well). Untainted iron is commonly used to ward off the effects of the residual and skilled alchemists can produce potions brewed with it that can protect from, or cleanse a body of, the residuals effects. 
%
%Whenever exposed to a residual infused environment (or consuming tainted food/drink) a creature must make a \textlf{resolve} check against a \textlf{difficulty} set by how saturated the area is. For example low contamination is \textlf{difficulty} 8, dangerous is 11, and deadly is 14. If the creature fails they gain a Residual point (\textlf{critical failure} adds extra points). The number of points is added to the \textlf{difficulty} of subsequent checks against residual contamination. At 5 points the creature begins to feel light-headed, weak, and sweaty (has an \textlf{edge} penalty on all actions). At 10 points a creature dies. Residual points last until removed. Armour made of untainted iron grants an \textlf{edge} on all such \textlf{resolve} checks and weapons of this metal have an \textlf{edge} bonus to damage rolls against changekin. 



\chapter{The highlands of Atla}
This area of land is entirely mountainous with the major city of Stormheight occupying the tallest peak in the region. Smaller towns and villages dot the mountain valleys around the region. The mountainous terrain is rich in aether, with great deposits of crystals being mined in many of taller peaks. This richness of aether has lead to Stormheight being a seat of technological advancement, with the study of aether being an integral part of its powerful religious structures.  

\section{Religion and society}
Stormheight itself is home to the Synod of Inquiry, an organisation dedicated to the reverence of the natural world as a matchless system of harmony (the religion itself is known as the Harmony). The synod is extremely influential, being the ruling body of the city in all but name. The heart of this religious body is a grand campus, where all study of the natural world and its aether is conducted in a suitably reverent manner. Needless to say, scholarship and aethermancy outside of this body is illegal. The rationale being that unrestricted inquiry might lead to disastrous abuses of nature. Under the Harmony, the duty of thinking beings is to match themselves to the harmony of nature, part of this is by achieving ``inner harmony'', the other being in harmony with society. Everyone in Stormheight is allocated a role and occupation according to the assessment of the synod. Those who cannot find harmony with their allocated position are confined to asylums and studied, the better to find a cure for such disharmonius afflictions. The influence of the synod is significantly lesser in the countryside, where folk festivals tend to match the religious calender of the Harmony, but the people aren't really invested in the wider system. Their only encounter with the synod being the yearly arrival of tax men, along with assessors who decide on the future occupations of the village youth.





\chapter{Aether}
Aether is the term used to describe the natural energy that flows throughout the world. All creatures are attuned to aether to some extent, but study and practice can allow one to directly manipulate their own aether, as well as draw it in from their environment. Aether manipulation can create a variety of effects, either aligned to the natural elements or altering the aetheric fields of living creatures.

\section{Aether points}
Every character has a maximum of 1 aether point at any one time. Any character can spend this on the \textlf{aether surge} ability. Holding additional points requires a device called an aether capacitor (see Section~\ref{sec:capacitor}).

\subsection{Aether surge}
The character concentrates their aether to grant themselves a burst of power. At a cost of 1 aether point they may gain an \textlf{edge} bonus on a roll of their choice.

\subsection{Recharging aether}
A one hour rest restores 1 aether point to any character that had none remaining. A hearty meal also restores up to 1 aether point. A full night's sleep restores up to 3 aether points. A character cannot gain aether if they would exceed their allowed maximum.

\subsection{Aether crystals}
Sometimes aether crystallises of its own accord. These crystals can be consumed by anyone to gain aether points. The number gained depends on the strength of the crystal. In addition to this, a character consuming a crystal experiences a rush of life and enhanced sensations for 10 minutes.
\subsubsection{Crystal jitters}
The consumption of aether crystals is addictive, withdrawal manifesting in the form of uncontrolled shaking motions throughout the victim's body. Each time a character uses a crystal they gain crystal points equal to the gained aether. Crystal points decay at a rate of 1 per week. A character must make a \textlf{resolve} check each day vs 10 + crystal points. If they fail, they are overcome with a need to acquire and use more crystals. This means they have an \textlf{edge} penalty on all checks until they consume a crystal.   




\chapter{Character creation}
These are additional rules providing the options for character species and additional backgrounds in this fantasy setting.

\section{Character species}

\subsection{Humans}
Humans are a dynamic species: adaptive, resilient and, as a group, will always stand firm even against the most difficult odds.

Humans are the most prevalent species in the highlands of Atla, making up the majority of the rural population. Stormheight, however, is considerably more diverse, with lots of peoples of other species both resident and passing through.

The average human lives for up to 80 years, 30 years being considered a mature adult, though humans are considered adults at the age of 16 in most societies. Though they lack the patience and long practice of the Elves they make up for it with a natural ability to adapt quickly to new situations or tasks. 

The elves' popular view of humans is that they are flighty and lacking in focus. Mus-folk stereotype humans as grim and unscrupulous. Trogs stereotype humankind as deceitful and conniving.  

\subsubsection*{Flexibility}
Humans are extremely adaptable and may select an extra associated skill for their background.


\subsection{Elves}
Though elves somewhat resemble humans at first glance, there is a world of difference on closer examination. Firstly, elves are incapable of digesting meat. They are vegetarian with a preferred diet of sweet, sugary things. Second, their manner of reproduction is completely different, elven eggs gestate externally in a similar manner to those of fish. Additionally elven blood is clear, as it contains no iron, and their hearts are far smaller than that of a man, as capillary action plays a major role in their circulatory systems.

Elves have high cheek bones and very narrow, angular features. Their skin varies in shade between green and light brown with a similar texture to smooth tree bark. Their eyes range in colour from yellow, through green and blue to white, the pupils of which are narrow and cat-like. Elves are short by human standards, and are also of a slighter build. All of them posses the curious ability to alter their hair colour at will. However, if they are not careful it will change on its own to reflect their mood. 

Elves' lives and metabolisms move far more slowly than those of humans and they are children till the age of about 40 years. They grow far more slowly than other species and tend only to reach their full growth by 60 years old. Additionally, they only need to eat or sleep once every two or three days. A small population of Elves exists within the city of Stormheight, but there are several predominantly elven villages around some of the more remote peaks. 

Humans tend to stereotype elves as vain, vapid, and hedonistic. Mus-folk and Trogs popularly assert elves to be crazy and lacking in any kind of sense. 

\subsubsection*{Perfectionists}
Elves live long lives and thus have time to pursue their skills to a point few others can reach. As such, elves may choose one additional skill proficiency.


\subsection{Mus-folk}
The mus-folk (or `Musmus' in their own language) resemble large rats. They reach a maximum height/length of 3 - 4 ft and tend to inhabit the margins of human settlements where they are regarded as a mixture of pest and second-class citizen (rude terms for them include `squeakers' and 'ratties'). Their own language consists of chirps and squeaks rather than words. However, they are fully capable of producing human and elvish speech. When they do, it comes out in a rapid torrent of words infused with poor grammar and often features repetition when they are excited. Mus-folk prefer underground dwellings and make burrows throughout the highland peaks, with a significant population in Stormheight itself. Their prolific digging means they gravitate towards mining for minerals and aether. They are often stereotyped as light-fingered and mischievous by other species.  

\subsubsection*{Sneak \& squeak}
Mus-folk are naturally stealthy. They may always add \textlf{stealth} to their listed of associated background skills. 

\subsubsection*{Small}
Mus-folk are \textlf{small} creatures, and thus get +1 \textlf{deflect}.


\subsection{Trogs}
Trogs are large (usually 7 ft tall) and bulky, but their most striking feature is their large eyes. Trog facial features tend to be heavy, with wide square jaws and large beak-like noses. Trogs have tough greyish, slightly scaly skin and are in general quite difficult to kill or injure. Traditionally, trog groups are familial and organised around a mother trog, a very large and formidable female (female trogs tend to be larger and stronger than males), with a harem of males who care for the young trogs. Plenty of young trogs find this authoritarian system stifling, so they form their own more egalitarian bands which often migrate to settlements like Stormheight. This form of living is scorned and belittled by their elders. Trogs typically prefer dark places, due to their sensitive eyes. This means they tend to live in close proximity to the Mus-folk, often forming a symbiotic system where Trogs provide the muscle and Mus-folk, the cunning.

Other species have a stereotyped view of Trogs as stupid and violent (Mus-folk also view them as useful).  

\subsubsection*{Tough as nails}
Trogs are extremely durable and thus have + 1 \textlf{endurance}.

\subsubsection*{Dark-dwellers}
Trogs have excellent eyesight in the dark as well as sensitive noses. They have an \textlf{edge} bonus on \textlf{awareness} checks in the dark.


%\subsection{The changed}
%These people are the long-term effects of the Catastrophe, sentient creatures whose forms have been altered by the warping magic of the residual. Unlike the feral changekin, their minds are intact and they continue to try and live within the society of their original people. The changed are regarded with suspicion and prejudice by all of the unchanged and are often openly discriminated against. Some of the changed form their own settlements to avoid this mistreatment. 
%
%Changed can be created by extreme residual exposure or born, to either changed parents, or those unchanged who have suffered a slow build up of residual exposure. As such changed can be born to any family, even within the walls of the last city. They are commonly cast out of their families to fend for themselves.
%
%Changed vary wildly in appearance, as they are formed from a base species: human, trog, elf, or mus which has then been warped with strange features. Some possibilities can be elongated limbs, clawed hands, feathers or scales on their skin, fangs instead of teeth, horned heads, animal heads (or elements of this like a beak or snout), hoofed or clawed feet, boney spines jutting from the body, a tentacle instead of a normal limb, extra eyes, a tail, or reverse-jointed legs (like a bird).
%
%\subsubsection*{Change-fire}
%The changed are infused with the warping energies of the residual. They can channel this energy to emit a gout of roiling, iridescent fire from a limb or orifice. This is a 1 action point ability that can be used on two adjacent targets (within 6 m) and deals a hit with normal \textlf{lethality} while also inflicting a Residual point if it causes damage. 
%
%\subsubsection*{Infused}
%The changed have an \textlf{edge} bonus on \textlf{resolve} checks against Residual contamination.


\section{Starting equipment}
A new character may choose one piece of armour and up to three weapons from Table~\ref{tab:start-gear}. All characters get a set of clothes.
\begin{table}[ht!]
	\centering
	\caption{New Adventurers Starting Gear.}
	\label{tab:start-gear}
	\begin{tabular}{|l|l|l|l|l|}
		\hline
		Name & Power & Hands &  Lethality & Notes\\
		\hline
		Rusty sword & - & 1 & N & -\\
		Hunting knife & - & 1 & N & Small\\
		Notched axe & +1 & 2 & N & -\\
		Worn crossbow & - & 2 & N & MD, Reload 2, Range 2\\
		Corroded pike & - & 2 & N & Reach, Rending\\
		Pitted hammer & - & 2/1 & N & Penetration 1/- \\
		Aged spear & - & 2/1 & N & Reach/Throw 1\\
		Creaky short bow & - & 2 & N & Range 1\\
		Old hunting bow & +1 & 2 & N & Range 2, Reload 1\\
		Scarred great sword & +1 & 2 & N & -\\
		Shabby pistol & +1 & 2 & N & Reload 1, Range 1 \\
		Ramshackle musket & +1 & 2 & N & Range 2, Reload 2, Penetration 1 \\
		Dented shield & - & 1 & - & Deflect +1 \\
		\hline
	\end{tabular}
	\begin{tabular}{|l|l|l|l|}	
		\hline
		Name & Toughness & Type & Notes\\
		\hline
		Battered breastplate and mail & +3 & M & - \\
		Rusted mail hauberk & +2 & M & - \\
		Tattered gambeson & +1 & L & - \\
		Travelling clothes & - & - & - \\
		\hline
	\end{tabular}
\end{table}


\section{Additional backgrounds}

\subsection{Student of the synod}
The character has spent time studying aether and its powers under the auspices of synod in Stormheight. Associated skills: Aethermancy, Mechanical, Aethertech, Infusary, History, Religion, Plants, Animals.

\subsection{Savant}
The character has learned the mysteries of aethermancy on their own. Such individuals often hide their talents from the synod for fear of being forced into conformity or having to leave their home village. Associated skills: Aethermancy, Infusary, Aethertech, Survival, Persuade, Decieve. 

\subsection{Exile}
The character has been exiled by the synod for repeated defiance of the Harmony's dictates. They now skulk on the margins of society, keeping away from the eyes of law enforcement in the far-flung rural districts of highlands. Some such individuals find sanctuary among the Mus-folk, who have little regard for the Harmony or synod, others turn to less than legal ways of making a living in the aethertech black-market. Associated skills: Stealth, Disguise, Deceive, Survival, Slight of hand, Aethertech, Awareness.   


\section{Additional convictions}

%Religious conviction for the setting
\subsection{The Harmony of all}
The character is firmly convinced of the teachings of the Harmony. They live their life in pursuit of harmony within themselves and with society as a whole. They tend to regard the synod and its laws as an essential ingredient in achieving harmony with the natural world.

\subsection{Free spirit}
The character detests the enforced conformity that the Harmony brings with it. Having these views publicly known can be dangerous. 

\subsection{Aethermancy unlimited}
Aethermancy and aethertech are forces for good, their exploration brings nothing but enlightenment. Some radical individuals might also hold that it should be free from the synod's interference. 

\subsection{Aether exploited}
Aethermancy and aethertech go too far in their rampant consumption of the world's aether. One cannot achieve the claimed Harmony while mercilessly extracting resources.




\chapter{Perks and proficiencies}

\section{General proficiencies}
These perks do \textbf{not} occupy equipment slots for passive or active \textlf{perk}, their effect is always active.

\subsection{Aetheric learning (3)}
(Requires \textlf{skill proficiency: aethermancy}) This may only be chosen at character creation. This grants the character an aether capacitor that can hold 1 ether point (see Section~\ref{sec:capacitor}). Additionally, they may choose two aether manifestations from the following list: Aetheric blast~\ref{spell:sorc-blast}, Bend light~\ref{spell:bend-light}, Resonant motion~\ref{spell:min-tele}, Overgrowth~\ref{spell:overgrowth}, Ignite~\ref{spell:ignite}, Illuminate~\ref{spell:illuminate}, Orb of aether~\ref{spell:orb-light}, Aetheric armour~\ref{spell:arcane-armour}, Blinding ray~\ref{spell:blind-ray}, Flare~\ref{spell:flare}, and Stone sense~\ref{spell:stone-sense}.

\subsection{Weapon proficiency: X (2)}
In this setting, X can be drawn from: unarmed, swords \& daggers, axes, blunt weapons, pole weapons, extended weapons, bows, crossbows, and black-powder weapons. A character who is not proficient with his equipped weapon counts his \textlf{aim} and \textlf{power} as $\dicenoprof$.

%\subsection{Aetheric reservoir (3)}
%(Requires \textlf{skill proficiency: aethermancy}) The character increases their maximum aether limit by 1 point.



\section{Active perks}
These open up new actions that a character can make and they must occupy an equipment slot for active \textlf{perks} to be usable.

%\subsection{Channelled casting (2)}
%(Requires \textlf{Arcane learning}) The character can cast a spell that normally costs 1 action point for 2 instead. If they do so they gain an \textlf{edge} on opposed checks associated with the spell. If the spell inflicts a damage roll then it gains \textlf{Burst} + 1.

\subsection{Hawk Talon (4)}
(Requires \textlf{Weapon proficiency: bows}). Allows the character to launch a pair of arrows from a bow with a single shot. At a cost of 1 action point, the character's next bow attack has \textlf{Burst} +1.

%\subsection{Rhythm (5)}
%(Requires \textlf{Archery}). Breathing and rhythm are vital to the art of the archer. If the character's attacks made with a bow cause damage, he may make an extra shooting attack this round. This bonus may only be claimed once per turn.

\subsection{Heart-seeker (3)}
(Requires \textlf{Weapon proficiency: bows}). A skilled bowman knows just where to place their shots for maximum penetration. An attack with a bow can be declared as a Heart-seeker, in which case it costs 1 extra action point but gains \textlf{Rending}.

\subsection{Main Gauche (3)}
There is an art to pairing weapons for fighting with both hands, this character has mastered it. When dual-wielding, if pairing a \textlf{Small} weapon and one that isn't, the character may elect not to attack with the \textlf{small} weapon. If they do so, they gain an \textlf{edge} bonus to \textlf{deflect} against one attack within a round.



\section{Passive}
These passively enhance the character and must occupy an equipment slot for passive \textlf{perks} to make their benefit usable.

\subsection{Aetheric mastery (6)}
(Requires \textlf{Skill proficiency: aethermancy}) The character can equip 2 aether manifestations per active perk slot.

\subsection{Aetheric savant (5)}
The character can spend an aether point to gain the benefits of any \textlf{perk} for 1 round.

\subsection{Reach of the aether (3)}
(Requires \textlf{Skill proficiency: aethermancy}) Aetheric manifestations have + 1 range.

\subsection{Hammer Time (4)}
\textlf{Critical hits} from blunt weapons make the victim \textlf{vulnerable} to the next damaging hit.

\subsection{Lethal Thrust (4)}
The character delivers their killing blows with the point of a blade. While wielding a sword, the character's attacks gain \textlf{Rending} against victims with at least one face-up wound card.

\subsection{Momentum (4)}
When wielding cleaving weapons, like axes or glaives, the character's attacks gain \textlf{Heavy Weapon} +1 if their target failed a \textlf{Deflect} check against their attacks last round. This bonus stacks and lasts till an attack is \textlf{Deflected} or combat ends.

\subsection{Multi-tasker (4)}
(Requires \textlf{Skill proficiency: aethermancy}) The character has learned to concentrate on many things at once. This allows them to maintain two \textlf{persistent} effects at the same time.

\subsection{Reloading drill (4)}
The character is well practised at rapidly preparing crossbows to fire. This reduces crossbow \textlf{reload} action costs by 1.

\subsection{Rifle drill (4)}
Black-powder weapons \textlf{reload} action costs are reduced by 1 for the character. 

\subsection{Shield bash (3)}
The character may use a shield to \textlf{shove} (add the \textlf{deflect} bonus to such checks). 

%\subsection{Slippery Mind (3)}
%This allows the character to make Resist checks against spells using their \textlf{Cunning} rather than \textlf{Resolve}.

\subsection{Staff Mastery (4)}
The character may use a quarter staff (or other pole weapon) as though it was a dual-ended weapon, gaining an extra damage roll but losing the \textlf{reach} property when used in this way. No \textlf{dual-wielding} penalties apply when using this \textlf{perk}.

\subsection{Weapon Specialisation (6)}
(Requires \textlf{Weapon proficiency: X}) This grants an \textlf{edge} bonus on \textlf{aim} checks with weapons of type X.

\subsection{Witch hunter (3)}
The character gains an \textlf{edge} bonus to \textlf{resist (R)} versus aether manifestations.

\subsubsection{Upgrade: Burn the witch (3)}
(Requires \textlf{Witch hunter}) Gain an \textlf{edge} bonus to damage rolls against aethermancers, but only with flaming weapons.
 





\chapter{Character Skills}

\section{Aethermancy (Wit)}
This is the skill invoked to produce aether manifestations. More details can be found in Chapter~\ref{chap:magic}.

\section{Aethertech (Cunning)}
This represents a character's knowledge of aether-based technology and other sophisticated devices. A character does not need to be an \textlf{aethermancer} to use this skill. Using this skill a character can
\begin{itemize}
	\item Operate aethertech devices (non-proficient users have an \textlf{edge} penalty)
	\begin{itemize}
		\item \textlf{critical failure} means the device fails catastrophically
	\end{itemize}
	\item Identify functions of aethertech devices (non-proficient users have an \textlf{edge} penalty)
	\begin{itemize}
		\item Success means the functions are correctly identified
		\item \textlf{critical success} gives the character \textlf{edge} on subsequent checks with the device
		\item Failure means the character is unsure about the device
		\item \textlf{critical failure} means the character is confident but wrong about the device
	\end{itemize}
\end{itemize}

\subsection{Weapon tinkering}
This is an additional function of the aethertech skill, it allows characters to modify equipment with aethertech enhancements. The required materials are aether crystals and precious metals (to conduct aether), the cost of which are listed with each enhancement. A single piece of equipment can only ever have one enhancement. These enhancements cost double the material cost from professional aethertechs.

\subsubsection{Arcing (15)}
The weapon carries complicated metal architecture that enhances aether emitted by the bearer. Aether manifestations of the wielder may add one extra target. Material cost: 17 g.

\subsubsection{Aggression (10)}
The weapon feels alive in your hand as aether thrums through inlaid metal veins. The weapon grants + 1 \textlf{Aim} on its first attack against a given target. Material cost: 3 g.

\subsubsection{Searing (11)}
The weapon unleashes searing aether into existing wounds on the target. \textlf{Bleeding} targets are \textlf{Vulnerable} for 1 round if damaged by the weapon. Material cost: 5 g.

\subsubsection{Butchery (13)}
The weapon is imbued with aether circuitry that vastly increases the power of its impacts. The weapon has \textlf{Heavy Weapon} +1. Material cost: 12 g.

\subsubsection{Culling (13)}
The weapon unleashes a burst of aether when it strikes weakened targets. Attacks made against targets which are \textlf{Crippled}, \textlf{Knocked Down}, or \textlf{Bleeding} benefit from a \textlf{lethality} upgrade. Material cost: 12 g.

\subsubsection{Deflection (12)}
The aether circuits in this weapon create a shimmering field around the weapon that can deflect incoming attacks. This allows the weapon's wielder to make a counter attack if they score a \textlf{Critical Deflect}. Cost 7 g.

\subsubsection{Devastation (14)}
The weapon emits blasts of energy whenever it strikes. The weapon has \textlf{Burst} +1. Material cost: 15 g.

\subsubsection{Executioner (15)}
The weapon has auxiliary systems that kick-in when it inflicts a telling strike to increase its power even further. When the weapon inflicts a \textlf{Critical Hit} the wielder is granted an extra damage roll. Material cost: 17 g.

\subsubsection{Finesse (11)}
The weapon is enhanced with guidance mechanisms to keep the wielder's aim true. \textlf{Critical Failures} made with the weapon may be re-rolled. Material cost: 5 g.

\subsubsection{Impaling (14)}
The weapons stabbing or cutting edges are augmented with active aether blades that bore into the target with a well-placed strike. The weapon gains \textlf{Rending}. Material cost: 15 g.

\subsubsection{Malevolent (12)}
Aether circuits zap the nervous system of its victims. The weapon inflicts an \textlf{edge} penalty to the next \textlf{Deflect} check made by its victims. Material cost: 7 g.

\subsubsection{Masterful (16)}
The weapon generates a resonance with aether manifestations produced by the wearer. The weapon applies an \textlf{edge} penalty to \textlf{Resist} attempts made against the wielder's aetheric manifestations. Material cost: 20 g.

\subsubsection{Penetrating (11)}
The weapon's edge uses aetheric fields to slice through even the thickest armour. The weapon gains \textlf{Penetration} + 1. Material cost: 5 g.

\subsubsection{Thirsting (13)}
Aetheric circuits draw energy out of targets wounded by this weapon. When the weapon causes at least 1 \textlf{endurance} damage to a victim, the wielder regains 1 missing \textlf{endurance} or aether point. Material cost: 15 g. 

\subsubsection{Thunderous (10)}
Powerful blows from this weapon unleash a booming aetheric disturbance. \textlf{Critical Hits} from this weapon knock the victim down. Material cost: 3 g.

\subsubsection{Vengeful (12)}
The aetheric circuits in the weapon register the signature of targets who have damaged the wielder. The weapons gains an \textlf{edge} bonus to \textlf{aim} and damage rolls against a foe that has damaged you within the last round of combat. Material cost: 7 g. 

\subsubsection{Vorpal (16)}
The weapon is imbued with aetheric fields that disrupt the target, making it far more lethal. Upgrades \textlf{Lethality} of the weapon on \textlf{critical hits}. Material cost: 20 g.


\subsection{Armour tinkering}
This is an additional function of the aethertech skill, it allows characters to modify equipment with aethertech enhancements. The required materials are aether crystals and precious metals (to conduct aether), the cost of which are listed with each enhancement. A single piece of equipment can only ever have one enhancement. The difficulty of the enhancement is given in brackets after the name. These enhancements cost double the material cost from professional aethertechs.

\subsubsection{Resilient (11)}
Aetheric circuitry runs through the armour, disrupting aetheric manifestations. Grants the wearer + 1 to \textlf{Resist} checks. Material cost: 5 g.

\subsubsection{Feather-light (12)}
The aetheric field of the armour makes its wearer lighter. This grants its wearer +1 range to their normal movement. Material cost: 7 g.

\subsubsection{Unflinching (15)}
An aetheric field generator softens blows dealt to the armour. This armour now negates the first \textlf{endurance} point lost each turn. Material cost: 17 g.

\subsubsection{Evasion (11)}
A flickering aetheric field surrounds the wearer, making it hard to determine their location. The armour grants + 1 \textlf{Deflect}. Material cost: 5 g.

\subsubsection{Mirrored (12)}
This can only be used upon a shield, it uses aether ciruits to make the shield reflect aetheric manifestations. While using the shield, a wielder may use their \textlf{Deflect} when making \textlf{Resist} attempts against aetheric manifestations. Material cost: 7 g.



\section{Infusary (Cunning)}
\textlf{Infusary} is the craft of creating aether-enhanced liquids that can have powerful effects. 

An infusion needs a container to hold it as well as a aether crystals and a signature ingredient to shape the purpose of the aether. To create infusions a character needs set of glass vessels and equipment for measuring, grinding and heating ingredients (infuser's tools) that can be purchased from an aethertech for 50 silver.
Table~\ref{tab:alch} displays some suggested infusions.

\begin{table}[ht!]
\caption{Aether infusions. D is the \textlf{difficulty}, time is how long the effects last and is given in minutes. The cost reflects that of the ingredients for a single draught of potion and the bottle to hold it, double this for the price charged by most infusers.}
\begin{tabular}{|l|l|l|l|l|l|}
\hline
Potion & D & Effects & Time & Ingredients & Cost \\
\hline
Might &  13 & +1 Might & 10 & Troll teeth & 25 s \\
Cunning & 13 & +1 Cunning & 10 & Fresh mus-folk whiskers & 25 s \\
Aethermancy & 12 &\textlf{edge} bonus - aethermancy & 10 & Sapphires & 30 s\\
Iron-flesh & 8 & +1 Toughness & 10 & Fine granite power & 10 s \\
Fire-blood & 11 & Enrage & - & Trog blood & 20 s \\
Hawk-eye & 12 & \textlf{edge} Awareness/Aim & 20 & Eagle feathers & 30 s \\
Troll-blood & 15 & Regeneration & 5 & Troll blood & 1 g \\
Warding & 10 & \textlf{edge} on \textlf{resist} checks & 10 & Powdered silver & 30 s \\ 
Invigoration & 11 & Restore 1 \textlf{endurance} & - & Mend-well root & 20 s \\
Restoration & 10 & Cure 1 Condition & - & Common herbs & 30 s \\
Peace & 14 & Cure all Conditions & - & Nightshade,  & 60 s \\
 & & & & Mend-well leaves & \\
Competence & 14 & \textlf{edge} bonus on a chosen skill & 60 & Gold, Silver & 2 g \\
Giant-blood & 16 & Become Large Creature  & 10 & Giant Blood & 2 g \\
 & & (+1 Might) & & & \\
Camoflage & 11 & \textlf{Stealth edge} bonus & 20 & Nightshade, Ivy root & 30 s \\
Invisibility & 15 & Invisibility & 5 & Distilled aether & 5 g \\
Haste & 15 & +1 action point per round & 5 & Adrenaline & 5 g \\ 
Explosive & 12 & radius 1, Burst 1, Power +1 & - & Acid, coal, salt-peter & 30 s \\
\hline
\end{tabular}
\label{tab:alch}
\end{table}



\chapter{Aethermancy}
\label{chap:magic}

Any character with sufficient knowledge of the flow and hum of the aether can alter the very fabric of the world.  The energy for this is drawn from the inner strength of the character themselves. Thus, a character can only exercise this powers sparingly, lest they exhaust the aether powering their body. The use of aether to create external effects is known as ``manifestation". Characters learn new manifestations either through being taught them or by spending Hero Experience to unlock them.

\section{Aethermantic mechanics}

\subsection{Aether capacitor}
\label{sec:capacitor}
A character's body can only hold so much aether. To transcend this limitation, aethermancers have developed a device called an `aether capacitor' which stores additional aether for them. This device is bulky and has wires running from it various points on the aethermancer's body. This means that one cannot wear armour and have an aether capacitor equipped. A standard aether capacitor costs 1 g and can hold 1 aether point. For each additional capacity point the cost increases by 3 g. The maximum possible capacity is 4 points.  

\subsection{Producing a manifestation}
Manifestations are actions like any other. However, they also costs 1 point of aether, unless otherwise specified. Manifestations usually just require that the caster and victim make a \textlf{resist} check to decide if they take effect. However, some manifestations produce projectiles and thus also involve a \textlf{deflect} check. A \textlf{critical failure} during casting releases aether in an uncontrolled reaction. 

\subsection{Resist(X)}
A Resist(X) check involves the victim making an opposed roll with their natural attribute X against the aethermancer, who adds -1 or \textlf{Might} depending on \textlf{magic} skill proficiency. 

\subsection{Dispel}
An aethermancer can use their own aether to nullify the manifestations of another. This can be used in place of \textlf{Resist} or as 1 action point prepared action, it involves an \textlf{aethermancy} check opposed with the other aethermancer. If the check succeeds the manifestation is nullified.

\subsection{Persistent manifestations}
These have a long duration but require that concentration be maintained to keep them functioning. Thus, only one can be employed at a time. If a character is disabled or killed then their \textlf{persistent} manifestation ends. Otherwise, if they suffer a successful damage roll, they must make a \textlf{resolve} check against $\dicediffbase$ plus the \textlf{power} of the attack. If they fail, the \textlf{persistent} manifestation ends.





\section{Aether manifestation}
\label{sec:spells}

To represent a character's development in aethermancy they can learn new manifestations by paying the experience point cost listed in square brackets after the name. No manifestations can be learned unless the character has \textlf{Skill proficiency: aethermancy}. All manifestations cost 1 action point unless otherwise specified. \textlf{persistent} manifestations are marked with (P) after their experience cost.

\subsection{Aether siphon [2]}
An aethermancer can establish a resonance with another creature such that the aether is stripped from their body. For 2 action points (but no aether points) a chosen target within range 0 (5 m) must \textlf{resist(C)} or lose an aether point. If the target has no aether points, it loses an \textlf{endurance} instead. If the aethermancer wins the \textlf{resist} check, they gain 1 aether point.


\subsection{Aetheric blast [2]}
\label{spell:sorc-blast}
This costs 1 action point and projects your aether out in a searing jet. A chosen target within range 1 (15 m) must \textlf{deflect}, if they fail then they must \textlf{resist(R)} or suffer damage with normal \textlf{lethality} (\textlf{critical success} for the aethermancer on \textlf{resist} increments damage severity). You have \textlf{Edge} on the \textlf{resist} roll. 
\subsubsection{Augment: Resounding blast [2]}
Aetheric blast now inflicts \textlf{knocked down} if it causes damage.


\subsection{Aetheric wall [3] (P)}
For 1 action point you manifest a wall of shimmering aether around a radius of 5 m (1 combat area) within a range of 2 (25 m). Passing through the wall costs any creature 1 action point and requires a \textlf{resist(M)} check, failure means they cannot traverse the wall and suffer damage with normal \textlf{lethality} (\textlf{critical success} for the aethermancer on the \textlf{resist} increments the damage severity). This lasts for 10 minutes and is a \textlf{persistent} manifestation.


\subsection{Animal aether [2]}
Every living creature hums with an aetheric field. By carefully tuning your own aetheric field into resonance you can hear the mood and emotions of animals.
\subsubsection{Augment: Animal resonance [2]}
Your mastery of the aetheric fields of animals allows you to communicate emotions to them as well as attempt to \textlf{persuade} them.


\subsection{Animate homunculus [2]}
You can shape a small creature out of any given material (30 cm is the maximum size), this creature is then imbued with aether. This grants it a kind of limited life while directly controlled by you (this requires your full concentration). The creature will only act when controlled, otherwise it remains inert. A homunculus has no skill proficiencies and cannot inflict damage in combat.
\subsubsection{Augment: Automaton [4]}
Your homunculi are always animated and will obey simple verbal commands. You may have 1 + \textlf{Wit} homunculi active at once. 
\subsubsection{Augment: Golem [5]}
The maximum size, for a single homunculus, is increased to medium (man-sized) creatures, these can wield weapons and make unarmed attacks in combat. 


\subsection{Arcing aether [3]}
You infuse your aether into a resonance with elemental lightning. For 2 action points, aetheric lighting leaps from your body and strikes a target within range 2 (25 m). The target must \textlf{resist(R)} or suffer damage with normal \textlf{lethality}. The lightning then leaps to a second target within range 1 of the first. It will only leap to the aethermancer's allies if no enemies are in range. 
\subsubsection{Augment: Resonant arcing [2]}
The lightning gains an extra leap every time it causes damage.
\subsubsection{Augment: Paralytic arcs [2]}
Victims damaged by this manifestation are \textlf{immobilised} for 1 round.

\subsection{Befuddle [2]}
A discordant aether infusion removes your target's ability to distinguish friends from foes. If the target fails a \textlf{resist(W)} check, they regard all creatures as hostile and dangerous. This means they can panic, run away, or lash-out at anything nearby. Choose combat targets and behaviour for the victim randomly. This lasts up to 1 minute. Each round after the first, the victim may re-attempt to \textlf{resist}.


\subsection{Bend light [3] (P)}
\label{spell:bend-light}
Aether flows from your fingers creating an aetheric lattice that bends light. At a distance up to range 1 (15 m), you can create a stationary illusion up to medium size. Anyone looking at this effect must \textlf{resist(W)} to decide if they are fooled. This is a \textlf{persistent} effect. The illusion is only visual, it makes no sounds, smells, and cannot be touched. 
\subsubsection{Augment: Major illusion [2]}
You can bend light with such dexterity that you can create illusions up to 5 m in size. 


\subsection{Blinding ray [2]}
\label{spell:blind-ray}
You pour aether into a light source you touch, focusing its illumination into a searingly bright beam. This \textlf{blinds} a victim that fails a \textlf{resist(R)} check, additionally the normal lighting from the light source is removed for 1 round but the area adjacent to the target is fully illuminated for this time. The blindness lasts 1 additional round per level of \textlf{critical success} for the aethermancer on \textlf{resist}.


\subsection{Clinging aether [3]}
For 1 action point you blast a 5 m radius (1 combat area) within a range of 1 (15 m) with crackling aether that anchors everything to the earth. All creatures in the area must \textlf{resist(C)} or be \textlf{immobilised} for 1 round (1 additional round per level of aethermancer \textlf{critical success} on the \textlf{resist} check).


\subsection{Compel [3]}
You infuse aether into a target by touch, forcing them to make an action chosen by you if they fail a \textlf{Resist(W)} check. This fails if the action would harm the target. This is a \textlf{persistent} effect. Specify one extra action per level of \textlf{critical success} for the aethermancer on \textlf{resist}. The victim is aware that they being forced to act.
\subsubsection{Augment: Dominate [3]}
Your control is so strong that Compel will not fail if the action would be harmful to the victim itself.
\subsubsection{Augment: Project aether [2]}
You can now manifest this power at range 1.


\subsection{Damping field [2] (P)}
\label{spell:arcane-armour}
For 1 action point you envelope yourself in a high-pressure field of aether that depletes the force of incoming attacks. This manifests as a shimmering in the air around you and increases your \textlf{toughness} by your \textlf{wit} score. You cannot manifest this while wearing armour, as this disrupts the widespread projection of your aether. This manifestation lasts for 10 minutes and is a \textlf{persistent} effect.
\subsubsection{Augment: Hardened aether [3]}
Damage rolls against you suffer an \textlf{edge} penalty while this manifestation endures.


\subsection{Earthquake [4] (P)}
You flood the ground with aether, making it roll and buck as though it was alive. For 2 action points you make the ground violently shake. All creatures adjacent to you must \textlf{resist(M)} or be knocked down. This can be maintained as a \textlf{persistent} effect for up to 1 minute. 
\subsubsection{Augment: Project aether [2]}
You can now manifest this power at range 2 (25 m), instead of shaking the ground adjacent to you.


\subsection{Eyes of aether [2]}
You can alter your senses to perceive flows of the aether. This renders you blind, instead you view the world as the flow of the aether. This costs no aether points to use and may be cancelled at any time.


\subsection{Fiery surge [3]}
You can fuel a flame via aether infusion, greatly increase the intensity of an existing fire within range 1 (15 m). Any creature that makes contact with such a flame must \textlf{resist(C)} or suffer damage with crushing \textlf{lethality}. This fire burns for 1 minute.


\subsection{Fear [2]}
You infuse a target, within range 1 (15 m), with aether, causing a cloud of fear to pass across the target's mind if they fail a \textlf{resist(R)} check. Such a target cannot approach the aethermancer or remain adjacent to them for 1 minute. Each round after the first the victim may re-attempt the \textlf{resist(R)}. 
\subsubsection{Augment: Terror [3]}
You have learned to afflict the mind with absolute terror. The target of fear suffers an \textlf{edge} penalty to all rolls while this effect persists. 



\subsection{Flare [2]}
\label{spell:flare}
Projecting aether suddenly into a flame within range 1 (15 m), you cause it to burst into in blinding white flash. Anyone who can see the flare is \textlf{blind} for 1 round and 1 additional round per level of \textlf{critical failure} on \textlf{resist(M)}. The explosion of the flare itself is harmless.
\subsubsection{Augment: Burning flare [3]}
Flare ignites all adjacent creatures, who suffer damage with normal \textlf{lethality} if they failed \textlf{resist(M)} (\textlf{critical success} for the aethermancer on \textlf{resist} increments damage severity).


\subsection{Fireball [4]}
This costs 2 action points. You condense your aether into an explosive fireball that you can hurl at a 5 m radius (1 combat area) within range 2 (25 m). Creatures within the blast must \textlf{deflect} or make a \textlf{resist(R)} check. Failure on the latter inflicts damage with normal \textlf{lethality} (\textlf{lethality} scales with levels of \textlf{critical success} for the aethermancer). This manifestation sets all flammable material in the radius on fire.
\subsubsection{Augment: Blaster Master [3]}
The fiery explosion burns with voracious intensity. This manifestation gains the \textlf{Heavy weapon} 1 effect.


\subsection{Freezing storm [4] (P)}
For 2 action points you choose a region of radius 5 m radius (1 combat area) within a range of 2 (25 m) and rain a shower of freezing ice shards upon it. Creatures within the area suffer a hit with normal \textlf{lethality}. For 10 minutes the area is \textlf{rough terrain}, anyone entering the area suffers a hit with normal \textlf{lethality}. This is a \textlf{persistent} effect.
\subsubsection{Augment: Sub-zero [2]}
The cost of this manifestation is reduced by 1 action point.


\subsection{Glacial ray [3]}
This costs 2 action points. A ray of freezing aether strikes a target within range 1 (15 m). They must \textlf{deflect} and then \textlf{resist(R)} if they fail. Failure on the second check means they are \textlf{immobilised} and \textlf{vulnerable} until damaged. 
\subsubsection{Augment: Creeping frost [2]}
Glacial ray applies its effects to all creatures adjacent to the primary target.
\subsubsection{Augment: Deep-freeze [2]}
After the effect of Glacial ray ends the ice shatters causing all creatures adjacent to the target take a hit with normal \textlf{lethality}.


\subsection{Guided motion [3]}
Aether streams out from you to enhance the movements of another body within range 1 (15 m). This confers an \textlf{edge} bonus to an allied target's next \textlf{Athletics} or \textlf{deflect} check.
\subsubsection{Alternate: Inhibit motion [2]}
Guided motion can target an enemy and confer an \textlf{edge} penalty instead if they fail a \textlf{resist(M)} check.


\subsection{Ignite [2]}
\label{spell:ignite}
You infuse your aether into generating a resonance between your target and the element of fire. This manifestation can be used on a chosen target within range 1 (15 m). The fire created by Ignite also suffers from normal physical restrictions., i.e. You may not set fire to a creature unless it is naturally flammable, covered in oil, or circumstantially vulnerable. Materials like metals can be heated by this effect to burn their bearer. Setting fire to the clothes of a foe does no great harm to him (it might cause weak enemies to panic).


\subsection{Illuminate [2]}
\label{spell:illuminate}
You infuse an object with aether, causing it to resonate with the aetheric field of light. A single small object you touch begins to emit a soft aetheric glow. This provides low-light illumination over a single combat area (around a 5 m radius). This lasts for up to 6 hours.


\subsection{Iron arm [4] (P)}
You infuse a creature with aether that resonates with the musculature of their body. A chosen creature within range 1 (15 m) gains +1 \textlf{power}. This is a \textlf{persistent} effect that lasts up to 1 minute.


\subsection{Mimic sound [2]}
You subtly leak aether into the air, causing it to vibrate in a chosen pattern. This allows you to produce any sound you can imagine, at volumes between a whisper and a shout. The convincingness of this is decided by results of listeners' \textlf{resist(W)} checks.


\subsection{Orb of aether [3] (P)}
\label{spell:orb-light}
You can infuse your aether into the aetheric field of light itself, creating a condensed orb of luminosity. This manifests as floating orb of aether that can move a distance of 15 m (range 1) each turn. The orb provides full illumination within 5 m (1 combat area) of itself and \textlf{low} light within 15 m (radius 1). This lasts up to 1 hour and is a \textlf{persistent} effect.


\subsection{Overgrowth [3]}
\label{spell:overgrowth}
Plants have their own variety of aetheric field, you can infuse aether into this field to create a resonance. You may choose a region of radius 5 m radius (1 combat area) within a range of 1 (15 m) and make plant growth explode from the ground. This region is now \textlf{rough terrain}.


\subsection{Paths in the aether [2]}
You can attune your own aether to the fields within the terrain around you, allowing effortless avoidance of obstructions. This manifestation allows the aethermancer to ignore natural forms of \textlf{rough} or \textlf{dangerous terrain} effects as well as an \textlf{edge} bonus on \textlf{stealth} and \textlf{awareness} checks while in wilderness. This lasts for 1 hour.


\subsection{Prediction [3]}
Time itself resonates with aetheric frequencies. You can use this to attempt to peer into possible futures. This manifestation takes 10 minutes to complete. Afterwards, roll 3d6 and put them to one side. In addition, make an \textlf{aethermancy} check vs \textlf{difficulty} 13. If you succeed, then at any point within the next 24 hours you may replace one 3d6 roll (made by any creature or character) with the 3d6 you set aside. On a \textlf{critical failure} the GM may instead choose when to make the roll substitution. You may only have one such set of predicted dice available at once.
\subsubsection{Augment: Forecasting [3]}
You can store 2 predictions at once.


\subsection{Resonant motion [2]}
\label{spell:min-tele}
You infuse aether into objects so that their aetheric fields resonate with motion. This allows you to exert the force of a single hand to perform simple actions on an object visible to you within range 1 (15 m).
\subsubsection{Augment: Major resonance [3]}
The aethermancer is far more attuned to the resonance of motion, allowing the aethermancer to manipulate an object up to a maximum weight of 20 kg per point of \textlf{might}. If the object is used as a weapon, it uses the aethermancer's \textlf{aethermancy} for \textlf{aim}. The bonus \textlf{Power} of such a weapon is given by the excess lifting force of the manifestation (that not required to actually move the object):  + 1 per 10 kg excess. The \textlf{power} for grabbing objects is calculated in the same manner. If multiple objects are controlled, the \textlf{Power} bonus from excess lifting force may be divided between them at the aethermancer's discretion. This manifestation requires that the aethermancer to maintain full concentration, in combat they must spend an action point to manipulate objects with this power.


\subsection{Root weaving [3] (P)}
By creating a complex aetheric resonance between yourself and nearby plants, you can call on their assistance to entrap foes. Provided there are plants or other natural growths nearby, this can be used to make roots \textlf{grapple} or \textlf{shove} a chosen creature or object within the foliage (this can be used once per turn but costs no action points). The roots use your \textlf{aethermancy} power score when making opposed checks. This is a \textlf{persistent} effect that can be maintained for up to 10 minutes.
\subsubsection{Augment: Bushwhack [3]}
Plants and roots can be made to strike at your foes, using your \textlf{aethermancy} for \textlf{aim}. For an action point you can activate a tree to strike at a nearby foe, this attack has crushing \textlf{lethality}.


\subsection{Sapping aether [4] (P)}
You infuse a discordant aetheric field into a target, weakening their body. If a target within range 1 (15 m) fails a \textlf{resist(R)} check then, whenever they lose \textlf{endurance} they lose 1 additional point. This lasts for up to 1 minute and  is a \textlf{persistent} effect.
\subsubsection{Augment: Field [2]}
If used for 2 action points this affects all creatures within a radius of 5 m (1 combat area).


\subsection{Shadows in the aether [3]}
Light has its own aetheric field and your aether can be used to create a counter vibrating field that extinguishes lights. All light disappears within a chosen radius of 1 (around 15 m). Creatures within the region fight using the \textlf{dark} lighting rules. This lasts until you cancel it.


\subsection{Transfix [3] (P)}
You infuse your aether into a creature, disrupting its aetheric field to fill its mind with a deluge of sensory information. A single target within range 1 (15 m) must \textlf{resist(R)} or stand transfixed (no actions allowed) for 10 minutes. A \textlf{resist} attempt may be made at the end of each turn to end the effect. This is \textlf{persistent} effect and any damage ends the effect immediately. 
\subsubsection{Augment: Field [2]}
If used for 2 action points this manifestation affects all creatures within a radius of 5 m (1 combat area).
\subsubsection{Augment: Amplification [3]}
If the victim of this effect is damaged, the manifestation lingers for 1 round before ending.


\subsection{Shape earth [3]}
The ground around you has an aetheric field that can be manipulated. A cunning flow of aether allows one to twist the very earth into any desired shape. The aethermancer may touch earth or stone and then manipulate up to 10 kg of earth, or 5 kg of stone, plus 10 kg earth or 5 kg stone per point of \textlf{wit}. 


\subsection{Sonic boom [3]}
By tuning your aether to resonate with the field of the air around you, a localised over-pressured pocket of air escapes outwards with a booming crash. This \textlf{Staggers} all creatures within a chosen area if they fail a \textlf{resist(R)} check. The chosen area must be within range 1 (15 m).
\subsubsection{Augment: Thunderous blast [2]}
This power now additionally inflicts damage with normal \textlf{lethality} on victims if they failed \textlf{resist} (\textlf{critical success} for the aethermancer on \textlf{resist} increments damage severity).


\subsection{Stone sense [2]}
\label{spell:stone-sense}
Your expert knowledge of aetheric vibrations allows you to extend your hearing through a continuous stone (or earthen) surface or structure. For this purpose, a wall of stone bricks is continuous but soil and dirt do not count (they do not conduct vibrations coherently enough). This power lasts for up to 1 hour.


\subsection{Stone skin [3]}
Manipulating the aetheric field of rock, you draw stone from the ground to clad you in armour. For 1 round damage rolls against you suffer a \textlf{lethality} downgrade (to a minimum of normal). 
\subsubsection{Augment: Earth-clad [2] (P)}
This makes \textlf{Stone skin} a \textlf{persistent} effect that lasts up to 10 minutes.


\subsection{Suggestion [3]}
A creature's own aetheric field can be infused with aether to make its mind more open and malleable. Make a single-sentence suggestion of an action to the target, on a failed \textlf{Resist(W)} check they follow the suggestion willingly. This suggestion will automatically fail if it would
be harmful to the victim themself or their friends/allies. Specify 1 additional suggestion per level of \textlf{critical success} for the aethermancer on \textlf{resist}. 		
\subsubsection{Augment: Manipulation [2]}
Your mastery of the aetheric fields in brains allows you to implant beliefs or feeling into a target via suggestion. In addition suggestion no longer fails if it would be harmful to the victims friends/allies.


\subsection{Time warp [5]}
Projecting your aether towards another creature, you make its own aetheric field discordant with the flow of time. Choose a creature within range 1 (15 m) to either gain or lose an action point each turn until the effect ends (maximum duration 1 minute). Targets may \textlf{resist(R)} each turn to end the effect. Only one action point may be gained or lost each turn in this way. \textlf{critical success} for the aethermancer on \textlf{resist} means the target cannot attempt \textlf{resist} next turn.
\subsubsection{Augment: Field [2]}
If used for 2 action points this affects all creatures within a targeted radius of 5 m (1 combat area).


\subsection{Vortex [3]}
Infusing a sudden burst of aether into the air creates a resonance that roars out as a powerful vortex of wind. This reverses the flight of projectiles entering a target region within range 1 (15 m) for 1 round. 
\subsubsection{Augment: Howling winds [2]}
Your mastery of the air means that this power can be  manifested with ferocious winds that knock down all creatures in the target area if they fail a \textlf{resist(M)} check.


\subsection{Warp space [5]}
You infuse your aether around a creature, warping the aetheric field of space itself. For 1 action point you can move a target creature, which can attempt to \textlf{resist(C)} if it wants to. This moves the target a distance up to range 2 (25 m). This cannot affect targets more than 1 size category larger than the aethermancer without 1 level of \textlf{critical success} for the aethermancer on \textlf{resist} per size additional category difference.


\subsection{Weather sense [2]}
You attune to the aetheric fields of air and water, thus you can predict the weather up to 1 day + \textlf{wit} in advance.
\subsubsection{Augment: Weather touch [4]}
Your mastery of air and water allows you to nudge the weather. This lets you alter the weather, with the effect occurring after 5 - X days where X is the number of aether points spent on this manifestation. 





\chapter{Arms and Armour}
\label{chap:arms}

\begin{table}[ht!]
	\centering
	\caption{Close-combat Weapons. Note that L signifies \textlf{lethality}, A is \textlf{aim}, D is \textlf{deflect}, and P is \textlf{power}. Pen indicates the \textlf{penetration} effect and MD is \textlf{massive damage}.}
	\label{tab:weps}
	\begin{tabular}{|l|l|l|l|l|l|l|l|}
		\hline
		Name & Cost & Hands & L & A & D & P & Special\\ [0.5ex]
		\hline 
		\textbf{Daggers} & & & & & & & \\
		\hline
		Dagger & 5 s & 1 & N & - & - & - & Small, Throw 1\\
		Stiletto & 10 s & 1 & N & - & - & - & Small, Rending \\
		\hline 
		\textbf{Swords} & & & & & & & \\
		\hline
		Arming Sword & 20 s & 1 & N & - & +1 & - & \\
		Long Sword & 60 s & 2/1 & N & - & +1 & +1/- & \\
		Falchion & 40 s & 1 & N & - & - & +1 & - \\ 
		Rapier & 50 s & 1 & N & - & +1 & - & Rending \\ 
		Sword-breaker & 10 s & 1 & N & - & +1 & - & Disarm\\
		Greatsword & 75 s & 2 & N & - & +1 & - & MD\\
		Zweihander & 2 g & 2 & C & - & - & +1 & Cumbersome\\ 
		\hline 
		\textbf{Axes} & & & & & & & \\
		\hline
		Battle Axe & 25 s & 1 & N & +1 & - & - & - \\
		Bearded Axe & 40 s & 1 & N & +1 & - & - & Disarm \\
		Throwing Axe & 10 s & 1 & N & +1 & - & - & Throw 2 \\
		Great Axe & 70 s & 2 & N & +1 & - & - & MD \\
		Long Axe & 70 s & 2 & N & +1 & - & - & MD, Reach \\
		Bearded Long Axe & 90 s & 2 & N & +1 & - & - & MD, Disarm, Reach \\
		\hline
		\textbf{Pole Weapons} & & & & & & & Trip \\
		\hline
		Javelin & 5 s & 1 & N & - & - & - & Rending, Throw 2\\
		Quarter Staff & 10 c & 2 &  N & - & - & - & Disarm,Reach\\
		Spear & 10 s & 2/1 & N & - & - & -/- & Rending, Reach/- \\
		Glaive & 40 s & 2 & N & - & - & +1 & Reach \\
		Halberd & 1 g 50 s & 2 & N & - & - & - & Multi-edged, Reach\\
		Lucerne Hammer & 1 g & 2 & N & - & - & - & Multi-edged, Reach\\ 
		Ranseur & 40 s & 2 & N & - & - & - & Rending, Disarm \\ 
		Partisan & 40 s & 2 & N & - & +1 & - & Rending\\
		Pole-axe & 1 g & 2 & N & - & - & - & Multi-edged \\
		Lance & 1 g & 2 & N & - & - & - & Rending, Reach\\
		\hline
		\textbf{Blunt Weapons} & & & & & & & \\
		\hline
		Cudgel & 20 c & 1  & N & - & - & - &  \\
		Club & 2 s & 2 & N & - & - & +1 & Pen 1 \\
		War Hammer & 40 s & 1 & N & - & - & - & Pen 2 \\ 
		Great Hammer & 60 s & 2 & N & - & - & - & Pen 2, MD \\
		Maul & 70 s & 2 & C & - & - & - & Cumbersome, Pen 1 \\
		Mace & 5 s & 1 & N & - & - & - & Pen 1 \\
		\hline
		\textbf{Extended Weapons} & & & & & & & \\
		\hline
		Flail & 50 s & 1 & N & - & - & +1 & Flail Fail \\ 
		Grand Flail & 1 g & 2 & N & - & - & +2 & Flail Fail, Trip, Reach \\
		Whip & 5 s & 1 & N & - & - & -1 & Trip, Disarm, Range 1\\
		Chain & 25 s & 2 & N & - & - & - & Trip, Disarm, Reach\\  
		\hline
	\end{tabular}
\end{table}
\begin{table}[ht!]
	\centering
	\caption{Ranged Weapons. Note that L signifies \textlf{lethality} and P is \textlf{power}. Pen indicates the \textlf{penetration} effect and MD is \textlf{massive damage}.}
	\begin{tabular}{|l|l|l|l|l|l|l|l|}
		\hline
		Name & Cost & Hands & Range & Reload & L & P & Special\\ [0.5ex]
		\hline
		\textbf{Bows} & & & & & & & \\
		\hline
		%Hunting Bow & 5 s & 2 & 2 & 1 & N & +1 & - \\
		Short Bow & 20 s & 2 & 1 & - & N & - & -\\
		Long Bow & 60 s & 2 & 3 & 1 & N & - & MD \\
		Recurve Bow & 1 g & 2 & 2 & - & N & + 1 & -\\
		\hline
		\textbf{Cross Bows} & & & &  & &  & \\
		\hline
		Hand Crossbow & 50 s & 1 & 1 & - & N & - & Small \\ 
		Light Crossbow &  50 s & 2 & 2 & 2 & N & +1 & MD\\
		Repeater Crossbow & 70 s & 2 & 1 & 1 & N & - & - \\  
		Heavy Crossbow & 3 g & 2 & 3 & 3 & C & +1 & MD \\
		\hline
		\textbf{Black Powder} & & & & & & & Powder-Shot \\
		\hline
		Pistol & 1 g & 1 & 1 & 1 & N & - & Pen 1, MD \\
		Musket & 3 g & 2 & 3 & 2 & N & - & Pen 2, MD \\
		Blunderbuss & 2 g & 2 & 1 & 2 & N & - & Burst 1, Cone, MD\\
		Bomb & 50 s & 1 & 1 & - & N & - & Burst 1, Throw, Radius 0, MD \\
		\hline 
	\end{tabular}
	\label{tab:range-weps}    
\end{table}

\begin{table}[ht]
	\centering
	\caption{Basic Armour}
	\begin{tabular}{|l|l|l|l|}
		\hline
		Name  & Cost & Type & Toughness\\
		\hline
		Light Armour & & &  \\
		\hline
		Gambeson & 20 s & L & +1  \\
		\hline
		Medium Armour & & &  \\
		\hline
		Mail hauberk and gambeson & 2 g & M & +3   \\ 
		Brigandine and mail  & 4 g & M & +4  \\
		\hline
		Heavy Armour & & &  \\
		\hline
		Brigandine and plate & 6 g & H & +5 \\
		Full Plate  & 10 g & H & +6  \\
		\hline
		Barding (Horse)& 6 g & H & +4 \\
		\hline
	\end{tabular}
	\caption{Shields}
	\begin{tabular}{|l|l|l|l|}
		\hline
		Name & Cost & Deflect & Special\\   
		\hline
		Shield & 30 s & +2 & - \\
		\hline
	\end{tabular}
\end{table}





\section{Weapon Special Rules}

\subsection{Multi-Edged}
A weapon like this has many different naughty ends. Which bit is being used must be decided before rolling each attack. A halberd can \textlf{trip} with its hook, slash with the axe-blade (+ 1 \textlf{Power}) or stab with the spear point (\textlf{Rending}). The Lucerne Hammer is similar but it has \textlf{Penetration} 2 when the hammer head is used. A pole-axe has a heavy axe blade with \textlf{massive damage}, a hammer-head with \textlf{penetration} 2, and a spear tip with \textlf{rending}.

\subsection{Flail}
A flail is a one-handed weapon consisting of a short handle attached to a heavy mace head on the end of a chain. The extension of the chain grants the weapon great power for a single-handed weapon. However, the chain can get tangled or caught on an opponent. Thus, any \textlf{Critical success} \textlf{Deflects} by targets result in the wielder being forced to untangle the flail before it can be used to attack, which results in a \textlf{Moment of Weakness}. 

\subsection{Lance}
A lance gains bonus \textlf{lethality} during a mounted charge action.

\subsection{Bows}
Bows are shooting-type weapons. Firing two arrows together incurs an \textlf{edge} penalty to \textlf{Aim} but adds an extra projectile to the shot.

\subsection{Crossbows}
All crossbows are shooting weapons. Unless specially modified, these cannot fire two bolts at once.

\subsubsection{Hand Crossbow}
These diminutive weapons are designed to be easily concealed and to unloaded at close range into an unsuspecting victim. As such, they are \textlf{small} and fast firing but are still tricky to reload when used in pairs (requiring a 1 action \textlf{Reload}).

\subsection{Powder-Shot}
Black powder weapons are powerful but quite dangerous to their wielder. As such, scoring a raw $5-$ on \textlf{aim} means the weapons firing goes dangerously awry. This causes the user a single damage roll (use \textlf{power} +2 and \textlf{lethality} N). In the case of the bomb, treat it as though it hit everyone inside the same area as the character (including themselves) and resolve hits as normal.

\subsubsection{Blunderbuss}
This weapon fires a scatter of shot in a cone, hitting up to 2 targets in a single combat area. The user can also declare using it point blank within range 0, then it hits one target only but has \textlf{Burst} +1.





\section{Armour Special Rules}
\label{sec:armspec}

\subsection{Armour Weights}
Armour is listed with a weight appropriate to medium sized armour, that is, armour made for medium sized creatures to wear. Armour for smaller creatures weighs half as much and armour for Large creatures weighs twice as much, larger armour weigh six times as much as medium armour.


\subsection{Putting On and Removing Armour}
Even heroes seldom sleep in full-plate armour, so there are then times when the speed at which a character puts on or removes armour might matter. It requires four action points to put on \textlf{light armour} or two to strap on a shield. However, it takes 1 minute to put on \textlf{medium armour} but it takes 5 minutes to put on \textlf{heavy armour} and requires that the wearer has assistance in putting it on. \textlf{Medium} or \textlf{light armour} can be put on in a rush, doing so means that armour might not be precisely adjusted in order function to it's full effect, this takes only four action points for medium armour or two action point for \textlf{light armour}, but reduces the \textlf{Toughness} of the armour by 1 while it is worn in this sloppy fashion.

Removing \textlf{light armour} takes 1 minute, \textlf{medium armour} takes 2 minutes, and \textlf{heavy armour} takes 5 minutes to remove.


\subsection{Shields}
\label{sec:shields}
Shields add a + 2 bonus to the bearer's \textlf{Deflect} score. 


\section{Weapon Enhancements}
A weapon may have one quality enhancement and one other enhancements. However, weapons cannot have more than one that is labelled `Exclusive'.

\subsection{Fine}
(Quality enhancement) This grants the item + 1 \textlf{power}. This doubles the cost of the item. 

\subsection{Masterly}
(Quality enhancement) This grants the item + 1 \textlf{Power} and \textlf{aim}. This quadruples the cost of the item.

\subsection{Expanded Bolt Rack}
(Crossbow only) The weapon is fitted with expanded armatures and space for two bolts. It can thus fire two bolts with a single trigger release. This grants \textlf{Burst} + 1 on attacks made with the crossbow. Cost: Base cost x 3 + 1 g. 

\subsection{Special Payload}
(Crossbow only) The armatures and bolt track of the bow are modified to carry larger and heavier bolts. This configures the crossbow to fire specialised ammunition that can carry explosives (bomb from infusary - Section~\ref{sec:alch}), chains/ropes/nets, or poisons and other chemicals that spray in a radius of 0 around the impact point. Cost: 5 g.

\subsection{Experimental Breach}
(Black Powder Weapons only) The weapon can be loaded through a opening in the barrel, rather than down the muzzle. This reduces the \textlf{Reload} time of the weapon by one rank (reduce paired pistols to \textlf{Reload} 1). However, if the user scores a \textlf{Critical Failure} with the weapon roll 3d6 in addition to normal \textlf{Black Powder} effects. On a score of $8-$ the weapon needs 50 s worth of repairs before it works again, on $9+$ it jams requiring 2 action points be spent to clear the jam. Cost: 10 g.

\subsection{Extra Barrel}
(Black Powder Weapons only) The weapon is fitted with an extra barrel. The barrels can be fired together adding an \textlf{edge} bonus to damage rolls, or separately, in which case the gun may be fired twice before re-loading. Cost: 3 g.




\section{Armour Enhancements}
Armour may have one quality enhancement and only one other enhancement.

\subsection{Fine}
(Quality enhancement) This grants the item + 1 Toughness. This doubles the cost of the item.

\subsection{Masterly}
(Quality enhancement) This grants the item + 2 Toughness. This quadruples the cost of the item.

\subsection{Thick Padding}
A hefty layer of cloth padding soaks up the impact of musket fire. This reduces crushing lethality \textlf{Black Powder} weapons to normal \textlf{lethality} but adds an \textlf{edge} penalty on \textlf{Athletics} and \textlf{Stealth}. Cost: 50 s.

\subsection{Intricate Scale-work}
Fine layers of scaled plates distribute impact. This incurs a -1 \textlf{Burst} penalty (minimum 1 damage roll) to any attacks made against the wearer. Cost: 12 g. \textlf{Medium} or \textlf{heavy armour} only.

\subsection{Heavily Reinforced}
Nothing like just adding lots of extra steel. Adds the \textlf{Adamant} rule. Cost: 10 g. \textlf{Medium} or \textlf{heavy armour} only.

\subsection{Cunning Strappage}
Careful use of straps and weight distribution makes the armour feel so light you could dance in it (this is still not recommended). This negates a chosen \textlf{edge} penalty incurred by the armour. Cost: 6 g. \textlf{Medium} or \textlf{heavy armour} only.

\subsection{Knuckle-Blades}
Punching people is easier with steel blades on the knuckles of your gloves. This granst + 1 \textlf{power} to unarmed damage rolls. Cost: 50 s.

\subsection{Armour Spikes}
Sharpened protrusions or heavy studs set in the surface of the armour can make attacking the wearer a real hazard. This can be applied to any \textlf{Medium} or \textlf{Heavy armour} piece, enemies grappling with the wearer have an \textlf{edge} penalty to \textlf{Grapple} checks. Cost 20 s.



\chapter{Other Equipment}

\section{Currency}
There are 10 copper pieces to a silver, and 100 silver to a gold.

\section{Cost}
This cost reflects an average price that the item would be purchasable for, from a merchant. This cost can go up and down dependent on how well a player haggles and on the circumstances of the city and/or merchant; that is, if something is in shortage it costs a lot more, or if the merchant is desperate to sell, the price goes down. A player can sell an item for a price dependent on the condition of the item, a merchant will offer him half the normal cost for a used item in good condition.

\section{Equipment}
\subsection{Tools}

\subsubsection*{Bandages}
Required by a healer to use his healing skill, each use of the skill consumes a bandage.

\subsubsection*{Infuser's tools}
A complex piece of equipment that allows an infuser to purify and distil various liquids, this is required for the creation of infusions (see Section~\ref{sec:alch}).

\subsubsection*{Glass Vial}
A simple glass vial with a stopper, used to hold liquids, this is needed to hold any brew created by an infuser (see Section~\ref{sec:alch}).

\subsubsection*{Lock-pick}
A small piece of cunningly bent metal that can be used to open a lock with a Mechanical skill check opposed to the difficulty of the lock, on a failure the lock-pick breaks.

\subsubsection*{Musical Instrument}
A simple instrument for making music.

\subsubsection*{Wood Axe}
An axe for chopping up wood.

\subsubsection*{Master-Crafted Tools}
These are finely crafted versions of any type of tool, this adds an \textlf{edge} bonus to any rolls made using the tools.

\subsection{Adventuring Gear}

\subsubsection*{Backpack}
A large but otherwise conventional satchel.

\subsubsection*{Barrel}
A large wooden barrel, bound with iron rings. The barrel can hold about 50 litres.

\subsubsection*{Basket}
A simple wicker basket for carrying small loads.

\subsubsection*{Candle}
A small candle that that produces \textlf{low} light in a 6 m radius, this candle can burn for up to 3 hours.

\subsubsection*{Canvas}
A sheet of water-proof canvas. The price depends on the size of the sheet.

\subsubsection*{Chain}
A metal chain made up of heavy steel links, the chain is strong enough to support very large weights, up to 500 kg.

\subsubsection*{Crowbar}
A simple steel crowbar used for levering open doors or hinges. Using a crowbar adds an \textlf{edge} bonus to rolls made using your Might to force open any hinged container or door.

\subsubsection*{Firewood}
Wood for keeping a fire burning for one day.

\subsubsection*{Fishing Tackle}
A fishing line, a hook, sinkers and lures. This can be used to attempt to catch fish with the \textlf{Survival} skill, the difficulty of this is dependent on the speed of the water and type of the fish.

\subsubsection*{Grappling Hook}
A heavy iron hook used for scaling near vertical surfaces. This hook will hold the attached rope in place after being successfully thrown to the point you wish to climb to. The hook also makes an effective weapon.

\subsubsection*{Hammer}
A simple hammer for hitting stuff and pitching tents.

\subsubsection*{Ladder}
A length of ladder which can be climbed without needing an Athletics check.

\subsubsection*{Lantern}
A hooded lantern used for projecting \textlf{full} light directionally, up to 13 m from the bearer. This uses 0.5 kg of oil to burn for 24 hours.

\subsubsection*{Lantern oil}
Enough oil to power a standard lantern for 24 hours. Highly flammable.

\subsubsection*{Mirror}
A simple shiny surface (glass or metal) that reflects light. Useful for looking around corners.

\subsubsection*{Needle}
A sharp needle for use in sewing or stitching wounds.

\subsubsection*{Paper}
A single sheet of paper for writing on.

\subsubsection*{Pick}
A simple pick used for breaking earth or skulls.

\subsubsection*{Pole}
A length of wooden or metal pole.

\subsubsection*{Pot}
A cooking pot, about 30 cm in diameter.

\subsubsection*{Quill}
A feather quill used for writing.

\subsubsection*{Rope}
A length of rope capable of supporting the weight of three people ($\sim$250 kg).

\subsubsection*{Sack}
A large rough cloth bag.

\subsubsection*{Spade}
A digging implement (can be used to hit people as well).

\subsubsection*{Spyglass}
A telescope capable of magnifying things from up to a mile away.

\subsubsection*{Tent}
A canvas tent to keep you dry at night, this can house up to two man-sized creatures.

\subsubsection*{Tinder Box}
A small box containing fast lighting twigs, used to rapidly start a fire.

\subsubsection*{Torch}
Provides \textlf{full} light up to a 6 m radius and \textlf{low} light within 12 m. The torch can burn for 4 hours.

\subsubsection*{Trail Rations}
Generally dried meat, bread and cheese. Long lasting food-stuffs to nourish a traveller.

\subsubsection*{Water-Skin}
A hide bag used for carrying water, this carries water for one person for 5 days.

\subsection{Mounts}
\subsubsection*{Pony}
A small hard working pony, capable of carrying an full grown small creature or the child of a man sized creature. A pony has a combat movement distance of 2. Such a creature has Might -1 but can carry weights of up to 100 kg while still being to walk or trot all day with no appreciable strain. Carrying weights of 150 kg or more risks injuring the horse if done for prolonged periods. This horse is not trained for war and may panic, it has Deflect -2 and Toughness 8.

\subsubsection*{Light Horse}
A swift if none too tough horse, a Light Horse has a combat movement distance of 4. A light horse can gallop up to 2 hours a day at a speed of five times faster than a man, any longer risks injuring the horse. This horse can only move a distance of 2 per movement action while carrying a heavily-armoured rider. A light horse has Might 0 but can carry weights of up to 80 kg while still being to walk or trot all day with no appreciable strain. Carrying weights of 130 kg or more risks injuring the horse if done for prolonged periods. This horse is not trained for war and may panic, it has Deflect 0 and Toughness 9.

\subsubsection*{Cart Horse}
A heavy farm horse capable of carrying heavy loads and working all day, however, it is not capable of maintaining very rapid speeds. A Cart Horse has a combat movement distance of 2. A cart horse has Might +2 but can carry weights of up to 200 kg while still being to walk or trot all day with no appreciable strain. Carrying weights of 300 kg or more risks injuring the horse if done for prolonged periods. This horse is not trained for war and may panic, it has Deflect -2 and Toughness 11.

\subsubsection*{Heavy Horse}
A strong and reasonably quick horse. This horse can carry a man in heavy armour while moving at full speed. This horse has a combat movement distance of 3. Such a creature has Might +1 but can carry weights of up to 120 kg while still being to walk or trot all day with no appreciable strain. Carrying weights of 200 kg or more risks injuring the horse if done for prolonged periods. This horse is not trained for war and may panic, it has Deflect -2 and Toughness 10.

\subsubsection*{Destrier (War Horse)}
A heavy horse trained for battle, the destrier is by far the largest breed of horse, standing over 2 metres at the shoulder. Its war training makes it a savage weapon in its own right and it has close-combat Aim +1, Deflect 0, Toughness 10, Power +3, and combat movement range 3. A destrier's damage rolls all have Massive Damage. A destrier can carry weights of up to 200 kg while still being to walk or trot all day with no appreciable strain. Carrying weights of 300 kg or more risks injuring the horse if done for prolonged periods.

\subsubsection*{Simple Saddle}
A rough worked leather saddle, needed for sitting astride your mount. A saddle is fastened with girth buckles and has attached stirrups.

\subsubsection*{War Saddle}
A heavy saddle needed to support a fully-armoured knight, this saddle is designed also to allow the knight to wield a lance one handed.

\subsubsection*{Racing Saddle}
A saddle built for race horses, it is light weight and fine craftsmanship add 1 to the horses' combat movement distance.

\subsubsection*{Tack}
Reins, bit, bridle and other equipment necessary for riding a mount. Riding bareback increases the difficulty of all ride checks by 2.

\subsubsection*{Feed}
Food for a mount for one day.

\chapter{Equipment Tables}
\begin{table}[ht]
	\centering
	\caption{Tools}
	\begin{tabular}{|l|l|}
		\hline
		Item & Cost\\ [0.5ex]
		\hline
		Bandages (5)& 2 s\\
		Infuser's tools & 1 g\\
		Glass Vials (5)& 5s\\
		Lock-picks (5) & 10 s\\
		Musical Instrument & 50 s\\
		Wood Axe & 5 s\\		
		\hline
		Master-Crafted Tools & +3 g\\
		\hline
	\end{tabular}
\end{table}

\begin{table}[ht]
	\centering
	\caption{Adventuring Gear}
	\begin{tabular}{|l|l|l|}
		\hline
		Item & Cost & Weight\\ [0.5ex]
		\hline
		Backpack & 5s & 1 kg\\
		Barrel & 10 s & \\
		Basket & 2 s & 0.2 kg\\
		Candle & 10 c & -\\
		Canvas (per sq. m) & 1 s & 4 kg\\
		Chain, 3 m & 10 s & 10 kg\\
		Crowbar & 25 s & 5 kg\\
		Firewood, 1 day & 20 c & 5 kg\\
		Fishing Tackle & 1 s & -\\
		Grappling Hook & 10 s & 5 kg\\
		Hammer & 5 s & 4 kg\\
		Ladder, 3 m & 10 s & 10 kg\\
		Lantern & 15 s & 1 kg\\
		Lantern oil & 5 s & 0.5 kg \\
		Mirror & 5 s & -\\
		Needle & 1 s & -\\
		Paper 1 sheet & 1c & -\\
		Pick & 5 s & 8 kg\\
		Pole, 3 m & 1 s & 5 kg\\
		Pot & 10 s & 0.5 kg\\
		Quill Pen & 10 c & -\\
		Rope, 3 m & 1 s & 0.2 kg\\
		Sack & 10 c & 0.2 kg\\
		Spade & 5 s & 4 kg\\
		Spyglass & 1 g & 1 kg\\
		Tent & 20 s & 3 kg\\
		Tinder Box & 5 s & -\\
		Torch & 2 c & 0.1 kg\\
		Trail Rations 1 day & 10 c & 0.1 kg\\
		Water-skin (full) & 1 s & 3 kg\\		
		\hline
	\end{tabular}
\end{table}

\begin{table}[ht]
	\centering
	\caption{Mounts}
	\begin{tabular}{|l|l|}
		\hline
		Item & Cost\\ [0.5ex]
		\hline
		Pony & 75 s\\
		Light Horse & 1 g 20 s\\
		Cart Horse & 4 g\\
		Heavy Horse & 4 g\\
		War Horse & 10 g\\
		%		Wolf & 2 g\\
		%		Boar & 3 g\\
		\hline
		Simple Saddle & 10 s\\
		War Saddle & 50 s\\
		Racing Saddle & 50 s\\
		Tack & 20 s\\
		Feed, 1 day & 5 c\\
		\hline
	\end{tabular}
\end{table}

\begin{table}[ht]
	\centering
	\caption{Clothes}
	\begin{tabular}{|l|l|}
		\hline
		Item & Cost\\ [0.5ex]
		\hline
		Peasant & 5 c\\
		Scholar & 10 s\\
		Cleric & 5 s\\
		Cold Weather & 20 s\\
		Noble & 2 g\\
		Merchant & 50 s\\
		\hline
	\end{tabular}
\end{table}





\listoftables


\end{document}
