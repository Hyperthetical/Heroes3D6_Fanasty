%Copyright (C) 2021  Geoff Beck
%
%This document is falls under the category of free software: you can redistribute it and/or modify
%it under the terms of the GNU General Public License as published by
%the Free Software Foundation, either version 3 of the License, or
%(at your option) any later version.
%
%This program is distributed in the hope that it will be useful,
%but WITHOUT ANY WARRANTY; without even the implied warranty of
%MERCHANTABILITY or FITNESS FOR A PARTICULAR PURPOSE.  See the
%GNU General Public License for more details.
%
%See https://www.gnu.org/licenses/ for more details

\documentclass[a4paper,11pt,oneside]{book}

\usepackage{fullpage}
\usepackage{amsmath}
\usepackage{import}
\usepackage{tocbibind}
\usepackage[bookmarks=true,plainpages=false]{hyperref}
\usepackage{titletoc,titlecaps}
\usepackage{caption}

%\usepackage{helvet}
%\usepackage{sansmath}
%\usepackage{sfmath}
%\renewcommand{\familydefault}{\sfdefault}

\usepackage[T1]{fontenc}
%\usepackage{palatino,mathpazo}
\usepackage{lmodern}


\newcommand{\dicediffbase}{11}
\newcommand{\dicenoprof}{-1}
\newcommand{\dicecritlvl}{4}

\newcommand{\textlf}[1]{\textbf{\titlecap{#1}}}
\newcommand{\textlfirst}[1]{\textbf{\textit{\titlecap{#1}}}}

\title{\textbf{\huge Heroes3D6\\Fantasy Role-playing Rules Add-on}}
\author{Geoff Beck}
\date{}

\begin{document}
\maketitle
\frontmatter
\tableofcontents
\mainmatter

%\chapter{The world and the `Last City'}
%The world of Krell is a devastated wasteland. An event in the distant past commonly referred to as the `Catastrophe' unleashed vast quantities of warping magical energies that laid waste to cities and ecosystems across the planet. The cause of this disaster is not clearly remembered, however, it has left a long lasting suspicion of magic in the few survivors that cling to this wrecked world. Over time the lingering magic, called `the residual', has produced new mutant life-forms by warping the small surviving populations. The `Last City' is the only major settlement in existence, built after the Catastrophe and housing peoples of all species, it teeters constantly the brink of collapse as a result of supply shortages as well as internal tensions. What holds the city together is the shelter it provides from the residual, this being a great wall built of a patchwork of `untainted' iron. This being a metal of great value, often used as currency outside the city, as it survived the Catastrophe without being infused with dangerous residual energies and now repels these same influences.    
%
%Small settlements dot the wastelands outside the Last City, these struggle constantly against the dangers of residual, mutant wild-life, bandits, and the terrifying changekin. These last are feral people twisted by the residual energies into nightmarish monsters. Life outside the city is a constant struggle to find food and water that are untainted by the residual, as too much exposure can eventually twist one into a changekin.
%
%\section{The last city}
%The only bastion of civilisation in a blighted world, the Last City stands as monument to the endurance of the people of Krell. Surrounded by a patch-work wall of untainted iron the city is largely protected from the warping power of the residual. Entrances to the city are guarded and those too contaminated by the residual are denied entrance, especially as a common punishment for crimes is being exposed to contamination and cast out of the city. Such is the suspicion of magic and its users that all mages must be registered and are branded with forehead marking, those who refuse are cast out as criminals. The mage brand brings much suspicion upon its bearers with common folk often spitting in the street at their passing, and some being unwilling to even do business with them. Often children with magical talent are disowned by their families and turned out onto the streets. 
%
%The city is ruled over by council of exclusively human `founding families', constituting a ruling aristocracy whose power is enforced by a small soldier corps. Low-level crime is endemic in the city, as it contains many desperate people of all species, and criminal gangs act as the defacto rulers of some districts and trades. The existence of the city is precarious, as it depends strongly on food grown in surrounding settlements, which are constantly under pressure from raiding bandits, changekin, and the residual itself. Despite this, it is the most sure shelter against the terrors of the wastelands.
%
%Within the city coins are used as currency but the alternative of untainted iron is also widely accepted (it is often the only currency used in the wastelands). A coin-sized disk of this metal is exchanged for 3 gold coins, such is its value in warding off the effects of residual when travelling outside the city.   
%
%\section{The changed and the changekin}  
%The changed are a phenomenon that began after the Catastrophe. They are beings that have been warped by the residual, and exhibit anything from minor changes like scales or feathers on their skin, up to limbs or heads being a mixture of animal and human (bird claws, goats hooves, beaks, animal jaws). The changed are, however, still people and although often viewed with suspicion and prejudice they are not attacked on sight. 
%
%Changekin are somewhat like the changed but have become feral, their minds rotted away by the residual. Their mutations tend to be even more extreme, often multiple additional limbs, or large bloated bodies. Changekin are savage and attack anything on sight, except for other changekin (most of the time). Such creatures form into roaming hunting bands that seek out anything weak enough to be preyed upon, attacking travellers or small settlements if their band grows large enough. Changekin seldom wield weapons or wear armour, their minds are too damaged for such rationality.  
%
%
%\section{The residual}
%The residual is remnant warping energy left over from the Catastrophe. Exposure to it is dangerous as it causes damage to exposed tissues and can, with sufficiently high doses, twist the bodies of the exposed in monstrous forms. The residual is undetectable to normal senses except in very high concentrations, where it feels like a heated crackle in the air (these concentration levels are immediately dangerous as well). Untainted iron is commonly used to ward off the effects of the residual and skilled alchemists can produce potions brewed with it that can protect from, or cleanse a body of, the residuals effects. 
%
%Whenever exposed to a residual infused environment (or consuming tainted food/drink) a creature must make a \textlf{resolve} check against a \textlf{difficulty} set by how saturated the area is. For example low contamination is \textlf{difficulty} 8, dangerous is 11, and deadly is 14. If the creature fails they gain a Residual point (\textlf{critical failure} adds extra points). The number of points is added to the \textlf{difficulty} of subsequent checks against residual contamination. At 5 points the creature begins to feel light-headed, weak, and sweaty (has an \textlf{edge} penalty on all actions). At 10 points a creature dies. Residual points last until removed. Armour made of untainted iron grants an \textlf{edge} on all such \textlf{resolve} checks and weapons of this metal have an \textlf{edge} bonus to damage rolls against changekin. 



\chapter{The highlands of Atla}
This area of land is entirely mountainous with the major city of Stormheight occupying the tallest peak in the region. Smaller towns and villages dot the mountain valleys around the region. The mountainous terrain is rich in aether, with great deposits of crystals being mined in many of taller peaks. This richness of aether has lead to Stormheight being a seat of technological advancement, with the study of aether being an integral part of its powerful religious structures.  

\section{Religion and society}
Stormheight itself is home to the Synod of Inquiry, an organisation dedicated to the reverence of the natural world as a matchless system of harmony (the religion itself is known as the Harmony). The synod is extremely influential, being the ruling body of the city in all but name. The heart of this religious body is a grand campus, where all study of the natural world and its aether is conducted in a suitably reverent manner. Needless to say, scholarship and aethermancy outside of this body is illegal. The rationale being that unrestricted inquiry might lead to disastrous abuses of nature. Under the Harmony, the duty of thinking beings is to match themselves to the harmony of nature, part of this is by achieving ``inner harmony'', the other being in harmony with society. Everyone in Stormheight is allocated a role and occupation according to the assessment of the synod. Those who cannot find harmony with their allocated position are confined to asylums and studied, the better to find a cure for such disharmonius afflictions. The influence of the synod is significantly lesser in the countryside, where folk festivals tend to match the religious calender of the Harmony, but the people aren't really invested in the wider system. Their only encounter with the synod being the yearly arrival of tax men, along with assessors who decide on the future occupations of the village youth.





\chapter{Aether}
Aether is the term used to describe the natural energy that flows throughout the world. All creatures are attuned to aether to some extent, but study and practice can allow one to directly manipulate their own aether, as well as draw it in from their environment. Aether manipulation can create a variety of effects, either aligned to the natural elements or altering the aetheric fields of living creatures.

\section{Aether points}
Every character has a maximum of 1 aether point at any one time. Any character can spend this on the \textlf{aether surge} ability. Holding additional points requires a device called an aether capacitor (see Section~\ref{sec:capacitor}).

\subsection{Aether surge}
The character concentrates their aether to grant themselves a burst of power. At a cost of 1 aether point they may gain an \textlf{edge} bonus on a roll of their choice.

\subsection{Recharging aether}
A one hour rest restores 1 aether point to any character that had none remaining. A hearty meal also restores up to 1 aether point. A full night's sleep restores up to 3 aether points. A character cannot gain aether if they would exceed their allowed maximum.

\subsection{Aether crystals}
Sometimes aether crystallises of its own accord. These crystals can be consumed by anyone to gain aether points. The number gained depends on the strength of the crystal. In addition to this, a character consuming a crystal experiences a rush of life and enhanced sensations for 10 minutes.
\subsubsection{Crystal jitters}
The consumption of aether crystals is addictive, withdrawal manifesting in the form of uncontrolled shaking motions throughout the victim's body. Each time a character uses a crystal they gain crystal points equal to the gained aether. Crystal points decay at a rate of 1 per week. A character must make a \textlf{resolve} check each day vs 10 + crystal points. If they fail, they are overcome with a need to acquire and use more crystals. This means they have an \textlf{edge} penalty on all checks until they consume a crystal.   




\chapter{Character creation}
These are additional rules providing the options for character species and additional backgrounds in this fantasy setting.

\section{Character species}

\subsection{Humans}
Humans are a dynamic species: adaptive, resilient and, as a group, will always stand firm even against the most difficult odds.

Humans are the most prevalent species in the highlands of Atla, making up the majority of the rural population. Stormheight, however, is considerably more diverse, with lots of peoples of other species both resident and passing through.

The average human lives for up to 80 years, 30 years being considered a mature adult, though humans are considered adults at the age of 16 in most societies. Though they lack the patience and long practice of the Elves they make up for it with a natural ability to adapt quickly to new situations or tasks. 

The elves' popular view of humans is that they are flighty and lacking in focus. Mus-folk stereotype humans as grim and unscrupulous. Trogs stereotype humankind as deceitful and conniving.  

\subsubsection*{Flexibility}
Humans are extremely adaptable and may select an extra associated skill for their background.


\subsection{Elves}
Though elves somewhat resemble humans at first glance, there is a world of difference on closer examination. Firstly, elves are incapable of digesting meat. They are vegetarian with a preferred diet of sweet, sugary things. Second, their manner of reproduction is completely different, elven eggs gestate externally in a similar manner to those of fish. Additionally elven blood is clear, as it contains no iron, and their hearts are far smaller than that of a man, as capillary action plays a major role in their circulatory systems.

Elves have high cheek bones and very narrow, angular features. Their skin varies in shade between green and light brown with a similar texture to smooth tree bark. Their eyes range in colour from yellow, through green and blue to white, the pupils of which are narrow and cat-like. Elves are short by human standards, and are also of a slighter build. All of them posses the curious ability to alter their hair colour at will. However, if they are not careful it will change on its own to reflect their mood. 

Elves' lives and metabolisms move far more slowly than those of humans and they are children till the age of about 40 years. They grow far more slowly than other species and tend only to reach their full growth by 60 years old. Additionally, they only need to eat or sleep once every two or three days. A small population of Elves exists within the city of Stormheight, but there are several predominantly elven villages around some of the more remote peaks. 

Humans tend to stereotype elves as vain, vapid, and hedonistic. Mus-folk and Trogs popularly assert elves to be crazy and lacking in any kind of sense. 

\subsubsection*{Perfectionists}
Elves live long lives and thus have time to pursue their skills to a point few others can reach. As such, elves may choose one additional skill proficiency.


\subsection{Mus-folk}
The mus-folk (or `Musmus' in their own language) resemble large rats. They reach a maximum height/length of 3 - 4 ft and tend to inhabit the margins of human settlements where they are regarded as a mixture of pest and second-class citizen (rude terms for them include `squeakers' and 'ratties'). Their own language consists of chirps and squeaks rather than words. However, they are fully capable of producing human and elvish speech. When they do, it comes out in a rapid torrent of words infused with poor grammar and often features repetition when they are excited. Mus-folk prefer underground dwellings and make burrows throughout the highland peaks, with a significant population in Stormheight itself. Their prolific digging means they gravitate towards mining for minerals and aether. They are often stereotyped as light-fingered and mischievous by other species.  

\subsubsection*{Sneak \& squeak}
Mus-folk are naturally stealthy. They may always add \textlf{stealth} to their listed of associated background skills. 

\subsubsection*{Pack rats}
Like their rat relatives, mus-folk work best in groups. Mus-folk get an \textlf{edge} bonus on \textlf{aim} when targeting the same foe as an ally. 


\subsection{Trogs}
Trogs are large (usually 7 ft tall) and bulky, but their most striking feature is their large eyes. Trog facial features tend to be heavy, with wide square jaws and large beak-like noses. Trogs have tough greyish, slightly scaly skin and are in general quite difficult to kill or injure. Traditionally, trog groups are familial and organised around a mother trog, a very large and formidable female (female trogs tend to be larger and stronger than males), with a harem of males who care for the young trogs. Plenty of young trogs find this authoritarian system stifling, so they form their own more egalitarian bands which often migrate to settlements like Stormheight. This form of living is scorned and belittled by their elders. Trogs typically prefer dark places, due to their sensitive eyes. This means they tend to live in close proximity to the Mus-folk, often forming a symbiotic system where Trogs provide the muscle and Mus-folk, the cunning.

Other species have a stereotyped view of Trogs as stupid and violent (Mus-folk also view them as useful).  

\subsubsection*{Tough as nails}
Trogs are extremely durable and thus have + 1 \textlf{endurance}.

\subsubsection*{Dark-dwellers}
Trogs have excellent eyesight in the dark as well as sensitive noses. They have an \textlf{edge} bonus on \textlf{awareness} checks in the dark.


%\subsection{The changed}
%These people are the long-term effects of the Catastrophe, sentient creatures whose forms have been altered by the warping magic of the residual. Unlike the feral changekin, their minds are intact and they continue to try and live within the society of their original people. The changed are regarded with suspicion and prejudice by all of the unchanged and are often openly discriminated against. Some of the changed form their own settlements to avoid this mistreatment. 
%
%Changed can be created by extreme residual exposure or born, to either changed parents, or those unchanged who have suffered a slow build up of residual exposure. As such changed can be born to any family, even within the walls of the last city. They are commonly cast out of their families to fend for themselves.
%
%Changed vary wildly in appearance, as they are formed from a base species: human, trog, elf, or mus which has then been warped with strange features. Some possibilities can be elongated limbs, clawed hands, feathers or scales on their skin, fangs instead of teeth, horned heads, animal heads (or elements of this like a beak or snout), hoofed or clawed feet, boney spines jutting from the body, a tentacle instead of a normal limb, extra eyes, a tail, or reverse-jointed legs (like a bird).
%
%\subsubsection*{Change-fire}
%The changed are infused with the warping energies of the residual. They can channel this energy to emit a gout of roiling, iridescent fire from a limb or orifice. This is a 1 action point ability that can be used on two adjacent targets (within 6 m) and deals a hit with normal \textlf{lethality} while also inflicting a Residual point if it causes damage. 
%
%\subsubsection*{Infused}
%The changed have an \textlf{edge} bonus on \textlf{resolve} checks against Residual contamination.


\section{Starting equipment}
A new character may choose one piece of armour and up to three weapons from Table~\ref{tab:start-gear}. All characters get a set of clothes.
\begin{table}[ht!]
	\centering
	\caption{New Adventurers Starting Gear.}
	\label{tab:start-gear}
	\begin{tabular}{|l|l|l|l|l|}
		\hline
		Name & Power & Hands &  Lethality & Notes\\
		\hline
		Rusty sword & - & 1 & N & -\\
		Hunting knife & - & 1 & N & Small\\
		Notched axe & +1 & 2 & N & -\\
		Worn crossbow & - & 2 & N & MD, Reload 2, Range 2\\
		Corroded pike & - & 2 & N & Reach, Rending\\
		Pitted hammer & - & 2/1 & N & Penetration 1/- \\
		Aged spear & - & 2/1 & N & Reach/Throw 1\\
		Creaky short bow & - & 2 & N & Range 1\\
		Old hunting bow & +1 & 2 & N & Range 2, Reload 1\\
		Scarred great sword & +1 & 2 & N & -\\
		Shabby pistol & +1 & 2 & N & Reload 1, Range 1 \\
		Ramshackle musket & +1 & 2 & N & Range 2, Reload 2, Penetration 1 \\
		Dented shield & - & 1 & - & Deflect +1 \\
		\hline
	\end{tabular}
	\begin{tabular}{|l|l|l|l|}	
		\hline
		Name & Toughness & Type & Notes\\
		\hline
		Battered breastplate and mail & +3 & M & - \\
		Rusted mail hauberk & +2 & M & - \\
		Tattered gambeson & +1 & L & - \\
		Travelling clothes & - & - & - \\
		\hline
	\end{tabular}
\end{table}


\section{Additional backgrounds}

\subsection{Student of the synod}
The character has spent time studying aether and its powers under the auspices of synod in Stormheight. Associated skills: Aethermancy, Mechanical, Aethertech, Infusary, History, Religion, Plants, Animals.

\subsection{Savant}
The character has learned the mysteries of aethermancy on their own. Such individuals often hide their talents from the synod for fear of being forced into conformity or having to leave their home village. Associated skills: Aethermancy, Infusary, Aethertech, Survival, Persuade, Decieve. 

\subsection{Exile}
The character has been exiled by the synod for repeated defiance of the Harmony's dictates. They now skulk on the margins of society, keeping away from the eyes of law enforcement in the far-flung rural districts of highlands. Some such individuals find sanctuary among the Mus-folk, who have little regard for the Harmony or synod, others turn to less than legal ways of making a living in the aethertech black-market. Associated skills: Stealth, Disguise, Deceive, Survival, Slight of hand, Aethertech, Awareness.   


\section{Additional convictions}

%Religious conviction for the setting
\subsection{The Harmony of all}
The character is firmly convinced of the teachings of the Harmony. They live their life in pursuit of harmony within themselves and with society as a whole. They tend to regard the synod and its laws as an essential ingredient in achieving harmony with the natural world.

\subsection{Free spirit}
The character detests the enforced conformity that the Harmony brings with it. Having these views publicly known can be dangerous. 

\subsection{Aethermancy unlimited}
Aethermancy and aethertech are forces for good, their exploration brings nothing but enlightenment. Some radical individuals might also hold that it should be free from the synod's interference. 

\subsection{Aether exploited}
Aethermancy and aethertech go too far in their rampant consumption of the world's aether. One cannot achieve the claimed Harmony while mercilessly extracting resources.




\chapter{Perks and proficiencies}

\section{General proficiencies}
These perks do \textbf{not} occupy equipment slots for passive or active \textlf{perk}, their effect is always active.

\subsection{Aetheric learning (3)}
(Requires \textlf{skill proficiency: aethermancy}) This may only be chosen at character creation. This grants the character an aether capacitor that can hold 1 ether point (see Section~\ref{sec:capacitor}). Additionally, they may choose two aether manifestations from the following list: Aetheric blast~\ref{spell:sorc-blast}, Bend light~\ref{spell:bend-light}, Resonant motion~\ref{spell:min-tele}, Overgrowth~\ref{spell:overgrowth}, Ignite~\ref{spell:ignite}, Illuminate~\ref{spell:illuminate}, Orb of aether~\ref{spell:orb-light}, Aetheric armour~\ref{spell:arcane-armour}, Blinding ray~\ref{spell:blind-ray}, Flare~\ref{spell:flare}, and Stone sense~\ref{spell:stone-sense}.

\subsection{Weapon proficiency: X (2)}
In this setting, X can be drawn from: unarmed, swords \& daggers, axes, blunt weapons, pole weapons, extended weapons, bows, crossbows, and black-powder weapons. A character who is not proficient with his equipped weapon counts his \textlf{aim} and \textlf{power} as $\dicenoprof$.

%\subsection{Aetheric reservoir (3)}
%(Requires \textlf{skill proficiency: aethermancy}) The character increases their maximum aether limit by 1 point.



\section{Active perks}
These open up new actions that a character can make and they must occupy an equipment slot for active \textlf{perks} to be usable.

%\subsection{Channelled casting (2)}
%(Requires \textlf{Arcane learning}) The character can cast a spell that normally costs 1 action point for 2 instead. If they do so they gain an \textlf{edge} on opposed checks associated with the spell. If the spell inflicts a damage roll then it gains \textlf{Burst} + 1.

\subsection{Hawk Talon (4)}
(Requires \textlf{Weapon proficiency: bows}). Allows the character to launch a pair of arrows from a bow with a single shot. At a cost of 1 action point, the character's next bow attack has \textlf{Burst} +1.

%\subsection{Rhythm (5)}
%(Requires \textlf{Archery}). Breathing and rhythm are vital to the art of the archer. If the character's attacks made with a bow cause damage, he may make an extra shooting attack this round. This bonus may only be claimed once per turn.

\subsection{Heart-seeker (3)}
(Requires \textlf{Weapon proficiency: bows}). A skilled bowman knows just where to place their shots for maximum penetration. An attack with a bow can be declared as a Heart-seeker, in which case it costs 1 extra action point but gains \textlf{Rending}.

\subsection{Main Gauche (3)}
There is an art to pairing weapons for fighting with both hands, this character has mastered it. When dual-wielding, if pairing a \textlf{Small} weapon and one that isn't, the character may elect not to attack with the \textlf{small} weapon. If they do so, they gain an \textlf{edge} bonus to \textlf{deflect} against one attack within a round.



\section{Passive}
These passively enhance the character and must occupy an equipment slot for passive \textlf{perks} to make their benefit usable.

\subsection{Aetheric mastery (6)}
(Requires \textlf{Skill proficiency: aethermancy}) The character can equip 2 aether manifestations per active perk slot.

\subsection{Aetheric savant (5)}
The character can spend an aether point to gain the benefits of any \textlf{perk} for 1 round.

\subsection{Reach of the aether (3)}
(Requires \textlf{Skill proficiency: aethermancy}) Aetheric manifestations have + 1 range.

\subsection{Hammer Time (4)}
\textlf{Critical hits} from blunt weapons make the victim \textlf{vulnerable} to the next damaging hit.

\subsection{Lethal Thrust (4)}
The character delivers their killing blows with the point of a blade. While wielding a sword, the character's attacks gain \textlf{Rending} against victims with at least one face-up wound card.

\subsection{Momentum (4)}
When wielding cleaving weapons, like axes or glaives, the character's attacks gain \textlf{Heavy Weapon} +1 if their target failed a \textlf{Deflect} check against their attacks last round. This bonus stacks and lasts till an attack is \textlf{Deflected} or combat ends.

\subsection{Multi-tasker (4)}
(Requires \textlf{Skill proficiency: aethermancy}) The character has learned to concentrate on many things at once. This allows them to maintain two \textlf{persistent} effects at the same time.

\subsection{Reloading drill (4)}
The character is well practised at rapidly preparing crossbows to fire. This reduces crossbow \textlf{reload} action costs by 1.

\subsection{Rifle drill (4)}
Black-powder weapons \textlf{reload} action costs are reduced by 1 for the character. 

\subsection{Shield bash (3)}
The character may use a shield to \textlf{shove} (add the \textlf{deflect} bonus to such checks). 

%\subsection{Slippery Mind (3)}
%This allows the character to make Resist checks against spells using their \textlf{Cunning} rather than \textlf{Resolve}.

\subsection{Staff Mastery (4)}
The character may use a quarter staff (or other pole weapon) as though it was a dual-ended weapon, gaining an extra damage roll but losing the \textlf{reach} property when used in this way. No \textlf{dual-wielding} penalties apply when using this \textlf{perk}.

\subsection{Weapon Specialisation (6)}
(Requires \textlf{Weapon proficiency: X}) This grants an \textlf{edge} bonus on \textlf{aim} checks with weapons of type X.

\subsection{Witch hunter (3)}
The character gains an \textlf{edge} bonus to \textlf{resist (R)} versus aether manifestations.

\subsubsection{Upgrade: Burn the witch (3)}
(Requires \textlf{Witch hunter}) Gain an \textlf{edge} bonus to damage rolls against aethermancers, but only with flaming weapons.
 





\chapter{Character Skills}

\section{Aethermancy (Wit)}
This is the skill invoked to produce aether manifestations. More details can be found in Chapter~\ref{chap:magic}.

\section{Aethertech (Cunning)}
This represents a character's knowledge of aether-based technology and other sophisticated devices. A character does not need to be an \textlf{aethermancer} to use this skill. Using this skill a character can
\begin{itemize}
	\item Operate aethertech devices (non-proficient users have an \textlf{edge} penalty)
	\begin{itemize}
		\item \textlf{critical failure} means the device fails catastrophically
	\end{itemize}
	\item Identify functions of aethertech devices (non-proficient users have an \textlf{edge} penalty)
	\begin{itemize}
		\item Success means the functions are correctly identified
		\item \textlf{critical success} gives the character \textlf{edge} on subsequent checks with the device
		\item Failure means the character is unsure about the device
		\item \textlf{critical failure} means the character is confident but wrong about the device
	\end{itemize}
\end{itemize}

\subsection{Weapon tinkering}
This is an additional function of the aethertech skill, it allows characters to modify equipment with aethertech enhancements. The required materials are aether crystals and precious metals (to conduct aether), the cost of which are listed with each enhancement. A single piece of equipment can only ever have one enhancement. These enhancements cost double the material cost from professional aethertechs.

\subsubsection{Arcing (15)}
The weapon carries complicated metal architecture that enhances aether emitted by the bearer. Aether manifestations of the wielder may add one extra target. Material cost: 17 g.

\subsubsection{Aggression (10)}
The weapon feels alive in your hand as aether thrums through inlaid metal veins. The weapon grants + 1 \textlf{Aim} on its first attack against a given target. Material cost: 3 g.

\subsubsection{Searing (11)}
The weapon unleashes searing aether into existing wounds on the target. \textlf{Bleeding} targets are \textlf{Vulnerable} for 1 round if damaged by the weapon. Material cost: 5 g.

\subsubsection{Butchery (13)}
The weapon is imbued with aether circuitry that vastly increases the power of its impacts. The weapon has \textlf{Heavy Weapon} +1. Material cost: 12 g.

\subsubsection{Culling (13)}
The weapon unleashes a burst of aether when it strikes weakened targets. Attacks made against targets which are \textlf{Crippled}, \textlf{Knocked Down}, or \textlf{Bleeding} benefit from a \textlf{lethality} upgrade. Material cost: 12 g.

\subsubsection{Deflection (12)}
The aether circuits in this weapon create a shimmering field around the weapon that can deflect incoming attacks. This allows the weapon's wielder to make a counter attack if they score a \textlf{Critical Deflect}. Cost 7 g.

\subsubsection{Devastation (14)}
The weapon emits blasts of energy whenever it strikes. The weapon has \textlf{Burst} +1. Material cost: 15 g.

\subsubsection{Executioner (15)}
The weapon has auxiliary systems that kick-in when it inflicts a telling strike to increase its power even further. When the weapon inflicts a \textlf{Critical Hit} the wielder is granted an extra damage roll. Material cost: 17 g.

\subsubsection{Finesse (11)}
The weapon is enhanced with guidance mechanisms to keep the wielder's aim true. \textlf{Critical Failures} made with the weapon may be re-rolled. Material cost: 5 g.

\subsubsection{Impaling (14)}
The weapons stabbing or cutting edges are augmented with active aether blades that bore into the target with a well-placed strike. The weapon gains \textlf{Rending}. Material cost: 15 g.

\subsubsection{Malevolent (12)}
Aether circuits zap the nervous system of its victims. The weapon inflicts an \textlf{edge} penalty to the next \textlf{Deflect} check made by its victims. Material cost: 7 g.

\subsubsection{Masterful (16)}
The weapon generates a resonance with aether manifestations produced by the wearer. The weapon applies an \textlf{edge} penalty to \textlf{Resist} attempts made against the wielder's aetheric manifestations. Material cost: 20 g.

\subsubsection{Penetrating (11)}
The weapon's edge uses aetheric fields to slice through even the thickest armour. The weapon gains \textlf{Penetration} + 1. Material cost: 5 g.

\subsubsection{Thirsting (13)}
Aetheric circuits draw energy out of targets wounded by this weapon. When the weapon causes at least 1 \textlf{endurance} damage to a victim, the wielder regains 1 missing \textlf{endurance} or aether point. Material cost: 15 g. 

\subsubsection{Thunderous (10)}
Powerful blows from this weapon unleash a booming aetheric disturbance. \textlf{Critical Hits} from this weapon knock the victim down. Material cost: 3 g.

\subsubsection{Vengeful (12)}
The aetheric circuits in the weapon register the signature of targets who have damaged the wielder. The weapons gains an \textlf{edge} bonus to \textlf{aim} and damage rolls against a foe that has damaged you within the last round of combat. Material cost: 7 g. 

\subsubsection{Vorpal (16)}
The weapon is imbued with aetheric fields that disrupt the target, making it far more lethal. Upgrades \textlf{Lethality} of the weapon on \textlf{critical hits}. Material cost: 20 g.


\subsection{Armour tinkering}
This is an additional function of the aethertech skill, it allows characters to modify equipment with aethertech enhancements. The required materials are aether crystals and precious metals (to conduct aether), the cost of which are listed with each enhancement. A single piece of equipment can only ever have one enhancement. The difficulty of the enhancement is given in brackets after the name. These enhancements cost double the material cost from professional aethertechs.

\subsubsection{Resilient (11)}
Aetheric circuitry runs through the armour, disrupting aetheric manifestations. Grants the wearer + 1 to \textlf{Resist} checks. Material cost: 5 g.

\subsubsection{Feather-light (12)}
The aetheric field of the armour makes its wearer lighter. This grants its wearer +1 range to their normal movement. Material cost: 7 g.

\subsubsection{Unflinching (15)}
An aetheric field generator softens blows dealt to the armour. This armour now negates the first \textlf{endurance} point lost each turn. Material cost: 17 g.

\subsubsection{Evasion (11)}
A flickering aetheric field surrounds the wearer, making it hard to determine their location. The armour grants + 1 \textlf{Deflect}. Material cost: 5 g.

\subsubsection{Mirrored (12)}
This can only be used upon a shield, it uses aether ciruits to make the shield reflect aetheric manifestations. While using the shield, a wielder may use their \textlf{Deflect} when making \textlf{Resist} attempts against aetheric manifestations. Material cost: 7 g.



\section{Infusary (Cunning)}
\textlf{Infusary} is the craft of creating aether-enhanced liquids that can have powerful effects. 

An infusion needs a container to hold it as well as a aether crystals and a signature ingredient to shape the purpose of the aether. To create infusions a character needs set of glass vessels and equipment for measuring, grinding and heating ingredients (infuser's tools) that can be purchased from an aethertech for 50 silver.
Table~\ref{tab:alch} displays some suggested infusions.

\begin{table}[ht!]
\caption{Aether infusions. D is the \textlf{difficulty}, time is how long the effects last and is given in minutes. The cost reflects that of the ingredients for a single draught of potion and the bottle to hold it, double this for the price charged by most infusers.}
\begin{tabular}{|l|l|l|l|l|l|}
\hline
Potion & D & Effects & Time & Ingredients & Cost \\
\hline
Might &  13 & +1 Might & 10 & Troll teeth & 25 s \\
Cunning & 13 & +1 Cunning & 10 & Fresh mus-folk whiskers & 25 s \\
Aethermancy & 12 &\textlf{edge} bonus - aethermancy & 10 & Sapphires & 30 s\\
Iron-flesh & 8 & +1 Toughness & 10 & Fine granite power & 10 s \\
Fire-blood & 11 & Enrage & - & Trog blood & 20 s \\
Hawk-eye & 12 & \textlf{edge} Awareness/Aim & 20 & Eagle feathers & 30 s \\
Troll-blood & 15 & Regeneration & 5 & Troll blood & 1 g \\
Warding & 10 & \textlf{edge} on \textlf{resist} checks & 10 & Powdered silver & 30 s \\ 
Invigoration & 11 & Restore 1 \textlf{endurance} & - & Mend-well root & 20 s \\
Restoration & 10 & Cure 1 Condition & - & Common herbs & 30 s \\
Peace & 14 & Cure all Conditions & - & Nightshade,  & 60 s \\
 & & & & Mend-well leaves & \\
Competence & 14 & \textlf{edge} bonus on a chosen skill & 60 & Gold, Silver & 2 g \\
Giant-blood & 16 & Become Large Creature  & 10 & Giant Blood & 2 g \\
 & & (+1 Might) & & & \\
Camoflage & 11 & \textlf{Stealth edge} bonus & 20 & Nightshade, Ivy root & 30 s \\
Invisibility & 15 & Invisibility & 5 & Distilled aether & 5 g \\
Haste & 15 & +1 action point per round & 5 & Adrenaline & 5 g \\ 
Explosive & 12 & radius 1, Burst 1, Power +1 & - & Acid, coal, salt-peter & 30 s \\
\hline
\end{tabular}
\label{tab:alch}
\end{table}



%\section{Metal Smith (Cunning or Might)}
%A Metal Smith is capable of crafting metal tools or suits of armour from the Basic Armour Table in the Arms and Armour Chapter. See the Crafting Armour Table \ref{tab:craft-armour}. Making armour \textlf{Finely-crafted} increases the \textlf{difficulty} by 3, \textlf{Master-crafted} by 6 (but requires the \textlf{master perk} for this skill). These quality increases can also be achieved with 1 or 2 levels of \textlf{critical} success respectively with the same \textlf{perk} restrictions. Double the time requirements for making armour and it takes an extra 8 hours of work, over and above the normal time requirements, to make \textlf{Heavy}-type armour.
%
%\begin{table}[!ht]
%  \centering
%  \caption{Crafting Armour}
%  \label{tab:craft-armour}
%  \begin{tabular}{|l|l|l|l|}
%    \hline
%    Name & Type & Materials/Cost & Difficulty\\ [0.5ex]
%    Mail hauberk and & M & Steel, Cloth, Hides/1 g 50 s & 10\\
%    gambeson & & & \\
%    Brigandine and mail & M & Steel, Hides/3 g & 11 \\  
%    \hline
%    Brigandine and plate & H & Steel, Hides/4 g & 12\\
%    Full plate & H & Hides, Steel/10 g & 15\\
%    \hline
%    Barding (for Horses) & H & Hides, Steel/7 g & 11\\
%    \hline
%    Shield & H & Wooden Planks, Leather, Iron Bands/30 s & 12\\%    Tower Shield & H & Steel, Leather, Straps/1 g 50 s & 15\\
%    \hline
%    \end{tabular}
%\end{table}

%
%
%\section{Weapon Smith (Cunning or Might)}
%A Weapon Smith can craft their own fine weaponry to use or sell. See the Crafting Weapons Table \ref{tab:craft-weps}. Making a weapon \textlf{finely-crafted} increases the \textlf{difficulty} by 3, \textlf{Master-crafted} by 6 (but requires the \textlf{master perk} for this skill). These quality increases can also be achieved with 1 or 2 levels of \textlf{critical} success respectively with the same \textlf{perk} restrictions.
%\begin{table}[!ht]
%  \centering
%  \caption{Crafting Weapons}
%  \label{tab:craft-weps}
%  \begin{tabular}{|l|l|l|}
%    \hline
%    Name & Materials/Cost & Difficulty\\ [0.5ex]
%    \hline
%    Arming Sword & Steel, Leather/10 s & 8\\ 
%    Dagger & Steel, Leather/2 s & 8\\
%    Long Sword & Steel, Leather/30 s & 10\\
%    Horse Sword & Lots of Steel, Leather/1 g & 14\\
%    Great Sword & Lots of Steel, Leather/35 s & 12\\
%    Rapier & Steel, Leather/15 s & 12\\
%    Sword-Breaker & Steel, Leather/10 s & 10 \\
%    \hline
%    Battle Axe & Steel, Leather/10 s & 9\\ 
%    Bearded Axe & Steel, Leather/10 s & 10\\ 
%    Throwing Axe & Steel, Leather/5 s & 11\\ 
%    Great Axe & Steel, Leather/40 s & 10\\ 
%    Long Axe & Steel, Pole, Leather/40 s & 11\\ 
%    Bearded Long Axe & Steel, Pole, Leather/40 s & 11\\ 
%    \hline
%    Javelin & Short Pole, Steel/2 s & 9\\ 
%    Quarter Staff & Wood/- & 7\\
%    Spear & 3 m Pole, Steel/6 s & 7\\
%    Pike & 4 m Pole, Steel/10 s & 10\\
%    Glaive & 2 m Pole, Steel/25 s & 12\\
%    Halberd & 2 m Pole, Steel/75 s & 14\\
%    Lucerne Hammer & 2 m Pole, Steel/70 s & 14\\
%    Ranseur & 2 m Pole, Steel/20 s & 11\\
%    Partisan & 2 m Pole, Steel/30 s & 11\\
%    Pole-axe & 2 m Pole, Steel/70 s & 14\\
%    Lance & Heavy Pole, Steel/40 s & 11\\
%    \hline
%    Cudgel & Wood/50 c & 6\\
%    Club & Wood/2 s & 7\\
%    War Hammer & Handle, Steel, Leather/20 s & 12 \\
%    Great Hammer & Pole Handle, Steel, Leather/30 s & 13 \\
%    Maul & Pole Handle, Steel, Leather/40 s & 13 \\
%    Mace & Handle, Steel, Wood, Leather/2 s & 7 \\
%    \hline
%    Whip & Hides/10 s & 9\\
%    Chain & Steel/15 s & 10\\
%    Flail & Steel/25 s & 12\\   
%    Grand Flail & Steel/70 s & 13\\ 
%    \hline
%    Short Bow & Wood, Bow String/10 s & 12\\
%    Long Bow & Yew Wood, Bow String/60 s & 14\\
%    Recurve bow & Wood, horn, Bow String/1g 50s & 16 \\
%    \hline
%    Hand Crossbow & Wood/40 s & 13 \\ 
%    Light Crossbow & Wood, Steel/60 s & 11\\
%    Repeater Crossbow & Wood, Steel/60 & 12 \\
%    Heavy Crossbow & Wood, Steel/4 g & 14\\
%    \hline
%    \end{tabular}
%\end{table}
%
%
%
%\section{Tailor (Cunning)}
%With appropriate tools a \textlf{Tailor} can craft clothing or any other cloth products. The \textlf{difficulty} is based on how fine the clothes should be. Wizards robes or other magical garments can also be enchanted, see Section~\ref{sec:enchant}. The design and exact look of the clothing is up to the tailor themself, the ingredients only determine the quality. See the Crafting Cloth Table \ref{tab:craft-clothes}. In general, clothes provide no bonus to defence (notable exception being a Gambeson) and the level of success affects how fine and fancy they appear.
%
%\begin{table}[!ht]
%	\centering
%	\label{tab:craft-clothes}
%	\caption{Crafting Cloth}
%	\begin{tabular}{|l|l|l|}
%		\hline
%		Name & Materials/Cost & Difficulty\\ [0.5ex]
%		\hline
%		Rough Garments & Rough Wool, Course thread/5 c & 8\\
%		Gamebson & Thick linen, Course Thread/14 s & 8 \\
%		Course Robe & Rough Wool, Course Thread/10 c & 9\\
%		Neat Garments & Smooth Wool or Linen, Course Thread/20 c & 10\\
%		Fine Garments & Fine Wool, Embroidered Thread/10 s & 12\\
%		Silk Robe & Silk, Silver Thread/50 s & 14\\
%		Rich Garments & Silk, Fine Wool, Gold Thread/1 g & 14\\
%		Wizard Lord Robe & Silk, Dragon Skin, Arcane Diamonds/50 g & 17\\
%		\hline
%	\end{tabular}
%\end{table}


%\section{Enchanting (Cunning)}
%\label{sec:enchant}
%This allows a character to place magical enchantments upon items. This typically requires magical ingredients and cannot be employed without the \textlf{Arcane learning perk}. An item cannot carry both enchantments and Runes.
%The cost of an enchanted item (from an NPC enchanter) is equal to five times the difficulty of the enchantment (in gold pieces). The cost of an enchanted item and the material costs (for self creation) are halved for \textlf{Small} items. \textlf{Critical success} on the skill use also reduces the cost of materials by half (when doing your own enchanting). 
%
%\subsection{Materials}
%The materials used in creating enchantments are precious metals and gems, which are consumed in the process of enchanting. The value of the materials needed by each enchantment are listed in their descriptions. The exact nature of the materials is unimportant and the price given is a guideline average price (essentially the number of gold coins that could be used).
%
%\subsection{Prefix and Suffix}
%An item may only be enchanted with one of each type of enchantment (one \textlf{Prefix} and one \textlf{Suffix}). Names of magic items are created through the formula \textlf{Prefix} + item type + \textlf{Suffix}. For instance one can have an Executioner's Great sword of Aggression or a Vorpal Dagger of Butchery. Both of which are as intimidating as their names suggest.
%
%\subsection{Bound Spells}
%A \textlf{Bound Spell} enchantment is of the \textlf{Suffix}-type. Items can be imbued with a spell, this can be of the activated or triggered type. Activated \textlf{bound spells} can be used for 1 action point and may only be cast X times before the enchantment dissipates (X is dictated by the cost of the item). This kind of \textlf{Bound spell} can only be cast by a character who has the \textlf{perk} to cast spells of the appropriate school. The material cost is 1 g per charge, thus being able to cast the spell X times requires materials that cost X g. Triggered-type \textlf{bound spells} are activated upon a chosen condition, for instance, being the target of a spell or when the enchanted weapon strikes a target. They do not require action points to activate but have the same limits upon their number of uses. These can be employed even by non-mages.
%
%The \textlf{difficulty} of the enchantment is $\dicediffbase$+Y, where Y is the \textlf{Magic} skill score that the spell will be cast with. To determine the material cost simply use that required for the number of charges desired. Damaging spells use the \textlf{Might} of the character wielding the item.
%
%\subsection{Animation}
%An object can be enchanted to become animated, allowing it move and act on command. Who is able to command such an item is dictated by who holds a focus item, called a `control key', that is linked to the animated object (this control item is enchanted at the same time as the object is animated). The force and power with which the animated object can move is dictated by the \textlf{difficulty} of the enchantment, this is calculated according to lifting power of the \textlf{Telekinesis} spell, being \textlf{difficulty} 5 + 1 per 10 kg of lifting power. The control key may function as the focus for multiple such enchantments but may not simultaneously bear any other type. The material cost is 2.5 times the \textlf{difficulty} (rounding down) in gold pieces.
%
%One (plus \textlf{Wit}) animated items, can be commanded at a cost of one action point. Commands can only be issued by the key holder, who does not need to be able to wield magic themselves to use the key.
%
%\subsection{Weapon Enchantments}
%
%\subsubsection{Blasting (11+X)}
%\textlf{Prefix} or \textlf{Suffix}. This allows the weapon to be used in a magical projectile attack. This is evaluated as a standard shooting attack with \textlf{power} equal to X. The type of magical projectile is chosen when the enchantment is made. Material cost: X*2.5 g.
%
%%\subsubsection{Channelling (12)}
%%\textlf{Prefix} or \textlf{Suffix}. This allows a magic user to channel his power into blasts of destructive energy. This allows the wielder to fire a magical projectile attack that offers a \textlf{Deflect} chance to avoid it, use the wielder's \textlf{Magic} skill score as \textlf{Aim}. The \textlf{Power} of the projectile is given by the wielder's \textlf{Spell Power}. The visual nature of the projectile should be suitable to the magic type being used. Material Cost: 30 g.
%
%
%\subsubsection{Assassin (8,+4)}
%\textlf{Prefix} (Assasin's). The weapon grants + 1 \textlf{Power} (+ 1 per 4 added \textlf{difficulty}) when evaluating \textlf{Penetrating Hits}. Material cost: 20 g plus 10 g per extra point.
%
%\subsubsection{Malevolent (10)}
%\textlf{Prefix}. The weapon inflicts an \textlf{edge} penalty to the next \textlf{Deflect} check made by its victims. Material cost: 25 g.
%
%\subsubsection{Penetrating (10,+5)}
%\textlf{Prefix}. The weapon's edge bites through even the thickest armour. The weapon gains \textlf{Penetration} 1 (+1 per 5 added \textlf{difficulty}). Material cost: 25 g + 12 g per extra point.
%
%\subsubsection{Rampant (11)}
%\textlf{Prefix}. After moving the weapon gains \textlf{Power} equal to the \textlf{Deflect} bonus of an equipped shield. Material Cost: 27 g.
%
%\subsubsection{Vengeful (11) }
%\textlf{Prefix}. The weapons gains an \textlf{edge} bonus to \textlf{aim} and damage rolls against a foe that has wounded you recently (within the last round of combat). Material cost: 27 g. 
%
%\subsubsection{Executioner (12)}
%\textlf{Prefix}. When the weapon inflicts a \textlf{Critical Hit} the wielder is granted an extra damage roll. Material cost: 30 g.
%
%\subsubsection{Thirsting (13)}
%\textlf{prefix}. When the weapon causes at least 1 \textlf{endurance} damage to a victim, the wielder regains 1 missing \textlf{endurance}. Material cost: 32 g. 
%
%%\subsubsection{Righteous (12)}
%%\textlf{Prefix}. The weapon's wielder gains 1 point of \textlf{Divine Favour} (provided he is eligible) when he defeats a foe in combat. Material cost: 30 g.
%
%\subsubsection{Masterful (13)}
%\textlf{Prefix}. The weapon applies an \textlf{edge} penalty to \textlf{Resist} attempts made against the wielder's spells or abilities. Material cost: 35 g.
%
%\subsubsection{Impaling (14)}
%\textlf{Prefix}. The weapon gains \textlf{Rending}. Material cost: 42 g.
%
%\subsubsection{Thundering (14)}
%\textlf{Prefix}. The weapon \textlf{Cripples} victims of its \textlf{Critical Hits}. Material cost: 42 g.
%
%\subsubsection{Vorpal (16)}
%\textlf{Prefix}. Upgrades \textlf{Lethality} of the weapon on \textlf{critical hits}. Material cost: 60 g.
%
%
%
%\subsubsection{Aggression (10)}
%\textlf{Suffix}. The weapon feels alive in your hand, its cuts are far surer and swifter than those you could normally make. The weapon grants + 1 \textlf{Aim} on its first attack against a given target. Material cost: 22 g.
%
%\subsubsection{Deflection (10)}
%\textlf{Suffix}. This allows the weapon's wielder to make a counter attack if they score a \textlf{Critical Deflect}. Material Cost 25 g.
%
%\subsubsection{Finesse (11)}
%\textlf{Suffix}. \textlf{Critical Failures} made with the weapon may be re-rolled. Material cost: 22 g.
%
%\subsubsection{Force (11)}
%\textlf{Suffix}. The weapon's attacks knock down victims that are \textlf{Staggered}. Material cost: 25 g.
%
%\subsubsection{Culling (12)}
%\textlf{Suffix}. Attacks made against targets which are \textlf{Crippled}, \textlf{Knocked Down}, or \textlf{Bleeding} benefit from a \textlf{lethality} upgrade. Material cost: 30 g.
%
%\subsubsection{Laceration (13)}
%\textlf{Suffix}. The weapon inflicts \textlf{Bleeding} when it succeeds on a damage roll. Material cost: 32 g.
%
%\subsubsection{Butchery (14)}
%\textlf{Suffix}. The weapon has \textlf{Heavy Weapon} +1. Material cost: 40 g.
%
%\subsubsection{Devastation (14)}
%\textlf{Suffix}. The weapon has \textlf{Burst} +1. Material cost: 40 g.
%
%\subsubsection{Arcing (14)}
%\textlf{Suffix}. The spells cast by the wielder may add one extra target. Material cost: 42 g.
%
%\subsubsection{Determination (15)}
%\textlf{Suffix}. Attacks with the enchanted weapon have an \textlf{edge} bonus to \textlf{Aim}. Material cost: 45 g.
%
%
%
%\subsection{Armour Enchantments}
%
%\subsubsection{Resilient (10)}
%\textlf{Prefix}. Grants the wearer + 1 to \textlf{Resist} checks. Material cost: 25 g.
%
%\subsubsection{Feather-light (11)}
%\textlf{Prefix}. The armour grants its wearer +1 range to their normal movement. Material cost: 30 g.
%
%\subsubsection{Unflinching (12)}
%\textlf{Prefix}. This armour now negates the first \textlf{endurance} point lost each turn. Material cost: 32 g.
%
%%\subsubsection{Martyrdom (13)}
%%\textlf{Prefix} (Martyr's). When the wearer suffers a wound he gains 1 action point (up to a maximum of 1 per turn). Material cost: 32 g. 
%
%\subsubsection{Skilful: X (14)}
%\textlf{Prefix}. This grants the armour's wearer + 1 to checks for skill X. Material Cost: 37 g. When naming the armour one can alter the prefix to suit the skill, for example: sneaky leather armour would add + 1 \textlf{Stealth}.
%
%\subsubsection{Unshakeable (14)}
%\textlf{Prefix}. This grants the armour the \textlf{Bulwark} rule. Material cost: 40 g.
%
%\subsubsection{Adamant (15)}
%\textlf{Prefix}. This grants the armour the \textlf{Adamant} rule. Material cost: 45 g.
%
%
%
%\subsubsection{Warding (10,+4)}
%\textlf{Suffix}. The armour grants +1 \textlf{resist} (+ 1 level per 4 added \textlf{difficulty}). Material cost: 22 g plus 10 g per extra point.
%
%\subsubsection{Evasion (11,+4)}
%\textlf{Suffix}. The armour grants + 1 \textlf{Deflect} (+ 1 per 4 added \textlf{difficulty}). Material cost: 25 g plus 10 g per extra point.
%
%\subsubsection{Mirrored (12)}
%\textlf{Suffix}. This enchantment can only be used upon a shield. While using the shield a wielder may use their \textlf{Deflect} in place of \textlf{Resolve} when making \textlf{Resist} attempts against spells. Material Cost: 27 g.
%
%\subsubsection{Destruction (14)}
%\textlf{Suffix}. The armour grants + 1 \textlf{Power} for spells. Material cost: 37 g.
%
%\subsubsection{Attribute (17)}
%\textlf{Suffix}. The armour grants the wearer + 1 \textlf{Attribute}. Material cost: 40 g. Where \textlf{Attribute} is a chosen \textlf{Natural Attribute} and the \textlf{suffix} name will be something like Gauntlets of Might. 
%





\chapter{Aethermancy}
\label{chap:magic}

Any character with sufficient knowledge of the flow and hum of the aether can alter the very fabric of the world.  The energy for this is drawn from the inner strength of the character themselves. Thus, a character can only exercise this powers sparingly, lest they exhaust the aether powering their body. The use of aether to create external effects is known as ``manifestation". Characters learn new manifestations either through being taught them or by spending Hero Experience to unlock them.

\section{Aethermantic mechanics}

\subsection{Aether capacitor}
\label{sec:capacitor}
A character's body can only hold so much aether. To transcend this limitation, aethermancers have developed a device called an `aether capacitor' which stores additional aether for them. This device is bulky and has wires running from it various points on the aethermancer's body. This means that one cannot wear armour and have an aether capacitor equipped. A standard aether capacitor costs 1 g and can hold 1 aether point. For each additional capacity point the cost increases by 3 g. The maximum possible capacity is 4 points.  

\subsection{Producing a manifestation}
Manifestations are actions like any other. However, they also costs 1 point of aether, unless otherwise specified. Manifestations usually just require that the caster and victim make a \textlf{resist} check to decide if they take effect. However, some manifestations produce projectiles and thus also involve a \textlf{deflect} check. A \textlf{critical failure} during casting releases aether in an uncontrolled reaction. 

\subsection{Resist(X)}
A Resist(X) check involves the victim making an opposed roll with their natural attribute X against the aethermancer, who adds -1 or \textlf{Might} depending on \textlf{magic} skill proficiency. 

\subsection{Dispel}
An aethermancer can use their own aether to nullify the manifestations of another. This can be used in place of \textlf{Resist} or as 1 action point prepared action, it involves an \textlf{aethermancy} check opposed with the other aethermancer. If the check succeeds the manifestation is nullified.

\subsection{Persistent manifestations}
These have a long duration but require that concentration be maintained to keep them functioning. Thus, only one can be employed at a time. If a character is disabled or killed then their \textlf{persistent} manifestation ends. Otherwise, if they suffer a successful damage roll, they must make a \textlf{resolve} check against $\dicediffbase$ plus the \textlf{power} of the attack. If they fail, the \textlf{persistent} manifestation ends.





\subimport{./}{spells_v10.tex}


\section{Magic items}

With long study and practice, a magically attuned person can gain the ability to imbue magic into physical objects. These enchanted items can be given special powers, however, the expertise required to create them comes at a steep price. A single item can only bear one enchantment.

\subsection{Infused items}
Items can be imbued with an aetheric manifestation this can be of the activated or triggered type. Activated bound manifestations can be used for 1 action point and may only be cast X times before the enchantment dissipates (X is dictated by the cost of the item). This kind of \textlf{Bound spell} can be cast by any character. Triggered-type bound manifestations are activated upon a chosen condition, for instance, being the target of a spell or when the enchanted weapon strikes a target. They do not require action points to activate but have the same limits upon their number of uses. 

The cost of these items is 3 g per charge.

%\subsection{Animation}
%An object can be enchanted to become animated, allowing it move and act on command. Who is able to command such an item is dictated by who holds a focus item, called a `control key', that is linked to the animated object (this control item is enchanted at the same time as the object is animated). The force and power with which the animated object can move is dictated by the \textlf{difficulty} of the enchantment, this is calculated according to lifting power of the \textlf{Telekinesis} spell, being \textlf{difficulty} 5 + 1 per 10 kg of lifting power. The control key may function as the focus for multiple such enchantments but may not simultaneously bear any other type. The material cost is 2.5 times the \textlf{difficulty} (rounding down) in gold pieces.
%
%One (plus \textlf{Wit}) animated items, can be commanded at a cost of one action point. Commands can only be issued by the key holder, who does not need to be able to wield magic themselves to use the key.







\subimport{./}{fantasy_gear.tex}




\listoftables


\end{document}
