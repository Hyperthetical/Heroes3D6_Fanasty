\documentclass[a4paper,11pt,oneside]{book}

\usepackage{fullpage}
\usepackage{amsmath}
\usepackage{import}
\usepackage{tocbibind}
\usepackage[bookmarks=true,plainpages=false]{hyperref}
\usepackage{titletoc,titlecaps}
\usepackage{caption}

%\usepackage{helvet}
%\usepackage{sansmath}
%\usepackage{sfmath}
%\renewcommand{\familydefault}{\sfdefault}

\usepackage[T1]{fontenc}
%\usepackage{palatino,mathpazo}
\usepackage{lmodern}


\subimport{../}{dice}

\newcommand{\textlf}[1]{\textbf{\titlecap{#1}}}
\newcommand{\textlfirst}[1]{\textbf{\textit{\titlecap{#1}}}}

\title{\textbf{\huge 3d6 Heroes \\Fantasy Role-playing Rules Add-on \\Version 10.1}}
\author{Geoff Beck}
\date{}

\begin{document}
\maketitle
\frontmatter
\tableofcontents
\mainmatter

\chapter{The world and the `Last City'}
The world of Krell is a devastated wasteland. An event in the distant past commonly referred to as the `Catastrophe' unleashed vast quantities of warping magical energies that laid waste to cities and ecosystems across the planet. The cause of this disaster is not clearly remembered, however, it has left a long lasting suspicion of magic in the few survivors that cling to this wrecked world. Over time the lingering magic, called `the residual', has produced new mutant life-forms by warping the small surviving populations. The `Last City' is the only major settlement in existence, built after the Catastrophe and housing peoples of all species, it teeters constantly the brink of collapse as a result of supply shortages as well as internal tensions. What holds the city together is the shelter it provides from the residual, this being a great wall built of a patchwork of `untainted' iron. This being a metal of great value, often used as currency outside the city, as it survived the Catastrophe without being infused with dangerous residual energies and now repels these same influences.    

Small settlements dot the wastelands outside the Last City, these struggle constantly against the dangers of residual, mutant wild-life, bandits, and the terrifying changekin. These last are feral people twisted by the residual energies into nightmarish monsters. Life outside the city is a constant struggle to find food and water that are untainted by the residual, as too much exposure can eventually twist one into a changekin.

\section{The last city}
The only bastion of civilisation in a blighted world, the Last City stands as monument to the endurance of the people of Krell. Surrounded by a patch-work wall of untainted iron the city is largely protected from the warping power of the residual. Entrances to the city are guarded and those too contaminated by the residual are denied entrance, especially as a common punishment for crimes is being exposed to contamination and cast out of the city. Such is the suspicion of magic and its users that all mages must be registered and are branded with forehead marking, those who refuse are cast out as criminals. The mage brand brings much suspicion upon its bearers with common folk often spitting in the street at their passing, and some being unwilling to even do business with them. Often children with magical talent are disowned by their families and turned out onto the streets. 

The city is ruled over by council of exclusively human `founding families', constituting a ruling aristocracy whose power is enforced by a small soldier corps. Low-level crime is endemic in the city, as it contains many desperate people of all species, and criminal gangs act as the defacto rulers of some districts and trades. The existence of the city is precarious, as it depends strongly on food grown in surrounding settlements, which are constantly under pressure from raiding bandits, changekin, and the residual itself. Despite this, it is the most sure shelter against the terrors of the wastelands.

Within the city coins are used as currency but the alternative of untainted iron is also widely accepted (it is often the only currency used in the wastelands). A coin-sized disk of this metal is exchanged for 3 gold coins, such is its value in warding off the effects of residual when travelling outside the city.   

\section{The changed and the changekin}  
The changed are a phenomenon that began after the Catastrophe. They are beings that have been warped by the residual, and exhibit anything from minor changes like scales or feathers on their skin, up to limbs or heads being a mixture of animal and human (bird claws, goats hooves, beaks, animal jaws). The changed are, however, still people and although often viewed with suspicion and prejudice they are not attacked on sight. 

Changekin are somewhat like the changed but have become feral, their minds rotted away by the residual. Their mutations tend to be even more extreme, often multiple additional limbs, or large bloated bodies. Changekin are savage and attack anything on sight, except for other changekin (most of the time). Such creatures form into roaming hunting bands that seek out anything weak enough to be preyed upon, attacking travellers or small settlements if their band grows large enough. Changekin seldom wield weapons or wear armour, their minds are too damaged for such rationality.  


\section{The residual}
The residual is remnant warping energy left over from the Catastrophe. Exposure to it is dangerous as it causes damage to exposed tissues and can, with sufficiently high doses, twist the bodies of the exposed in monstrous forms. The residual is undetectable to normal senses except in very high concentrations, where it feels like a heated crackle in the air (these concentration levels are immediately dangerous as well). Untainted iron is commonly used to ward off the effects of the residual and skilled alchemists can produce potions brewed with it that can protect from, or cleanse a body of, the residuals effects. 

Whenever exposed to a residual infused environment (or consuming tainted food/drink) a creature must make a \textlf{resolve} check against a \textlf{difficulty} set by how saturated the area is. For example low contamination is \textlf{difficulty} 8, dangerous is 11, and deadly is 14. If the creature fails they gain a Residual point (\textlf{critical failure} adds extra points). The number of points is added to the \textlf{difficulty} of subsequent checks against residual contamination. At 5 points the creature begins to feel light-headed, weak, and sweaty (has an \textlf{edge} penalty on all actions). At 10 points a creature dies. Residual points last until removed. Armour made of untainted iron grants an \textlf{edge} on all such \textlf{resolve} checks and weapons of this metal have an \textlf{edge} bonus to damage rolls against changekin. 



\chapter{Character creation}
These are additional rules providing the options for character species and additional backgrounds in this fantasy setting.

\section{Character species}

\subsection{Humans}
Humans are a dynamic species: adaptive, resilient and, as a group, will always stand firm even against the most difficult odds.

Humans settlements tend to be sprawling affairs full of the bustle of life and rich in diversity. Often with lots of peoples of other species passing through. Either staying there for trade, or just because they prefer the fast pace and constant change of atmosphere in human society.

The average human lives for up to 80 years, 30 years being considered a mature adult, though humans are considered adults at the age of 16 in most societies. Humans can excel in any career they put their minds to. Though they lack the patience and long practice of the Elves they make up for it with a natural ability to adapt quickly to new situations or tasks. 

The elves' popular view of humans is that they are flighty and lacking-in-seriousness. Mus-folk stereotype humans as grim and unscrupulous. Trogs stereotype humankind as deceitful and conniving.  

\subsubsection*{Flexibility}
Humans are extremely adaptable and may select an extra associated skill for their background.

\subsection{Elves}
Though elves somewhat resemble humans at first glance, there is a world of difference on closer examination. Firstly, elves are incapable of digesting meat. They are vegetarian with a preferred diet of sweet, sugary things. Second, their manner of reproduction is completely different, elven eggs gestate externally in a similar manner to those of fish. Additionally elven blood is clear, it contains no iron, and their hearts are far smaller than that of a man, as capillary action plays a major role in their circulatory systems.

Elves have high cheek bones and very narrow, angular features. Their skin varies in shade between green and light brown and has a similar texture to smooth tree bark. Their eyes range in colour from yellow, through green and blue to white, the pupils of which are narrow and cat-like. Elves are short by human standards, and are also of a slighter build. All of them posses the curious ability to alter their hair colour at will. However, if they are not careful it will change on its own to reflect their mood. 

Elves' lives and metabolisms move far more slowly than those of humans and they are children till the age of about 40 years. They grow far more slowly than other species and tend only to reach their full growth by 60 years old. Additionally, they only need to eat or sleep once every two or three days. Elven travellers often pass through the last city but they mostly prefer the company of their kind in the remains of the coastal coral forests.

Humans tend to stereotype elves as vain, vapid, and hedonistic. Mus-folk and Trogs popularly assert elves to be crazy and lacking in any kind of sense. 

\subsubsection*{Perfectionists}
Elves live long lives and thus have time to pursue their skills to a point few others can reach. As such, elves may choose on additional skill proficiency.


\subsection{Mus-folk}
The mus-folk (or `Musmus' in their own language) resemble large rats. They reach a maximum height/length of 3 - 4 ft and tend to inhabit the margins of human settlements where they are regarded as a mixture of pest and second-class citizen (rude terms for them include `squeakers' and 'ratties'). Their own language consists of chirps and squeaks rather than words. However, they are fully capable of producing human and elvish speech. When they do, it comes out in a rapid torrent of words infused with poor grammar and often features repetition when they are excited. Mus-folk prefer underground dwellings and make burrows underneath human settlements. They are often stereotyped as light-fingered and mischievous. Due to their mistreatment within human societies the mus-folk tend to find employment on the sidelines of human economies.    

\subsubsection*{Sneak \& squeak}
Mus-folk are naturally stealthy. They may always add \textlf{stealth} to their listed of associated background skills. 

\subsubsection*{Pack rats}
Like their rat relatives, mus-folk work best in groups. Mus-folk get an \textlf{edge} bonus on \textlf{aim} when targeting the same foe as an ally. 


\subsection{Trogs}
Trogs are large (usually 7 ft tall) and bulky, but their most striking feature is their large eyes. Trog facial features tend to be heavy, with wide square jaws and large beak-like noses. Trogs have tough greyish, slightly scaly skin and are in general quite difficult to kill or injure. Traditionally, trog groups are familial and organised around a mother trog, a very large and formidable female (female trogs tend to be larger and stronger than males), with a harem of males who care for the young trogs. Plenty of young trogs find this authoritarian system stifling, so they form their own more egalitarian bands. This form of living is often sconed and belittled by their elders. Trogs typically prefer dark places, due to their sensitive eyes. However, in the last city they mix with other species in a limited fashion, as they make good bruisers and bouncers due to their comparative size.

Other species have a stereotyped view of Trogs as stupid and violent.  

\subsubsection*{Tough as nails}
Trogs are extremely durable and thus have the \textlf{hardy perk}.

\subsubsection*{Dark-dwellers}
Trogs have excellent eyesight in the dark as well as sensitive noses. They have an \textlf{edge} bonus on \textlf{awareness} checks in the dark.


\subsection{The changed}
These people are the long-term effects of the Catastrophe, sentient creatures whose forms have been altered by the warping magic of the residual. Unlike the feral changekin, their minds are intact and they continue to try and live within the society of their original people. The changed are regarded with suspicion and prejudice by all of the unchanged and are often openly discriminated against. Some of the changed form their own settlements to avoid this mistreatment. 

Changed can be created by extreme residual exposure or born, to either changed parents, or those unchanged who have suffered a slow build up of residual exposure. As such changed can be born to any family, even within the walls of the last city. They are commonly cast out of their families to fend for themselves.

Changed vary wildly in appearance, as they are formed from a base species: human, trog, elf, or mus which has then been warped with strange features. Some possibilities can be elongated limbs, clawed hands, feathers or scales on their skin, fangs instead of teeth, horned heads, animal heads (or elements of this like a beak or snout), hoofed or clawed feet, boney spines jutting from the body, a tentacle instead of a normal limb, extra eyes, a tail, or reverse-jointed legs (like a bird).

\subsubsection*{Change-fire}
The changed are infused with the warping energies of the residual. They can channel this energy to emit a gout of roiling, iridescent fire from a limb or orifice. This is a 1 action point ability that can be used on two adjacent targets (within 6 m) and deals a hit with normal \textlf{lethality} while also inflicting a Residual point if it causes damage. 

\subsubsection*{Infused}
The changed have an \textlf{edge} bonus on \textlf{resolve} checks against Residual contamination.


\section{Starting equipment}
A new character may choose one piece of armour and up to three weapons from Table~\ref{tab:start-gear}. All characters get a set of clothes.
\begin{table}[ht!]
	\centering
	\caption{New Adventurers Starting Gear.}
	\label{tab:start-gear}
	\begin{tabular}{|l|l|l|l|l|}
		\hline
		Name & Power & Hands &  Lethality & Notes\\
		\hline
		Rusty sword & - & 1 & N & -\\
		Hunting knife & - & 1 & N & Small\\
		Notched axe & - & 2 & N & MD\\
		Worn crossbow & - & 2 & N & MD, Reload 2, Range 2\\
		Corroded pike & - & 2 & N & Reach, Rending\\
		Pitted hammer & +1/- & 2/1 & N & Penetration 1 \\
		Aged spear & - & 2/1 & N & Reach/Throw 1\\
		Creaky short bow & - & 2 & N & Range 1\\
		Old hunting bow & +1 & 2 & N & Range 2, Reload 1\\
		Scarred great sword & - & 2 & N & MD\\
		Shabby pistol & - & 2 & C & Reload 3, Range 1 \\
		Ramshackle musket & - & 2 & C & Range 2, Reload 3, Penetration 1 \\
		Dented shield & - & 1 & - & Deflect +1 \\
		\hline
	\end{tabular}
	\begin{tabular}{|l|l|l|l|}	
		\hline
		Name & Toughness & Type & Notes\\
		\hline
		Battered breastplate and mail & 12 & H & - \\
		Rusted mail hauberk & 11 & M & - \\
		Tattered gambeson & 10 & L & - \\
		Travelling clothes & 9 & - & - \\
		\hline
	\end{tabular}
\end{table}


\section{Additional backgrounds}

\subsection{Student of the arcane}
The character has spent time studying magic under the tutelage of a mentor wizard or within a school of magic. The character may sacrifice one skill bonus in order to learn one additional spell with an experience cost of 3 or less. Associated skills: Magic, Arcana, Mechanical, Alchemy, History, Religion, Plants, Animals.

\subsection{Hedge wizard}
The character has learned to wield magic on their own. They tend to keep their abilities quiet as they are usually unregistered magic users, and wander from place to place making use of their skills to earn a living. Associated skills: Magic, Healer, Alchemy, Survival, Persuade, Decieve. 


\section{Additional convictions}

\subsection{It's the end of the world as we know it, and I feel fine}
The character is a true citizen of the apocalypse, the world ended a while ago and its just taking everyone else a long time to realise this. You have no qualms about doing what has to be done, survival is all that matters.

\subsection{Civilisation starts with being civil}
The world may have ended but things can be rebuilt, all it requires is not resorting to killing each other for scraps. The character believes that a civilised world can be restored by individual civility.

\subsection{Magic means mistakes}
Reckless magic use lead to the Catastrophe. Even if we can't remember exactly how, it's enough reason to regard it like a dangerous animal. The character is deeply suspicious of magic and its users.

\subsection{Magic makes amends}
Magic may have lead to the Catastrophe but that was the fault of the users. The character believes magic is a vital tool, a dangerous one to be sure, but fascinating beyond measure. 


\chapter{Perks and proficiencies}

\section{General proficiencies}
These perks do \textbf{not} occupy equipment slots for passive or active \textlf{perk}, their effect is always active.

\subsection{Archery (3)}
The character has trained in the use of the bow, no longer suffering untrained penalties.

\subsection{Arcane learning (3/7)}
The character has experience wielding magic and thus can learn spells from Chapter~\ref{chap:magic} for experience points. The character may now spend up to 3 additional experience on spells. If this is chosen at character creation it costs 3 experience, the cost is 7 otherwise. 

A character who picks this may choose 3 spells to learn from the following list: Sorcerous blast~\ref{spell:sorc-blast},Bend light~\ref{spell:bend-light}, Minor telekinesis~\ref{spell:min-tele}, Summon familiar~\ref{spell:familiar}, Overgrowth~\ref{spell:overgrowth}, Ignite~\ref{spell:ignite}, Illuminate~\ref{spell:illuminate}, Transfix~\ref{spell:transfix}, Orb of light~\ref{spell:orb-light}, Arcane armour~\ref{spell:arcane-armour}, Blind ray~\ref{spell:blind-ray}, Flare~\ref{spell:flare}, and Stone sense~\ref{spell:stone-sense}.

\section{Active perks}
These open up new actions that a character can make and they must occupy an equipment slot for active \textlf{perks} to be usable.

\subsection{Hawk Talon (3)}
(Requires \textlf{Eagle Eye} and \textlf{Archery}). Allows the character to launch a pair of arrows from a bow with a single shot at no \textlf{Aim} penalty. This grants the shot \textlf{Burst} +1.

%\subsection{Rhythm (5)}
%(Requires \textlf{Archery}). Breathing and rhythm are vital to the art of the archer. If the character's attacks made with a bow cause damage, he may make an extra shooting attack this round. This bonus may only be claimed once per turn.

\subsection{Heart-seeker (3)}
(Requires \textlf{Archery}). A skilled bowman knows just where to place his shots for maximum penetration. An attack with a bow can be declared as a Heart-seeker, in which case it costs 1 extra action point but gains \textlf{Rending}.

\subsection{Channelled casting (2)}
(Requires \textlf{Arcane learning}) The character can cast a spell that normally costs 1 action point for 2 instead. If they do so they gain an \textlf{edge} on opposed checks associated with the spell. If the spell inflicts a damage roll then it gains \textlf{Burst} + 1.

\subsection{Main Gauche (3)}
There is an art to pairing weapons for fighting with both hands, this character has mastered it. When dual-wielding, if pairing a \textlf{Small} weapon and one that isn't, the character may elect not to attack with the \textlf{small} weapon. If he does so he gains an \textlf{edge} bonus to \textlf{deflect}.



\section{Passive}
These passively enhance the character and must occupy an equipment slot for passive \textlf{perks} to make their benefit usable.

\subsection{Magical mastery (6)}
(Requires \textlf{Arcane learning}) The character can equip 2 spells per active perk slot.

\subsection{Multi-tasker (4)}
(Requires \textlf{Arcane learning}) The character has learned to concentrate on many things at once. This allows them to maintain two \textlf{persistent} spells effects at the same time.

\subsection{Shield bash (3)}
The character may use a shield to \textlf{shove} (add the \textlf{deflect} bonus to such checks). 

\subsection{Polished shield (3)}
The character can add a shield's \textlf{deflect} bonus to \textlf{resist} checks.

\subsection{Lethal Thrust (4)}
The character delivers their killing blows with the point of a blade. While wielding a sword, the character's attacks gain \textlf{Rending} against victims with at least one face-up wound card.

\subsection{Hammer Time (4)}
\textlf{Critical hits} from blunt weapons make the victim \textlf{vulnerable} to the next damaging hit.

\subsection{Momentum (4)}
When wielding cleaving weapons, like axes or glaives, the character's attacks gain \textlf{Heavy Weapon} +1 if their target failed a \textlf{Deflect} check against their attacks last round. This bonus stacks and lasts till an attack is \textlf{Deflected} or combat ends.

\subsection{Reloading drill (4)}
The character is well practised at rapidly preparing crossbows to fire. This reduces crossbow \textlf{reload} action costs by 1. 

\subsection{Pole Position (4)}
When wielding thrusting-type pole weapons, like spears or pikes, successful damage rolls can be used to move their victims out of the current combat area. A character can use this effect only on creatures of the same size or smaller.

\subsection{Expert gunner (3)}
The character is highly familiar with black-powder weapons, as such they no longer suffer the \textlf{edge} penalty to \textlf{aim} with them.

\subsection{Rifle drill (4)}
(Requires \textlf{expert gunner}) Black-powder weapons \textlf{reload} action costs are reduced by 1 for the character. 

%\subsection{Slippery Mind (3)}
%This allows the character to make Resist checks against spells using their \textlf{Cunning} rather than \textlf{Resolve}.

\subsection{Weapon Specialisation (5)}
This \textlf{perk} from the core rules may apply to the following weapon categories: Bladed (sword-type), Axes, Pole-weapons, Blunt (hammers,clubs,maces,cudgels), Extended (whip,chain,flail), Crossbows, Bows, Throwing (applies to all thrown weapons), and Shield (an \textlf{edge} bonus to \textlf{deflect} with shields).

\subsection{Witch hunter (3)}
The character gains an \textlf{edge} bonus to \textlf{resist} and \textlf{deflect} spells.

\subsubsection{Upgrade: Burn the witch (3)}
(Requires \textlf{Witch hunter}) Gain an \textlf{edge} bonus to damage rolls against magic users.
 





\chapter{Character Skills}

\section{Working Professionally}
A character makes 16 copper per hour per skill level while performing his profession. If he has the \textlf{Master perk} for the profession the income rate triples.

\section{Magic (Perception)}
\textlf{Magic} is the skill invoked to cast spells. More details can be found in Chapter~\ref{chap:magic}.

\section{Alchemy (Cunning)}
\textlf{Alchemy} is the craft of potion brewing, creating chemical mixtures to achieve almost miraculous effects, or simply mistakes that produce spectacular explosions, the alchemist can whip up a potion to suit any need.

Creating a Potion requires suitable ingredients of course (this is up to the Game Master but try make them sensible, like an ogre's tooth for a strength potion; that kind of thing). All potions have Base Solution as an ingredient. A potion also needs a container and general a set of glass vessels and equipment for measuring, grinding and heating ingredients (alchemist's tools) that can be purchased from an alchemist for 50 silver.
Table~\ref{tab:alch} displays some potion suggestions.

\begin{table}[ht!]
\caption{Alchemist Potions. D is the \textlf{difficulty}, time is how long the effects last and is given in minutes. The cost reflects that of the ingredients for a single draught of potion and the bottle to hold it, double this for the price charged by most alchemists.}
\begin{tabular}{|l|l|l|l|l|l|}
\hline
Potion & D & Effects & Time & Ingredients & Cost \\
\hline
Base Solution & 4 & Ingredient for all below & - & Water, Alcohol,  & 1 s \\
 & & & & Copper Sulphate & \\
\hline
Might &  9/14 & +1/+2 Might & 10 & Troll teeth & 25/50 s \\
Cunning & 9/14 & +1/+2 Cunning & 10 & Mus-folk hair & 25/50 s \\
%Swiftness & 9/14 & +1/+2 Agility & 10 & Snake scales & 25/50 s \\
Sorcery & 10/15 &\textlf{edge} bonus - spells & 10 & Residual crystals & 50/100 s\\
Iron-flesh & 8/13 & +1/+2 Toughness & 10 & Fine granite power & 15/30 s \\
Fire-blood & 11 & Enrage & - & Trog blood & 35 s \\
Hawk-eye & 12 & \textlf{edge} Awareness/Aim & 20 & Eagle feathers & 15 s \\
Troll-blood & 15 & Regeneration & 5 & Troll blood & 2 g \\
Purging & 11 & -2 Residual point & - & Untainted iron & 2 g\\
Warding & 10 & \textlf{edge} on Residual checks & 10 & Untainted iron & 1 g \\ 
Healing & 11 & Remove a Wounded card & - & Mend-well root & 25 s \\
Restoration & 10 & Cure 1 Condition & - & Common herbs & 2 s \\
Peace & 14 & Cure all Conditions & - & Nightshade,  & 30 s \\
 & & & & Mend-well leaves & \\
Competence & 10/15 & \textlf{edge} bonus Professions & 60 & Gold, Silver & 50/100 s \\
Giant-blood & 16 & Become Large Creature  & 10 & Giant Blood & 5 g \\
 & & (+2 Might) & & & \\
Camoflage & 7/11 & \textlf{Stealth edge} bonus & 20 & Nightshade, Ivy root & 5 s \\
Invisibility & 18 & Invisibility & 5 & Ectoplasm & 10 g \\
Haste & 18 & +1 action point per round & 5 & Dire-wolf Heart & 5 g \\ 
Explosive & 14 & radius 1, Burst 1, Power +1 & - & Acid, coal, salt-peter & 50 s \\
\hline
\end{tabular}
\label{tab:alch}
\end{table}



\section{Metal Smith (Cunning or Might)}
A Metal Smith is capable of crafting metal tools or suits of armour from the Basic Armour Table in the Arms and Armour Chapter. See the Crafting Armour Table \ref{tab:craft-armour}. Making armour \textlf{Finely-crafted} increases the \textlf{difficulty} by 3, \textlf{Master-crafted} by 6 (but requires the \textlf{master perk} for this skill). These quality increases can also be achieved with 1 or 2 levels of \textlf{critical} success respectively with the same \textlf{perk} restrictions. Double the time requirements for making armour and it takes an extra 8 hours of work, over and above the normal time requirements, to make \textlf{Heavy}-type armour.

\begin{table}[!ht]
  \centering
  \caption{Crafting Armour}
  \label{tab:craft-armour}
  \begin{tabular}{|l|l|l|l|}
    \hline
    Name & Type & Materials/Cost & Difficulty\\ [0.5ex]
    \hline
    Gambeson & L & Hides, Cloth/30 s & 8\\
    \hline
    Mail hauberk and & M & Steel Bars, Cloth, Hides/1 g 50 s & 10\\
    gambeson & & & \\
    Brigandine and mail & M & Steel Bars, Hides/3 g & 12 \\  
    \hline
    Brigandine and plate & H & Steel Bars, Hides/4 g & 14\\
    Full plate & H & Hides, Steel Bars/10 g & 17\\
    \hline
    Barding (for Horses) & H & Hides, Steel Bars/7 g & 13\\
    \hline
    Shield & H & Wooden Planks, Leather, Iron Bands/30 s & 13\\
%    Tower Shield & H & Steel Bars, Leather, Straps/1 g 50 s & 15\\
    \hline
    \end{tabular}
\end{table}



\section{Weapon Smith (Cunning or Might)}
A Weapon Smith can craft his own fine weaponry to use or sell. See the Crafting Weapons Table \ref{tab:craft-weps}. Making a weapon \textlf{finely-crafted} increases the \textlf{difficulty} by 3, \textlf{Master-crafted} by 6 (but requires the \textlf{master perk} for this skill). These quality increases can also be achieved with 1 or 2 levels of \textlf{critical} success respectively with the same \textlf{perk} restrictions.
\begin{table}[!ht]
  \centering
  \caption{Crafting Weapons}
  \label{tab:craft-weps}
  \begin{tabular}{|l|l|l|}
    \hline
    Name & Materials/Cost & Difficulty\\ [0.5ex]
    \hline
    Short Sword & Steel Bars, Leather/10 s & 8\\ 
    Dagger & Steel Bars, Leather/2 s & 6\\
    Long Sword & Steel Bars, Leather/30 s & 10\\
    Horse Sword & Lots of Steel Bars, Leather/1 g & 14\\
    Great Sword & Lots of Steel Bars, Leather/35 s & 12\\
    Rapier & Steel Bars, Leather/15 s & 12\\
    Sword-Breaker & Steel bars, Leather/10 s & 11 \\
    \hline
    Battle Axe & Steel Bars, Leather/10 s & 9\\ 
    Bearded Axe & Steel Bars, Leather/10 s & 10\\ 
    Throwing Axe & Steel Bars, Leather/5 s & 11\\ 
    Great Axe & Steel Bars, Leather/40 s & 10\\ 
    Long Axe & Steel Bars, Pole, Leather/40 s & 11\\ 
    Bearded Long Axe & Steel Bars, Pole, Leather/40 s & 11\\ 
    \hline
    Javelin & Short Pole, Steel Bars/2 s & 9\\ 
    Quarter Staff & Wood/- & 7\\
    Spear & 3 m Pole, Steel Bars/6 s & 7\\
    Pike & 4 m Pole, Steel Bars/10 s & 10\\
    Glaive & 2 m Pole, Steel Bars/25 s & 12\\
    Halberd & 2 m Pole, Steel Bars/75 s & 14\\
    Lucerne Hammer & 2 m Pole, Steel Bars/70 s & 14\\
    Ranseur & 2 m Pole, Steel Bars/20 s & 11\\
    Spetum & 2 m Pole, Steel Bars/30 s & 13\\
    Lance & Heavy Pole, Steel Bars/40 s & 11\\
    \hline
    Cudgel & Wood/50 c & 6\\
    Club & Wood/2 s & 8\\
    War Hammer & Handle, Steel Bars, Leather/20 s & 13 \\
    Great Hammer & Pole Handle, Steel Bars, Leather/30 s & 15 \\
    Maul & Pole Handle, Steel Bars, Leather/40 s & 13 \\
    Mace & Handle, Steel Bars, Wood, Leather/2 s & 6 \\
    Great Mace & Pole Handle, Steel Bars, Leather/30 s & 9 \\
    \hline
    Short Bow & Wood, Bow String/10 s & 13\\
    Long Bow & Yew Wood, Bow String/60 s & 16\\
    \hline
    Hand Crossbow & Wood/10 s & 14 \\ 
    Light Crossbow & Wood, Steel Bars/30 s & 11\\
    Heavy Crossbow & Wood, Steel Bars/50 s & 16\\
    \hline
    Whip & Hides/10 s & 9\\
    Chain & Steel Bars/15 s & 10\\
    Flail & Steel Bars/25 s & 13\\    
    \hline
    \end{tabular}
\end{table}



\section{Tailor (Cunning)}
With appropriate tools a \textlf{Tailor} can craft clothing or any other cloth products. The \textlf{difficulty} is based on how fine the clothes should be. Wizards robes or other magical garments can also be enchanted, see Section~\ref{sec:enchant}. The design and exact look of the clothing is up to the tailor himself, the ingredients only determine the quality. See the Crafting Cloth Table \ref{tab:craft-clothes}. In general, clothes provide no bonus to defence and the level of success affects how fine and fancy they appear.

\begin{table}[!ht]
	\centering
	\label{tab:craft-clothes}
	\caption{Crafting Cloth}
	\begin{tabular}{|l|l|l|}
		\hline
		Name & Materials/Cost & Difficulty\\ [0.5ex]
		\hline
		Rough Garments & Rough Wool, Course thread/5 c & 8\\
		Course Robe & Rough Wool, Course Thread/10 c & 9\\
		Neat Garments & Smooth Wool, Course Thread/20 c & 10\\
		Fine Garments & Fine Wool, Embroidered Thread/10 s & 12\\
		Silk Robe & Silk, Silver Thread/50 s & 14\\
		Rich Garments & Silk, Fine Wool, Gold Thread/1 g & 14\\
		Wizard Lord Robe & Silk, Dragon Skin, Arcane Diamonds/50 g & 17\\
		\hline
	\end{tabular}
\end{table}


\section{Enchanting (Cunning)}
\label{sec:enchant}
This allows a character to place magical enchantments upon items. This typically requires magical ingredients and cannot be employed without the \textlf{Arcane learning perk}. An item cannot carry both enchantments and Runes.
The cost of an enchanted item (from an NPC enchanter) is equal to five times the difficulty of the enchantment (in gold pieces). The cost of an enchanted item and the material costs (for self creation) are halved for \textlf{Small} items. \textlf{Critical success} on the skill use also reduces the cost of materials by half (when doing your own enchanting). 

\subsection{Materials}
The materials used in creating enchantments are precious metals and gems, which are consumed in the process of enchanting. The value of the materials needed by each enchantment are listed in their descriptions. The exact nature of the materials is unimportant and the price given is a guideline average price (essentially the number of gold coins that could be used).

\subsection{Prefix and Suffix}
An item may only be enchanted with one of each type of enchantment (one \textlf{Prefix} and one \textlf{Suffix}). Names of magic items are created through the formula \textlf{Prefix} + item type + \textlf{Suffix}. For instance one can have an Executioner's Great sword of Aggression or a Vorpal Dagger of Butchery. Both of which are as intimidating as their names suggest.

\subsection{Bound Spells}
A \textlf{Bound Spell} enchantment is of the \textlf{Suffix}-type. Items can be imbued with a spell, this can be of the activated or triggered type. Activated \textlf{bound spells} can be used for 1 action point and may only be cast X times before the enchantment dissipates (X is dictated by the cost of the item). This kind of \textlf{Bound spell} can only be cast by a character who has the \textlf{perk} to cast spells of the appropriate school. The material cost is 1 g per charge, thus being able to cast the spell X times requires materials that cost X g. Triggered-type \textlf{bound spells} are activated upon a chosen condition, for instance, being the target of a spell or when the enchanted weapon strikes a target. They do not require action points to activate but have the same limits upon their number of uses. These can be employed even by non-mages.

The \textlf{difficulty} of the enchantment is $\dicediffbase$+Y, where Y is the \textlf{Magic} skill score that the spell will be cast with. To determine the material cost simply use that required for the number of charges desired. Damaging spells use the \textlf{Might} of the character wielding the item.

\subsection{Animation}
An object can be enchanted to become animated, allowing it move and act on command. Who is able to command such an item is dictated by who holds a focus item, called a `control key', that is linked to the animated object (this control item is enchanted at the same time as the object is animated). The force and power with which the animated object can move is dictated by the \textlf{difficulty} of the enchantment, this is calculated according to lifting power of the \textlf{Telekinesis} spell, being \textlf{difficulty} 5 + 1 per 10 kg of lifting power. The control key may function as the focus for multiple such enchantments but may not simultaneously bear any other type. The material cost is 2.5 times the \textlf{difficulty} (rounding down) in gold pieces.

One (plus \textlf{Perception}) animated items, can be commanded at a cost of one action point. Commands can only be issued by the key holder, who does not need to be able to wield magic himself to use the key.

\subsection{Weapon Enchantments}

\subsubsection{Blasting (11+X)}
\textlf{Prefix} or \textlf{Suffix}. This allows the weapon to be used in a magical projectile attack. This is evaluated as a standard shooting attack with \textlf{power} equal to X. The type of magical projectile is chosen when the enchantment is made. Material cost: X*2.5 g.

%\subsubsection{Channelling (12)}
%\textlf{Prefix} or \textlf{Suffix}. This allows a magic user to channel his power into blasts of destructive energy. This allows the wielder to fire a magical projectile attack that offers a \textlf{Deflect} chance to avoid it, use the wielder's \textlf{Magic} skill score as \textlf{Aim}. The \textlf{Power} of the projectile is given by the wielder's \textlf{Spell Power}. The visual nature of the projectile should be suitable to the magic type being used. Material Cost: 30 g.


\subsubsection{Assassin (8,+4)}
\textlf{Prefix} (Assasin's). The weapon grants + 1 \textlf{Power} (+ 1 per 4 added \textlf{difficulty}) when evaluating \textlf{Penetrating Hits}. Material cost: 20 g plus 10 g per extra point.

\subsubsection{Malevolent (10)}
\textlf{Prefix}. The weapon inflicts an \textlf{edge} penalty to the next \textlf{Deflect} check made by its victims. Material cost: 25 g.

\subsubsection{Penetrating (10,+5)}
\textlf{Prefix}. The weapon's edge bites through even the thickest armour. The weapon gains \textlf{Penetration} 1 (+1 per 5 added \textlf{difficulty}). Material cost: 25 g + 12 g per extra point.

\subsubsection{Rampant (11)}
\textlf{Prefix}. After moving the weapon gains \textlf{Power} equal to the \textlf{Deflect} bonus of an equipped shield. Material Cost: 27 g.

\subsubsection{Vengeful (11) }
\textlf{Prefix}. The weapons gains an \textlf{edge} bonus to \textlf{aim} and damage rolls against a foe that has wounded you recently (within the last round of combat). Material cost: 27 g. 

\subsubsection{Executioner (12)}
\textlf{Prefix}. When the weapon inflicts a \textlf{Critical Hit} the wielder is granted an extra damage roll. Material cost: 30 g.

\subsubsection{Thirsting (13)}
\textlf{prefix}. When the weapon causes at least 1 \textlf{endurance} damage to a victim, the wielder regains 1 missing \textlf{endurance}. Material cost: 32 g. 

%\subsubsection{Righteous (12)}
%\textlf{Prefix}. The weapon's wielder gains 1 point of \textlf{Divine Favour} (provided he is eligible) when he defeats a foe in combat. Material cost: 30 g.

\subsubsection{Masterful (14)}
\textlf{Prefix}. The weapon applies an \textlf{edge} penalty to \textlf{Resist} attempts made against the wielder's spells or abilities. Material cost: 35 g.

\subsubsection{Impaling (17)}
\textlf{Prefix}. The weapon gains \textlf{Rending}. Material cost: 42 g.

\subsubsection{Thundering (17)}
\textlf{Prefix}. The weapon \textlf{Cripples} victims of its \textlf{Critical Hits}. Material cost: 42 g.

\subsubsection{Vorpal (20)}
\textlf{Prefix}. Upgrades \textlf{Lethality} of the weapon on \textlf{critical hits}. Material cost: 60 g.



\subsubsection{Aggression (10)}
\textlf{Suffix}. The weapon feels alive in your hand, its cuts are far surer and swifter than those you could normally make. The weapon grants + 1 \textlf{Aim} on its first attack against a given target. Material cost: 22 g.

\subsubsection{Deflection (10)}
\textlf{Suffix}. This allows the weapon's wielder to make a counter attack if he scores a \textlf{Critical Deflect}. Material Cost 25 g.

\subsubsection{Finesse (11)}
\textlf{Suffix}. \textlf{Critical Failures} made with the weapon may be re-rolled. Material cost: 22 g.

\subsubsection{Force (12)}
\textlf{Suffix}. The weapon's attacks knock down victims that are \textlf{Staggered}. Material cost: 25 g.

\subsubsection{Culling (12)}
\textlf{Suffix}. Attacks made against targets which are \textlf{Crippled}, \textlf{Knocked Down}, or \textlf{Bleeding} benefit from a \textlf{lethality} upgrade. Material cost: 30 g.

\subsubsection{Laceration (14)}
\textlf{Suffix}. The weapon inflicts \textlf{Bleeding} when it succeeds on a damage roll. Material cost: 32 g.

\subsubsection{Butchery (16)}
\textlf{Suffix}. The weapon has \textlf{Heavy Weapon} +1. Material cost: 40 g.

\subsubsection{Devastation (16)}
\textlf{Suffix}. The weapon has \textlf{Burst} +1. Material cost: 40 g.

\subsubsection{Arcing (17)}
\textlf{Suffix}. The spells cast by the wielder may add one extra target. Material cost: 42 g.

\subsubsection{Determination (18)}
\textlf{Suffix}. Attacks with the enchanted weapon have an \textlf{edge} bonus to \textlf{Aim}. Material cost: 45 g.



\subsection{Armour Enchantments}

\subsubsection{Resilient (10)}
\textlf{Prefix}. Grants the wearer + 1 to \textlf{Resolve} checks. Material cost: 25 g.

\subsubsection{Feather-light (12)}
\textlf{Prefix}. The armour grants the wearer +1 range to his normal movement. Material cost: 30 g.

\subsubsection{Unflinching (13)}
\textlf{Prefix}. This grants the armour the ability to ignore the first \textlf{endurance} point lost each turn. Material cost: 32 g.

%\subsubsection{Martyrdom (13)}
%\textlf{Prefix} (Martyr's). When the wearer suffers a wound he gains 1 action point (up to a maximum of 1 per turn). Material cost: 32 g. 

\subsubsection{Skilful: X (15)}
\textlf{Prefix}. This grants the armour's wearer + 1 to checks for skill X. Material Cost: 37 g. When naming the armour one can alter the prefix to suit the skill, for example: sneaky leather armour would add + 1 \textlf{Stealth}.

\subsubsection{Unshakeable (16)}
\textlf{Prefix}. This grants the armour the \textlf{Bulwark} rule. Material cost: 40 g.

\subsubsection{Adamant (18)}
\textlf{Prefix}. This grants the armour the \textlf{Adamant} rule. Material cost: 45 g.



\subsubsection{Warding (10,+4)}
\textlf{Suffix}. The armour grants +1 \textlf{resist} (+ 1 level per 4 added \textlf{difficulty}). Material cost: 22 g plus 10 g per extra point.

\subsubsection{Evasion (11,+4)}
\textlf{Suffix}. The armour grants + 1 \textlf{Deflect} (+ 1 per 4 added \textlf{difficulty}). Material cost: 25 g plus 10 g per extra point.

\subsubsection{Mirrored (13)}
\textlf{Suffix}. This enchantment can only be used upon a shield. While using the shield a wielder may use his \textlf{Deflect} in place of \textlf{Resolve} when making \textlf{Resist} attempts against spells. Material Cost: 27 g.

\subsubsection{Destruction (15)}
\textlf{Suffix}. The armour grants + 1 \textlf{Power} for spells. Material cost: 37 g.

\subsubsection{Attribute (16)}
\textlf{Suffix}. The armour grants the wearer + 1 \textlf{Attribute}. Material cost: 40 g. Where \textlf{Attribute} is a chosen \textlf{Natural Attribute} and the \textlf{suffix} name will be something like Gauntlets of Might. 






\chapter{Magic}
\label{chap:magic}

Any character with sufficient force of will can alter the very fabric of the world around him, this magic is drawn from the inner strength of the character himself, and so is tiring to perform. Characters learn new spells either through being taught them or by spending Hero Experience to unlock them.

\section{Magic Mechanics}

\subsection{Casting a Spell}
Spells are cast as actions like any other. Their complexity means that the caster must succeed ona \textlf{magic} skill check in order to cast correctly. A failure means the spell fizzles, a \textlf{critical failure} means the spell effect goes haywire. What happens next is up to the GM, but a suggestion is that the spell affects the wrong target (the fireball detonates on the caster rather than his intended victim, for example). If the spell is cast successfully, apply the spell description effects. A \textlf{critical success} on casting grants an \textlf{edge} bonus on subsequent rolls made as part of the spell.  

\subsection{Long casts}
If a spell costs 2 or more action points then if the caster suffers any successful damage rolls during the round, he has an \textlf{edge} penalty on all subsequent rolls associated with the spell.

\subsection{Resist}
A Resist check involves the victim making a \textlf{Resolve} check against \dicediffbase + the caster's magic \textlf{power}. 

\subsection{Dispel}
A caster can use his spell-weaving skills to nullify the spells cast by another magic user. This can be used in place of \textlf{Resist} or as 1 action point prepared action, it involves a \textlf{magic} check with the score the spell was cast on as \textlf{difficulty}. If the check succeeds the spell is nullified.

\subsection{Deflect}
Spells which fire a projectile or project over an area allow the victim the chance to dodge, this involves a \textlf{Deflect} check using the caster's magic \textlf{aim} score.

\subsection{Persistent spells}
These spells have a long duration but concentration must be maintained to keep them functioning so only one can be functioning at a time. If a character is disabled or killed then his \textlf{persistent} spell ends. Otherwise if he suffers a successful damage roll he must make a \textlf{resolve} check against $\dicediffbase$ plus the \textlf{power} of the attack, if he fails the \textlf{persistent} spell ends.





\subimport{./}{spells_v10.tex}




\subimport{./}{fantasy_gear.tex}




\listoftables


\end{document}
